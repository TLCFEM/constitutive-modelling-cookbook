\chapter{Metal}
In this chapter, several frameworks suitable for developing metal models are introduced. The basic one is the von Mises framework, which is also called J2 model in some literature as it adopts the second invariant of the deviatoric stress to characterise yield function. The intermediate one is the VAFCRP model, in which a Voce type nonlinear isotropic hardening, a multiplicative Armstrong--Fredrick type kinematic hardening and a Peric type viscosity are implemented. Thus, this model can account for dynamic effects. The third model is a general framework developed based on the Hoffman criterion, it is suitable for orthotropic materials.
\section{von Mises Framework}
Here the uniaxial combined isotropic/kinematic model introduced in \secref{sec:isotropic/kinematic} is reformulated in 3D space. Some difference will be observed, but the final local residual is a scalar equation.
\subsection{Theory}
\subsubsection{Yield Function}
The von Mises yield criterion is adopted,
\begin{gather}
f=\norm{\beeta}-\sqrt{\dfrac{2}{3}}\sigma^y,
\end{gather}
with $\beeta=\bs-\balpha$ is the shifted stress with $\balpha$ denoting the back stress and $\bs=\dev{\bsigma}$ denoting the deviatoric stress.
By definition, the back stress $\balpha$ is also a deviatoric stress, thus $\tr{\balpha}=0$.
It shifts the centre of yield surface in the deviatoric stress space.

The only purpose of the constant $\sqrt{\dfrac{2}{3}}$ is to produce consistent response under uniaxial loading with the same set of model parameters.
To see this point, consider the uniaxial loading case, which can be represented by the following stress tensor
\begin{gather}
    \bsigma=\begin{bmatrix}
        \sigma & 0 & 0 \\
        0      & 0 & 0 \\
        0      & 0 & 0
    \end{bmatrix},
\end{gather}
in absence of $\balpha$, the shifted stress $\beeta$ is
\begin{gather}
    \beeta=\begin{bmatrix}
        \dfrac{2}{3}\sigma & 0                   & 0                   \\
        0                  & -\dfrac{1}{3}\sigma & 0                   \\
        0                  & 0                   & -\dfrac{1}{3}\sigma
    \end{bmatrix},
\end{gather}
thus, its norm can be computed as
\begin{gather}
    \norm{\beeta}=\sqrt{\left(\dfrac{2}{3}\right)^2+\left(\dfrac{1}{3}\right)^2+\left(\dfrac{1}{3}\right)^2}\sigma=\sqrt{\dfrac{2}{3}}\sigma.
\end{gather}
Inserting it back, the yield function effectively defines $\sigma=\sigma^y$.

There are many equivalent alternative forms of the von Mises yield function.
For example, one can also write it as
\begin{gather}
    f=\sqrt{\dfrac{3}{2}}\norm{\beeta}-\sigma^y=\sqrt{3J_2}-\sigma^y,\qquad
    f=3J_2-\left(\sigma^y\right)^2,
\end{gather}
given that $J_2=\dfrac{1}{2}\beeta:\beeta$.
One may find all those forms (and maybe more) in the literature.
In principle, one can choose any of those forms if the yield function is the only factor considered.
However, accounting for the associated numerical implementation, which involves computing the derivative of $f$, some forms are more convenient than the others.
\subsubsection{Flow Rule}
Assuming associative rule, the flow rule is
\begin{gather}\label{eq:j2_flow}
\dot{\bvarepsilon^p}=\gamma\pdfrac{f}{\bsigma}=\gamma\pdfrac{\norm{\beeta}}{\bsigma}=\gamma\dfrac{1}{2}\dfrac{2\mathbb{I}^\text{dev}:\beeta}{\norm{\beeta}}=\gamma\dfrac{\beeta}{\norm{\beeta}}=\gamma\bn.
\end{gather}
All analytical formulations are based on tensor notation.
If the above expression is expressed explicitly in tensor components, one could obtain
\begin{gather}
\underbrace{\begin{bmatrix}
\dot{\varepsilon^p}_{11}&\dot{\varepsilon^p}_{12}&\dot{\varepsilon^p}_{31}\\
\dot{\varepsilon^p}_{12}&\dot{\varepsilon^p}_{22}&\dot{\varepsilon^p}_{23}\\
\dot{\varepsilon^p}_{31}&\dot{\varepsilon^p}_{23}&\dot{\varepsilon^p}_{33}
\end{bmatrix}}_{\dot{\bvarepsilon^p}}
=
\gamma
\underbrace{\begin{bmatrix}
n_{11}&n_{12}&n_{31}\\
n_{12}&n_{22}&n_{23}\\
n_{31}&n_{23}&n_{33}
\end{bmatrix}}_{\bn},\qquad\text{in tensor components.}
\end{gather}
We define the scaling vector $\mb{c}=\begin{bmatrix}
1&1&1&2&2&2
\end{bmatrix}^\mT$ and let $\circ$ be the Hadamard (element--wise) product operator, then the above expression is equivalent to
\begin{gather}
\underbrace{\begin{bmatrix}
\dot{\varepsilon^p}_{11}\\
\dot{\varepsilon^p}_{22}\\
\dot{\varepsilon^p}_{33}\\
2\dot{\varepsilon^p}_{12}\\
2\dot{\varepsilon^p}_{23}\\
2\dot{\varepsilon^p}_{31}
\end{bmatrix}}_{\dot{\bvarepsilon^p}}
=
\gamma\begin{bmatrix}
1&&&&&\\
&1&&&&\\
&&1&&&\\
&&&2&&\\
&&&&2&\\
&&&&&2
\end{bmatrix}
\underbrace{
\begin{bmatrix}
n_{11}\\
n_{22}\\
n_{33}\\
n_{12}\\
n_{23}\\
n_{31}
\end{bmatrix}}_{\bn}=\gamma~\mb{c}\circ\bn,\qquad\text{in the Voigt notation.}
\end{gather}
This agrees with \eqsref{eq:norm_derivative2}.
Such a difference has practical implications.
Due to symmetry of both strain and stress tensors, it is often more performant to just use vectors and matrices to represent second--order and fourth--order tensors in implementation.
This means tensor quantities are expressed in compressed vector/matrix form.
For a given expression, one shall always question what operation it implies and how it would be eventually computed for operands expressed in different forms.
For example, if $\dot{\bvarepsilon^p}$ is used in intermediate computation, is can simply be replaced by $\gamma\bn$ --- the mathematics would be correct as long as we are using the consistent tensor notation.
However, if $\bvarepsilon^p$ needs to be stored in the Voigt notation, one shall carefully scale the shear components by a factor of $2$ when committing new $\bvarepsilon^p$.

It can be observed that $\dot{\bvarepsilon^p}$ has a magnitude of $\gamma$ while $\bn$ is a unit tensor in $\mathbb{R}^3\times\mathbb{R}^3$ hyper space. Thus $\bn$ serves as a direction indicator, $\bvarepsilon^p$ evolves towards $\bn$ by the amount of $\gamma$. Since $\bn$ is deviatoric, $\dot{\bvarepsilon^p}$ is also deviatoric (so as $\bvarepsilon^p$).
Furthermore, since $\bn$ is a unit tensor,
\begin{gather}
\norm{\dot{\bvarepsilon^p}}=\gamma.
\end{gather}
\subsubsection{Hardening Law}
For internal variable $q$, the hardening law takes the accumulated magnitude of $\bvarepsilon^p$.
\begin{gather}
\dot{q}=\sqrt{\dfrac{2}{3}}\norm{\dot{\bvarepsilon^p}}=\sqrt{\dfrac{2}{3}}\gamma.
\end{gather}

For isotropic hardening, $\sigma^y$ is defined as a general function of $q$,
\begin{gather}
\sigma^y=\sigma^y\left(q\right).
\end{gather}

For kinematic hardening, the evolution law of back stress $\balpha$ is defined to be
\begin{gather}
\dot{\balpha}=\sqrt{\dfrac{2}{3}}\dot{H}\bn.
\end{gather}
in which $H=H\left(q\right)$ is now a scalar--valued function of $q$ that controls the development of $\norm{\balpha}$, $\balpha$ always evolves towards $\bn$ by the amount $\dot{H}=H\left(q_{n+1}\right)-H\left(q_n\right)$ characterising the increment of $H$. The fraction $\sqrt{\dfrac{2}{3}}$ is introduced for consistent response as stated earlier. Since $\dot{\balpha}$ and $\bn$ are coaxial, $\balpha$ stays deviatoric but may not be coaxial with all $\bn$ through the loading process.
\subsection{Formulation}
The summation of the von Mises model is listed as follows.
\begin{table}[ht]
\centering
\begin{tabular}{rl}
\toprule
Constitutive Law&$\bsigma=\mb{D}:\left(\bvarepsilon-\bvarepsilon^p\right)$\\
Yield Function&$f=\norm{\beeta}-\sqrt{\dfrac{2}{3}}\sigma^y$\\
Flow Rule&$\dot{\bvarepsilon^p}=\gamma\bn$\\
Hardening Law&$\dot{q}=\sqrt{\dfrac{2}{3}}\gamma$\\
&$\dot{\balpha}=\sqrt{\dfrac{2}{3}}\dot{H}\bn$\\\bottomrule
\end{tabular}
\end{table}
\subsubsection{Elastic Loading/Unloading}
The trial stress can be computed such as
\begin{gather}\label{eq:j2_tsigma}
\bsigma^\text{trial}=\mb{D}:\left(\bvarepsilon_{n+1}-\bvarepsilon^p_n\right)=\bsigma_n+\mb{D}:\left(\bvarepsilon_{n+1}-\bvarepsilon_n\right).
\end{gather}
In matrix representation, it is
\begin{gather}
\underbrace{
\begin{bmatrix}
\sigma^\text{trial}_{11}\\
\sigma^\text{trial}_{22}\\
\sigma^\text{trial}_{33}\\
\sigma^\text{trial}_{12}\\
\sigma^\text{trial}_{23}\\
\sigma^\text{trial}_{31}
\end{bmatrix}}_{\bsigma^\text{trial}}
=
\underbrace{\begin{bmatrix}
\lambda+2G&\lambda&\lambda&\cdot&\cdot&\cdot\\
\lambda&\lambda+2G&\lambda&\cdot&\cdot&\cdot\\
\lambda&\lambda&\lambda+2G&\cdot&\cdot&\cdot\\
\cdot&\cdot&\cdot&2G&\cdot&\cdot\\
\cdot&\cdot&\cdot&\cdot&2G&\cdot\\
\cdot&\cdot&\cdot&\cdot&\cdot&2G
\end{bmatrix}}_{\mb{D}}
\underbrace{\left(\begin{bmatrix}
        \varepsilon_{11,n+1} \\
        \varepsilon_{22,n+1} \\
        \varepsilon_{33,n+1} \\
        \varepsilon_{12,n+1} \\
        \varepsilon_{23,n+1} \\
        \varepsilon_{31,n+1}
    \end{bmatrix}-\begin{bmatrix}
        \varepsilon^p_{11,n} \\
        \varepsilon^p_{22,n} \\
        \varepsilon^p_{33,n} \\
        \varepsilon^p_{12,n} \\
        \varepsilon^p_{23,n} \\
        \varepsilon^p_{31,n}
    \end{bmatrix}
    \right).}_{\bvarepsilon_{n+1}-\bvarepsilon^p_n}
\end{gather}
Using engineering strains, the above must be adjusted to
\begin{gather}
\underbrace{
    \begin{bmatrix}
        \sigma^\text{trial}_{11} \\
        \sigma^\text{trial}_{22} \\
        \sigma^\text{trial}_{33} \\
        \sigma^\text{trial}_{12} \\
        \sigma^\text{trial}_{23} \\
        \sigma^\text{trial}_{31}
    \end{bmatrix}}_{\bsigma^\text{trial}}
=
\underbrace{\begin{bmatrix}
\lambda+2G&\lambda&\lambda&\cdot&\cdot&\cdot\\
\lambda&\lambda+2G&\lambda&\cdot&\cdot&\cdot\\
\lambda&\lambda&\lambda+2G&\cdot&\cdot&\cdot\\
\cdot&\cdot&\cdot&G&\cdot&\cdot\\
\cdot&\cdot&\cdot&\cdot&G&\cdot\\
\cdot&\cdot&\cdot&\cdot&\cdot&G
\end{bmatrix}}_{\mb{D}\text{ or conventionally }\mb{E}}
\underbrace{\left(\begin{bmatrix}
\varepsilon_{11,n+1}\\
\varepsilon_{22,n+1}\\
\varepsilon_{33,n+1}\\
     \gamma_{12,n+1}\\
     \gamma_{23,n+1}\\
     \gamma_{31,n+1}
\end{bmatrix}-\begin{bmatrix}
\varepsilon^p_{11,n}\\
\varepsilon^p_{22,n}\\
\varepsilon^p_{33,n}\\
     \gamma^p_{12,n}\\
     \gamma^p_{23,n}\\
     \gamma^p_{31,n}
\end{bmatrix}
\right).}_{\bvarepsilon_{n+1}-\bvarepsilon^p_n}
\end{gather}

Then $\beeta^\text{trial}=\dev{\bsigma^\text{trial}}-\balpha_n$, which gives trial yield function
\begin{gather}\label{eq:j2_tf}
f^\text{trial}=\norm{\beeta^\text{trial}}-\sqrt{\dfrac{2}{3}}\sigma^y_n
\end{gather}
with $\sigma^y_n=\sigma^y\left(q_n\right)$ evaluated with current $q_n$.
The above expression simply assumes the current loading increment $\Delta\bvarepsilon=\bvarepsilon_{n+1}-\bvarepsilon_n$ is purely elastic, thus all plasticity related variables remain unchanged.
The yield function evaluated in this way gives an upper bound of the actual yield function.
If this value is negative, the actual state must be elastic, otherwise, plasticity evolution must occur.
\subsubsection{Plastic Evolution}
By default, we present the formulation with the implicit Euler method.

If plasticity evolution occurs, that is, $f^\text{trial}>0$, $\gamma$ must be non-zero and thus needs to be solved.
\begin{gather}
\beeta_{n+1}=\dev{\bsigma_{n+1}}-\balpha_{n+1}.
\end{gather}
Accounting for
\begin{gather}
\bsigma_{n+1}=\bsigma^\text{trial}-\mb{D}:\dot{\bvarepsilon^p}=\bsigma^\text{trial}-\gamma2G\bn,\\
\balpha_{n+1}=\balpha_n+\sqrt{\dfrac{2}{3}}\dot{H}\bn,
\end{gather}
one can further derive
\begin{gather}
\beeta_{n+1}=\dev{\bsigma^\text{trial}-\gamma2G\bn}-\balpha_n-\sqrt{\dfrac{2}{3}}\dot{H}\bn.
\end{gather}
Since $\bn$ is already deviatoric, meaning $\dev{\bn}=\bn$, then,
\begin{gather}
\begin{split}
\beeta_{n+1}&=\dev{\bsigma^\text{trial}}-\gamma2G\bn-\balpha_n-\sqrt{\dfrac{2}{3}}\dot{H}\bn\\
&=\beeta^\text{trial}-\left(2G\gamma+\sqrt{\dfrac{2}{3}}\dot{H}\right)\bn.
\end{split}
\end{gather}
Noting that, by definition, $\bn=\dfrac{\beeta_{n+1}}{\norm{\beeta_{n+1}}}$, thus,
\begin{gather}
\underbrace{\left(1+\dfrac{2G\gamma+\sqrt{\dfrac{2}{3}}\dot{H}}{\norm{\beeta_{n+1}}}\right)}_{\text{scalar}}\beeta_{n+1}=\beeta^\text{trial}.
\end{gather}
The above expression implies that $\beeta_{n+1}$ and $\beeta^\text{trial}$ are \textbf{coaxial}. Thus, $\beeta_{n+1}$ always points to the direction of $\beeta^\text{trial}$, which is a constant in each time step within that time step.
Only its magnitude would be adjusted in local iterations.
Thus, $\bn=\dfrac{\beeta_{n+1}}{\norm{\beeta_{n+1}}}=\dfrac{\beeta^\text{trial}}{\norm{\beeta^\text{trial}}}$ is also a constant during the local iteration.
This is a quite convenient property, although it may not exist in other complex models.
However, when it is available, it can be exploited to simplify the formulation.
One can further derive
\begin{gather}
\norm{\beeta_{n+1}}+2G\gamma+\sqrt{\dfrac{2}{3}}\dot{H}=\norm{\beeta^\text{trial}}.
\end{gather}

The yield function evaluated at the new state thus reads
\begin{gather}\label{eq:j2_f}
f=\norm{\beeta^\text{trial}}-\left(2G\gamma+\sqrt{\dfrac{2}{3}}\left(H_{n+1}-H_n\right)\right)-\sqrt{\dfrac{2}{3}}\sigma^y_{n+1}.
\end{gather}

The Jacobian reads
\begin{gather}
\pdfrac{f}{\gamma}=-2G-\sqrt{\dfrac{2}{3}}\ddfrac{H_{n+1}}{q_{n+1}}\ddfrac{q_{n+1}}{\gamma}-\sqrt{\dfrac{2}{3}}\ddfrac{\sigma^y_{n+1}}{q_{n+1}}\ddfrac{q_{n+1}}{\gamma}.
\end{gather}
Since $q_{n+1}=q_n+\sqrt{\dfrac{2}{3}}\gamma$, $\ddfrac{q_{n+1}}{\gamma}=\sqrt{\dfrac{2}{3}}$. Hence,
\begin{gather}\label{eq:j2_df}
\pdfrac{f}{\gamma}=-2G-\dfrac{2}{3}\ddfrac{H_{n+1}}{q_{n+1}}-\dfrac{2}{3}\ddfrac{\sigma^y_{n+1}}{q_{n+1}}.
\end{gather}
\subsubsection{Consistent Tangent Stiffness}
From $\bsigma_{n+1}=\bsigma^\text{trial}-\gamma2G\bn$, as $\bn=\dfrac{\beeta}{\norm{\beeta}}=\dfrac{\beeta^\text{trial}}{\norm{\beeta^\text{trial}}}$, the consistent tangent stiffness can be computed via the chain rule as
\begin{gather}
\pdfrac{\bsigma_{n+1}}{\bvarepsilon_{n+1}}=\pdfrac{\bsigma_{n+1}^\text{trial}}{\bvarepsilon_{n+1}}-2G\pdfrac{\left(\gamma\bn\right)}{\bvarepsilon_{n+1}}=\mb{D}-2G\left(\bn\otimes\pdfrac{\gamma}{\bvarepsilon_{n+1}}+\gamma\pdfrac{\bn}{\bvarepsilon_{n+1}}\right).
\end{gather}
In which, according to \eqsref{eq:unit_derivative},
\begin{gather}
\begin{split}
\pdfrac{\bn}{\bvarepsilon_{n+1}}&=\dfrac{1}{\norm{\beeta^\text{trial}}}\left(\mathbb{I}-\bn\otimes\bn\right):\pdfrac{\mb{\beeta}^\text{trial}}{\bvarepsilon_{n+1}}\\
&=\dfrac{2G}{\norm{\beeta^\text{trial}}}\left(\mathbb{I}-\bn\otimes\bn\right):\mathbb{I}^\text{dev}\\
&=\dfrac{2G}{\norm{\beeta^\text{trial}}}\left(\mathbb{I}^\text{dev}-\bn\otimes\bn\right).
\end{split}
\end{gather}
Here, $\bn$ is expressed as $\bn=\dfrac{\beeta^\text{trial}}{\norm{\beeta^\text{trial}}}$ due to coaxiality between $\beeta_{n+1}$ and $\beeta^\text{trial}$.
This simplifies the computation.

From converged local residual (yield function),
\begin{gather}
\begin{split}
\pdfrac{\gamma}{\bvarepsilon_{n+1}}&=-\left(\pdfrac{f}{\gamma}\right)^{-1}\pdfrac{f}{\bvarepsilon_{n+1}}=-\left(\pdfrac{f}{\gamma}\right)^{-1}2G\bn.
\end{split}
\end{gather}

Thus the final expression for consistent tangent stiffness can be assembled as
\begin{gather}\label{eq:j2_stiffness}
\begin{split}
\pdfrac{\bsigma_{n+1}}{\bvarepsilon_{n+1}}&=\mb{D}-2G\left(-2G\left(\pdfrac{f}{\gamma}\right)^{-1}\bn\otimes\bn+\gamma\dfrac{2G}{\norm{\beeta^\text{trial}}}\left(\mathbb{I}^\text{dev}-\bn\otimes\bn\right)\right)\\
&=\mb{D}+4G^2\left(\pdfrac{f}{\gamma}\right)^{-1}\bn\otimes\bn+\dfrac{4G^2\gamma}{\norm{\beeta^\text{trial}}}\left(\bn\otimes\bn-\mathbb{I}^\text{dev}\right)\\
&=\mb{D}+4G^2\left(\left(\pdfrac{f}{\gamma}\right)^{-1}+\dfrac{\gamma}{\norm{\beeta^\text{trial}}}\right)\bn\otimes\bn-\dfrac{4G^2\gamma}{\norm{\beeta^\text{trial}}}\mathbb{I}^\text{dev}\\
&=\mb{D}+4G^2\left(\dfrac{\gamma}{\norm{\beeta^\text{trial}}}-\dfrac{1}{2G+\dfrac{2}{3}\ddfrac{H_{n+1}}{q_{n+1}}+\dfrac{2}{3}\ddfrac{\sigma^y_{n+1}}{q_{n+1}}}\right)\bn\otimes\bn-\dfrac{4G^2\gamma}{\norm{\beeta^\text{trial}}}\mathbb{I}^\text{dev}.
\end{split}
\end{gather}
It is a symmetric tensor/matrix.
It shall be noted that $\mathbb{I}^\text{dev}$ takes the form as presented in \eqsref{eq:dev_tensor_se}.
Readers are strongly suggested to derive it via both tensor notation and compressed matrix representation as a practice.
Both shall lead to identical results.

Since the local iteration is a scalar function, the closed--form of consistent tangent stiffness is relatively easy to compute. It will be seen in more complex models that closed--forms do not always possess simple forms.

As a general framework, the above formulation does not require explicit forms of $H\left(q\right)$ and $\sigma^y\left(q\right)$. Thus, various types of scalar--valued functions can be adopted.
\subsection{Implementation}
The state determination algorithm of this general model incorporating von Mises criterion is given in \algoref{algo:j2_model}.
\begin{breakablealgorithm}
\caption{state determination of general von Mises model}\label{algo:j2_model}
\begin{algorithmic}
\State \textbf{Parameter}: $\lambda$, $G$
\State \textbf{Input}: $\bvarepsilon_{n+1}$, $\bvarepsilon_n$, $\bvarepsilon^p_n$, $\bsigma_n$, $\balpha_n$, $q_n$
\State \textbf{Output}: $\mb{D}_{n+1}$, $\bvarepsilon^p_{n+1}$, $\bsigma_{n+1}$, $\balpha_{n+1}$, $q_{n+1}$
\State compute $\bsigma^\text{trial}$, $\beeta^\text{trial}$, $\bn$ and $f^\text{trial}$\Comment{\eqsref{eq:j2_tsigma} and \eqsref{eq:j2_tf}}
\If {$f^\text{trial}\geqslant0$}
\State $\gamma=0$
\While{true}
\State compute $f$ and $\pdfrac{f}{\gamma}$\Comment{\eqsref{eq:j2_f} and \eqsref{eq:j2_df}}
\State update $\Delta{}H=H\left(q_{n+1}\right)-H\left(q_n\right)$
\State $\Delta\gamma=\left(\pdfrac{f}{\gamma}\right)^{-1}f$
\If {$\abs{\Delta\gamma}<\text{tolerance}$}
\State break
\EndIf
\State $\gamma\leftarrow\gamma-\Delta\gamma$
\State $q_{n+1}=q_n+\sqrt{\dfrac{2}{3}}\gamma$
\EndWhile
\State $\bsigma_{n+1}=\bsigma^\text{trial}-\gamma2G\bn$
\State $\bvarepsilon^p_{n+1}=\bvarepsilon^p_n+\gamma\bn$
\State $\balpha_{n+1}=\balpha_n+\sqrt{\dfrac{2}{3}}\Delta{}H\bn$
\State $q_{n+1}=q_n+\sqrt{\dfrac{2}{3}}\gamma$
\State compute $\mb{D}_{n+1}$\Comment{\eqsref{eq:j2_stiffness}}
\Else
\State $\bsigma_{n+1}=\bsigma^\text{trial}$
\State $\bvarepsilon^p_{n+1}=\bvarepsilon^p_n$
\State $\balpha_{n+1}=\balpha_n$
\State $q_{n+1}=q_n$
\State $\mb{D}_{n+1}=\mb{D}$
\EndIf
\end{algorithmic}
\end{breakablealgorithm}
It shall be noted that the algorithm does not implement $H\left(q\right)$ and $\sigma^y\left(q\right)$. It is assumed those two functions are defined somewhere else and are able to provide derivatives.

Depending on the choice of updating stress as in \eqsref{eq:j2_tsigma}, the history of $\bvarepsilon^p$ is \textbf{not} compulsory in this model if the latter expression is used instead of the former.
\subsection{Closing Remarks}
As the first 3D material model introduced, the von Mises framework allows beginners to get familiar with multiaxial constitutive modelling with a relatively smooth learning curve. The formulation is expressed in tensor notation. Readers are strongly encouraged to derive the formulation from three governing equations independently in both tensor and compressed matrix notions separately. It is a good practice to get each tiny detail correct.
\begin{cppcode}
NonlinearJ2::update_trial_status
\end{cppcode}

Finally, in \eqsref{eq:j2_flow}, we have discussed that the same expression
\begin{gather}
\dot{\bvarepsilon^p}=\gamma\bn
\end{gather}
may have different explicit forms, depending on how those quantities are represented.

The $\dot{\bvarepsilon^p}$ is further used in
\begin{gather}\label{eq:j2_two_form}
\bs^\text{trial}-\bs_{n+1}=2G\dot{\bvarepsilon^p}.
\end{gather}
The above expression assumes the tensor notation.
If the matrix notation is used, it can be explicitly written as
\begin{gather}
\begin{bmatrix}
s^\text{trial}_{11}-s_{11,n+1}\\
s^\text{trial}_{22}-s_{22,n+1}\\
s^\text{trial}_{33}-s_{33,n+1}\\
s^\text{trial}_{12}-s_{12,n+1}\\
s^\text{trial}_{23}-s_{23,n+1}\\
s^\text{trial}_{31}-s_{31,n+1}
\end{bmatrix}=\gamma\begin{bmatrix}
2G&&&&&\\
&2G&&&&\\
&&2G&&&\\
&&&G&&\\
&&&&G&\\
&&&&&G
\end{bmatrix}\diag{\mb{c}}\bn.
\end{gather}
One immediately realises that
\begin{gather}
\diag{\begin{matrix}
2G\\2G\\2G\\G\\G\\G
\end{matrix}}\circ\diag{\mb{c}}\bn=\diag{\begin{matrix}
2G\\2G\\2G\\2G\\2G\\2G
\end{matrix}}\bn=2G\bn,
\end{gather}
thus, \eqsref{eq:j2_two_form} becomes
\begin{gather}
\bs^\text{trial}-\bs_{n+1}=\gamma2G\bn.
\end{gather}
No matter which notation is used, the same expression holds.
This is based on the fact that $\dot{\bvarepsilon^p}$ is used as an intermediate variable.
The intermediate scaling matrices cancel out and eventually the same expression is obtained.

This is a beautiful attribute, though could be confusing for beginners.
\section{Extended Subloading Surface Model}\label{sec:subloading}
The model is published \cite{Chang2025}.
\subsection{Subloading Surface}
Conventional plasticity models often adopt the concept of a yield surface, denoted by $f\left(\bsigma,\cdots\right)$, which defines the frontier/boundary of plasticity.
For example,
\begin{gather}
    f=\norm{\beeta}-F,
\end{gather}
in which $\beeta=\beeta\left(\bsigma,\cdots\right)$ is known as the shifted stress that shall be a function of the stress tensor $\bsigma$ and other internal variables, $F=F\left(q\right)$ characterises the size of this surface and could evolve with the development of plasticity driven by some internal variable $q$.
Within the yield surface $f$, the material is elastic, while plasticity would develop once the stress state reaches and exceeds $f$, in which case, the surface itself would also expand to enclose the current stress state accordingly.
Thus, any admissible state shall satisfy the inequality $f\leqslant0$.

The original proposal of the subloading surface model introduces a new internal variable, named as the normal yield ratio, denoted by $z$ in this work, which effectively scales the yield surface $f$ down by a factor $z$, the new surface is call the subloading surface $f_s$ and it is concentric with the yield surface $f$.
\begin{gather}
    f_s=\norm{\beeta}-zF.
\end{gather}
The normal yield ratio $0\leqslant{}z\leqslant{}1$ shall be updated in a manner such that $f_s=0$ is enforced not only for plastic loading but also for elastic unloading.
With such a construct, any given stress state would always lie on the subloading surface, meaning that $f_s=0$ is always satisfied.

The main shortcoming of the original subloading surface model is that it exhibits excessive strain accumulation under cyclic loading.
To address this issue, an additional internal field, denoted by $\bc$, is adopted to define the elastic core surface in the extended subloading surface model, which is concentric with the yield surface (centred at the back stress $\bbeta$).
The subloading surface is then pulled from the original centre $\bbeta$ towards $\bc$ by a certain amount governed by the normal yield ratio $z$.
\begin{figure}[htb]
    \centering
    \begin{tikzpicture}[>=latex]
        \newcommand{\FF}{4}
        \newcommand{\SF}{2.5}
        \draw[->](-.5,0)--(3,0);
        \draw[->](0,-.5)--(0,3);
        \coordinate(A)at(2.8,3);
        \coordinate(B)at($(A)+(10:3)$);
        \coordinate(C)at($(B)!\SF/\FF!(A)$);
        \node[fill=black,circle,inner sep=0,minimum size=2mm]at(A){};
        \node[fill=black,circle,inner sep=0,minimum size=2mm]at(B){};
        \node[fill=black,circle,inner sep=0,minimum size=2mm]at(C){};
        \draw[dashed,draw,line width=.3mm](A)circle(\FF);
        \draw[dotted,draw,line width=.3mm](A)circle(\SF);
        \draw[draw,line width=.6mm](C)circle(\SF);
        \draw[->,d73027,line width=.4mm](0,0)--(A)node[midway,fill=white,sloped]{$\bbeta$};
        \draw[->,d73027,line width=.6mm](A)--(B);
        \draw[->,d73027,line width=.4mm](0,0)--(B)node[midway,fill=white,sloped]{$\bc$};
        \draw[->,4575b4,line width=.8mm](A)--(C);
        \draw[|<->|](A)--++(140:\FF)node[midway,fill=white,sloped]{$F$};
        \draw[|<->|](C)--++(140:\SF)node[midway,fill=white,sloped]{$zF$};
        \draw[<-]($(A)+(60:\FF)$)--++(60:1.4)--++(1,0)node[right,fill=white]{yield surface $f$};
        \draw[<-]($(A)+(60:\SF)$)--++(60:.8)--++(2,0)node[right,fill=white,align=left]{subloading surface\\in the original proposal};
        \draw[<-]($(C)+(60:\SF)$)--++(60:1.8)--++(1,0)node[right,fill=white,align=left]{subloading surface\\in the extended version};
        \draw[<-]($(C)!.5!(B)$)--++(-60:1.5)--++(2,0)node[right,fill=white]{$z\left(\bc-\bbeta\right)$};
        \draw[<-]($(A)!.5!(C)$)--++(-60:2)--++(2.5,0)node[right,fill=white]{$\left(1-z\right)\left(\bc-\bbeta\right)$};
        \draw[black!20]($(A)+(140:\FF)$)--(B);
    \end{tikzpicture}
    \caption{graphical representation of the surfaces used in the subloading surface model}\label{fig:surfaces}
\end{figure}
A graphical illustration is shown in \figref{fig:surfaces}.
By definition, $\bc$ denotes the homothetic centre between $f$ and $f_s$.
Then, according to similarity, the new centre of the subloading surface is $\bbeta+\left(1-z\right)\left(\bc-\bbeta\right)$, that is $z\bbeta+\left(1-z\right)\bc$.
\subsection{Original Evolution Rules}
The evolution rule for the elastic core has undergone several iterations, see, for example, the initial version \cite{Hashiguchi1989}, one of the variations \cite{Hashiguchi2015}, the most recent version \cite{Anjiki2019}.
The original formulation further imposes the requirement that the size of the elastic core shall be limited to a certain fraction of that of the yield surface.
This is achieved by the following evolution rule for $\bc$ \cite{Anjiki2019,Hashiguchi2023},
\begin{gather}\label{eq:original_core}
    \dot{\bc}=c_e\norm{\dot{\bvarepsilon^p}}\left(z_eF\bn-\left(\bc-\bbeta\right)\right)+\dot{\bbeta}+\dfrac{\dot{F}}{F}\left(\bc-\bbeta\right).
\end{gather}
In which, $0\leqslant{}c_e$ is a constant that controls the evolution rate, $0\leqslant{}z_e\leqslant1$ is a constant that controls the limit size of the elastic core and $\bn=\pdfrac{f_s}{\bsigma}/\norm{\pdfrac{f_s}{\bsigma}}=\dfrac{\beeta}{\norm{\beeta}}$ is unit gradient of the subloading surface.
It shall be pointed out that the term $z_eF\bn$ is expressed as $z_e\dfrac{\beeta}{z}$ in the original formulation.
They are identical due to the fact that $\norm{\beeta}=zF$ since $f_s\equiv0$ while $F\bn$ has clear numerical advantages over $\dfrac{\beeta}{z}$.

The evolution rule for the normal yield ratio $z$ is given by
\begin{gather}\label{eq:original_z}
    \dot{z}=U\dot{q},
\end{gather}
where $U=U\left(z\right)$ is a function that shall satisfy certain conditions, and $q$ is the accumulated plastic strain.

Both of the above evolution rules will be discussed and refined in the following sections.
Here, we only present them in their original forms for completeness.
\subsection{Revisions}\label{sec:revised}
\subsubsection{Preliminary Remarks}
In the scalar form, the Armstrong--Frederick evolution rule \cite{Frederick2007} is equivalent to the following ordinary differential equation (ODE),
\begin{equation}\label{eq:af_ode}
    y'=b\left(a-y\right),
\end{equation}
it can be explicitly integrated to give the solution for the initial condition $y(0)=0$,
\begin{equation}
    y(x)=a\left(1-\exp\left(-bx\right)\right),
\end{equation}
which defines a curve that asymptotically approaches the constant bound $a$.

To employ a non-constant bound, it is possible to simply replace the constant $a$ by a function $a(x)$ in the ODE \eqsref{eq:af_ode} such that $y'=b\left(a\left(x\right)-y\right)$.
However, as noted in the literature, the corresponding solution is not strictly bounded by $a(x)$.
\figref{fig:examples} presents both hardening and softening examples to illustrate this point.
\begin{figure}[htb]
    \centering
    \begin{subfigure}{.48\textwidth}
        \centering
        \begin{tikzpicture}
            \begin{axis}[axis lines=middle,xlabel=$x$,ylabel=$y$,grid=both,width=6.5cm,height=4.5cm,legend pos=south east,ymin=0,ymax=1.5,scale only axis,enlargelimits=false]
                \pgfmathsetmacro{\aa}{-.05}
                \pgfmathsetmacro{\bb}{1}
                \addplot[line width=2pt,dashed,domain=0:10,samples=100,color=fc8d59]{\aa*x+\bb};
                \addplot[line width=1pt,domain=0:10,samples=100,color=d73027]{\aa*x+\bb+(\aa-\bb)*exp(-x)-\aa};
                \addlegendentry{$a(x)=\aa{}x+\bb$}
                \addlegendentry{$\dot{y}=a(x)-y$}
                \addlegendentry{$\dot{y}=1-y/a(x)$}
            \end{axis}
        \end{tikzpicture}
        \caption{softening}
    \end{subfigure}\hfill
    \begin{subfigure}{.48\textwidth}
        \centering
        \begin{tikzpicture}
            \begin{axis}[axis lines=middle,xlabel=$x$,ylabel=$y$,grid=both,width=6.5cm,height=4.5cm,legend pos=south east,ymin=0,ymax=1.5,scale only axis,enlargelimits=false]
                \pgfmathsetmacro{\aa}{0.05}
                \pgfmathsetmacro{\bb}{1}
                \addplot[line width=2pt,dashed,domain=0:10,samples=100,color=fc8d59]{\aa*x+\bb};
                \addplot[line width=1pt,domain=0:10,samples=100,color=d73027]{\aa*x+\bb+(\aa-\bb)*exp(-x)-\aa};
                \addlegendentry{$a(x)=\aa{}x+\bb$}
                \addlegendentry{$\dot{y}=a(x)-y$}
                \addlegendentry{$\dot{y}=1-y/a(x)$}
            \end{axis}
        \end{tikzpicture}
        \caption{hardening}
    \end{subfigure}
    \caption{asymptotic behaviour with hardening and softening bounds}\label{fig:examples}
\end{figure}
As can be seen from the figure, the solution may overshoot for softening bounds.
Furthermore, in both cases, the solution $y$ does \textbf{not} asymptotically approach the bound $a(x)$.
That is, the following condition does not hold for arbitrarily small $\epsilon$,
\begin{gather}
    \lim\limits_{x\to\infty}\abs{y-a(x)}<\epsilon.
\end{gather}
It is worth noting that the following form
\begin{gather}
    y'=b\left(1-\dfrac{1}{a\left(x\right)}y\right)=\dfrac{b}{a\left(x\right)}\left(a\left(x\right)-y\right)
\end{gather}
has a similar solution to that of $y'=b\left(a\left(x\right)-y\right)$.
The $y'$ is governed by the difference $a\left(x\right)-y$, while the approaching rate changes from $b$ to $b/a\left(x\right)$.
They are equivalent in terms of the asymptotic behaviour and share the same shortcoming.
This form has been used in defining the evolution rule for the back stress in the subloading surface model \cite{Anjiki2019,Hashiguchi2023,Hashiguchi2024}.

To ensure that the solution is bounded by $a(x)$, let's consider the following function,
\begin{gather}\label{eq:strict_bound_core}
    y=a(x)\left(1-\exp\left(-bx\right)\right),
\end{gather}
since $1-\exp\left(-bx\right)<1$, it is guaranteed that $y<a(x)$.
On the other hand, the corresponding ODE can be derived as follows,
\begin{gather}\label{eq:strict_bound}
    y'=ab\exp\left(-bx\right)+a'\left(1-\exp\left(-bx\right)\right)=b\left(a-y\right)+\dfrac{a'}{a}y.
\end{gather}
After careful comparison, one could observe that \eqsref{eq:strict_bound} is in fact the scalar unidirectional version of the evolution rule presented in \eqsref{eq:original_core}.
They bear the same format.
Thus, one can conclude that \eqsref{eq:strict_bound_core} is mathematically equivalent to \eqsref{eq:strict_bound} as it is the solution to the latter with the initial condition $y(0)=0$, however, the latter is much more cumbersome.

On the other hand, consider the function
\begin{gather}\label{eq:polar_decomposition}
    \bc-\bbeta=F\bd,
\end{gather}
in which $\bd$ is some quantity, its derivative can be expressed as
\begin{gather}
    \dot{\bc}-\dot{\bbeta}=F\dot{\bd}+\dot{F}\bd=F\dot{\bd}+\dfrac{\dot{F}}{F}\left(\bc-\bbeta\right).
\end{gather}
By comparing with \eqsref{eq:original_core}, one could conclude that \eqsref{eq:polar_decomposition} must be mathematically equivalent to \eqsref{eq:original_core} if $\bd$ is governed by the following evolution rule,
\begin{gather}
    \dot{\bd}=c_e\norm{\dot{\bvarepsilon^p}}\left(z_e\bn-\bd\right).
\end{gather}
If the associative flow rule is adopted, the above expression to be further expressed as
\begin{gather}
    \dot{\bd}=c_e\left(z_e\dot{\bvarepsilon^p}-\norm{\dot{\bvarepsilon^p}}\bd\right).
\end{gather}
This is effectively an Armstrong--Frederick type evolution rule for $\bd$.
\subsubsection{Revised Evolution Rule (Elastic Core)}
The evolution rule for $\bc$ shown in \eqsref{eq:original_core} is derived at the limit state where the subloading surface coincides with the limit surface --- which would not occur due to its asymptotic nature.
It possesses a cumbersome form that involves both $F$ and its derivative.
As a result, it is not straightforward to implement numerically.
Instead, based on the above observations, one can apply the following decomposition,
\begin{gather}\label{eq:decomposition_core}
    \boxed{\bc-\bbeta=F\bd\qquad\rightarrow\qquad\bc=\bbeta+F\bd,}
\end{gather}
then the evolution of $\bc$, which characterises a point on the elastic core surface, can be equivalently described by the evolution of $\bd$, which could take a very simple form as the contribution of $F$ is separated,
\begin{gather}\label{eq:revised_core}
    \boxed{\dot{\bd}=c_e\left(z_e\dot{\bvarepsilon^p}-\norm{\dot{\bvarepsilon^p}}\bd\right).}
\end{gather}
Since the magnitude of $\bd$ defined in this manner is bounded by the constant $z_e$, it is then guaranteed that $\norm{\bc-\bbeta}=F\norm{\bd}<z_eF$.
Based on the above discussion, it can be asserted that \eqsref{eq:decomposition_core} with \eqsref{eq:revised_core} is mathematically equivalent to \eqsref{eq:original_core}.
However, such a decomposition separates the directional vector from the scalar magnitude, which allows much simpler evolution rules for both parts, although the vector here is not a unit vector thus does not possess a fixed unit length.
\eqsref{eq:decomposition_core} may be called the `\textit{pseudo-polar coordinate}' representation while $\bd$ may be called the `\textit{similarity vector}' as it points to the similarity centre $\bc$.
\subsubsection{Revised Evolution Rule (Back Stress)}
Now since \eqsref{eq:decomposition_core} is adopted, a similar strategy can also be applied to $\bbeta$ to allow its bound to evolve with the development of plasticity, that is,
\begin{gather}\label{eq:decomposition_back}
    \boxed{\bbeta=H\balpha,}
\end{gather}
with $H=H\left(q\right)$ being the bound and $\balpha$ being the normalised back stress (normalised by $H$) which could have the following evolution rule.
\begin{gather}\label{eq:revised_back}
    \boxed{\dot{\balpha}=b\left(\dot{\bvarepsilon^p}-\norm{\dot{\bvarepsilon^p}}\balpha\right),}
\end{gather}
where $b$ is a constant that controls the evolution rate.
Similar to $\norm{\bc-\bbeta}$, here it is guaranteed that $\norm{\bbeta}=H\norm{\balpha}<H$ as $\norm{\balpha}<1$.

It is possible to set $H$ to a fraction of $F$.
Here, to allow more flexibility, $H$ is defined as an independent quantity.
Nevertheless, if one wishes, it is possible to associate $H$ with $F$ as done in other work \cite{Anjiki2019}.
However, one shall note that the specific form $\bbeta=b\left(\dot{\bvarepsilon^p}-\norm{\dot{\bvarepsilon^p}}\dfrac{\bbeta}{H}\right)$, which is used by \cite{Anjiki2019,Anjiki2021}, shall not be used as the back stress may go beyond the intended bound ($\norm{\bbeta}>H$) for softening materials, see the discussion around \figref{fig:examples}.
\subsubsection{Shifted Stress}
Since now the new centre of the subloading surface is shifted to $\bbeta+\left(1-z\right)\left(\bc-\bbeta\right)$, the original formulation uses the following expression to compute the shifted stress $\beeta$,
\begin{gather}
    \beeta=\bs-\left(\bbeta+\left(1-z\right)\left(\bc-\bbeta\right)\right),
\end{gather}
where $\bs=\bs\left(\bsigma\right)$ is some function of the stress tensor $\bsigma$.
Account for the proposed decompositions \eqsref{eq:decomposition_core} and \eqsref{eq:decomposition_back}, the above expression can be rewritten as
\begin{gather}
    \boxed{\beeta=\bs-H\balpha+\left(z-1\right)F\bd,}
\end{gather}
\figref{fig:revised_surfaces} presents a graphical representation of the quantities used in this revised formulation.
\begin{figure}[htb]
    \centering
    \begin{tikzpicture}[>=latex]
        \newcommand{\DD}{80}
        \newcommand{\FF}{4}
        \newcommand{\SF}{2.5}
        \draw[->](-.5,0)--(3,0);
        \draw[->](0,-.5)--(0,3);
        \coordinate(A)at(2.8,.4);
        \coordinate(B)at($(A)+(30:3)$);
        \coordinate(C)at($(B)!\SF/\FF!(A)$);
        \coordinate(D)at($(C)+(\DD:\SF)$);
        \node[fill=black,circle,inner sep=0,minimum size=2mm]at(A){};
        \node[fill=black,circle,inner sep=0,minimum size=2mm]at(B){};
        \node[fill=black,circle,inner sep=0,minimum size=2mm]at(C){};
        \node[fill=black,circle,inner sep=0,minimum size=2mm]at(D){};
        \draw[dashed,draw,line width=.3mm](A)circle(\FF);
        \draw[draw,line width=.6mm](C)circle(\SF);
        \draw[->,d73027,line width=1mm](0,0)--(A)node[midway,fill=white,sloped]{$\bbeta=H\balpha$};
        \draw[->,dotted,d73027,line width=.4mm](C)--(B);
        \draw[->,d73027,line width=1mm](A)--(C);
        \draw[->,d73027,line width=1mm](C)--(D)node[midway,fill=white]{$\beeta$};
        \draw[->,d73027,line width=1mm](0,0)--(D)node[midway,fill=white,sloped]{$\bs$};
        \draw[|<->|](C)--++(50:\SF)node[midway,fill=white,sloped]{$zF$};
        \draw[<-]($(A)+(60:\FF)$)--++(60:1.4)--++(1,0)node[right,fill=white]{yield surface $f$};
        \draw[<-]($(C)+(60:\SF)$)--++(60:1.7)--++(1,0)node[right,fill=white]{subloading surface};
        \draw[<-]($(C)!.5!(B)$)--++(-60:1.5)--++(2,0)node[right,fill=white]{$zF\bd$};
        \draw[<-]($(A)!.5!(C)$)--++(-60:2)--++(2.5,0)node[right,fill=white]{$\left(1-z\right)F\bd$};
    \end{tikzpicture}
    \caption{graphical representation of the surfaces using proposed definitions}\label{fig:revised_surfaces}
\end{figure}
It could also be noted that, since $\norm{\bd}<1$, then $\norm{zF\bd}<zF$, the homothetic centre is always inside the subloading surface $f_s$, which itself is inside the yield surface $f$.
\subsection{Loading/Unloading Criterion}\label{sec:correct_loading_criterion}
The conventional Kuhn--Tucker condition cannot be applied in the subloading surface model since, even for elastic unloading, the history variable $z$ shall be updated for every load increment/step.
For arbitrary loading directions and magnitudes, there exists possibilities that the normal yield ratio $z$ may decrease first --- the subloading surface shrinks, then increase --- the subloading surface expands.
When it expands, the plasticity can only be correctly computed with the correct initial value of $z$.
An inward load increment resulting in $f_s^\text{trial}<0$ does not necessarily imply that it is a pure elastic unloading step.
A graphical representation is shown in \figref{fig:loading_criterion}.
\begin{figure}[htb]
    \centering
    \begin{tikzpicture}[>=latex]
        \newcommand{\DD}{70}
        \newcommand{\FF}{4}
        \newcommand{\SF}{2.5}
        \newcommand{\BF}{3.3}
        \newcommand{\LF}{1.7}
        \draw[->](-.5,0)--(2,0);
        \draw[->](0,-.5)--(0,2);
        % yield surface
        \coordinate(A)at(2.8,3);
        \node[fill=black,circle,inner sep=0,minimum size=2mm]at(A){};
        \draw[dashed,draw,line width=.3mm](A)circle(\FF);
        \draw[<-]($(A)+(60:\FF)$)--++(60:1.4)--++(1,0)node[right,fill=white]{yield surface $f$};
        % similarity centre
        \coordinate(B)at($(A)+(10:3)$);
        \draw[black!40](B)--($(A)+(\DD:\FF)$);
        \node[fill=black,circle,inner sep=0,minimum size=2mm]at(B){};
        \draw[dotted,d73027,line width=.4mm]($(B)!2.5!(A)$)--($(A)!1.5!(B)$)node[black,right,align=left]{centre of the subloading\\surface must lie on this\\fixed line};
        % current subloading surface
        \coordinate(C)at($(B)!\SF/\FF!(A)$);
        \node[fill=black,circle,inner sep=0,minimum size=2mm]at(C){};
        \draw[draw,line width=.6mm](C)circle(\SF);
        \draw[<-]($(C)+(60:\SF)$)--++(60:1.7)--++(.5,0)node[right,fill=white]{current subloading surface};
        \coordinate(D)at($(C)+(\DD:\SF)$);
        \node[fill=black,circle,inner sep=0,minimum size=2mm]at(D){};
        % bigger subloading surface
        \coordinate(CB)at($(B)!\BF/\FF!(A)$);
        \node[fill=black,circle,inner sep=0,minimum size=2mm]at(CB){};
        \node[fill=black,circle,inner sep=0,minimum size=2mm]at($(CB)+(\DD:\BF)$){};
        \draw[|<->|](CB)--++(\DD:\BF)node[midway,fill=white,sloped]{$z\uparrow$};
        \draw[black!40,draw,line width=.6mm](CB)circle(\BF);
        % smaller subloading surface
        \coordinate(CS)at($(B)!\LF/\FF!(A)$);
        \node[fill=black,circle,inner sep=0,minimum size=2mm]at(CS){};
        \node[fill=black,circle,inner sep=0,minimum size=2mm]at($(CS)+(\DD:\LF)$){};
        \draw[|<->|](CS)--++(\DD:\LF)node[midway,fill=white,sloped]{$z\downarrow$};
        \draw[black!40,draw,line width=.6mm](CS)circle(\LF);
        % arrows
        \draw[->,d73027,line width=1mm](0,0)--(A)node[midway,fill=white,sloped]{$\bbeta=H\balpha$};
        \draw[->,d73027,line width=1mm](A)--(C);
        \draw[->,d73027,line width=1mm](C)--(D)node[midway,fill=white,sloped]{$\beeta$};
        \draw[<-]($(A)!0.3!(C)$)--++(-60:2)--++(2.5,0)node[right,fill=white]{$\left(1-z\right)F\bd$};
        \draw[dotted,line width=.4mm]($(D)+(180:6)$)--($(D)+(0:2)$)node[right,align=left]{search along the loading\\direction to minimise $z$};
        \draw[->,4575b4,line width=1mm](D)--++(180:2.8)node[fill=black,circle,inner sep=0,minimum size=2mm]{};
        \draw[<-]($(D)+(180:2)$)--++(135:2.5)--++(-1,0)node[left,align=right]{load increment $\Delta\bs$\\$z$ decreases first\\then increases};
    \end{tikzpicture}
    \caption{arbitrary loading direction with frozen plasticity}\label{fig:loading_criterion}
\end{figure}

In the trial state, by freezing plasticity, viz., $H$, $F$, $\balpha$ and $\bd$ are all kept constant, it can be concluded that the centre of the subloading surface must lie on the line defined by centre of the yield surface $\bbeta$ and the similarity centre $\bc=\bbeta+F\bd$.
Then, for a given load increment $\Delta\bs=\bs_{n+1}-\bs_n$, assuming proportional loading, there exists a scalar $r$ such that the corresponding stress state $\bs_n+r\Delta\bs$ would minimise $z$, we denote this $z$ as $z^\text{trial}$ since it is computed in the trial state.
In mathematical terms, it can be expressed as
\begin{gather}
    \boxed{\min\limits_{r\in\mathbb{R}}z^\text{trial}\qquad\text{s.t.}\qquad{}f_s^\text{trial}\left(z^\text{trial},r\right)=0,}
\end{gather}
where $z^\text{trial}$ shall be computed by enforcing the corresponding $f_s^\text{trial}=0$.
% \begin{gather}\label{eq:trial_criterion}
%     \begin{split}
%         f_s^\text{trial} & =\norm{\beeta^\text{trial}}-z^\text{trial}F_n                                                 \\
%                          & =\norm{\bs_n+r\Delta\bs-H_n\balpha_n+\left(z^\text{trial}-1\right)F_n\bd_n}-z^\text{trial}F_n \\
%                          & =\norm{\bar{\beeta}+z^\text{trial}F_n\bd_n}-z^\text{trial}F_n                                 \\
%                          & =0,
%     \end{split}
% \end{gather}
% in which $\bar{\beeta}=\bar{\beeta}\left(r\right)=\bs_n-H_n\balpha_n-F_n\bd_n+r\Delta\bs$.
% The subscript $\left(\cdot\right)_n$ denotes the converged state at the beginning of the current loading step.
Since no extra constraints are imposed on $r$ for the moment, a solution is guaranteed to exist.
However, there are three distinctive cases.
\begin{enumerate}
    \item If $0<r<1$, the given load increment $\Delta\bs$ contains both elastic unloading and plastic loading.
          Since the elastic unloading portion ($0\to{}r$) reduces $z$ to a value smaller than the converged $z_n$, the plastic loading portion ($r\to{}1$) shall use this minimum $z$ (smaller than $z_n$, not necessarily zero in 3D models), instead of $z_n$, as the starting value to correctly compute plasticity development.
    \item If $r\geqslant1$, the given load increment $\Delta\bs$ involves only elastic unloading.
    \item If $r\geqslant0$, it is a pure loading case. Nothing additional needs to be done. $z_n$ is the correct starting value for return mapping.
\end{enumerate}
Thus, for any given load increment, one could 1) find the corresponding scalar $r$ that minimises $z^\text{trial}$, 2) use the above criteria to determine whether plasticity occurs, if so, also correct the starting value of $z$ if necessary.
We defer the presentation of explicit expressions for the above procedure after introducing the von Mises framework.
\subsection{A von Mises Model for Metals}
\subsubsection{Subloading Surface}
For the von Mises criterion, $\bs$ shall take the deviatoric part of the stress tensor $\bs=\dev{\bsigma}$, then
\begin{gather}
    \beeta=\dev{\bsigma}-H\balpha+\left(z-1\right)F\bd.
\end{gather}
The subloading surface shall be expressed as follows by replacing $F$ with $\sqrt{\dfrac{2}{3}}\sigma^y$.
\begin{gather}\label{eq:mises_subloading_surface}
    f_s=\norm{\beeta}-z\sqrt{\dfrac{2}{3}}\sigma^y.
\end{gather}
\subsubsection{Flow Rule}
Assuming associative plasticity, the flow rule is given by
\begin{gather}\label{eq:subloading_flow}
    \dot{\bvarepsilon^p}=\gamma\pdfrac{f_s}{\bsigma}=\gamma\dfrac{\beeta}{\norm{\beeta}}=\gamma\bn,
\end{gather}
where $\gamma$ denotes the plastic multiplier.
\subsubsection{Evolution Rules}
\paragraph{Accumulated Plastic Strain}
The accumulated plastic strain $q$ simply takes the accumulated magnitude of $\bvarepsilon^p$, thus, its rate form is
\begin{gather}\label{eq:eqv_strain}
    \dot{q}=\sqrt{\dfrac{2}{3}}\norm{\dot{\bvarepsilon^p}}=\sqrt{\dfrac{2}{3}}\gamma.
\end{gather}
\paragraph{Isotropic Hardening}
The evolution of $\sigma^y$ depends on $q$, here we adopt general form that combines linear hardening and nonlinear saturation.
\begin{gather}\label{eq:iso_bone}
    \sigma^y=\sigma^i+k_\text{iso}q+\sigma^s_\text{iso}\left(1-e^{-m^s_\text{iso}q}\right),
\end{gather}
where $\sigma^i$ is the initial yield stress, $k_\text{iso}$ is the isotropic hardening modulus, $\sigma^s_\text{iso}$ is the saturation stress, and $m^s_\text{iso}$ is the saturation rate.
\paragraph{Kinematic Hardening}
The kinematic hardening bound $H$ is replaced with a discrete function $a^y$ that takes a similar form to $\sigma^y$,
\begin{gather}\label{eq:kin_bone}
    a^y=a^i+k_\text{kin}q+a^s_\text{kin}\left(1-e^{-m^s_\text{kin}q}\right),
\end{gather}
where $a^i$ is the initial stress, $k_\text{kin}$ is the kinematic hardening modulus, $a^s_\text{kin}$ is the saturation stress, and $m^s_\text{kin}$ is the saturation rate.

The normalised back stress $\balpha$ adopts an Armstrong--Frederick type \cite{Frederick2007} evolution rule as shown in \eqsref{eq:revised_back}.
In this particular case, it shall be written as
\begin{gather}\label{eq:normalised_back_rule}
    \dot{\balpha}=b\left(\sqrt{\dfrac{2}{3}}\bn-\balpha\right)\gamma.
\end{gather}
It must be pointed out that, strictly speaking, \eqsref{eq:normalised_back_rule} shall be expressed as \cite{Simo1998}
\begin{gather}
    \dot{\balpha}=b\left(\dfrac{2}{3}\dot{\bvarepsilon^p}-\balpha\dot{q}\right)=b\sqrt{\dfrac{2}{3}}\left(\sqrt{\dfrac{2}{3}}\bn-\balpha\right)\gamma.
\end{gather}
With \eqsref{eq:normalised_back_rule}, the extra constant $\sqrt{\dfrac{2}{3}}$ is combined with the constant $b$ to make it a little more compact.
\paragraph{Elastic Core}
Similarly, the new field $\bd$ also adopts an Armstrong--Frederick type evolution rule as shown in \eqsref{eq:revised_core}.
\begin{gather}
    \dot{\bd}=c_e\left(\sqrt{\dfrac{2}{3}}z_e\bn-\bd\right)\gamma.
\end{gather}
\subsubsection{Summary}
The key expressions for both uniaxial and 3D models are summarised in \tabref{tab:summary}.
Note the constant $\sqrt{\dfrac{2}{3}}$ is added to the evolution rules for $\alpha$ and $d$ in the uniaxial version to ensure the same set of parameters used for both models would yield the same results.
They can be removed if such a consistency is not required.
\begin{table}[htb]
    \centering\footnotesize
    \caption{summary of the subloading surface model}\label{tab:summary}
    \begin{tabular}{rlll}
        \toprule
                           & uniaxial                                                                                                     & 3D                                                                 &              \\ \midrule
        Constitutive Law   & $\sigma=E\left(\varepsilon-\varepsilon^p\right)$                                                             & $\bsigma=\mb{D}_e:\left(\bvarepsilon-\bvarepsilon^p\right)$        &              \\
        Subloading Surface & $f_s=\abs{\eta}-z\sigma^y$                                                                                   & $f_s=\norm{\beeta}-z\sqrt{\dfrac{2}{3}}\sigma^y$                   &              \\
                           & $\eta=\sigma-a^y\alpha+\left(z-1\right)\sigma^yd$                                                            & $\beeta=\dev{\bsigma}-a^y\balpha+\left(z-1\right)\sigma^y\bd$      & \faLightbulb \\
        Flow Rule          & $\dot{\varepsilon^p}=\gamma{}n$                                                                              & $\dot{\bvarepsilon^p}=\gamma{}\bn$                                 &              \\
                           & $n=\sign{\eta}$                                                                                              & $\bn=\pdfrac{f_s}{\bsigma}/\norm{\pdfrac{f_s}{\bsigma}}$           &              \\
        Evolution Rules    & $\dot{q}=\gamma$                                                                                             & $\dot{q}=\sqrt{\dfrac{2}{3}}\gamma$                                &              \\
                           & $\dot{\alpha}=\sqrt{\dfrac{2}{3}}b\left(n-\alpha\right)\gamma$                                               & $\dot{\balpha}=b\left(\sqrt{\dfrac{2}{3}}\bn-\balpha\right)\gamma$ & \faLightbulb \\
                           & $\dot{d}=\sqrt{\dfrac{2}{3}}c_e\left(z_en-d\right)\gamma$                                                    & $\dot{\bd}=c_e\left(\sqrt{\dfrac{2}{3}}z_e\bn-\bd\right)\gamma$    & \faLightbulb \\
                           & \multicolumn{2}{l}{$\dot{z}=U\dot{q}$}                                                                       &                                                                                   \\
                           & \multicolumn{2}{l}{$\sigma^y=\sigma^i+k_\text{iso}q+\sigma^s_\text{iso}\left(1-e^{-m^s_\text{iso}q}\right)$} &                                                                                   \\
                           & \multicolumn{2}{l}{$a^y=a^i+k_\text{kin}q+a^s_\text{kin}\left(1-e^{-m^s_\text{kin}q}\right)$}                &                                                                                   \\ \midrule
        \multicolumn{4}{l}{Items labelled by \faLightbulb{} are novel proposals compared to the existing formulation \cite{Hashiguchi2024}.}                                                                                  \\ \bottomrule
    \end{tabular}
\end{table}

For brevity, the Armstrong--Frederick evolution rule is used for $\balpha$ and $\bd$.
More complex definitions can be used if necessary.
The main takeaway is the proposed decomposition, namely \eqref{eq:decomposition_core} and \eqsref{eq:decomposition_back}.
It satisfies the enclosing/bounding condition as long as $\balpha$ and $\bd$ are bounded.
It allows simple and concise formulation that is beneficial for both theoretical interpretation and numerical implementation.
\subsection{Formulation}
\subsubsection{Trial State}
Similar to the conventional return mapping algorithm, the trial state is computed by assuming the given load increment is fully elastic, then one could compute the trial stress $\bsigma^\text{trial}$ via the constitutive law.
\begin{gather}
    \bsigma^\text{trial}=\bsigma_n+\mb{D}_e:\left(\bvarepsilon_{n+1}-\bvarepsilon_n\right)=\bsigma_n+\mb{D}_e:\Delta\bvarepsilon.
\end{gather}
The incremental of the deviatoric part $\bs^\text{trial}=\dev{\bsigma^\text{trial}}$ can be computed accordingly.
\begin{gather}
    \Delta\bs=\bs^\text{trial}-\bs_n=\dev{\bsigma^\text{trial}-\bsigma_n}=2G\cdot\dev{\Delta\bvarepsilon},
\end{gather}
where $G$ is the shear modulus.
\subsubsection{Loading/Unloading Criterion}
At the beginning of each load increment, accounting for \eqsref{eq:mises_subloading_surface} and $\bar{\beeta}=\bs_n-H_n\balpha_n-F_n\bd_n+r\Delta\bs$, noting that $f_s^\text{trial}=0$ implies
\begin{gather}
    \left(\bar{\beeta}+z^\text{trial}F_n\bd_n\right):\left(\bar{\beeta}+z^\text{trial}F_n\bd_n\right)=\dfrac{2}{3}\left(z^\text{trial}\right)^2F_n^2,
\end{gather}
which is effectively,
\begin{gather}
    \bar{\beeta}:\bar{\beeta}+2F_n\bar{\beeta}:\bd_nz^\text{trial}+F_n^2\left(\bd_n:\bd_n-\dfrac{2}{3}\right)\left(z^\text{trial}\right)^2=0.
\end{gather}
By the quadratic formula,
\begin{gather}\label{eq:z_trial}
    z^\text{trial}\left(r\right)=\dfrac{\bar{\beeta}:\bd_n+\sqrt{\left(\bar{\beeta}:\bd_n\right)^2+\left(\bar{\beeta}:\bar{\beeta}\right)\left(\dfrac{2}{3}-\bd_n:\bd_n\right)}}{F_n\left(\dfrac{2}{3}-\bd_n:\bd_n\right)}.
\end{gather}
Noting that $\norm{\bd_n}<\sqrt{\dfrac{2}{3}}z_e\leqslant\sqrt{\dfrac{2}{3}}$, then, $\dfrac{2}{3}-\bd_n:\bd_n>0$, thus, the denominator is positive.
Due to the same reason, $\sqrt{\left(\bar{\beeta}:\bd_n\right)^2+\left(\bar{\beeta}:\bar{\beeta}\right)\left(\dfrac{2}{3}-\bd_n:\bd_n\right)}>\abs{\bar{\beeta}:\bd_n}$.
The other root is discarded since it may be negative.

To minimise $z^\text{trial}$, which is a univariate function of $r$ in the trial state, one could derive
\begin{gather}
    \ddfrac{z^\text{trial}}{r}=0\quad\rightarrow\quad
    \ddfrac{}{r}\left(\bar{\beeta}:\bd_n+\sqrt{\left(\bar{\beeta}:\bd_n\right)^2+\left(\bar{\beeta}:\bar{\beeta}\right)\left(\dfrac{2}{3}-\bd_n:\bd_n\right)}\right)=0.
\end{gather}
Its explicit form can be further expressed as follows,
\begin{gather}
    \bd_n:\Delta\bs+\dfrac{\left(\bd_n:\bar{\beeta}\right)\left(\bd_n:\Delta\bs\right)+\left(\dfrac{2}{3}-\bd_n:\bd_n\right)\left(\bar{\beeta}:\Delta\bs\right)}{\sqrt{\left(\bd_n:\bar{\beeta}\right)^2+\left(\dfrac{2}{3}-\bd_n:\bd_n\right)\left(\bar{\beeta}:\bar{\beeta}\right)}}=0,
\end{gather}
which, in a general context, can be solved by the Newton--Raphson method.

Let $R_r$ denote the residual
\begin{multline}\label{eq:z_residual}
    R_r=\left(\bd_n:\Delta\bs\right)\sqrt{\left(\bd_n:\bar{\beeta}\right)^2+\left(\dfrac{2}{3}-\bd_n:\bd_n\right)\left(\bar{\beeta}:\bar{\beeta}\right)}\\
    +\left(\bd_n:\bar{\beeta}\right)\left(\bd_n:\Delta\bs\right)+\left(\dfrac{2}{3}-\bd_n:\bd_n\right)\left(\bar{\beeta}:\Delta\bs\right).
\end{multline}
It's derivative with regard to $r$ is
\begin{multline}\label{eq:z_derivative}
    J_r=\ddfrac{R_r}{r}=\left(\bd_n:\Delta\bs\right)\dfrac{\left(\bd_n:\bar{\beeta}\right)\left(\bd_n:\Delta\bs\right)+\left(\dfrac{2}{3}-\bd_n:\bd_n\right)\left(\bar{\beeta}:\Delta\bs\right)}{\sqrt{\left(\bd_n:\bar{\beeta}\right)^2+\left(\dfrac{2}{3}-\bd_n:\bd_n\right)\left(\bar{\beeta}:\bar{\beeta}\right)}}\\
    +\left(\bd_n:\Delta\bs\right)^2+\left(\dfrac{2}{3}-\bd_n:\bd_n\right)\left(\Delta\bs:\Delta\bs\right).
\end{multline}

Compared to the recent work \cite{Hashiguchi2018,Anjiki2019}, here we have presented an equivalent but simpler approach to compute the starting normal yield ratio for arbitrary loading increments.
\algoref{algo:loading_criterion} presents a general algorithm to detect partial elastic unloading and correct the starting value of the normal yield ratio if necessary.
\begin{breakablealgorithm}
    \caption{correct starting value of normal yield ratio $z$}\label{algo:loading_criterion}
    \begin{algorithmic}[1]
        \State \textbf{Input}: $\bsigma^\text{trial}$, $\bsigma_n$, $\balpha_{n}$, $\bd_n$, $q_n$, $z_n$
        \State \textbf{Output}: $r$
        \State compute $\Delta\bs=\bs^\text{trial}-\bs_n$ using $\bsigma^\text{trial}$ and $\bsigma_n$
        \State $r=0$
        \While{true}\Comment{perform line search}
        \State $\bar{\beeta}=\bs_n-H_n\balpha_n-F_n\bd_n+r\Delta\bs$
        \State formulate $R_r$ and $J_r$\Comment{\eqsref{eq:z_residual}, \eqsref{eq:z_derivative}}
        \State $\delta{}r=R_r/J_r$
        \If{$\abs{\delta{}r}<\text{tolerance}$}
        \If{$0<r<1$}\Comment{current loading increment contains elastic unloading}
        \State update $z_{n}=z^\text{trial}$\Comment{compute $z^\text{trial}$ using $r$ via \eqsref{eq:z_trial}}
        \EndIf
        \State \textbf{return} $r$
        \EndIf
        \State $r\leftarrow{}r-\delta{}r$
        \EndWhile
    \end{algorithmic}
\end{breakablealgorithm}
\subsubsection{Some Simplifications}
Noting that $\dot{\balpha}$ is a linear function of $\balpha$, taking advantage of the implicit Euler method, one could express $\balpha$ in its explicit form.
\begin{gather}\label{eq:explicit_alpha}
    \balpha_n+b\left(\sqrt{\dfrac{2}{3}}\bn-\balpha\right)\gamma=\balpha\qquad\rightarrow\qquad
    \balpha=\dfrac{\gamma\sqrt{\dfrac{2}{3}}b\bn+\balpha_n}{1+\gamma{}b}.
\end{gather}
Similarly, $\bd$ can be expressed as
\begin{gather}\label{eq:explicit_d}
    \bd_n+c_e\left(\sqrt{\dfrac{2}{3}}z_e\bn-\bd\right)\gamma=\bd\qquad\rightarrow\qquad\bd=\dfrac{\gamma{}\sqrt{\dfrac{2}{3}}c_ez_e\bn+\bd_n}{1+\gamma{}c_e}.
\end{gather}
On the other hand, the deviatoric stress $\bs$ shall be computed via the trial stress $\bsigma^\text{trial}$,
\begin{gather}
    \bs=\dev{\bsigma}=\dev{\bsigma^\text{trial}-\gamma2G\bn}=\dev{\bsigma^\text{trial}}-\gamma2G\bn=\bs^\text{trial}-\gamma2G\bn.
\end{gather}

The shifted stress $\beeta$ is then
\begin{gather}\label{eq:explicit_eta}
    \begin{split}
        \beeta & =\bs-a^y\balpha+\left(z-1\right)\sigma^y\bd                                                                                                                                                \\
               & =\bs^\text{trial}-\gamma2G\bn-a^y\dfrac{\gamma{}\sqrt{\dfrac{2}{3}}b\bn+\balpha_n}{1+\gamma{}b}+\left(z-1\right)\sigma^y\dfrac{\gamma{}\sqrt{\dfrac{2}{3}}c_ez_e\bn+\bd_n}{1+\gamma{}c_e}.
    \end{split}
\end{gather}
Rearranging gives
\begin{multline}
    \beeta+\gamma2G\bn+\sqrt{\dfrac{2}{3}}b\dfrac{\gamma{}a^y}{1+\gamma{}b}\bn+\sqrt{\dfrac{2}{3}}c_ez_e\left(1-z\right)\dfrac{\gamma{}\sigma^y}{1+\gamma{}c_e}\bn=\\\bs^\text{trial}-\dfrac{a^y}{1+\gamma{}b}\balpha_n+\left(z-1\right)\dfrac{\sigma^y}{1+\gamma{}c_e}\bd_n.
\end{multline}
One could denote the right hand side as
\begin{gather}\label{eq:zeta}
    \bzeta=\bs^\text{trial}-\dfrac{a^y}{1+\gamma{}b}\balpha_n+\left(z-1\right)\dfrac{\sigma^y}{1+\gamma{}c_e}\bd_n,
\end{gather}
such that
\begin{gather}
    \left(\norm{\beeta}+\gamma2G+\sqrt{\dfrac{2}{3}}b\dfrac{\gamma{}a^y}{1+\gamma{}b}+\sqrt{\dfrac{2}{3}}c_ez_e\left(1-z\right)\dfrac{\gamma{}\sigma^y}{1+\gamma{}c_e}\right)\bn=\bzeta.
\end{gather}
It can be noted that each term in the bracket is non-negative by definition, thus it is correct to conclude that
\begin{gather}\label{eq:zeta_norm}
    \norm{\beeta}+\gamma2G+\sqrt{\dfrac{2}{3}}b\dfrac{\gamma{}a^y}{1+\gamma{}b}+\sqrt{\dfrac{2}{3}}c_ez_e\left(1-z\right)\dfrac{\gamma{}\sigma^y}{1+\gamma{}c_e}=\norm{\bzeta},
\end{gather}
and
\begin{gather}
    \bn=\dfrac{\beeta}{\norm{\beeta}}=\dfrac{\bzeta}{\norm{\bzeta}}.
\end{gather}
The above simplifications are valid under the assumption of associative plasticity flow.
Although it is not possible to simplify more, one shall realise that $\bzeta$ is a function of $\bvarepsilon$, $\gamma$ and $z$ only.
Its derivatives can be computed as follows.
\begin{gather}\label{eq:zeta_d1}
    \pdfrac{\bzeta}{\gamma}=-\dfrac{\ddfrac{a^y}{\gamma}\left(1+\gamma{}b\right)-ba^y}{\left(1+\gamma{}b\right)^2}\balpha_n+\left(z-1\right)\dfrac{\ddfrac{\sigma^y}{\gamma}\left(1+\gamma{}c_e\right)-c_e\sigma^y}{\left(1+\gamma{}c_e\right)^2}\bd_n,\\\label{eq:zeta_d2}
    \pdfrac{\bzeta}{z}=\dfrac{\sigma^y}{1+\gamma{}c_e}\bd_n,\\\label{eq:zeta_d3}
    \pdfrac{\bzeta}{\bvarepsilon}=2G\mathbb{I}^\text{dev}.
\end{gather}
\subsubsection{Local System}
The local system consists of two residuals, $f_s$ and $z$.
\begin{gather}
    \mb{R}=\left\{
    \begin{array}{l}
        \norm{\beeta}-\sqrt{\dfrac{2}{3}}z\sigma^y, \\[2mm]
        z-z_n-\sqrt{\dfrac{2}{3}}U\gamma.
    \end{array}
    \right.
\end{gather}
Taking the local variable as $\mb{x}=\begin{bmatrix}
        \gamma & z
    \end{bmatrix}^\mT$, the Jacobian $\mb{J}$ shall be written as
\begin{gather}
    \mb{J}=\begin{bmatrix}
        \pdfrac{\norm{\beeta}}{\gamma}-\sqrt{\dfrac{2}{3}}z\ddfrac{\sigma^y}{\gamma} & \pdfrac{\norm{\beeta}}{z}-\sqrt{\dfrac{2}{3}}\sigma^y \\[4mm]
        -\sqrt{\dfrac{2}{3}}U                                                        & 1-\sqrt{\dfrac{2}{3}}\gamma{}\ddfrac{U}{z}
    \end{bmatrix}.
\end{gather}
It is possible to avoid cumbersome computation of derivatives of $\norm{\beeta}$ by taking advantage of \eqsref{eq:zeta_norm}.
By noting that the derivatives of $\norm{\bzeta}$ can be further expanded as
\begin{gather}
    \pdfrac{\norm{\bzeta}}{}=\bn:\pdfrac{\bzeta}{},
\end{gather}
then
\begin{gather}\label{eq:eta_d1}
    \pdfrac{\norm{\beeta}}{\gamma}=\bn:\pdfrac{\bzeta}{\gamma}-2G-\pdfrac{}{\gamma}\left(\sqrt{\dfrac{2}{3}}b\dfrac{\gamma{}a^y}{1+\gamma{}b}+\sqrt{\dfrac{2}{3}}c_ez_e\left(1-z\right)\dfrac{\gamma{}\sigma^y}{1+\gamma{}c_e}\right),\\\label{eq:eta_d2}
    \pdfrac{\norm{\beeta}}{z}=\bn:\pdfrac{\bzeta}{z}+\sqrt{\dfrac{2}{3}}c_ez_e\dfrac{\gamma{}\sigma^y}{1+\gamma{}c_e}.
\end{gather}
\subsubsection{Consistent Tangent Operator}
It is possible to differentiate the local equilibrium $\mb{R}=\mb{0}$ to obtain
\begin{gather}
    \pdfrac{\mb{R}}{\bvarepsilon}+\pdfrac{\mb{R}}{\mb{x}}\ddfrac{\mb{x}}{\bvarepsilon}=\mb{0}.
\end{gather}
This gives
\begin{gather}\label{eq:local_equilibrium}
    \ddfrac{\mb{x}}{\bvarepsilon}=-\left(\pdfrac{\mb{R}}{\mb{x}}\right)^{-1}\pdfrac{\mb{R}}{\bvarepsilon}=-\mb{J}^{-1}\pdfrac{\mb{R}}{\bvarepsilon}.
\end{gather}
Noting that
\begin{gather}
    \pdfrac{\norm{\beeta}}{\bvarepsilon}=\bn:\pdfrac{\bzeta}{\bvarepsilon}=2G\bn:\mathbb{I}^\text{dev}=2G\bn,
\end{gather}
then in the matrix form,
\begin{gather}
    \pdfrac{\mb{R}}{\bvarepsilon}=2G\begin{bmatrix}
        \bn^\mT \\\mb{0}
    \end{bmatrix}.
\end{gather}
By solving \eqsref{eq:local_equilibrium}, one could obtain $\ddfrac{\mb{x}}{\bvarepsilon}$, we are interested in the first row, which can be explicitly written as
\begin{gather}
    \ddfrac{\gamma}{\bvarepsilon}=-2G\dfrac{1-\sqrt{\dfrac{2}{3}}\gamma{}\ddfrac{U}{z}}{\det\mb{J}}\bn^\mT.
\end{gather}
Now from the stress update expression $\bsigma=\bsigma^\text{trial}-\gamma2G\bn$, it is possible to derive the consistent tangent operator as
\begin{gather}
    \ddfrac{\bsigma}{\bvarepsilon}=\mb{D}_e-2G\left(\pdfrac{\bn}{\bvarepsilon}\gamma+\bn\ddfrac{\gamma}{\bvarepsilon}\right),
\end{gather}
where $\mb{D}_e$ is the elastic stiffness.
In the meantime, by $\bn=\dfrac{\bzeta}{\norm{\bzeta}}$, the derivative of $\bn$ with respect to $\bvarepsilon$ can be shown as
\begin{gather}
    \pdfrac{\bn}{\bvarepsilon}=\dfrac{2G}{\norm{\bzeta}}\left(\mathbb{I}^\text{dev}-\bn\otimes\bn\right),
\end{gather}
after some algebra, the consistent tangent operator can be written as
\begin{gather}\label{eq:consistent_tangent}
    \ddfrac{\bsigma}{\bvarepsilon}=\mb{D}_e-\dfrac{4G^2\gamma}{\norm{\bzeta}}\mathbb{I}^\text{dev}+4G^2\left(\dfrac{\gamma}{\norm{\bzeta}}+\dfrac{1-\sqrt{\dfrac{2}{3}}\gamma{}\ddfrac{U}{z}}{\det\mb{J}}\right)\bn\otimes\bn.
\end{gather}
It is a symmetric tensor/matrix.
One can also observe some resemblance between \eqsref{eq:j2_stiffness} and \eqsref{eq:consistent_tangent}.
\subsection{Implementation}
The complete algorithm is presented in \algoref{algo:subloading_steel}.
It adopts a standard Newton--Raphson scheme (implicit, closest point projection) to solve the local equilibrium.
Noting that the local iteration does not involve complex tensor algebra, the algorithm can be implemented in a very efficient and concise manner.
\begin{breakablealgorithm}
    \caption{state determination of the subloading surface model}\label{algo:subloading_steel}
    \begin{algorithmic}[1]
        \State \textbf{Parameter}: $E$, $\nu$, $\sigma^i$, $k_\text{iso}$, $\sigma^s_\text{iso}$, $m^s_\text{iso}$, $a^i$, $k_\text{kin}$, $a^s_\text{kin}$, $m^s_\text{kin}$, $b$, $u$, $c_e$, $z_e$
        \State \textbf{Input}: $\bvarepsilon_{n+1}$, $\bvarepsilon_n$, $\bsigma_n$, $\balpha_{n}$, $\bd_n$, $q_n$, $z_n$
        \State \textbf{Output}: $\mb{D}_{n+1}$, $\bsigma_{n+1}$, $\balpha_{n+1}$, $\bd_{n+1}$, $q_{n+1}$, $z_{n+1}$
        \State compute trial stress $\bsigma^\text{trial}=\bsigma_n+\mb{D}_e\left(\bvarepsilon_{n+1}-\bvarepsilon_n\right)$\Comment{$\mb{D}_e$: elastic stiffness}
        \State set history variables $\left(\cdot\right)_{n+1}=\left(\cdot\right)_{n}$ for $\balpha_{n+1}$, $\bd_{n+1}$, $q_{n+1}$, $z_{n+1}$
        \State check loading criterion\Comment{\algoref{algo:loading_criterion}}
        \If{$r\geqslant1$}\Comment{elastic unloading}
        \State $\bsigma_{n+1}=\bsigma^\text{trial}$
        \State $\mb{D}_{n+1}=\mb{D}_e$
        \State update $z$ by solving $\norm{\bs_{n+1}-a^y_{n+1}\balpha_{n+1}+\left(z-1\right)\sigma^y_{n+1}\bd_{n+1}}=\sqrt{\dfrac{2}{3}}z\sigma^y_{n+1}$\Comment{$0\leqslant{}z\leqslant1$}
        \State \textbf{return}
        \EndIf
        \State $\gamma=0$\Comment{plastic multiplier}
        \While{true}
        \State update $q_{n+1}$ according to $\gamma$\Comment{\eqsref{eq:eqv_strain}}
        \State update $a^y_{n+1}$ and $\sigma^y_{n+1}$ according to $q_{n+1}$\Comment{\eqsref{eq:iso_bone}, \eqsref{eq:kin_bone}}
        \State compute $\bzeta_{n+1}$ and $\bn_{n+1}$\Comment{\eqsref{eq:zeta}}
        \State update $\balpha_{n+1}$, $\bd_{n+1}$\Comment{\eqsref{eq:explicit_alpha}, \eqsref{eq:explicit_d}}
        \State formulate $\mb{R}$\Comment{$\norm{\beeta}$ can be directly computed using \eqsref{eq:explicit_eta}}
        \State formulate $\mb{J}$\Comment{\eqsref{eq:eta_d1} via \eqsref{eq:zeta_d1}, \eqsref{eq:eta_d2} via \eqsref{eq:zeta_d2}}
        \State $\Delta=\mb{J}^{-1}\mb{R}$\Comment{$\Delta=\begin{bmatrix}\delta\gamma&\delta{}z\end{bmatrix}$}
        \If{$\norm{\Delta}<\text{tolerance}$}
        \State compute stress $\bsigma_{n+1}=\bsigma^\text{trial}-\gamma2G\bn_{n+1}$\Comment{$G$: shear modulus}
        \State compute consistent tangent stiffness $\mb{D}_{n+1}$\Comment{\eqsref{eq:consistent_tangent}}
        \State \textbf{return}
        \EndIf
        \State $\gamma\leftarrow\gamma-\delta\gamma$
        \State $z\leftarrow{}z-\delta{}z$
        \EndWhile
    \end{algorithmic}
\end{breakablealgorithm}

The following is a reference implementation in CPP.
\begin{cppcode}
Subloading::update_trial_status
\end{cppcode}
\subsubsection{Remarks}
Compared to the previous implementation \cite{Fincato2017,Anjiki2019}, the formulation proposed in this work dramatically reduces the size of the local system from \num{20} to \num{2}.
We also present the explicit expression of the consistent tangent operator, which is not available in the previous works.
If \eqsref{eq:original_z} can be explicitly integrated by choosing a specific function $U$ (such as the cotangent function used in the original proposal), the local system can be further reduced to a scalar residual.
\section{Orthotropic Hoffman Model}\label{sec:hoffman}
Here we introduce an anisotropic model that adopts the orthotropic Hoffman yielding criterion.
This framework resembles the isotropic von Mises model.
As a matter of fact, the von Mises model can be seen as a degenerated version of the Hoffman model.
It can be used to model orthtropic materials such sheet steel and timber.
One can further refer to the corresponding section (\S~10.3) in \cite{SouzaNeto2008}.
\subsection{Theory}
\subsubsection{Yield Function}
The yield function adopts the Hoffman criterion, which is expressed in compressed matrix form,
\begin{gather}\label{eq:orthotropic_yield_function_proto}
f=\dfrac{1}{2}\bsigma^\mT\mb{P}\bsigma+\mb{q}^\mT\bsigma-\sigma_y^2,
\end{gather}
where $\mb{P}=\mb{P}^\mT$ and $\bq$ are constant scaling matrix/vector.
Various other orthotropic yielding criteria can also fit in this form, see \cite{Oller2003} for details.
Apparently, $\mb{P}$ and $\bq$ represent two tensors of order four and two, respectively.
But since \cite{Oller2003} uses matrix notation, we shall follow this convention.
For example, the Hoffman criterion can be expressed as
\begin{tiny}
\begin{gather}\label{eq:hoffman_criterion}
\dfrac{1}{2}\mb{P}=\begin{bmatrix}
T_1&-\dfrac{T_1+T_2-T_3}{2}&-\dfrac{T_3+T_1-T_2}{2}&&&\\[4mm]
-\dfrac{T_1+T_2-T_3}{2}&T_2&-\dfrac{T_2+T_3-T_1}{2}&&&\\[4mm]
-\dfrac{T_3+T_1-T_2}{2}&-\dfrac{T_2+T_3-T_1}{2}&T_3&&&\\[4mm]
&&&\dfrac{1}{f_{12}^2}&&\\[4mm]
&&&&\dfrac{1}{f_{23}^2}&\\[4mm]
&&&&&\dfrac{1}{f_{13}^2}
\end{bmatrix},\quad
\mb{q}=\begin{bmatrix}
\left(f_{11}^c-f_{11}^t\right)T_1\\[4mm]
\left(f_{22}^c-f_{22}^t\right)T_2\\[4mm]
\left(f_{33}^c-f_{33}^t\right)T_3\\[4mm]
0\\[4mm]
0\\[4mm]
0
\end{bmatrix},
\end{gather}
\end{tiny}
in which,
\begin{gather}
T_1=\dfrac{1}{f_{11}^tf_{11}^c},\qquad
T_2=\dfrac{1}{f_{22}^tf_{22}^c},\qquad
T_3=\dfrac{1}{f_{33}^tf_{33}^c},
\end{gather}
with $f_{ij}^\aleph$ representing the yielding stress along different directions.

Unlike other models introduced in this book, because the yield function is expressed in matrix form, all the following are thus mainly based in matrix notation.
An equivalent tensor notation may also exist, but as it is not used in literature, we shall not introduce it here.
\subsubsection{Flow Rule}
The associated plasticity is assumed such that the plastic potential $g$ is identical to $f$. The plastic flow direction is then
\begin{gather}
\bn=\pdfrac{g}{\bsigma}=\pdfrac{f}{\bsigma}=\mb{P}\bsigma+\mb{q}.
\end{gather}
The flow rule can be defined as
\begin{gather}
\dot{\bvarepsilon^p}=\gamma\bn.
\end{gather}
It is worth noting that since the yield function is expressed as a function of the stress (Voigt) vector rather than the stress tensor, the partial derivative $\mb{P}\bsigma+\mb{q}$ is the derivative with regard to the vector $\bsigma$, which already contains the scaling factor for shear terms.
This is different from other models that use the tensor notation.
\subsubsection{Hardening Law}
The reference yield stress $\sigma_y$ is defined as a function of the accumulated equivalent plastic strain $\varepsilon_{p}$.
\begin{gather}
\sigma_y=H\left(\varepsilon^{p}\right).
\end{gather}
It must be noted that here it is called the `reference' yield stress --- there is no unique uniaxial equivalence as along different directions the yield stress may vary.

The evolution of $\varepsilon^{p}$ is defined by
\begin{gather}
\dot{\varepsilon^p}=\sqrt{\dfrac{2}{3}\dot{\bvarepsilon^p}:\dot{\bvarepsilon^p}}=\gamma\sqrt{\dfrac{2}{3}\bn:\bn},
\end{gather}
where $\bn:\bn$ shall be treated as double contraction of strain tensors, that is,
\begin{gather}
\underbrace{\bn:\bn}_\text{tensor}=\underbrace{\bn^\mT\mb{T}\bn}_\text{vector},
\end{gather}
with $\mb{T}=\diag{\begin{matrix}
1&1&1&\dfrac{1}{2}&\dfrac{1}{2}&\dfrac{1}{2}
\end{matrix}}$.
This is due to the fact that \eqsref{eq:orthotropic_yield_function_proto} is expressed in vectors/matrices rather than tensors!

Although \textbf{not} correct, we'll use $\norm{\bn}_e$ to represent $\sqrt{\dfrac{2}{3}\bn:\bn}$ for brevity in the following.
\subsection{Formulation}
To some extend, the model is even simpler than the von Mises model as there is no back stress to support kinematic hardening.
This is reasonable as there is no experimental evidence that timber exhibits the Bauschinger effect.
For other materials, specific models can be developed, in which kinematic hardening can be introduced.
Furthermore, the yield function involves only matrix--vector operations, the corresponding derivatives are relatively easy to compute.
\begin{table}[ht]
\centering
\begin{tabular}{rl}
\toprule
Constitutive Law&$\bsigma=\mb{E}\left(\bvarepsilon-\bvarepsilon^p\right)$\\
Yield Function&$f=\dfrac{1}{2}\bsigma^\mT\mb{P}\bsigma+\mb{q}^\mT\bsigma-\sigma_y^2$\\
Flow Rule&$\dot{\bvarepsilon^p}=\gamma\bn$\\
Hardening Law&$\dot{\varepsilon^p}=\gamma\norm{\bn}_e$\\\bottomrule
\end{tabular}
\end{table}
\subsubsection{Elastic Loading/Unloading}
Compared with the von Mises framework, there is no essential difference in terms of elastic loading/unloading, the plasticity is frozen at the beginning of each substep, allowing one to compute the trial stress such that
\begin{gather}
\bsigma^\text{trial}=\mb{E}\left(\bvarepsilon_{n+1}-\bvarepsilon^p_n\right)\xcancel{=\bsigma_n+\mb{E}\left(\bvarepsilon_{n+1}-\bvarepsilon_n\right)}.
\end{gather}
One shall note that the second form is not used here.
In most cases, both forms are equivalent.
However, some models may further apply damage mechanics to the result of plasticity, making the second form incorrect (as $\bsigma_n$ may contain damage reductions).
Here we demonstrate the usage of the first form.

The trial yield function can then be computed using the unchanged plastic strain.
\begin{gather}
f^\text{trial}=\dfrac{1}{2}\bsigma^\text{trial,T}\mb{P}\bsigma^\text{trial}+\mb{q}^\mT\bsigma^\text{trial}-\sigma_{y,n}^2.
\end{gather}
\subsubsection{Plastic Evolution}
The discretised evolution of plastic strain is written as
\begin{gather}
\bvarepsilon^p_{n+1}=\bvarepsilon^p_{n}+\Delta\bvarepsilon^p=\bvarepsilon^p_{n}+\gamma\bn_{m}.
\end{gather}
Note that we do not directly use $\bn_{n+1}$ here, instead, we use $\bn_{m}$ to allow multiple different implementations.
In specific, if $\bn_{m}=\bn_{n+1}$, it is effectively the implicit Euler method.
\subsubsection{Local Residual}
The residual is chosen as follows. For brevity, all subscripts $\left(\cdot\right)_{n+1}$ are dropped.
\begin{gather}
\mb{R}=\left\{
\begin{array}{l}
\dfrac{1}{2}\bsigma^\mT\mb{P}\bsigma+\mb{q}^\mT\bsigma-\sigma_y^2,\\[4mm]
\bsigma+\gamma\mb{E}\bn_m-\bsigma^\text{trial}.
\end{array}
\right.
\end{gather}
We use $\mb{E}$ to denote the elasticity tensor $\mb{D}$ expressed in matrix form.
They represent the same quantity.

By choosing $\mb{x}=\begin{bmatrix}
\gamma&\bsigma
\end{bmatrix}^\mT$ as the independent variables, the Jacobian can be then computed as
\begin{gather}
\mb{J}=\pdfrac{\mb{R}}{\mb{x}}=\begin{bmatrix}
-2\sigma_y\ddfrac{\sigma_y}{\varepsilon_{p}}\norm{\bn_m}_e&\bn^\mT-2\sigma_y\ddfrac{\sigma_y}{\varepsilon_{p}}\gamma\ddfrac{\norm{\bn_m}_e}{\bsigma}\\[4mm]
\mb{E}\bn_m&\mb{I}+\gamma\mb{E}\ddfrac{\bn_m}{\bsigma}
\end{bmatrix}.
\end{gather}

Some references would further derive a scalar local residual at the cost of complicating gradient. Here we choose to increase the size of local system in order to express the Jacobian in a simpler form. Performance wise, a scalar local residual does not necessarily leads to faster state determination.
\subsubsection{Consistent Tangent Stiffness}
The consistent tangent stiffness can be directly computed from the local residual, given that $\bsigma_{n+1}$ is chosen as the variable. Differentiating $\mb{R}$ at equilibrium/convergence $\mb{R}=\mb{0}$ gives
\begin{gather}\label{eq:universal_consistent_tangent}
    \pdfrac{\mb{R}}{\mb{x}}\pdfrac{\mb{x}}{\bvarepsilon_{n+1}}+\pdfrac{\mb{R}}{\bvarepsilon_{n+1}}=\mb{0},
\end{gather}
rearranging which gives
\begin{gather}
\pdfrac{\mb{x}}{\bvarepsilon_{n+1}}=-\left(\pdfrac{\mb{R}}{\mb{x}}\right)^{-1}\pdfrac{\mb{R}}{\bvarepsilon_{n+1}}=-\mb{J}^{-1}\pdfrac{\mb{R}}{\bvarepsilon_{n+1}}.
\end{gather}

Since the left hand side $\pdfrac{\mb{x}}{\bvarepsilon_{n+1}}$ contains
\begin{gather}
    \pdfrac{\mb{x}}{\bvarepsilon_{n+1}}=\begin{bmatrix}
        \pdfrac{\gamma}{\bvarepsilon_{n+1}} \\[4mm]
        \pdfrac{\bsigma_{n+1}}{\bvarepsilon_{n+1}}
    \end{bmatrix},
\end{gather}
then,
\begin{gather}\label{eq:ortho_consistent_tangent}
\pdfrac{\bsigma_{n+1}}{\bvarepsilon_{n+1}}=-\left(\mb{J}^{-1}\pdfrac{\mb{R}}{\bvarepsilon_{n+1}}\right)^{\langle2-7\rangle},
\end{gather}
where $\left(\cdot\right)^{\langle2-7\rangle}$ denotes the second to the seventh row of target quantity $\left(\cdot\right)$.

Unlike the von Mises framework, in which the analytical expression for the consistent tangent stiffness matrix is derived. Here we take advantage of the fact that when the local equilibrium is achieved, $\mb{R}=\mb{0}$ or at least $\mb{R}\approx\mb{0}$, allowing one to take full differentiation to obtain some useful quantities that otherwise may be difficult to compute. If $\bsigma_{n+1}$ is directly involved as one of the independent variables in local iteration, the consistent stiffness can be directly obtained. Otherwise, often additional simple steps (chain rule) shall be applied to the stress update formula to compute $\pdfrac{\bsigma_{n+1}}{\bvarepsilon_{n+1}}$.

This method avoid the computation of lengthy, cumbersome analytical expressions of consistent tangent. In most cases, it is also very simple to implement as $\mb{J}$ is already available when the local iteration converges, and $\pdfrac{\mb{R}}{\bvarepsilon_{n+1}}$ often is very easy to compute. Readers shall try to grasp the beauty of \eqsref{eq:universal_consistent_tangent}, as this method will be frequently used in the models introduced later in this book.
\subsection{Implementation}
\subsubsection{Trapezoidal Integration}
In this particular model, we demonstrate the less common trapezoidal rule.
By the trapezoidal rule, we mean that the update of the plastic strain shall be expressed as
\begin{gather}
    \bvarepsilon^p_{n+1}=\bvarepsilon^p_{n}+\gamma\bn_{m},
\end{gather}
as presented earlier.
In a conventional context, it can be further expressed as
\begin{gather}
    \bvarepsilon^p_{n+1}=\bvarepsilon^p_{n}+\gamma\dfrac{\bn_{n}+\bn_{n+1}}{2},\qquad\text{(WRONG!)}
\end{gather}
such that $\bn_{n}$ and $\bn_{n+1}$ are the flow directions at the beginning and the end of the step, respectively.
However, one must notice that the current state $\left(\cdot\right)_n$ may be elastic!
The above expression is correct only when the current state is on the yield surface.
To be precise, as we know that the new state $\left(\cdot\right)_{n+1}$ must be plastic, the above expression has to be modified as
\begin{gather}
    \bvarepsilon^p_{n+1}=\bvarepsilon^p_{n}+\gamma\dfrac{\bn_\text{onset}+\bn_{n+1}}{2},
\end{gather}
where $\bn_\text{onset}$ is the flow direction at the onset of plasticity in the current step.
It thus implies that
\begin{gather}
    \bn_{m}=\dfrac{\bn_\text{onset}+\bn_{n+1}}{2}.
\end{gather}

Assuming proportional loading, finding $\bn_\text{onset}$ is equivalent to finding a ratio $r$ such that the onset stress $\bsigma_\text{onset}$, which is defined as
\begin{gather}
    \bsigma_\text{onset}=\mb{E}\left(\bvarepsilon_n+r\Delta\bvarepsilon-\bvarepsilon_n^p\right),
\end{gather}
is on the yield surface, that is,
\begin{gather}
    f_\text{onset}=f\left(\bsigma_\text{onset}\right)=\dfrac{1}{2}\bsigma_\text{onset}^\mT\mb{P}\bsigma_\text{onset}+\mb{q}^\mT\bsigma_\text{onset}-\sigma_{y,n}^2=0.
\end{gather}
Note in this procedure, plasticity is frozen, the only variable is the ratio $r$.
This is a well bounded problem --- $r$ is guaranteed to be greater than zero and less than one.
The solution $r$ can be computed using a root finding method.
Alternatively, noting that $f_\text{onset}$ is a quadratic function of $r$, one can compute the roots of the equation directly.
Once $r$ is known, $\bsigma_\text{onset}$ can then be computed, as well as $\bn_\text{onset}$.

It is necessary to compute the derivative as well.
From the equilibrium of $f_\text{onset}$, one has
\begin{gather}
    \pdfrac{f_\text{onset}}{r}\md{r}+\pdfrac{f_\text{onset}}{\bvarepsilon_{n+1}}\md{\bvarepsilon_{n+1}}=0,
\end{gather}
thus,
\begin{gather}
    \ddfrac{r}{\bvarepsilon_{n+1}}=\dfrac{-1}{\pdfrac{f_\text{onset}}{r}}\pdfrac{f_\text{onset}}{\bvarepsilon_{n+1}},
\end{gather}
with
\begin{gather}
    \pdfrac{f_\text{onset}}{r}=\bn_\text{onset}^\mT\mb{E}\Delta\bvarepsilon,\qquad
    \pdfrac{f_\text{onset}}{\bvarepsilon_{n+1}}=\bn_\text{onset}^\mT\mb{E}r.
\end{gather}
With the above, one can further derive
\begin{gather}
    \ddfrac{\bn_\text{onset}}{\bvarepsilon_{n+1}}=\mb{PE}\left(\Delta\bvarepsilon\ddfrac{r}{\bvarepsilon_{n+1}}+r\mb{I}\right)=r\left(\mb{PE}-\mb{PE}\dfrac{\Delta\bvarepsilon\bn_\text{onset}^\mT\mb{E}}{\bn_\text{onset}^\mT\mb{E}\Delta\bvarepsilon}\right).
\end{gather}
A slight rearrangement gives
\begin{gather}
    \ddfrac{\bn_\text{onset}}{\bvarepsilon_{n+1}}=r\mb{P}\left(\mb{I}-\dfrac{\mb{E}\Delta\bvarepsilon\bn_\text{onset}^\mT}{\bn_\text{onset}^\mT\mb{E}\Delta\bvarepsilon}\right)\mb{E}=r\mb{P}\left(\mb{I}-\dfrac{\Delta\bsigma\bn_\text{onset}^\mT}{\bn_\text{onset}^\mT\Delta\bsigma}\right)\mb{E}.
\end{gather}
It is clear that when $r\rightarrow0$, the derivative approaches constant as $\ddfrac{\bn_\text{onset}}{\bvarepsilon_{n+1}}\rightarrow\mb{0}$.

With $\bn_\text{onset}$, it is possible to compute the explicit expression of $\pdfrac{\mb{R}}{\bvarepsilon_{n+1}}$ as
\begin{gather}
    \pdfrac{\mb{R}}{\bvarepsilon_{n+1}}=\begin{bmatrix}
        -2\sigma_y\ddfrac{\sigma_y}{\varepsilon_{p}}\gamma\ddfrac{\norm{\bn_m}_e}{\bvarepsilon_{n+1}} \\[4mm]
        \gamma\mb{E}\ddfrac{\bn_m}{\bvarepsilon_{n+1}}-\mb{E}
    \end{bmatrix}=\begin{bmatrix}
        -\sigma_y\ddfrac{\sigma_y}{\varepsilon_{p}}\gamma\ddfrac{\norm{\bn_m}_e}{\bn_m}r\mb{P}\left(\mb{I}-\dfrac{\Delta\bsigma\bn_\text{onset}^\mT}{\bn_\text{onset}^\mT\Delta\bsigma}\right)\mb{E} \\[4mm]
        \dfrac{\gamma}{2}\mb{E}r\mb{P}\left(\mb{I}-\dfrac{\Delta\bsigma\bn_\text{onset}^\mT}{\bn_\text{onset}^\mT\Delta\bsigma}\right)\mb{E}-\mb{E}
    \end{bmatrix}.
\end{gather}
Rearranging the above gives
\begin{gather}
    \pdfrac{\mb{R}}{\bvarepsilon_{n+1}}=\begin{bmatrix}
        -\sigma_yr\gamma\ddfrac{\sigma_y}{\varepsilon_{p}}\ddfrac{\norm{\bn_m}_e}{\bn_m}\mb{P}\left(\mb{I}-\dfrac{\Delta\bsigma\bn_\text{onset}^\mT}{\bn_\text{onset}^\mT\Delta\bsigma}\right) \\[4mm]
        \dfrac{r\gamma}{2}\mb{EP}\left(\mb{I}-\dfrac{\Delta\bsigma\bn_\text{onset}^\mT}{\bn_\text{onset}^\mT\Delta\bsigma}\right)-\mb{I}
    \end{bmatrix}\mb{E}.
\end{gather}
The above can be inserted into \eqsref{eq:ortho_consistent_tangent} to obtain the consistent tangent stiffness.
\subsubsection{Line Search}
Since it is an anisotropic model, the local iteration may have difficulties in convergence, especially when a high anisotropy is defined.
Some implementations \cite{Krasnovskiy2004,SouzaNeto2008,Scherzinger2017} adopt line search, which effectively becomes a damped Newton--Raphson method.
It indeed improves numerical stability in certain cases, but does not guarantee global convergence.
The additional cost of searching is not necessarily negligible, as a desired step size is not guaranteed to be found in a fixed number of searches.
What we end up with is a more costly algorithm that may not converge.

Instead of line search, noting that there must exist a solution for some $\gamma>0$, it is possible to use a bracketing method that guarantees a solution can be found for arbitrary specified accuracy.
The idea can be explained as follows.
For a given $\gamma$, it is possible to compute the stress $\bsigma$ via
\begin{gather}
    \bsigma=\bsigma^\text{trial}-\gamma\mb{E}\bn_m,
\end{gather}
thus
\begin{multline}
    \left(\mb{I}+\dfrac{\gamma}{2}\mb{E}\mb{P}\right)\bsigma=\bsigma^\text{trial}-\dfrac{\gamma}{2}\mb{E}\left(\bn_\text{onset}+\mb{q}\right) \\\longrightarrow\qquad
    \bsigma=\left(\mb{I}+\dfrac{\gamma}{2}\mb{E}\mb{P}\right)^{-1}\left(\bsigma^\text{trial}-\dfrac{\gamma}{2}\mb{E}\left(\bn_\text{onset}+\mb{q}\right)\right).
\end{multline}
Once $\bsigma$ is known, the yield function can be evaluated.
This effectively turns $f=f\left(\gamma\right)$ into a univariate function of $\gamma$, and noting that utilizing the above procedure, no iteration is required to evaluate $f$ for a given $\gamma$, and such an evaluation is precise.
Knowing that $f\left(0\right)>0$, to use a root finding method, one may start with a small $\gamma_b>0$ and compute $f\left(\gamma_b\right)$.
If $f\left(\gamma_b\right)>0$, double $\gamma_b$ by, for example, $\gamma_b\leftarrow2\gamma_b$ and continue evaluating $f\left(\gamma_b\right)$ until $f\left(\gamma_b\right)\leqslant0$.
This ensures that the solution shall be bracketed by $0$ and $\gamma_b$.
One can then use root finding methods to find the solution $\gamma$.
\subsubsection{Return Mapping Implementation}
The following is a reference implementation that switches between the implicit Euler method and the trapezoidal method based on the size of strain increment.
The criterion used to determine which method to use is purely heuristic, and it is not guaranteed to be optimal.
The general idea is to use the second order method when the strain increment is small so that the asymptotic convergence can be retained.
\begin{cppcode}
NonlinearOrthotropic::update_trial_status
\end{cppcode}

The two integration methods follow a similar structure.
Note the computation of the onset state.
\begin{cppcode}
NonlinearOrthotropic::trapezoidal_return
\end{cppcode}

The Euler method is simpler.
\begin{cppcode}
NonlinearOrthotropic::euler_return
\end{cppcode}
\section{YLD0418P}
In \cite{SouzaNeto2008}, an anisotropic yield criterion named as the Barlat--Lian criterion \cite{Barlat1989} is discussed in the context of plane stress for sheet metals.
The yield criterion was further generalised \cite{Barlat2003} and extended to the 3D space \cite{Barlat2005}, which is known as the YLD2004-18P criterion since it was formulated in 2004 and has 18 parameters.
Here, we use a slightly different notation, YLD0418P, to denote this criterion.
\subsection{Miscellaneous Aspects}
Before introducing the yield criterion, we shall first introduce some miscellaneous aspects to make it a complete plasticity model.
Namely, one needs to provide the corresponding elasticity, flow rule and hardening law.

For elasticity, we assume either isotropic or orthotropic elasticity such that
\begin{gather}
    \bsigma=\mb{E}\left(\bvarepsilon-\bvarepsilon^p\right),
\end{gather}
where $\mb{E}$ is the elastic stiffness matrix that can be either isotropic or orthotropic.
Because in the following both matrix and tensor notations may be used, the elasticity may also be equivalently expresssed in tensor notation as
\begin{gather}
    \bsigma=\mb{D}:\left(\bvarepsilon-\bvarepsilon^p\right).
\end{gather}
The convention is simple: when $\mb{D}$ is used, it represents a tensor, thus tensor operations are required, while when $\mb{E}$ is used, it represents a matrix, thus matrix operations are required.
For the flow rule, we assume the associative plasticity such that the plastic potential is identical to the yield function.
\begin{gather}
    \dot{\bvarepsilon^p}=\gamma\bn=\gamma\pdfrac{f}{\bsigma}.
\end{gather}
For the hardening rule, let the isotropic yield stress be a function of the accumulated equivalent plastic strain $\varepsilon^p$.
\begin{gather}
    \sigma_y=H\left(\varepsilon^{p}\right),
\end{gather}
with
\begin{gather}
    \dot{\varepsilon^p}=\sqrt{\dfrac{2}{3}\dot{\bvarepsilon^p}:\dot{\bvarepsilon^p}}=\gamma\sqrt{\dfrac{2}{3}\underbrace{\bn:\bn}_\text{tensor}}=\gamma\sqrt{\dfrac{2}{3}\underbrace{\bn^\mT\mb{T}\bn}_\text{vector}},
\end{gather}
with $\mb{T}=\diag{\begin{matrix}
            1 & 1 & 1 & \dfrac{1}{2} & \dfrac{1}{2} & \dfrac{1}{2}
        \end{matrix}}$.
Both rules are effectively identical to the ones introduced previously in \S~\ref{sec:hoffman}.
However, since we only focus on the yield criterion in this section, we are not going to explore various alternatives.
Nevertheless, one shall be aware of the fact that based on the framework presented here, it is relatively easy to employ more complex hardening rules.
\subsection{Derivatives of Eigenvalues and Eigenvectors}\label{sec:derivative_eigen}
Now let $\lambda_i$ be the eigenvalues of the stress tensor $\mb{\aleph}$ and $\bv_i$ be the corresponding normalized eigenvectors, one has
\begin{gather}\label{eq:eigenvalue_def}
    \lambda_i=\underbrace{\bv_i^\mT\mb{\aleph}\bv_i}_{\text{vector/matrix}}=\underbrace{\bv_i\cdot\mb{\aleph}\cdot\bv_i}_{\text{tensor contraction}}=\underbrace{\left(\bv_i\otimes\bv_i\right):\mb{\aleph}}_{\text{tensor contraction}},\qquad
    \mb{\aleph}=\sum_{i}\lambda_i\bv_i\otimes\bv_i.
\end{gather}
The first attribute to note is that since $\bv_i$ is normalized, viz., $\bv_i\cdot\bv_i=1$, then
\begin{gather}
    \md{\left(\bv_i\cdot\bv_i\right)}=\md{\bv_i}\cdot\bv_i+\bv_i\cdot\md{\bv_i}=2\bv_i\cdot\md{\bv_i}=\md{1}=0\qquad\longrightarrow\qquad\bv_i\cdot\md{\bv_i}=0.
\end{gather}
Without loss of generality, one arrives at the conclusion that the change of a unit vector (or first order tensor) is perpendicular to the vector/tensor itself, i.e., $\bv_i\perp\md{\bv_i}$.
With this, it is possible to further write
\begin{gather}
    \md{\bv_i}\cdot\mb{\aleph}\cdot\bv_i=\lambda_i\cdot\md{\bv_i}\cdot\bv_i=0,
\end{gather}
thus, the differentiation of $\lambda_i$ can be expressed as
\begin{gather}\label{eq:eigenvalue_diff_def}
    \md{\lambda_i}=\md{\bv_i}\cdot\mb{\aleph}\cdot\bv_i+\bv_i\cdot\md{\mb{\aleph}}\cdot\bv_i+\bv_i\cdot\mb{\aleph}\cdot\md{\bv_i}=\bv_i\cdot\md{\mb{\aleph}}\cdot\bv_i+2\md{\bv_i}\cdot\mb{\aleph}\cdot\bv_i=\bv_i\cdot\md{\mb{\aleph}}\cdot\bv_i.
\end{gather}
Comparing \eqsref{eq:eigenvalue_def} and \eqsref{eq:eigenvalue_diff_def}, it is quite interesting to note that the mapping between $\lambda_i$ and $\mb{\aleph}$ is identical to that between $\md{\lambda_i}$ and $\md{\mb{\aleph}}$.
Alternatively, the above expressions are equivalent to the following,
\begin{gather}
    \md{\lambda_i}=\left(\bv_i\otimes\bv_i\right):\md{\mb{\aleph}}=\mb{P}_{ii}:\md{\mb{\aleph}}.
\end{gather}
We have defined $\mb{P}_{ii}=\bv_i\otimes\bv_i$ as a second order tensor.
Just like the stress/strain tensor, since $\mb{P}_{ii}$ is symmetric, it can also be expressed in the Voigt vector form.
In matrix form,
\begin{gather}
    \lambda_i=\underbrace{\mb{P}_{ii}}_{1\times6}\underbrace{\mb{\aleph}}_{6\times1},\qquad
    \md{\lambda_i}=\underbrace{\mb{P}_{ii}}_{1\times6}\underbrace{\md{\mb{\aleph}}}_{6\times1}.
\end{gather}
Note the double contraction is embedded in $\mb{P}_{ii}$, thus, the explicit matrix form of $\mb{P}_{ii}$ shall contain the corresponding scaling factor, which shall be
\begin{gather}
    \lambda_i=\underbrace{\mb{P}_{ii}}_{1\times6}\underbrace{\mb{\aleph}}_{6\times1}=\begin{bmatrix}
        \bv_{i,1}^2 & \bv_{i,2}^2 & \bv_{i,3}^2 & 2\bv_{i,1}\bv_{i,2} & 2\bv_{i,2}\bv_{i,3} & 2\bv_{i,3}\bv_{i,1}
    \end{bmatrix}\begin{bmatrix}
        \mb{\aleph}_{11} \\\mb{\aleph}_{22}\\\mb{\aleph}_{33}\\\mb{\aleph}_{12}\\\mb{\aleph}_{23}\\\mb{\aleph}_{31}
    \end{bmatrix}.
\end{gather}
One can also keep both the Voigt representation and the double contraction by defining
\begin{gather}
    \lambda_i=\underbrace{\mb{P}_{ii}}_{1\times6}:\underbrace{\mb{\aleph}}_{6\times1}=\begin{bmatrix}
        \bv_{i,1}^2 & \bv_{i,2}^2 & \bv_{i,3}^2 & \bv_{i,1}\bv_{i,2} & \bv_{i,2}\bv_{i,3} & \bv_{i,3}\bv_{i,1}
    \end{bmatrix}\diag{\begin{matrix}1\\1\\1\\2\\2\\2\end{matrix}}\begin{bmatrix}
        \mb{\aleph}_{11} \\\mb{\aleph}_{22}\\\mb{\aleph}_{33}\\\mb{\aleph}_{12}\\\mb{\aleph}_{23}\\\mb{\aleph}_{31}
    \end{bmatrix}.
\end{gather}
Such a mixed notation may appear to be a bit confusing, but it is actually quite useful in practice.

For $\md{\bv_i}$ itself, it can be derived, see \cite{Wu2006} and the references therein, and explicitly expressed as
\begin{gather}\label{eq:eigenvector_diff_def}
    \md{\bv_i}=\sum_{j=1,i\neq{}j}^3\dfrac{\mb{P}_{ij}:\md{\mb{\aleph}}}{\lambda_i-\lambda_j}\bv_j
\end{gather}
with $\mb{P}_{ij}=\dfrac{1}{2}\left(\bv_i\otimes\bv_j+\bv_j\otimes\bv_i\right)$ being also a symmetric second order tensor.
It can be observed that $ \md{\bv_i}$ is a linear combination of the other two eigenvectors.
Since $\bv_i\cdot\bv_j=\delta_{ij}$, indeed $\md{\bv_i}\cdot\bv_i=0$.
\eqsref{eq:eigenvector_diff_def} allows the following.
\begin{gather}
    \begin{split}
        \md{\mb{P}_{ii}}=\md{\left(\bv_i\otimes\bv_i\right)}=2\md{\bv_i}\otimes\bv_i & =2\sum_{j=1,i\neq{}j}^3\dfrac{\mb{P}_{ij}:\md{\mb{\aleph}}}{\lambda_i-\lambda_j}\bv_j\otimes\bv_i \\&=2\sum_{j=1,i\neq{}j}^3\dfrac{\mb{P}_{ij}:\md{\mb{\aleph}}}{\lambda_i-\lambda_j}\mb{P}_{ij}=2\sum_{j=1,i\neq{}j}^3\dfrac{\mb{P}_{ij}\otimes\mb{P}_{ij}}{\lambda_i-\lambda_j}:\md{\mb{\aleph}}.
    \end{split}
\end{gather}
It is assumed that $\lambda_i\neq\lambda_j$, that is, the eigenvalues are distinct.
For degenerated cases, the above expression is \textbf{not} valid.
But is it actually a problem?
In practice, it's likely that the algorithm used to perform eigen analysis is iterative, exactly identical eigenvalues are not likely to be obtained.
This is also guaranteed by the fact that $\mb{\aleph}$ are transformed tensors.
Unless carefully designed, to obtain very close eigenvalues is practically difficult, let alone exactly identical eigenvalues.
We will revisit this issue later.
\subsection{Yield Function}\label{sec:yld0418p_yield_function}
The YLD0418P yield criterion is expressed as a function of a set of \textbf{transformed} stress tensors, that is, in its most general form,
\begin{gather}
    f=f\left(\balpha,\bbeta,\bgamma,\cdots\right),
\end{gather}
where $\balpha$, $\bbeta$, $\bgamma$ are the transformed stress tensors that can be obtained by linear transformations of the deviatoric stress tensor $\bs=\dev{\bsigma}$.
In matrix form, for example,
\begin{gather}
    \mb{\aleph}=\mb{C_\aleph}\bs,
\end{gather}
where $\mb{\aleph}$ is the transformed stress tensor (any of $\balpha$, $\bbeta$, etc.), and $\mb{C_\aleph}$ is the corresponding constant transformation matrix that has the following structure.
\begin{gather}
    \mb{C_\aleph}=
    \begin{bmatrix}
                              & -C^{\mb{\aleph}}_{12} & -C^{\mb{\aleph}}_{13} &                      &                      &                      \\[2mm]
        -C^{\mb{\aleph}}_{21} &                       & -C^{\mb{\aleph}}_{23} &                      &                      &                      \\[2mm]
        -C^{\mb{\aleph}}_{31} & -C^{\mb{\aleph}}_{32} &                       &                      &                      &                      \\[2mm]
                              &                       &                       & C^{\mb{\aleph}}_{44} &                      &                      \\[2mm]
                              &                       &                       &                      & C^{\mb{\aleph}}_{55} &                      \\[2mm]
                              &                       &                       &                      &                      & C^{\mb{\aleph}}_{66}
    \end{bmatrix}.
\end{gather}
As mainly discussed in \cite{Barlat2005}, the YLD0418P yield function only focus on two of those transformed stress tensors, namely $\balpha$ and $\bbeta$.
Let $\alpha_i$ and $\beta_i$ ($i\in\{1,2\}$ for 2D and $i\in\{1,2,3\}$ for 3D) be the principal values of $\balpha$ and $\bbeta$, viz., the eigenvalues of the corresponding tensors.
In 3D stress space, the yield function is then expressed as
\begin{gather}\label{eq:yld0418p_yield_function}
    f=\sum_{i=1,j=1}^3\abs{\alpha_i-\beta_j}^m-4\sigma_y^m.
\end{gather}
In above, the exponent $m$ is a model parameter that controls the shape of the yield surface.
One can spot a few interesting features of \eqsref{eq:yld0418p_yield_function}.
\begin{enumerate}
    \item The shape is controlled by principal stresses.
    \item The summation involves nine terms in total.
    \item Depending on the parameters used to define the transformation matrices, the transformed stresses $\balpha$ and $\bbeta$ may or may not share the same principal directions.
    \item Noting that $\bs$ is a deviatoric stress tensor, thus the yield function is hydrostatic pressure independent. However, $\balpha$ and $\bbeta$ may or may not be deviatoric --- depending on the transformation matrices.
\end{enumerate}

It is important to also notice that $f$ has a unit of that of stress.
Considering that $m$ could be potentially (very) large, up to \num{40} or so, to avoid numerical issues, we use the following revision in this section.
\begin{gather}\label{eq:yld0418p_yield_function_normalized}
    f=\sum_{i=1}^3\sum_{j=1}^3\abs{\dfrac{\alpha_i-\beta_j}{\sigma_\text{ref}}}^m-4\sigma_y^m,
\end{gather}
where $\sigma_\text{ref}$ is a reference stress.
As a result, $\sigma_y$ shall be dimensionless and evaluates to unity for zero plastic strain.

It is necessary to derive the gradients of $f$ in order to implement the plasticity model.
Using the chain rule, by denoting $\Delta_{ij}=\dfrac{\alpha_i-\beta_j}{\sigma_\text{ref}}$, it is straightforward to derive the following.
\begin{gather}
    \pdfrac{f}{\alpha_i}=\dfrac{m}{\sigma_\text{ref}}\sum_{j=1}^3\abs{\Delta_{ij}}^{m-1}\sign{\Delta_{ij}}=\dfrac{m}{\sigma_\text{ref}}\sum_{j=1}^3\abs{\Delta_{ij}}^{m-2}\Delta_{ij},\\
    \pdfrac{f}{\beta_j}=-\dfrac{m}{\sigma_\text{ref}}\sum_{i=1}^3\abs{\Delta_{ij}}^{m-1}\sign{\Delta_{ij}}=-\dfrac{m}{\sigma_\text{ref}}\sum_{i=1}^3\abs{\Delta_{ij}}^{m-2}\Delta_{ij},
\end{gather}
and the second order derivatives
\begin{gather}
    \pdfrac{^2f}{\alpha_i^2}=\dfrac{m\left(m-1\right)}{\sigma_\text{ref}^2}\sum_{j=1}^3\abs{\Delta_{ij}}^{m-2},\\
    \pdfrac{^2f}{\beta_j^2}=\dfrac{m\left(m-1\right)}{\sigma_\text{ref}^2}\sum_{i=1}^3\abs{\Delta_{ij}}^{m-2},\\
    \pdfrac{^2f}{\alpha_i\partial\beta_j}=-\dfrac{m\left(m-1\right)}{\sigma_\text{ref}^2}\abs{\Delta_{ij}}^{m-2}.
\end{gather}

We have prepared all necessary ingredients to compute the full derivatives of (the summation term of) the yield function in the previous sections.
\begin{gather}
    \begin{split}
        \pdfrac{f}{\bs} & =\sum_{i=1}^{3}\pdfrac{f}{\alpha_i}\ddfrac{\alpha_i}{\balpha}:\ddfrac{\balpha}{\bs}+\sum_{i=1}^{3}\pdfrac{f}{\beta_i}\ddfrac{\beta_i}{\bbeta}:\ddfrac{\bbeta}{\bs} \\
                        & =\sum_{i=1}^{3}\pdfrac{f}{\alpha_i}\mb{P}_{ii}^{\balpha}:\mb{C}_{\balpha}+\sum_{i=1}^{3}\pdfrac{f}{\beta_i}\mb{P}_{ii}^{\bbeta}:\mb{C}_{\bbeta}.
    \end{split}
\end{gather}
Note that in the above we only consider the contribution of the summation term, while the contribution of $\sigma_y$ is not included.
This is also the reason why we use partial derivative notation in the above expression.
It is possible to rewrite the above expression in a more compact form, using the following notation.
\begin{gather}
    \pdfrac{f}{\balpha}=\underbrace{\begin{bmatrix}
            \pdfrac{f}{\alpha_1} & \pdfrac{f}{\alpha_2} & \pdfrac{f}{\alpha_3}
        \end{bmatrix}}_{1\times3}\underbrace{\begin{bmatrix}
            \mb{P}_{11}^{\balpha} \\[2mm]\mb{P}_{22}^{\balpha}\\[2mm]\mb{P}_{33}^{\balpha}
        \end{bmatrix}}_{3\times6},\qquad
    \pdfrac{f}{\bbeta}=\underbrace{\begin{bmatrix}
            \pdfrac{f}{\beta_1} & \pdfrac{f}{\beta_2} & \pdfrac{f}{\beta_3}
        \end{bmatrix}}_{1\times3}\underbrace{\begin{bmatrix}
            \mb{P}_{11}^{\bbeta} \\[2mm]\mb{P}_{22}^{\bbeta}\\[2mm]\mb{P}_{33}^{\bbeta}
        \end{bmatrix}}_{3\times6}.
\end{gather}
Thus,
\begin{gather}
    \pdfrac{f}{\bs}=\pdfrac{f}{\balpha}:\mb{C}_{\balpha}+\pdfrac{f}{\bbeta}:\mb{C}_{\bbeta}.
\end{gather}
One shall be aware of the fact that in the above expression, double contraction stemming from tensor operations is required between quantities expressed in matrix form.
In explicit matrix form, (the transpose of) the above can be expressed as a column vector
\begin{gather}
    \underbrace{\pdfrac{f}{\bs}}_{6\times1}=\sum_{\mb{\aleph}\in\{\balpha,\bbeta\}}\mb{C}_{\mb{\aleph}}^\mT\underbrace{\begin{bmatrix}
        1 &   &   &   &   &   \\
          & 1 &   &   &   &   \\
          &   & 1 &   &   &   \\
          &   &   & 2 &   &   \\
          &   &   &   & 2 &   \\
          &   &   &   &   & 2
    \end{bmatrix}}_{\text{double contraction scaling}}\underbrace{\begin{bmatrix}
            \mb{P}_{11}^{\mb{\aleph},\mT} & \mb{P}_{22}^{\mb{\aleph},\mT} & \mb{P}_{33}^{\mb{\aleph},\mT}
        \end{bmatrix}}_{6\times3}\underbrace{\begin{bmatrix}
            \pdfrac{f}{\aleph_1} \\[4mm] \pdfrac{f}{\aleph_2} \\[4mm] \pdfrac{f}{\aleph_3}
        \end{bmatrix}}_{3\times1}.
\end{gather}
Accordingly, the plastic flow direction is simply the deviatoric part, that is,
\begin{gather}
    \pdfrac{f}{\bsigma}=\dev{\pdfrac{f}{\bs}}.
\end{gather}

Furthermore,
\begin{gather}\label{eq:yld0418p_dfdaleph}
    \begin{split}
        \pdfrac{^2f}{\mb{\aleph}^2} & =\sum_{i=1}^{3}\left(\pdfrac{^2f}{\aleph_i^2}\mb{P}_{ii}^{\mb{\aleph}}\otimes\mb{P}_{ii}^{\mb{\aleph}}+2\pdfrac{f}{\aleph_i}\sum_{j=1,i\neq{}j}^3\dfrac{\mb{P}_{ij}^{\mb{\aleph}}\otimes\mb{P}_{ij}^{\mb{\aleph}}}{\aleph_i-\aleph_j}\right)                                   \\
                                    & =\sum_{i=1}^{3}\pdfrac{^2f}{\aleph_i^2}\mb{P}_{ii}^{\mb{\aleph}}\otimes\mb{P}_{ii}^{\mb{\aleph}}+\sum_{(i,j)\in\{(12),(23),(31)\}}\dfrac{2}{\aleph_i-\aleph_j}\left(\pdfrac{f}{\aleph_i}-\pdfrac{f}{\aleph_j}\right)\mb{P}_{ij}^{\mb{\aleph}}\otimes\mb{P}_{ij}^{\mb{\aleph}}.
    \end{split}
\end{gather}
In matrix form, if we define
\begin{gather}
    \mb{P}^{\mb{\aleph}}=\underbrace{\begin{bmatrix}
            \mb{P}_{11}^{\mb{\aleph}} \\[3mm]
            \mb{P}_{22}^{\mb{\aleph}} \\[3mm]
            \mb{P}_{33}^{\mb{\aleph}} \\[3mm]
            \mb{P}_{12}^{\mb{\aleph}} \\[3mm]
            \mb{P}_{23}^{\mb{\aleph}} \\[3mm]
            \mb{P}_{31}^{\mb{\aleph}}
        \end{bmatrix}}_{6\times6},
\end{gather}
\eqsref{eq:yld0418p_dfdaleph} can be equivalently expressed as
\begin{gather}
    \pdfrac{^2f}{\mb{\aleph}^2}= \mb{P}^{\mb{\aleph},\mT}\diag{\begin{matrix}
            \pdfrac{^2f}{\aleph_1^2}                                                           \\[3mm]
            \pdfrac{^2f}{\aleph_2^2}                                                           \\[3mm]
            \pdfrac{^2f}{\aleph_3^2}                                                           \\[3mm]
            \dfrac{2}{\aleph_1-\aleph_2}\left(\pdfrac{f}{\aleph_1}-\pdfrac{f}{\aleph_2}\right) \\[3mm]
            \dfrac{2}{\aleph_2-\aleph_3}\left(\pdfrac{f}{\aleph_2}-\pdfrac{f}{\aleph_3}\right) \\[3mm]
            \dfrac{2}{\aleph_3-\aleph_1}\left(\pdfrac{f}{\aleph_3}-\pdfrac{f}{\aleph_1}\right)
        \end{matrix}}\mb{P}^{\mb{\aleph}}.
\end{gather}
What a compact and elegant expression!
The matrix $\mb{P}^{\mb{\aleph}}$ is a projection matrix that purely depends on the eigenvectors.
The diagonal matrix accounts for both the direct second order partial derivatives and the contribution of the rotating eigenvectors.
If we further define $\mb{H}^{\mb{\aleph}}=\mb{P}^{\mb{\aleph}}:\mb{C}_{\mb{\aleph}}$, then,
\begin{gather}
    \pdfrac{^2f}{\bs^2}=\mb{H}^{\mb{\aleph},\mT}\diag{\cdots}\mb{H}^{\mb{\aleph}},
\end{gather}
in which for brevity the middle diagonal matrix is not repeated.

Now let's study the degenerated case when $\alpha_i$ approaches $\alpha_j$.
By examining the expression,
\begin{gather}
    \pdfrac{f}{\alpha_i}=\dfrac{m}{\sigma_\text{ref}}\sum_{j=1}^3\abs{\Delta_{ij}}^{m-2}\Delta_{ij},
\end{gather}
it is easy to see that $\pdfrac{f}{\alpha_i}$ is a function of $\alpha_i$ only (in the sense that the other two principal stresses in $\balpha$ do not appear in the expression).
When $\alpha_i\approx\alpha_j$, the fraction essentially approximates the second order derivative.
\begin{gather}
    \lim_{\aleph_1\approx\aleph_2}\dfrac{2}{\aleph_i-\aleph_j}\left(\pdfrac{f}{\aleph_i}-\pdfrac{f}{\aleph_j}\right)=2\pdfrac{^2f}{\aleph_i^2}\Big|_{\aleph_j}=2\pdfrac{^2f}{\aleph_j^2}\Big|_{\aleph_i}.
\end{gather}
To account for symmetry, it is possible to use $\pdfrac{^2f}{\aleph_i^2}\Big|_{\aleph_j}+\pdfrac{^2f}{\aleph_j^2}\Big|_{\aleph_i}$ to approximate it.

Finally, it must be pointed out that the above derivation, without loss of generality, can be applied to 2D cases \cite{Barlat2003} with minor modifications.
\subsection{Formulation}
The expressions are summarised in the following table.
\begin{table}[H]
    \centering
    \begin{tabular}{rl}
        \toprule
        Constitutive Law & $\bsigma=\mb{E}\left(\bvarepsilon-\bvarepsilon^p\right)$                \\[2mm]
        Yield Function   & $\displaystyle{}f=\sum_{i=1}^3\sum_{j=1}^3\abs{\alpha_i-\beta_j}^m-4\sigma_y^m$ \\[2mm]
        Flow Rule        & $\dot{\bvarepsilon^p}=\gamma\bn=\gamma\cdot\dev{\pdfrac{f}{\bs}}$       \\[4mm]
        Hardening Law    & $\displaystyle{}\dot{\varepsilon^p}=\gamma\norm{\bn}_e=\gamma\sqrt{\dfrac{2}{3}\bn:\bn}$ \\\bottomrule
    \end{tabular}
\end{table}
\subsubsection{Local System}
It is very important to note that since the yield function $f$ is defined in the deviatoric stress space, it is \textbf{not} possible to take $\bsigma$ as an independent variable as done in the previous orthotropic model.
This is reasonable as it is \textbf{not} possible to determine $\bsigma$ from $\bs$ as the spherical part remains unknown.
Thus, the local variable is taken as $\mb{x}=\underbrace{\begin{bmatrix}\gamma&\bs\end{bmatrix}^\mT}_{7\times1}$, the local system consists of two residuals.
\begin{gather}
    \mb{R}=\left\{\begin{array}{l}\displaystyle
       \sum_{i=1}^3\sum_{j=1}^3\abs{\alpha_i-\beta_j}^m-4\sigma_y^m, \\[4mm]
       \bs+\mb{E}\bn\gamma-\bs^\text{trial}.
    \end{array}\right.
\end{gather}
Only when the material is isotropic, the second residual can be simplified as $\bs+2G\bn\gamma-\bs^\text{trial}$ given that $\bn$ is deviatoric.
For orthotropic materials, there is no concise way to simplify it anymore.
The Jacobian is thus
\begin{gather}
    \mb{J}=\underbrace{\begin{bmatrix}
    -4m\sigma_y^{m-1}\ddfrac{\sigma_y}{\varepsilon^p}\norm{\bn}_e & \pdfrac{f}{\bs}-4m\sigma_y^{m-1}\ddfrac{\sigma_y}{\varepsilon^p}\gamma\ddfrac{\norm{\bn}_e}{\bn}:\mathbb{I}^\text{dev}:\pdfrac{^2f}{\bs^2} \\[4mm]
    \mb{E}\bn                                                   & \mb{I}+\mb{E}\mathbb{I}_\text{dev}:\pdfrac{^2f}{\bs^2}
    \end{bmatrix}}_{7\times7}.
    \end{gather}
    \subsubsection{Consistent Tangent Stiffness}
    From the stress update expression,
    \begin{gather}
        \bsigma=\bsigma^\text{trial}-\gamma\mb{E}\bn,
    \end{gather}
    the consistent tangent stiffness can be derived as
    \begin{gather}
        \begin{split}
            \ddfrac{\bsigma}{\bvarepsilon} & =\mb{E}-\mb{E}\left(\gamma\ddfrac{\bn}{\bvarepsilon}+\bn\ddfrac{\gamma}{\bvarepsilon}\right)=\mb{E}-\mb{E}\left(\gamma\mathbb{I}^\text{dev}:\pdfrac{^2f}{\bs^2}\ddfrac{\bs}{\bvarepsilon}+\bn\ddfrac{\gamma}{\bvarepsilon}\right) \\
                                           & =\mb{E}-\mb{E}\begin{bmatrix}
                                                               \bn & \gamma\mathbb{I}^\text{dev}:\pdfrac{^2f}{\bs^2}
                                                           \end{bmatrix}\begin{bmatrix}
                                                                            \ddfrac{\gamma}{\bvarepsilon} \\[4mm]
                                                                            \ddfrac{\bs}{\bvarepsilon}
                                                                        \end{bmatrix}=\mb{E}-\mb{E}\mathbb{I}^\text{dev}:\begin{bmatrix}
                                                                                                                            \pdfrac{f}{\bs} & \gamma\pdfrac{^2f}{\bs^2}
                                                                                                                        \end{bmatrix}\ddfrac{\mb{x}}{\bvarepsilon}.
        \end{split}
    \end{gather}
    In the above, $\ddfrac{\mb{x}}{\bvarepsilon}$ can be obtained from the local equilibrium,
    \begin{gather}
        \md{\mb{R}}=\pdfrac{\mb{R}}{\mb{x}}\md{\mb{x}}+\pdfrac{\mb{R}}{\bvarepsilon}\md{\bvarepsilon}=\mb{0}\qquad\longrightarrow\qquad\ddfrac{\mb{x}}{\bvarepsilon}=-\mb{J}^{-1}\pdfrac{\mb{R}}{\bvarepsilon},
    \end{gather}
    with
    \begin{gather}
        \pdfrac{\mb{R}}{\bvarepsilon}=\begin{bmatrix}
            \mb{0} \\[4mm]
            -\mathbb{I}^\text{dev}\mb{E}
        \end{bmatrix},
    \end{gather}
    thus, the final expression of the consistent tangent stiffness is
    \begin{gather}
    \ddfrac{\bsigma}{\bvarepsilon}=\mb{E}-\mb{E}\mathbb{I}^\text{dev}:\underbrace{\begin{bmatrix}
            \pdfrac{f}{\bs} & \gamma\pdfrac{^2f}{\bs^2}
        \end{bmatrix}}_{6\times7}\mb{J}^{-1}\underbrace{\begin{bmatrix}
            \mb{0} \\[4mm]
            \mathbb{I}^\text{dev}\mb{E}
        \end{bmatrix}}_{7\times6}.
\end{gather}
It is also worth noting that $\mathbb{I}^\text{dev}\mb{E}=\mb{E}\mathbb{I}^\text{dev}$ regardless of orthotropic or isotropic elasticity.
\subsection{Kinematic Hardening Extension}
Common orthotropic materials such as timber, bones, etc., do not exhibit kinematic hardening, this is the reason that kinematic hardening was not introduced in the previous sections.
However, as the YLD0418P yield criterion is widely used in the context of sheet metals, extending the model to include kinematic hardening is meaningful.

With the introduction of the back stress $\ba$, the corresponding shifted stress shall be defined as the difference between $\bs$ and $\ba$ as
\begin{gather}
\beeta=\bs-\ba.
\end{gather}
The evolution of $\ba$ can be defined as
\begin{gather}
\dot{\ba}=k_r\left(\sqrt{\dfrac{2}{3}}k_b\dfrac{\bn}{\norm{\bn}}-\ba\right)\gamma=k_r\left(\sqrt{\dfrac{2}{3}}k_b\bu-\ba\right)\gamma.
\end{gather}
It is very important to point out that the above definition is isotropic, whether it is appropriate with an anisotropic yield criterion remains to be investigated.

Expressing the above evolution rule in its discretised form using the implicit Euler method, one has
\begin{gather}
    \ba=\ba_n+k_r\left(\sqrt{\dfrac{2}{3}}k_b\bu-\ba\right)\gamma.
\end{gather}
Rearranging gives
\begin{gather}
    \left(1+\gamma{}k_r\right)\ba=\ba_n+\gamma\sqrt{\dfrac{2}{3}}k_rk_b\bu.
\end{gather}
Denoting
\begin{gather}
    R_{\ba}=\ba_n+\gamma\sqrt{\dfrac{2}{3}}k_rk_b\bu-\left(1+\gamma{}k_r\right)\ba,
\end{gather}
then the corresponding derivatives are
\begin{gather}
    \pdfrac{R_{\ba}}{\gamma}=\sqrt{\dfrac{2}{3}}k_rk_b\bu-k_r\ba,\\
    \pdfrac{R_{\ba}}{\bs}=\gamma\sqrt{\dfrac{2}{3}}k_rk_b\pdfrac{\bu}{\bs},\\
    \pdfrac{R_{\ba}}{\ba}=\gamma\sqrt{\dfrac{2}{3}}k_rk_b\pdfrac{\bu}{\ba}-\left(1+\gamma{}k_r\right)\mb{I}.
\end{gather}
The term $\pdfrac{\bu}{\bs}$ can be explicitly derived using the chain rule,
\begin{gather}
    \pdfrac{\bu}{\bs}=\pdfrac{\bu}{\bn}:\pdfrac{\bn}{\bs}=\dfrac{1}{\norm{\bn}}\left(\mathbb{I}-\bu\otimes\bu\right):\mathbb{I}^\text{dev}:\pdfrac{^2f}{\bs^2}.
\end{gather}
One must pay extra attention to two double contractions in the above expression, are they identical?
Interested readers are encouraged to verify this by explicitly writing down the matrix form of the above expression.

With the presence of $\ba$, one shall see that now
\begin{gather}
    \mb{\aleph}=\mb{C_\aleph}\beeta,
\end{gather}
and
\begin{gather}
    \pdfrac{\mb{\aleph}}{\bs}=-\pdfrac{\mb{\aleph}}{\ba}.
\end{gather}
Furthermore, since
\begin{gather}
    \pdfrac{\beeta}{\bs}=\mb{I}\quad\text{(Voigt vector notation)},\qquad
    \pdfrac{\beeta}{\bs}=\mathbb{I}\quad\text{(tensor notation)},
\end{gather}
the explicit expression of $\pdfrac{f}{\bs}$ remains unchanged, and
\begin{gather}
    \pdfrac{f}{\ba}=-\pdfrac{f}{\bs}.
\end{gather}
\subsubsection{Local System}
Taking $\ba$ as the additional local variable such that $\mb{x}=\underbrace{\begin{bmatrix}\gamma&\bs&\ba\end{bmatrix}^\mT}_{13\times1}$, now the local system becomes
\begin{gather}
    \mb{R}=\left\{\begin{array}{l}\displaystyle
        \sum_{i=1}^3\sum_{j=1}^3\abs{\alpha_i-\beta_j}^m-4\sigma_y^m, \\[4mm]
        \bs+\mb{E}\bn\gamma-\bs^\text{trial},                         \\[4mm]
        \ba_n+\gamma\sqrt{\dfrac{2}{3}}k_rk_b\bu-\left(1+\gamma{}k_r\right)\ba.
    \end{array}\right.
\end{gather}
The Jacobian is thus
\begin{gather}
    \mb{J}=\underbrace{\begin{bmatrix}
            -4m\sigma_y^{m-1}\ddfrac{\sigma_y}{\varepsilon^p}\norm{\bn}_e & \pdfrac{R_f}{\bs}                                            & -\pdfrac{R_f}{\bs}                                                                \\[4mm]
            \mb{E}\bn                                                     & \mb{I}+\gamma\mb{E}\mathbb{I}^\text{dev}:\pdfrac{^2f}{\bs^2} & -\gamma\mb{E}\mathbb{I}^\text{dev}:\pdfrac{^2f}{\bs^2}                            \\[4mm]
            \sqrt{\dfrac{2}{3}}k_rk_b\bu-k_r\ba                           & \gamma\sqrt{\dfrac{2}{3}}k_rk_b\pdfrac{\bu}{\bs}             & \gamma\sqrt{\dfrac{2}{3}}k_rk_b\pdfrac{\bu}{\ba}-\left(1+\gamma{}k_r\right)\mb{I}
        \end{bmatrix}}_{13\times13}.
\end{gather}
In which the following is used,
\begin{gather}
\pdfrac{R_f}{\bs}=\pdfrac{f}{\bs}-4m\sigma_y^{m-1}\ddfrac{\sigma_y}{\varepsilon^p}\gamma\ddfrac{\norm{\bn}_e}{\bn}:\mathbb{I}^\text{dev}:\pdfrac{^2f}{\bs^2}.
\end{gather}

As a practice, please derive the consistent tangent stiffness following a similar approach used in the previous section.
\subsection{Implementation}
The following method computes the yield function and its corresponding derivatives/gradients.
\begin{cppcode}
YLD0418P::compute_yield_surface
\end{cppcode}

The local returning mapping further uses two algorithms depending on whether kinematic hardening is used.
\begin{cppcode}
YLD0418P::update_trial_status
\end{cppcode}

With kinematic hardening enabled, the local system has a size of \num{13}.
\begin{cppcode}
YLD0418P::with_kinematic
\end{cppcode}

With kinematic hardening disabled, the local system has a size of \num{7}.
This makes the Jacobian \SI{70}{\percent} smaller compared to the one with kinematic hardening.
\begin{cppcode}
YLD0418P::without_kinematic
\end{cppcode}
