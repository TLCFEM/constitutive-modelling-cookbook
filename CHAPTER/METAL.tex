\chapter{Metal}
In this chapter, several frameworks suitable for developing metal models are introduced. The basic one is the von Mises framework, which is also called J2 model in some literature as it adopts the second invariant of the deviatoric stress to characterise yield function. The intermediate one is the VAFCRP model, in which a Voce type nonlinear isotropic hardening, a multiplicative Armstrong--Fredrick type kinematic hardening and a Peric type viscosity are implemented. Thus, this model can account for dynamic effects. The third model is a general framework developed based on the Hoffman criterion, it is suitable for orthotropic materials.
\section{von Mises Framework}
Here the uniaxial combined isotropic/kinematic model introduced in \secref{sec:isotropic/kinematic} is reformulated in 3D space. Some difference will be observed, but the final local residual is a scalar equation.
\subsection{Theory}
\subsubsection{Yield Function}
The von Mises yield criterion is adopted,
\begin{gather}
f=\norm{\beeta}-\sqrt{\dfrac{2}{3}}\sigma^y,
\end{gather}
with $\beeta=\bs-\balpha$ is the shifted stress with $\balpha$ denoting the back stress and $\bs=\dev{\bsigma}$ denoting the deviatoric stress. The only purpose of the constant $\sqrt{\dfrac{2}{3}}$ is to produce consistent response under uniaxial loading compared to the uniaxial version with the same set of model parameters. By definition, the back stress $\balpha$ is also a deviatoric stress, thus $\tr{\balpha}=0$.
It shifts the centre of yield surface in the deviatoric stress space.
\subsubsection{Flow Rule}
Assuming associative rule, the flow rule is
\begin{gather}\label{eq:j2_flow}
\dot{\bvarepsilon^p}=\gamma\pdfrac{f}{\bsigma}=\gamma\pdfrac{\norm{\beeta}}{\bsigma}=\gamma\dfrac{1}{2}\dfrac{2\mathbb{I}^\text{dev}:\beeta}{\norm{\beeta}}=\gamma\dfrac{\beeta}{\norm{\beeta}}=\gamma\bn.
\end{gather}
All analytical formulations are based on tensor notation.
If the above expression is expressed explicitly in tensor components, one could obtain
\begin{gather}
\underbrace{\begin{bmatrix}
\dot{\varepsilon^p}_{11}&\dot{\varepsilon^p}_{12}&\dot{\varepsilon^p}_{31}\\
\dot{\varepsilon^p}_{12}&\dot{\varepsilon^p}_{22}&\dot{\varepsilon^p}_{23}\\
\dot{\varepsilon^p}_{31}&\dot{\varepsilon^p}_{23}&\dot{\varepsilon^p}_{33}
\end{bmatrix}}_{\dot{\bvarepsilon^p}}
=
\gamma
\underbrace{\begin{bmatrix}
n_{11}&n_{12}&n_{31}\\
n_{12}&n_{22}&n_{23}\\
n_{31}&n_{23}&n_{33}
\end{bmatrix}}_{\bn},\qquad\text{in tensor components.}
\end{gather}
We define the scaling vector $\mb{c}=\begin{bmatrix}
1&1&1&2&2&2
\end{bmatrix}^\mT$ and let $\circ$ be the Hadamard (element--wise) product operator, then the above expression is equivalent to
\begin{gather}
\underbrace{\begin{bmatrix}
\dot{\varepsilon^p}_{11}\\
\dot{\varepsilon^p}_{22}\\
\dot{\varepsilon^p}_{33}\\
2\dot{\varepsilon^p}_{12}\\
2\dot{\varepsilon^p}_{23}\\
2\dot{\varepsilon^p}_{31}
\end{bmatrix}}_{\dot{\bvarepsilon^p}}
=
\gamma\begin{bmatrix}
1&&&&&\\
&1&&&&\\
&&1&&&\\
&&&2&&\\
&&&&2&\\
&&&&&2
\end{bmatrix}
\underbrace{
\begin{bmatrix}
n_{11}\\
n_{22}\\
n_{33}\\
n_{12}\\
n_{23}\\
n_{31}
\end{bmatrix}}_{\bn}=\gamma~\mb{c}\circ\bn,\qquad\text{in the Voigt notation.}
\end{gather}
This agrees with \eqsref{eq:norm_derivative2}.
Such a difference has practical implications.
Due to symmetry of both strain and stress tensors, it is often more performant to just use vectors and matrices to represent second--order and fourth--order tensors in implementation.
This means tensor quantities are expressed in compressed vector/matrix form.
For a given expression, one shall always question what operation it implies and how it would be eventually computed for operands expressed in different forms.

It can be observed that $\dot{\bvarepsilon^p}$ has a magnitude of $\gamma$ while $\bn$ is a unit tensor in $\mathbb{R}^3\times\mathbb{R}^3$ hyper space. Thus $\bn$ serves as a direction indicator, $\bvarepsilon^p$ evolves towards $\bn$ by the amount of $\gamma$. Since $\bn$ is deviatoric, $\bvarepsilon^p$ is also deviatoric.
Furthermore, since $\bn$ is a unit tensor,
\begin{gather}
\norm{\dot{\bvarepsilon^p}}=\gamma.
\end{gather}
\subsubsection{Hardening Law}
For internal variable $q$, the hardening law takes the accumulated magnitude of $\bvarepsilon^p$.
\begin{gather}
\dot{q}=\sqrt{\dfrac{2}{3}}\norm{\dot{\bvarepsilon^p}}=\sqrt{\dfrac{2}{3}}\gamma.
\end{gather}

For isotropic hardening, $\sigma^y$ is defined as a general function of $q$,
\begin{gather}
\sigma^y=\sigma^y\left(q\right).
\end{gather}

For kinematic hardening, the evolution law of back stress $\balpha$ is defined to be
\begin{gather}
\dot{\balpha}=\sqrt{\dfrac{2}{3}}\dot{H}\bn.
\end{gather}
in which $H=H\left(q\right)$ is now a scalar--valued function of $q$ that controls the development of $\norm{\balpha}$, $\balpha$ always evolves towards $\bn$ by the amount $\dot{H}=H\left(q_{n+1}\right)-H\left(q_n\right)$ characterising the increment of $H$. The fraction $\sqrt{\dfrac{2}{3}}$ is introduced for consistent response as stated earlier. Since $\dot{\balpha}$ and $\bn$ are coaxial, $\balpha$ stays deviatoric but may not be coaxial with all $\bn$ through the loading process.
\subsection{Formulation}
The summation of the von Mises model is listed as follows.
\begin{table}[ht]
\centering
\begin{tabular}{rl}
\toprule
Constitutive Law&$\bsigma=\mb{D}:\left(\bvarepsilon-\bvarepsilon^p\right)$\\
Yield Function&$f=\norm{\beeta}-\sqrt{\dfrac{2}{3}}\sigma^y$\\
Flow Rule&$\dot{\bvarepsilon^p}=\gamma\bn$\\
Hardening Law&$\dot{q}=\sqrt{\dfrac{2}{3}}\gamma$\\
&$\dot{\balpha}=\sqrt{\dfrac{2}{3}}\dot{H}\bn$\\\bottomrule
\end{tabular}
\end{table}
\subsubsection{Elastic Loading/Unloading}
The trial stress can be computed such as
\begin{gather}\label{eq:j2_tsigma}
\bsigma^\text{trial}=\mb{D}:\left(\bvarepsilon_{n+1}-\bvarepsilon^p_n\right)=\bsigma_n+\mb{D}:\left(\bvarepsilon_{n+1}-\bvarepsilon_n\right).
\end{gather}
In matrix representation, it is
\begin{gather}
\underbrace{
\begin{bmatrix}
\sigma^\text{trial}_{11}\\
\sigma^\text{trial}_{22}\\
\sigma^\text{trial}_{33}\\
\sigma^\text{trial}_{12}\\
\sigma^\text{trial}_{23}\\
\sigma^\text{trial}_{31}
\end{bmatrix}}_{\bsigma^\text{trial}}
=
\underbrace{\begin{bmatrix}
\lambda+2G&\lambda&\lambda&\cdot&\cdot&\cdot\\
\lambda&\lambda+2G&\lambda&\cdot&\cdot&\cdot\\
\lambda&\lambda&\lambda+2G&\cdot&\cdot&\cdot\\
\cdot&\cdot&\cdot&G&\cdot&\cdot\\
\cdot&\cdot&\cdot&\cdot&G&\cdot\\
\cdot&\cdot&\cdot&\cdot&\cdot&G
\end{bmatrix}}_{\mb{D}}
\underbrace{\left(\begin{bmatrix}
\varepsilon_{11,n+1}\\
\varepsilon_{22,n+1}\\
\varepsilon_{33,n+1}\\
     \gamma_{12,n+1}\\
     \gamma_{23,n+1}\\
     \gamma_{31,n+1}
\end{bmatrix}-\begin{bmatrix}
\varepsilon^p_{11,n}\\
\varepsilon^p_{22,n}\\
\varepsilon^p_{33,n}\\
     \gamma^p_{12,n}\\
     \gamma^p_{23,n}\\
     \gamma^p_{31,n}
\end{bmatrix}.
\right)}_{\dot{\bvarepsilon_{n+1}}-\dot{\bvarepsilon^p_n}}
\end{gather}

Then $\beeta^\text{trial}=\dev{\bsigma^\text{trial}}-\balpha_n$, which gives trial yield function
\begin{gather}\label{eq:j2_tf}
f^\text{trial}=\norm{\beeta^\text{trial}}-\sqrt{\dfrac{2}{3}}\sigma^y_n
\end{gather}
with $\sigma^y_n=\sigma^y\left(q_n\right)$ evaluated with current $q_n$.
\subsubsection{Plastic Evolution}
By default, we present the formulation with the implicit Euler method.

If plasticity evolution occurs,
\begin{gather}
\beeta_{n+1}=\dev{\bsigma_{n+1}}-\balpha_{n+1}.
\end{gather}
Accounting for
\begin{gather}
\bsigma_{n+1}=\bsigma^\text{trial}-\mb{D}:\dot{\bvarepsilon^p}=\bsigma^\text{trial}-\gamma2G\bn,\\
\balpha_{n+1}=\balpha_n+\sqrt{\dfrac{2}{3}}\dot{H}\bn,
\end{gather}
one can further derive
\begin{gather}
\beeta_{n+1}=\dev{\bsigma^\text{trial}-\gamma2G\bn}-\balpha_n-\sqrt{\dfrac{2}{3}}\dot{H}\bn.
\end{gather}
Since $\bn$ is already deviatoric, meaning $\dev{\bn}=\bn$, thus,
\begin{gather}
\begin{split}
\beeta_{n+1}&=\dev{\bsigma^\text{trial}}-\gamma2G\bn-\balpha_n-\sqrt{\dfrac{2}{3}}\dot{H}\bn\\
&=\beeta^\text{trial}-\left(2G\gamma+\sqrt{\dfrac{2}{3}}\dot{H}\right)\bn.
\end{split}
\end{gather}
Noting that, by definition, $\bn=\dfrac{\beeta_{n+1}}{\norm{\beeta_{n+1}}}$, thus,
\begin{gather}
\underbrace{\left(1+\dfrac{2G\gamma+\sqrt{\dfrac{2}{3}}\dot{H}}{\norm{\beeta_{n+1}}}\right)}_{\text{scalar}}\beeta_{n+1}=\beeta^\text{trial}.
\end{gather}
The above expression implies that $\beeta_{n+1}$ and $\beeta^\text{trial}$ are \textbf{coaxial}. Thus, $\beeta_{n+1}$ always points to the direction of $\beeta^\text{trial}$, which is a constant in each time step within that time step. Only its magnitude would be adjusted in local iterations.
One can further derive
\begin{gather}
\norm{\beeta_{n+1}}+2G\gamma+\sqrt{\dfrac{2}{3}}\dot{H}=\norm{\beeta^\text{trial}}.
\end{gather}

The yield function evaluated at the new state reads
\begin{gather}\label{eq:j2_f}
f=\norm{\beeta^\text{trial}}-\left(2G\gamma+\sqrt{\dfrac{2}{3}}\left(H_{n+1}-H_n\right)\right)-\sqrt{\dfrac{2}{3}}\sigma^y_{n+1}.
\end{gather}

The Jacobian reads
\begin{gather}
\pdfrac{f}{\gamma}=-2G-\sqrt{\dfrac{2}{3}}\pdfrac{H_{n+1}}{q_{n+1}}\pdfrac{q_{n+1}}{\gamma}-\sqrt{\dfrac{2}{3}}\pdfrac{\sigma^y_{n+1}}{q_{n+1}}\pdfrac{q_{n+1}}{\gamma}.
\end{gather}
Since $q_{n+1}=q_n+\sqrt{\dfrac{2}{3}}\gamma$, $\pdfrac{q_{n+1}}{\gamma}=\sqrt{\dfrac{2}{3}}$. Hence,
\begin{gather}\label{eq:j2_df}
\pdfrac{f}{\gamma}=-2G-\dfrac{2}{3}\pdfrac{H_{n+1}}{q_{n+1}}-\dfrac{2}{3}\pdfrac{\sigma^y_{n+1}}{q_{n+1}}.
\end{gather}
\subsubsection{Consistent Tangent Stiffness}
From $\bsigma_{n+1}=\bsigma^\text{trial}-\gamma2G\bn$, as $\bn=\dfrac{\beeta}{\norm{\beeta}}=\dfrac{\beeta^\text{trial}}{\norm{\beeta^\text{trial}}}$, the consistent tangent stiffness can be computed via the chain rule as
\begin{gather}
\pdfrac{\bsigma_{n+1}}{\bvarepsilon_{n+1}}=\pdfrac{\bsigma_{n+1}^\text{trial}}{\bvarepsilon_{n+1}}-2G\pdfrac{\left(\gamma\bn\right)}{\bvarepsilon_{n+1}}=\mb{D}-2G\left(\bn\otimes\pdfrac{\gamma}{\bvarepsilon_{n+1}}+\gamma\pdfrac{\bn}{\bvarepsilon_{n+1}}\right).
\end{gather}
In which, according to \eqsref{eq:unit_derivative},
\begin{gather}
\begin{split}
\pdfrac{\bn}{\bvarepsilon_{n+1}}&=\dfrac{1}{\norm{\beeta^\text{trial}}}\left(\mathbb{I}-\bn\otimes\bn\right):\pdfrac{\mb{\beeta}^\text{trial}}{\bvarepsilon_{n+1}}\\
&=\dfrac{2G}{\norm{\beeta^\text{trial}}}\left(\mathbb{I}-\bn\otimes\bn\right):\mathbb{I}^\text{dev}\\
&=\dfrac{2G}{\norm{\beeta^\text{trial}}}\left(\mathbb{I}^\text{dev}-\bn\otimes\bn\right).
\end{split}
\end{gather}
Here, $\bn$ is expressed as $\bn=\dfrac{\beeta^\text{trial}}{\norm{\beeta^\text{trial}}}$ due to coaxiality between $\beeta_{n+1}$ and $\beeta^\text{trial}$.
This simplifies the computation.

From converged local residual (yield function),
\begin{gather}
\begin{split}
\pdfrac{\gamma}{\bvarepsilon_{n+1}}&=-\left(\pdfrac{f}{\gamma}\right)^{-1}\pdfrac{f}{\bvarepsilon_{n+1}}=-\left(\pdfrac{f}{\gamma}\right)^{-1}2G\bn.
\end{split}
\end{gather}

Thus the final expression for consistent tangent stiffness can be assembled as
\begin{gather}\label{eq:j2_stiffness}
\begin{split}
\pdfrac{\bsigma_{n+1}}{\bvarepsilon_{n+1}}&=\mb{D}-2G\left(-2G\left(\pdfrac{f}{\gamma}\right)^{-1}\bn\otimes\bn+\gamma\dfrac{2G}{\norm{\beeta^\text{trial}}}\left(\mathbb{I}^\text{dev}-\bn\otimes\bn\right)\right)\\
&=\mb{D}+4G^2\left(\pdfrac{f}{\gamma}\right)^{-1}\bn\otimes\bn+\dfrac{4G^2\gamma}{\norm{\beeta^\text{trial}}}\left(\bn\otimes\bn-\mathbb{I}^\text{dev}\right)\\
&=\mb{D}+4G^2\left(\left(\pdfrac{f}{\gamma}\right)^{-1}+\dfrac{\gamma}{\norm{\beeta^\text{trial}}}\right)\bn\otimes\bn-\dfrac{4G^2\gamma}{\norm{\beeta^\text{trial}}}\mathbb{I}^\text{dev}\\
&=\mb{D}+4G^2\left(\dfrac{\gamma}{\norm{\beeta^\text{trial}}}-\dfrac{1}{2G+\dfrac{2}{3}\pdfrac{H_{n+1}}{q_{n+1}}+\dfrac{2}{3}\pdfrac{\sigma^y_{n+1}}{q_{n+1}}}\right)\bn\otimes\bn-\dfrac{4G^2\gamma}{\norm{\beeta^\text{trial}}}\mathbb{I}^\text{dev}.
\end{split}
\end{gather}
It shall be noted that $\mathbb{I}^\text{dev}$ takes the form as presented in \eqsref{eq:dev_tensor_se}. Readers are strongly suggested to derive it via both tensor notation and compression matrix representation as a practice. Both leads to identical results.

Since the local iteration is a scalar function, the closed--form of consistent tangent stiffness is relatively easy to compute. It will be seen in more complex models that closed--forms do not always possess simple forms.

As a general framework, the above formulation does not require explicit forms of $H\left(q\right)$ and $\sigma^y\left(q\right)$. Thus, various types of scalar--valued functions can be adopted.
\subsection{Implementation}
The state determination algorithm of this general model incorporating von Mises criterion is given in \algoref{algo:j2_model}.
\begin{breakablealgorithm}
\caption{state determination of general von Mises model}\label{algo:j2_model}
\begin{algorithmic}
\State \textbf{Parameter}: $\lambda$, $G$
\State \textbf{Input}: $\bvarepsilon_{n+1}$, $\bvarepsilon_n$, $\bvarepsilon^p_n$, $\bsigma_n$, $\balpha_n$, $q_n$
\State \textbf{Output}: $\mb{D}_{n+1}$, $\bvarepsilon^p_{n+1}$, $\bsigma_{n+1}$, $\balpha_{n+1}$, $q_{n+1}$
\State compute $\bsigma^\text{trial}$, $\beeta^\text{trial}$, $\bn$ and $f^\text{trial}$\Comment{\eqsref{eq:j2_tsigma} and \eqsref{eq:j2_tf}}
\If {$f^\text{trial}\geqslant0$}
\State $\gamma=0$
\While{true}
\State compute $f$ and $\pdfrac{f}{\gamma}$\Comment{\eqsref{eq:j2_f} and \eqsref{eq:j2_df}}
\State update $\Delta{}H=H\left(q_{n+1}\right)-H\left(q_n\right)$
\State $\Delta\gamma=\left(\pdfrac{f}{\gamma}\right)^{-1}f$
\If {$\abs{\Delta\gamma}<\text{tolerance}$}
\State break
\EndIf
\State $\gamma\leftarrow\gamma-\Delta\gamma$
\State $q_{n+1}=q_n+\sqrt{\dfrac{2}{3}}\gamma$
\EndWhile
\State $\bsigma_{n+1}=\bsigma^\text{trial}-\gamma2G\bn$
\State $\bvarepsilon^p_{n+1}=\bvarepsilon^p_n+\gamma\bn$
\State $\balpha_{n+1}=\balpha_n+\sqrt{\dfrac{2}{3}}\Delta{}H\bn$
\State $q_{n+1}=q_n+\sqrt{\dfrac{2}{3}}\gamma$
\State compute $\mb{D}_{n+1}$\Comment{\eqsref{eq:j2_stiffness}}
\Else
\State $\bsigma_{n+1}=\bsigma^\text{trial}$
\State $\bvarepsilon^p_{n+1}=\bvarepsilon^p_n$
\State $\balpha_{n+1}=\balpha_n$
\State $q_{n+1}=q_n$
\State $\mb{D}_{n+1}=\mb{D}$
\EndIf
\end{algorithmic}
\end{breakablealgorithm}
It shall be noted that the algorithm does not implement $H\left(q\right)$ and $\sigma^y\left(q\right)$. It is assumed those two functions are defined somewhere else and are able to provide derivatives.

Depending on the choice of updating stress as in \eqsref{eq:j2_tsigma}, the history of $\bvarepsilon^p$ is \textbf{not} compulsory in this model if the latter expression is used instead of the former.
\subsection{Closing Remarks}
As the first 3D material model introduced, the von Mises framework allows beginners to get familiar with multiaxial constitutive modelling with a relatively smooth learning curve. The formulation is expressed in tensor notation. Readers are strongly encouraged to derive the formulation from three governing equations independently in both tensor and compressed matrix notions separately. It is a good practice to get each tiny detail correct.
\begin{cppcode}
NonlinearJ2::update_trial_status
\end{cppcode}

Finally, in \eqsref{eq:j2_flow}, we have discussed that the same expression
\begin{gather}
\dot{\bvarepsilon^p}=\gamma\bn
\end{gather}
may have different explicit forms, depending on how those quantities are represented.

The $\dot{\bvarepsilon^p}$ is further used in
\begin{gather}\label{eq:j2_two_form}
\bs^\text{trial}-\bs_{n+1}=2G\dot{\bvarepsilon^p}.
\end{gather}
The above expression assumes the tensor notation.
If the matrix notation is used, it can be explicitly written as
\begin{gather}
\begin{bmatrix}
s^\text{trial}_{11}-s_{11,n+1}\\
s^\text{trial}_{22}-s_{22,n+1}\\
s^\text{trial}_{33}-s_{33,n+1}\\
s^\text{trial}_{12}-s_{12,n+1}\\
s^\text{trial}_{23}-s_{23,n+1}\\
s^\text{trial}_{31}-s_{31,n+1}
\end{bmatrix}=\gamma\begin{bmatrix}
2G&&&&&\\
&2G&&&&\\
&&2G&&&\\
&&&G&&\\
&&&&G&\\
&&&&&G
\end{bmatrix}\diag{\mb{c}}\bn.
\end{gather}
One immediately realises that
\begin{gather}
\diag{\begin{matrix}
2G\\2G\\2G\\G\\G\\G
\end{matrix}}\circ\diag{\mb{c}}\bn=\diag{\begin{matrix}
2G\\2G\\2G\\2G\\2G\\2G
\end{matrix}}\bn=2G\bn,
\end{gather}
thus, \eqsref{eq:j2_two_form} becomes
\begin{gather}
\bs^\text{trial}-\bs_{n+1}=\gamma2G\bn.
\end{gather}
No matter which notation is used, the same expression holds.
This is a beautiful attribute, though could be confusing for beginners.
\section{Subloading Surface Model}
In this section, we formulate the 3D version of the subloading surface model.
\subsection{Theory}
\subsubsection{Subloading Surface}
The subloading surface can be expressed as
\begin{gather}
f_s=\norm{\beeta}-\sqrt{\dfrac{2}{3}}z\sigma^y,
\end{gather}
noting now $\beeta=\dev{\bsigma}-a^y\balpha+\left(z-1\right)\sigma^y\bd$.
\subsubsection{Flow Rule}
Again, assuming associative flow, the flow rule is
\begin{gather}\label{eq:subloading_flow}
\dot{\bvarepsilon^p}=\gamma\pdfrac{f_s}{\bsigma}=\gamma\dfrac{\beeta}{\norm{\beeta}}=\gamma\bn.
\end{gather}
\subsubsection{Evolution Laws}
The evolution laws depend on the internal variable $q$ as before, which takes the accumulated magnitude of $\bvarepsilon^p$.
\begin{gather}
\dot{q}=\sqrt{\dfrac{2}{3}}\norm{\dot{\bvarepsilon^p}}=\sqrt{\dfrac{2}{3}}\gamma.
\end{gather}
Similar to what have been introduced in \secref{sec:uniaxial_subloading}, the evolution laws are
\begin{gather}
\sigma^y=\sigma^i+k_\text{iso}q+\sigma^s_\text{iso}\left(1-e^{-m^s_\text{iso}q}\right),\\
a^y=a^i+k_\text{kin}q+a^s_\text{kin}\left(1-e^{-m^s_\text{kin}q}\right),\\
\dot{\balpha}=b\left(\sqrt{\dfrac{2}{3}}\bn-\balpha\right)\gamma,\\
\dot{\bd}=c_e\left(\sqrt{\dfrac{2}{3}}z_e\bn-\bd\right)\gamma.
\end{gather}
\subsection{Formulation}
We shall again express $\balpha$ and $\bd$ in their explicit forms, taking advantage of implicit Euler method.
\begin{gather}
    \balpha=\dfrac{\gamma{}\sqrt{\dfrac{2}{3}}b\bn+\balpha_n}{1+\gamma{}b},\qquad
    \bd=\dfrac{\gamma{}\sqrt{\dfrac{2}{3}}c_ez_e\bn+\bd_n}{1+\gamma{}c_e}.
\end{gather}

The shifted stress is then
\begin{gather}
    \beeta=\bs^\text{trial}-\gamma2G\bn-a^y\dfrac{\gamma{}\sqrt{\dfrac{2}{3}}b\bn+\balpha_n}{1+\gamma{}b}+\left(z-1\right)\sigma^y\dfrac{\gamma{}\sqrt{\dfrac{2}{3}}c_ez_e\bn+\bd_n}{1+\gamma{}c_e}.
\end{gather}
Rearranging gives
\begin{multline}
    \beeta+\gamma2G\bn+\sqrt{\dfrac{2}{3}}b\dfrac{\gamma{}a^y\bn}{1+\gamma{}b}+\sqrt{\dfrac{2}{3}}c_ez_e\left(1-z\right)\dfrac{\gamma{}\sigma^y\bn}{1+\gamma{}c_e}=\\\bs^\text{trial}-\dfrac{a^y}{1+\gamma{}b}\balpha_n+\left(z-1\right)\dfrac{\sigma^y}{1+\gamma{}c_e}\bd_n.
\end{multline}
Let's denote
\begin{gather}
    \bzeta=\bs^\text{trial}-\dfrac{a^y}{1+\gamma{}b}\balpha_n+\left(z-1\right)\dfrac{\sigma^y}{1+\gamma{}c_e}\bd_n.
\end{gather}
Since $\bn=\dfrac{\beeta}{\norm{\beeta}}$, one could conclude that
\begin{gather}
    \norm{\beeta}+\gamma2G+\sqrt{\dfrac{2}{3}}b\dfrac{\gamma{}a^y}{1+\gamma{}b}+\sqrt{\dfrac{2}{3}}c_ez_e\left(1-z\right)\dfrac{\gamma{}\sigma^y}{1+\gamma{}c_e}=\norm{\bzeta},
\end{gather}
and
\begin{gather}
    \bn=\dfrac{\beeta}{\norm{\beeta}}=\dfrac{\bzeta}{\norm{\bzeta}}.
\end{gather}
Although it is not possible to simplify more, one shall realise that $\bzeta$ is a function of $\bvarepsilon$, $\gamma$ and $z$ only.
Then we proceed to compute its derivatives.
\begin{gather}
    \pdfrac{\bzeta}{\gamma}=-\dfrac{\ddfrac{a^y}{\gamma}\left(1+\gamma{}b\right)-ba^y}{\left(1+\gamma{}b\right)^2}\balpha_n+\left(z-1\right)\dfrac{\ddfrac{\sigma^y}{\gamma}\left(1+\gamma{}c_e\right)-c_e\sigma^y}{\left(1+\gamma{}c_e\right)^2}\bd_n,\\
    \pdfrac{\bzeta}{z}=\dfrac{\sigma^y}{1+\gamma{}c_e}\bd_n.
\end{gather}

Similar to the uniaxial version, the local system consists of two residuals, $f_s$ and $z$.
\begin{gather}
    \mb{R}=\left\{
    \begin{array}{l}
        \norm{\beeta}-\sqrt{\dfrac{2}{3}}z\sigma^y, \\[2mm]
        z-z_n-\sqrt{\dfrac{2}{3}}U\gamma.
    \end{array}
    \right.
\end{gather}
Taking local variables $\mb{x}=\begin{bmatrix}
        \gamma & z
    \end{bmatrix}^\mT$, the Jacobian $\mb{J}$ shall be written as
\begin{gather}
    \mb{J}=\begin{bmatrix}
        \pdfrac{\norm{\beeta}}{\gamma}-\sqrt{\dfrac{2}{3}}z\ddfrac{\sigma^y}{\gamma} & \pdfrac{\norm{\beeta}}{z}-\sqrt{\dfrac{2}{3}}\sigma^y \\[4mm]
        -\sqrt{\dfrac{2}{3}}U                                                        & 1-\sqrt{\dfrac{2}{3}}\gamma{}\ddfrac{U}{z}
    \end{bmatrix}.
\end{gather}
In which,
\begin{gather}
\pdfrac{\norm{\beeta}}{\gamma}=\pdfrac{\norm{\bzeta}}{\gamma}-2G-\pdfrac{}{\gamma}\left(\sqrt{\dfrac{2}{3}}b\dfrac{\gamma{}a^y}{1+\gamma{}b}+\sqrt{\dfrac{2}{3}}c_ez_e\left(1-z\right)\dfrac{\gamma{}\sigma^y}{1+\gamma{}c_e}\right),\\
\pdfrac{\norm{\beeta}}{z}=\pdfrac{\norm{\bzeta}}{z}+\sqrt{\dfrac{2}{3}}c_ez_e\dfrac{\gamma{}\sigma^y}{1+\gamma{}c_e}.
\end{gather}
The derivatives of $\norm{\bzeta}$ can be further expanded as
\begin{gather}
\pdfrac{\norm{\bzeta}}{}=\bn:\pdfrac{\bzeta}{},
\end{gather}
so that
\begin{gather}
\pdfrac{\norm{\beeta}}{\gamma}=\bn:\pdfrac{\bzeta}{\gamma}-2G-\pdfrac{}{\gamma}\left(\sqrt{\dfrac{2}{3}}b\dfrac{\gamma{}a^y}{1+\gamma{}b}+\sqrt{\dfrac{2}{3}}c_ez_e\left(1-z\right)\dfrac{\gamma{}\sigma^y}{1+\gamma{}c_e}\right),\\
\pdfrac{\norm{\beeta}}{z}=\bn:\pdfrac{\bzeta}{z}+\sqrt{\dfrac{2}{3}}c_ez_e\dfrac{\gamma{}\sigma^y}{1+\gamma{}c_e}.
\end{gather}
\subsection{Implementation}
\begin{cppcode}
SubloadingMetal::update_trial_status
\end{cppcode}
\section{Hoffman J2 Model}\label{sec:hoffman}
Here we introduce an anisotropic model that adopts the Hoffman yielding criterion. This framework resembles the isotropic von Mises model. It can be used to model orthtropic materials such sheet steel and timber.
One can further refer to the corresponding section (\S~10.3) in \cite{SouzaNeto2008}.
\subsection{Theory}
\subsubsection{Yield Function}
The yield function adopts the Hoffman criterion, which is expressed in matrix form,
\begin{gather}
f=\dfrac{1}{2}\bsigma^\mT\mb{P}\bsigma+\mb{q}^\mT\bsigma-\sigma_y^2,
\end{gather}
where $\mb{P}=\mb{P}^\mT$ and $\bq$ are constant scaling matrix/vector of various forms \cite{Oller2003} that depend on material constants.
Apparently, $\mb{P}$ and $\bq$ represent two tensors of order four and two, respectively.
But since \cite{Oller2003} uses matrix notation, we shall follow this convention.

For example, the Hoffman criterion can be expressed as
\begin{tiny}
\begin{gather}\label{eq:hoffman_criterion}
\mb{P}=\begin{bmatrix}
T_1&-\dfrac{T_1+T_2-T_3}{2}&-\dfrac{T_3+T_1-T_2}{2}&&&\\[4mm]
-\dfrac{T_1+T_2-T_3}{2}&T_2&-\dfrac{T_2+T_3-T_1}{2}&&&\\[4mm]
-\dfrac{T_3+T_1-T_2}{2}&-\dfrac{T_2+T_3-T_1}{2}&T_3&&&\\[4mm]
&&&\dfrac{1}{f_{12}^2}&&\\[4mm]
&&&&\dfrac{1}{f_{23}^2}&\\[4mm]
&&&&&\dfrac{1}{f_{13}^2}
\end{bmatrix},\quad
\mb{q}=\begin{bmatrix}
\left(f_{11}^c-f_{11}^t\right)T_1\\[4mm]
\left(f_{22}^c-f_{22}^t\right)T_2\\[4mm]
\left(f_{33}^c-f_{33}^t\right)T_3\\[4mm]
0\\[4mm]
0\\[4mm]
0
\end{bmatrix},
\end{gather}
\end{tiny}
in which,
\begin{gather}
T_1=\dfrac{1}{f_{11}^tf_{11}^c},\qquad
T_2=\dfrac{1}{f_{22}^tf_{22}^c},\qquad
T_3=\dfrac{1}{f_{33}^tf_{33}^c},
\end{gather}
with $f_{ij}^\aleph$ representing the yielding stress along different directions.
\subsubsection{Flow Rule}
The associated plasticity is assumed such that the plastic potential $g$ is identical to $f$. The plastic flow direction is then
\begin{gather}
\bn=\pdfrac{g}{\bsigma}=\pdfrac{f}{\bsigma}=\mb{P}\bsigma+\mb{q}.
\end{gather}
The flow rule can be defined as
\begin{gather}
\dot{\bvarepsilon^p}=\gamma\bn.
\end{gather}
\subsubsection{Hardening Law}
The reference yield stress $\sigma_y$ is defined as a function of the accumulated equivalent plastic strain $\varepsilon_{p}$.
\begin{gather}
\sigma_y=H\left(\varepsilon^{p}\right).
\end{gather}
The evolution of $\varepsilon^{p}$ is driven by the norm of $\dot{\bvarepsilon^p}$.
\begin{gather}
\dot{\varepsilon^p}=\norm{\dot{\bvarepsilon^p}}=\gamma\norm{\bn},
\end{gather}
where $\norm{\bn}$, in matrix form, can be expressed as
\begin{gather}
\norm{\bn}=\sqrt{\dfrac{2}{3}\bn^\mT\mb{T}\bn},
\end{gather}
with $\mb{T}=\diag{\begin{matrix}
1&1&1&\dfrac{1}{2}&\dfrac{1}{2}&\dfrac{1}{2}
\end{matrix}}$.
Here, one shall treat $\norm{\cdot}$ as a `special' operator that is defined as above.
\subsection{Formulation}
To some extend, the model is even simpler than the von Mises model as there is no back stress to support kinematic hardening. Furthermore, the yield function involves only matrix--vector operations, the corresponding derivatives are relatively easy to compute.
\begin{table}[ht]
\centering
\begin{tabular}{rl}
\toprule
Constitutive Law&$\bsigma=\mb{D}:\left(\bvarepsilon-\bvarepsilon^p\right)$\\
Yield Function&$f=\dfrac{1}{2}\bsigma^\mT\mb{P}\bsigma+\mb{q}^\mT\bsigma-\sigma_y^2$\\
Flow Rule&$\dot{\bvarepsilon^p}=\gamma\bn$\\
Hardening Law&$\dot{\varepsilon^p}=\gamma\norm{\bn}$\\\bottomrule
\end{tabular}
\end{table}

We have used both tensor and matrix notations in the above table.
Can you identify which expressions use which notation?
This is important as different notations imply differences in implementation.
\subsubsection{Elastic Loading/Unloading}
Compared with the von Mises framework, there is no essential difference in terms of elastic loading/unloading, the plasticity is frozen at the beginning of each substep, allowing one to compute the trial stress such that
\begin{gather}
\bsigma^\text{trial}=\mb{D}:\left(\bvarepsilon_{n+1}-\bvarepsilon^p_n\right)\xcancel{=\bsigma_n+\mb{D}:\left(\bvarepsilon_{n+1}-\bvarepsilon_n\right)}.
\end{gather}
One shall note that the second form is not used here. In most cases, both forms are equivalent. However, some models may further apply damage mechanics to the result of plasticity, making the second form incorrect (as $\bsigma_n$ may contain damage reductions).

The trial yield function can then be computed using the unchanged plastic strain.
\begin{gather}
f^\text{trial}=\dfrac{1}{2}\bsigma^\text{trial,T}\mb{P}\bsigma^\text{trial}+\mb{q}^\mT\bsigma^\text{trial}-\sigma_{y,n}^2.
\end{gather}
\subsubsection{Plastic Evolution}
Since it is an anisotropic model, the local iteration may have difficulties in convergence, especially when a high anisotropy is defined. Some implementations \cite{Krasnovskiy2004,SouzaNeto2008} adopt line search, which does mitigate the local convergence issue but does not address it at the global level.

In this work, we choose another approach to alleviate the problem. Instead of the first order accurate backward Euler method for numerical integration, the higher order accurate method is used. In specific, the discretised evolution of plastic strain is written as
\begin{gather}
\bvarepsilon^p_{n+1}=\bvarepsilon^p_{n}+\Delta\bvarepsilon^p=\bvarepsilon^p_{n}+\gamma\bn_{m},
\end{gather}
where $\bn_{m}$ is $\bn$ evaluated at the middle of the substep. Since it is a linear function of $\sigma$,
\begin{gather}
\bn_m=\dfrac{\bn_n+\bn_{n+1}}{2}=\mb{P}\dfrac{\bsigma_n+\bsigma_{n+1}}{2}+\mb{q}.
\end{gather}
We effectively use the trapezoidal rule here.
It it second order accurate.
One shall note that replacing $\bn_m$ with $\bn_{n+1}=\mb{P}\bsigma_{n+1}+\mb{q}$ gives the evolution formula of the implicit Euler method.
\subsubsection{Local Residual}
The residual is chosen as follows. For brevity, all subscripts $\left(\cdot\right)_{n+1}$ are dropped.
\begin{gather}
\mb{R}=\left\{
\begin{array}{l}
\dfrac{1}{2}\bsigma^\mT\mb{P}\bsigma+\mb{q}^\mT\bsigma-\sigma_y^2,\\[4mm]
\bsigma+\gamma\mb{E}\bn_m-\bsigma^\text{trial}.
\end{array}
\right.
\end{gather}
We use $\mb{E}$ to denote $\mb{D}$ expressed in the matrix form.
They represent the same quantity.

By choosing $\mb{x}=\begin{bmatrix}
\gamma&\bsigma
\end{bmatrix}^\mT$ as the independent variables, the Jacobian can be then computed as
\begin{gather}
\mb{J}=\pdfrac{\mb{R}}{\mb{x}}=\begin{bmatrix}
-2\sigma_y\ddfrac{\sigma_y}{\varepsilon_{p}}\norm{\bn_m}&\bn^\mT-\sigma_y\ddfrac{\sigma_y}{\varepsilon_{p}}\gamma\ddfrac{\norm{\bn_m}}{\bn_m}\mb{P}\\[4mm]
\mb{E}\mb{n}_m&\mb{I}+\dfrac{\gamma}{2}\mb{E}\mb{P}
\end{bmatrix}.
\end{gather}
In the above expression,
\begin{gather}
\ddfrac{\norm{\bn_m}}{\bn_m}=\dfrac{2}{3}\dfrac{\bn_m^\mT\mb{T}}{\norm{\bn_m}}.
\end{gather}

Some references would further derive a scalar local residual at the cost of complicating gradient. Here we choose to increase the size of local system in order to express the Jacobian in a simpler form. Performance wise, a scalar local residual does not necessarily leads to faster state determination.
\subsubsection{Consistent Tangent Stiffness}
The consistent tangent stiffness can be directly computed from the local residual, given that $\bsigma_{n+1}$ is chosen as the variable. Differentiating $\mb{R}$ at equilibrium $\mb{R}=\mb{0}$ gives
\begin{gather}\label{eq:universal_consistent_tangent}
\pdfrac{\mb{R}}{\mb{x}}\pdfrac{\mb{x}}{\bvarepsilon_{n+1}}+\pdfrac{\mb{R}}{\bvarepsilon_{n+1}}=\mb{0},
\end{gather}
rearranging which gives
\begin{gather}
\pdfrac{\mb{x}}{\bvarepsilon_{n+1}}=-\left(\pdfrac{\mb{R}}{\mb{x}}\right)^{-1}\pdfrac{\mb{R}}{\bvarepsilon_{n+1}}=-\mb{J}^{-1}\pdfrac{\mb{R}}{\bvarepsilon_{n+1}}.
\end{gather}

The right hand side $\pdfrac{\mb{R}}{\bvarepsilon_{n+1}}$ can be computed as
\begin{gather}
\pdfrac{\mb{R}}{\bvarepsilon_{n+1}}=\begin{bmatrix}
0\\
-\mb{E}
\end{bmatrix},
\end{gather}
while the left hand side $\pdfrac{\mb{x}}{\bvarepsilon_{n+1}}$ contains
\begin{gather}
\pdfrac{\mb{x}}{\bvarepsilon_{n+1}}=\begin{bmatrix}
\pdfrac{\gamma}{\bvarepsilon_{n+1}}\\[4mm]
\pdfrac{\bsigma_{n+1}}{\bvarepsilon_{n+1}}
\end{bmatrix}.
\end{gather}
Thus,
\begin{gather}
\pdfrac{\bsigma_{n+1}}{\bvarepsilon_{n+1}}=\left(\mb{J}^{-1}\begin{bmatrix}
0\\
\mb{E}
\end{bmatrix}\right)^{\langle2-7\rangle},
\end{gather}
where $\left(\cdot\right)^{\langle2-7\rangle}$ denotes the second to the seventh row of target quantity $\left(\cdot\right)$.

Unlike the von Mises framework, in which the analytical expression for the consistent tangent stiffness matrix is derived. Here we take advantage of the fact that when the local equilibrium is achieved, $\mb{R}=\mb{0}$ or at least $\mb{R}\approx\mb{0}$, allowing one to take full differentiation to obtain some useful quantities that otherwise may be difficult to compute. If $\bsigma_{n+1}$ is directly involved as one of the independent variables in local iteration, the consistent stiffness can be directly obtained. Otherwise, often additional simple steps (chain rule) shall be applied to the stress update formula to compute $\pdfrac{\bsigma_{n+1}}{\bvarepsilon_{n+1}}$.

This method avoid the computation of lengthy, cumbersome analytical expressions of consistent tangent. In most cases, it is also very simple to implement as $\mb{J}$ is already available when the local iteration converges, and $\pdfrac{\mb{R}}{\bvarepsilon_{n+1}}$ often is very easy to compute. Readers shall try to grasp the beauty of \eqsref{eq:universal_consistent_tangent}, as this method will be frequently used in the models introduced later in this book.
\subsection{Implementation}
\begin{cppcode}
NonlinearHoffman::update_trial_status
\end{cppcode}