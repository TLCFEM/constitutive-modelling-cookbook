\chapter{Uniaxial Metal Models}
In this chapter, several models suitable for modelling metals are presented. We shall start from simple von Mises criterion based models, to more advanced uniaxial models suitable for modelling Bauschinger effect, ratcheting and buckling restrained bracing members.
\section{Linear Isotropic Hardening Model}
\subsection{Theory}
For uniaxial models, the constitutive equation \eqsref{eq:strain_decomposition} simplifies to
\begin{gather}
\sigma=E\left(\varepsilon-\varepsilon^p\right)
\end{gather}
where $E$ is Young's modulus.
\subsubsection{Yield Function}
The (probably) simplest yield function is
\begin{gather}\label{eq:isotropic_yield_function}
f=\abs{\sigma}-\sigma^y
\end{gather}
where $\sigma^y=\sigma^y\left(\sigma,~q\right)$ is the yield stress which is a function of some internal variable $q$ thus would evolve.
The above function simply states that, as long as the magnitude of $\sigma$ exceeds a certain level $\sigma^y$, plasticity would occur.

For example, a linear function can be chosen so that
\begin{gather}
\sigma^y=\sigma^i+Kq,
\end{gather}
where $\sigma^i$ is the initial yield stress (non-negative) and $K$ is the isotropic hardening modulus (either positive for hardening or negative for softening).
Typically, it is further constrained that $\sigma^y\geqslant0~\forall{}q$.
\subsubsection{Flow Rule}
It is reasonable to define that the direction of plastic flow coincides with the direction of stress, that is
\begin{gather}\label{eq:isotropic_flow_rule}
\dot{\varepsilon^p}=\gamma~\sign{\sigma}.
\end{gather}
Physically, it simply states that plastic strain evolves towards the direction of stress. If stress is positive, plastic strain increases and vice versa.

Noting that according to \eqsref{eq:isotropic_yield_function}, $\pdfrac{f}{\sigma}=\sign{\sigma}$, the flow rule can also be expressed as
\begin{gather}
\dot{\varepsilon^p}=\gamma\pdfrac{f}{\sigma}.
\end{gather}
In 3D space, such a flow rule is often called the \textbf{associative} rule, as the flow of plastic strain is now associated with the gradient of the yield function defined in stress space.
It shall be emphasised that, the direction of plastic flow does not necessarily be associative.
In many models, plastic flow follows the gradient of another surface, which can be a slight variant of the yield surface $f$, or even totally different.
\subsubsection{Hardening Law}
For the internal hardening variable $q$, the simplest case would be
\begin{gather}\label{eq:isotropic_hardening_law}
\dot{q}=\abs{\dot{\varepsilon^p}}=\gamma.
\end{gather}
Thus $q$ characterises the accumulated magnitude of plastic strain. Given that $\gamma\geqslant0$, it is clear that $q$ is a non-decreasing (strictly increasing) function.

We are ready to formulate and implement our first plastic model.
\subsection{Formulation}
In this book, subscript $\left(\cdot\right)_n$ is adopted to indicate initial conditions (current state or converged state) and subscript $\left(\cdot\right)_{n+1}$ is adopted to indicate solution (new state or trial state). Sometimes $\left(\cdot\right)_{n+1}$ is omitted for simplicity.

For this example, the initial conditions are stress $\sigma_n$, total strain $\varepsilon_n$, plastic strain $\varepsilon_n^p$ and hardening variable $q_n$. For a given $\varepsilon_{n+1}$, the model shall compute new $\sigma_{n+1}$, $\varepsilon^p_{n+1}$ and $q_{n+1}$.

The summation of this simple isotropic hardening model is listed as follows.
\begin{table}[ht]
\centering
\begin{tabular}{rl}
\toprule
Constitutive Law&$\sigma=E\left(\varepsilon-\varepsilon^p\right)$\\
Yield Function&$f=\abs{\sigma}-\left(\sigma^i+Kq\right)$\\
Flow Rule&$\dot{\varepsilon^p}=\gamma~\sign{\sigma}$\\
Hardening Law&$\dot{q}=\gamma$\\\bottomrule
\end{tabular}
\end{table}
\subsubsection{Elastic Loading/Unloading}
By freezing plasticity, one can first check if the new state is elastic.
Here, `freezing plasticity' means we assume the plasticity related variables do not evolve, thus the increment of strain $\Delta\varepsilon=\varepsilon_{n+1}-\varepsilon_n$ is purely elastic strain.
Since it may not be the final new state, we denote the computed stress as the trial stress.
According to
\begin{gather}
\sigma_{n+1}^\text{trial}=E\left(\varepsilon_{n+1}-\varepsilon^p_n\right),
\end{gather}
or equivalently,
\begin{gather}\label{eq:uniaxial_trial_stress}
\sigma_{n+1}^\text{trial}=\sigma_n+E\left(\varepsilon_{n+1}-\varepsilon_n\right)=\sigma_n+E\Delta\varepsilon,
\end{gather}
the yield function becomes
\begin{gather}
f^\text{trial}=\abs{\sigma^\text{trial}}-\sigma^y=E\abs{\varepsilon_{n+1}-\varepsilon^p_n}-\left(\sigma^i+Kq_n\right),\\
f^\text{trial}=\abs{\sigma^\text{trial}}-\sigma^y=\abs{\sigma_n+E\left(\varepsilon_{n+1}-\varepsilon_n\right)}-\left(\sigma^i+Kq_n\right).
\end{gather}

Here all we are doing is assuming the current state is elastic (regardless of the actual state), and applying the increment $\Delta\varepsilon$ to obtain the trial state.
If $f^\text{trial}<0$, meaning the trial state is still in the elastic region, indicating elastic loading/unloading, the new state is simply $\varepsilon^p_{n+1}=\varepsilon^p_n$, $q_{n+1}=q_n$ and $\sigma_{n+1}=\sigma_{n+1}^\text{trial}$.
\subsubsection{Plastic Evolution}
Otherwise the new state consists of new evolution of plasticity. In this case,
\begin{gather}
\sigma_{n+1}=E\left(\varepsilon_{n+1}-\varepsilon^p_{n+1}\right).
\end{gather}
The yield function is
\begin{gather}
f_{n+1}=\abs{\sigma_{n+1}}-\sigma^y_{n+1}=E\abs{\varepsilon_{n+1}-\varepsilon^p_{n+1}}-\left(\sigma^i+Kq_{n+1}\right)=0.
\end{gather}
According to \eqsref{eq:consistency_condition}, since now we know plasticity evolves, $f_{n+1}=0$.

The flow rule and hardening law shall be expressed as
\begin{gather}
\varepsilon^p_{n+1}=\varepsilon^p_n+\dot{\varepsilon^p}=\varepsilon^p_n+\gamma~\sign{\sigma_{n+1}},\\
q_{n+1}=q_n+\dot{q}=q_n+\gamma.
\end{gather}
Noting that
\begin{gather}\label{eq:isotropic_alt}
\begin{split}
\sigma_{n+1}=\sigma_{n+1}^\text{trial}-E\dot{\varepsilon^p}\quad&\longrightarrow\quad
\sigma_{n+1}+E\gamma~\sign{\sigma_{n+1}}=\sigma_{n+1}^\text{trial}\\&\longrightarrow\quad
\left(\abs{\sigma_{n+1}}+E\gamma\right)~\sign{\sigma_{n+1}}=\abs{\sigma_{n+1}^\text{trial}}~\sign{\sigma_{n+1}^\text{trial}},
\end{split}
\end{gather}
since both $E$ and $\gamma$ are non-negative, $\sign{\sigma_{n+1}}=\sign{\sigma_{n+1}^\text{trial}}$ and $\abs{\sigma_{n+1}}+E\gamma=\abs{\sigma_{n+1}^\text{trial}}$.
Thus,
\begin{gather}
\abs{\sigma_{n+1}}=\abs{\sigma_{n+1}^\text{trial}}-E\gamma,
\end{gather}
inserting the above expression into the yield function, one can obtain
\begin{gather}
\abs{\sigma_{n+1}^\text{trial}}-E\gamma-\left(\sigma^i+K\left(q_n+\gamma\right)\right)=0.
\end{gather}

The consistency parameter can be solved as
\begin{gather}
\gamma=\dfrac{\abs{\sigma_{n+1}^\text{trial}}-\left(\sigma^i+Kq_n\right)}{E+K}=\dfrac{f^\text{trial}}{E+K}.
\end{gather}
Once $\gamma$ is obtained, $\varepsilon^p_{n+1}$ and $q_{n+1}$ can be updated accordingly.

It is easy to further compute
\begin{gather}
\pdfrac{\gamma}{\varepsilon_{n+1}}=\dfrac{1}{E+K}\pdfrac{f^\text{trial}}{\varepsilon_{n+1}}=\dfrac{E}{E+K}.
\end{gather}
\subsection{Implementation}
The implementation is quite straightforward for such a simple model. \algoref{algo:isotropic} summarised the state determination algorithm for the above isotropic hardening model.
\begin{breakablealgorithm}
\caption{state determination of uniaxial isotropic hardening model}\label{algo:isotropic}
\begin{algorithmic}[1]
\State \textbf{Parameter}: $E$, $K$
\State \textbf{Input}: $\varepsilon_{n+1}$, $\varepsilon_n$, $\varepsilon^p_n$, $\sigma_n$, $q_n$
\State \textbf{Output}: $E_{n+1}$, $\varepsilon^p_{n+1}$, $\sigma_{n+1}$, $q_{n+1}$
\State compute $\sigma^\text{trial}$ and $f^\text{trial}$
\If {$f^\text{trial}\geqslant0$}
	\State $\gamma=\dfrac{f^\text{trial}}{E+K}$
	\State $\varepsilon^p_{n+1}=\varepsilon^p_n+\gamma~\sign{\sigma^\text{trial}_{n+1}}$
	\State $q_{n+1}=q_n+\gamma$
	\State $\sigma_{n+1}=\sigma_{n+1}^\text{trial}-E\gamma~\sign{\sigma_{n+1}^\text{trial}}$
	\State $E_{n+1}=E-\dfrac{E^2}{E+K}=\dfrac{EK}{E+K}$
\Else
	\State $\varepsilon^p_{n+1}=\varepsilon^p_n$
	\State $q_{n+1}=q_n$
	\State $\sigma_{n+1}=\sigma_{n+1}^\text{trial}$
	\State $E_{n+1}=E$
\EndIf
\end{algorithmic}
\end{breakablealgorithm}

It is clear that no local iteration is required, since simple linear functions are used as an introductory example.
Typical plasticity models often adopt nonlinear functions, local iterations are often required to compute the new state.
\section{Combined Isotropic/Kinematic Hardening Model}\label{sec:isotropic/kinematic}
Isotropic hardening controls the size of yield surface without changing its location. Kinematic hardening, on contrary, changes yield surface location but does not touch its size. Combining isotropic hardening and kinematic hardening allows flexible response to be modelled.
\subsection{Theory}
\subsubsection{Yield Function}
Since $\sigma^y$ characterises the size of yield surface, a natural approach to allow yield surface to move around is to introduce the explicit location of its centre.
\begin{gather}
f=\abs{\eta}-\sigma^y,
\end{gather}
with $\eta=\sigma-\alpha$ is defined to be the shifted stress with $\alpha$ denoting the back stress. The additional stress quantity $\alpha$ characterises the centre of yield surface.
\subsubsection{Flow Rule}
Assuming associative rule, the flow rule shall be updated as
\begin{gather}
\dot{\varepsilon^p}=\gamma\pdfrac{f}{\sigma}=\gamma~\sign{\eta}.
\end{gather}
\subsubsection{Hardening Law}
The additional internal variable $\alpha$ shall evolve as well. It can take a similar form as follows.
\begin{gather}
\dot{\alpha}=H\dot{\varepsilon^p}=\gamma{}H~\sign{\eta}.
\end{gather}
in which $H$ denotes the kinematic hardening modulus. The existing hardening law for $q$ does not alter.
\begin{gather}
\dot{q}=\abs{\dot{\varepsilon^p}}=\gamma.
\end{gather}
\subsection{Formulation}
The summation of this combined isotropic/kinematic hardening model is listed as follows.
\begin{table}[ht]
\centering
\begin{tabular}{rl}
\toprule
Constitutive Law&$\sigma=E\left(\varepsilon-\varepsilon^p\right)$\\
Yield Function&$f=\abs{\sigma-\alpha}-\left(\sigma^i+Kq\right)$\\
Flow Rule&$\dot{\varepsilon^p}=\gamma~\sign{\sigma-\alpha}$\\
Hardening Law&$\dot{q}=\gamma$\\
&$\dot{\alpha}=\gamma{}H~\sign{\sigma-\alpha}$\\\bottomrule
\end{tabular}
\end{table}

The algorithm aims to compute new $\sigma_{n+1}$, $\varepsilon^p_{n+1}$, $\alpha_{n+1}$ and $q_{n+1}$ based on current $\sigma_n$, $\varepsilon_n$, $\varepsilon_n^p$, $\alpha_n$, $q_n$ and new $\varepsilon_{n+1}$.
\subsubsection{Elastic Loading/Unloading}
The trial stress can be computed following \eqsref{eq:uniaxial_trial_stress}. Then by denoting $\eta^\text{trial}=\sigma^\text{trial}-\alpha_n$, the yield function becomes
\begin{gather}
f^\text{trial}=\abs{\eta^\text{trial}}-\left(\sigma^i+Kq_n\right).
\end{gather}
Again, if $f^\text{trial}\leqslant0$, it is elastic loading/unloading; if $f^\text{trial}>0$, the new state requires plasticity evolution.
\subsubsection{Plastic Evolution}
With new state variables, let $\eta_{n+1}=\sigma_{n+1}-\alpha_{n+1}$, compute the yield function as follows.
\begin{gather}
\begin{split}
f&=\abs{\sigma_{n+1}-\alpha_{n+1}}-\left(\sigma^i+Kq_{n+1}\right)\\
&=\abs{\sigma_{n+1}^\text{trial}-E\gamma~\sign{\eta_{n+1}}-\alpha_n-H\gamma~\sign{\eta_{n+1}}}-\left(\sigma^i+Kq_n+K\gamma\right)\\
&=\abs{\eta_{n+1}^\text{trial}-\left(E+H\right)\gamma~\sign{\eta_{n+1}}}-\left(\sigma^i+Kq_n+K\gamma\right)=0.
\end{split}
\end{gather}
Similar to \eqsref{eq:isotropic_alt}, it can be derived that
\begin{gather}
\sign{\eta_{n+1}^\text{trial}}=\sign{\eta_{n+1}},\qquad
\abs{\eta_{n+1}^\text{trial}}=\abs{\eta_{n+1}}+\left(E+H\right)\gamma.
\end{gather}
Thus,
\begin{gather}\label{eq:linear_combined}
f=\abs{\eta_{n+1}^\text{trial}}-\left(E+H\right)\gamma-\left(\sigma^i+Kq_n+K\gamma\right)=0.
\end{gather}
From which $\eta$ can be solved.
\begin{gather}
\eta=\dfrac{\abs{\eta_{n+1}^\text{trial}}-\left(\sigma^i+Kq_n\right)}{E+H+K}=\dfrac{f^\text{trial}}{E+H+K}.
\end{gather}
\subsection{Implementation}
The state determination algorithm resembles the previous one for isotropic hardening model.
\begin{breakablealgorithm}
\caption{state determination of uniaxial combined isotropic/kinematic hardening model}\label{algo:isotropic/kinematic}
\begin{algorithmic}[1]
\State \textbf{Parameter}: $E$, $H$, $K$
\State \textbf{Input}: $\varepsilon_{n+1}$, $\varepsilon_n$, $\varepsilon^p_n$, $\sigma_n$, $\alpha_n$, $q_n$
\State \textbf{Output}: $E_{n+1}$, $\varepsilon^p_{n+1}$, $\sigma_{n+1}$, $\alpha_{n+1}$, $q_{n+1}$
\State compute $\sigma^\text{trial}$, $\eta^\text{trial}$ and $f^\text{trial}$
\If {$f^\text{trial}\geqslant0$}
\State $\gamma=\dfrac{f^\text{trial}}{E+H+K}$
\State $\varepsilon^p_{n+1}=\varepsilon^p_n+\gamma~\sign{\eta_{n+1}^\text{trial}}$
\State $q_{n+1}=q_n+\gamma$
\State $\alpha_{n+1}=\alpha_n+H\gamma~\sign{\eta_{n+1}^\text{trial}}$
\State $\sigma_{n+1}=\sigma_{n+1}^\text{trial}-E\gamma~\sign{\eta_{n+1}^\text{trial}}$
\State $E_{n+1}=E-\dfrac{E^2}{E+H+K}=\dfrac{E\left(H+K\right)}{E+H+K}$
\Else
\State $\varepsilon^p_{n+1}=\varepsilon^p_n$
\State $q_{n+1}=q_n$
\State $\alpha_{n+1}=\alpha_n$
\State $\sigma_{n+1}=\sigma_{n+1}^\text{trial}$
\State $E_{n+1}=E$
\EndIf
\end{algorithmic}
\end{breakablealgorithm}
\section{Armstrong--Fredrick Hardening Model}
So far, two simple models have been introduced. The linear isotropic/kinematic hardening law is adopted so that the local residual is a linear function which can be solved within one step. However, linear hardening has limited applications.

To allow more versatile applications, in this section, a metal model incorporating Armstrong--Fredrick type kinematic hardening \cite{Frederick2007} and Voce type isotropic hardening \cite{Voce1955} is introduced. Both hardening types are nonlinear.
\subsection{Theory}
\subsubsection{Yield Function and Flow Rule}
The same yield function and flow rule used in the previous combined isotropic/kinematic hardening model are adopted.
\begin{gather}
f=\abs{\eta}-\sigma^y,\\
\dot{\varepsilon^p}=\gamma\pdfrac{f}{\sigma}=\gamma{}n=\gamma~\sign{\eta},
\end{gather}
in which $\eta=\sigma-\alpha$ is the shifted stress, $\alpha=\alpha\left(q\right)$ is the back stress, $\sigma^y=\sigma^y\left(q\right)$ is the isotropic hardening stress and $n=\sign{\eta}$.
\subsubsection{Hardening Law}
The same hardening law for $q$ is adopted as well.
\begin{gather}
\dot{q}=\abs{\dot{\varepsilon^p}}=\gamma.
\end{gather}
\paragraph{Isotropic Hardening}
Instead of using a simple linear function, a Voce type \cite{Voce1955} function is adopted for $\sigma^y$.
\begin{gather}
\sigma^y=\sigma^i+Kq+\sigma^s\left(1-e^{-m^sq}\right),
\end{gather}
in which $K$ denotes the linear hardening modulus, $\sigma^s$ is the saturated stress that denotes the size of additional yield stress caused by exponential hardening and $m^s$ is a model parameter that controls the rate of exponential part of hardening.

If $K=0$, it can be seen that
\begin{gather}
\lim\limits_{q\rightarrow\infty}\sigma^y=\sigma^i+\sigma^s,\qquad
\lim\limits_{q\rightarrow0}\sigma^y=\sigma^i.
\end{gather}
Also,
\begin{gather}
\dot{\sigma^y}=m^s\sigma^se^{-m^sq}=m^s\left(\sigma^i+\sigma^s-\sigma^y\right).
\end{gather}
This rate form is often presented in other literature. One must be aware of the fact that it can be explicitly integrated.
Furthermore, one could propose the following
\begin{gather}
\sigma^y=\sigma^i+Kq+\sigma^s\dfrac{\sum_{j=1}^{n}s_j\left(1-e^{-m^s_jq}\right)}{\sum_{j=1}^{n}s_j}.
\end{gather}
The fraction $\dfrac{\sum_{j=1}^{n}s_j\left(1-e^{-m^s_jq}\right)}{\sum_{j=1}^{n}s_j}$ defines a nonlinear curve that originates at point $\left(0,1\right)$ and asymptotically approaches zero when $q$ approaches infinity.
The parameter pairs $m^s_j$ and $s_j$ can be tweaked to approximate curves of \textbf{any} shape, theoretically.
An equivalent form would be
\begin{gather}
\sigma^y=\sigma^i+Kq+\sum_{j=1}^{n}\sigma^s_j\left(1-e^{-m^s_jq}\right).
\end{gather}
\paragraph{Kinematic Hardening}
A multiplicative formulation \cite{Chaboche1989} for back stress is adopted with the Armstrong--Fredrick type hardening rule. The back stress is defined to be the summation of several back stresses that evolve independently with different rates. That is,
\begin{gather}
\alpha=\sum_{i=1}^n\alpha_i
\end{gather}
with
\begin{gather}\label{eq:af_kinematic}
\dot{\alpha_i}=a_i\dot{\varepsilon^p}-b_i\alpha_i\dot{q}
\end{gather}
where $a_i$ and $b_i$ are two sets of model parameters. In terms of $\gamma$, it is
\begin{gather}
\dot{\alpha_i}=a_i\gamma{}n-b_i\alpha_i\gamma.
\end{gather}
\subsection{Formulation}
The summation of this AF model is listed as follows.
\begin{table}[ht]
\centering
\begin{tabular}{rl}
\toprule
Constitutive Law&$\sigma=E\left(\varepsilon-\varepsilon^p\right)$\\
Yield Function&$f=\abs{\sigma-\alpha}-\left(\sigma^i+Kq+\sigma^s\left(1-e^{-m^sq}\right)\right)$\\
Flow Rule&$\dot{\varepsilon^p}=\gamma{}n=\gamma~\sign{\sigma-\alpha}$\\
Hardening Law&$\dot{q}=\gamma$\\
&$\alpha=\sum_{i=1}^n\alpha_i$\\
&$\dot{\alpha_i}=a_i\gamma{}n-b_i\alpha_i\gamma$\\\bottomrule
\end{tabular}
\end{table}
\subsubsection{Elastic Loading/Unloading}
The trial stress can be computed following \eqsref{eq:uniaxial_trial_stress}. Then by denoting $\eta^\text{trial}=\sigma^\text{trial}-\alpha_n$, the yield function becomes
\begin{gather}
f^\text{trial}=\abs{\eta^\text{trial}}-\left(\sigma^i+Kq_n+\sigma^s\left(1-e^{-m^sq_n}\right)\right).
\end{gather}
\subsubsection{Plastic Evolution}
It is now clear that nonlinearity is introduced since $\dot{\alpha_i}=g\left(\alpha_i,\cdots\right)$ is a function of $\alpha_i$ and other variables. Various first order numerical methods can be applied. For example,
\begin{itemize}
\item explicit/forward Euler method
\begin{gather}
\alpha_{i,n+1}=\alpha_{i,n}+g\left(\alpha_{i,n},\cdots\right),
\end{gather}
\item implicit/backward Euler method
\begin{gather}
\alpha_{i,n+1}=\alpha_{i,n}+g\left(\alpha_{i,n+1},\cdots\right),
\end{gather}
\item mid-point method
\begin{gather}
\alpha_{i,n+1}=\alpha_{i,n}+\dfrac{g\left(\alpha_{i,n},\cdots\right)+g\left(\alpha_{i,n+1},\cdots\right)}{2}.
\end{gather}
\end{itemize}

Theoretically, other higher order numerical methods for ODEs can also be applied.
However, higher order methods often involves complex expressions, making the computation of the corresponding derivative difficult.
We mainly use the implicit Euler method in our plastic models, although it is significantly more complex than the explicit one.

Then $\alpha_{i,n+1}$ can be expressed as
\begin{gather}
\alpha_{i,n+1}=\alpha_{i,n}+\dot{\alpha_i}=\alpha_{i,n}+a_i\gamma{}n-b_i\alpha_{i,n+1}\gamma.
\end{gather}
Thus,
\begin{gather}
\alpha_{i,n+1}=\dfrac{\alpha_{i,n}+a_i\gamma{}n}{1+b_i\gamma}.
\end{gather}
Then,
\begin{gather}
\alpha_{n+1}=\sum\dfrac{\alpha_{i,n}+a_i\gamma{}n}{1+b_i\gamma}.
\end{gather}
The shifted stress can be then computed as
\begin{gather}
\eta_{n+1}=\sigma_{n+1}-\alpha_{n+1}=\sigma^\text{trial}-E\gamma{}n-\sum\dfrac{\alpha_{i,n}+a_i\gamma{}n}{1+b_i\gamma}.
\end{gather}
Rearranging leads to
\begin{gather}
\eta_{n+1}+E\gamma{}n+\sum\dfrac{a_i\gamma{}n}{1+b_i\gamma}=\sigma^\text{trial}-\sum\dfrac{\alpha_{i,n}}{1+b_i\gamma}.
\end{gather}
Given that $n=\sign{\eta_{n+1}}$ and $\gamma\geqslant0$,
\begin{gather}
\left(\abs{\eta_{n+1}}+E\gamma+\sum\dfrac{a_i\gamma{}}{1+b_i\gamma}\right)n=\abs{\sigma^\text{trial}-\sum\dfrac{\alpha_{i,n}}{1+b_i\gamma}}~\sign{\sigma^\text{trial}-\sum\dfrac{\alpha_{i,n}}{1+b_i\gamma}}.
\end{gather}
Thus,
\begin{gather}
n=\sign{\eta_{n+1}}=\sign{\sigma^\text{trial}-\sum\dfrac{\alpha_{i,n}}{1+b_i\gamma}},\\
\abs{\eta_{n+1}}+E\gamma+\sum\dfrac{a_i\gamma{}}{1+b_i\gamma}=\abs{\sigma^\text{trial}-\sum\dfrac{\alpha_{i,n}}{1+b_i\gamma}}.
\end{gather}
This is such a nice expression that gives the magnitude of the shifted stress as a function of sole $\gamma$.

With the above expressions, the yield function can be explicitly evaluated in terms of $\gamma$ as
\begin{gather}\label{eq:af_f}
\begin{split}
f\left(\gamma\right)&=\abs{\eta_{n+1}}-\left(\sigma^i+Kq_{n+1}+\sigma^s\left(1-e^{-m^sq_{n+1}}\right)\right)\\
&=\abs{\sigma^\text{trial}-\sum\dfrac{\alpha_{i,n}}{1+b_i\gamma}}-E\gamma-\sum\dfrac{a_i\gamma{}}{1+b_i\gamma}\\&-\sigma^i-Kq_n-K\gamma-\sigma^s\left(1-e^{-m^s\left(q_n+\gamma\right)}\right)\\
&=0.
\end{split}
\end{gather}

The Newton--Raphson method shall be used to solve this nonlinear equation. The Jacobian reads
\begin{gather}\label{eq:af_pfpg}
\pdfrac{f}{\gamma}=\sum\dfrac{b_i\alpha_{i,n}n-a_i}{\left(1+b_i\gamma\right)^2}-E-K-m^s\sigma^se^{-m^s\left(q_n+\gamma\right)}.
\end{gather}

The tangent modulus shall be computed via the chain rule.
\begin{gather}
\pdfrac{\sigma_{n+1}}{\varepsilon_{n+1}}=\pdfrac{\sigma^\text{trial}}{\varepsilon_{n+1}}+En\pdfrac{\gamma}{\varepsilon_{n+1}}=E+En\pdfrac{\gamma}{\varepsilon_{n+1}}.
\end{gather}
Following the general procedure \eqsref{eq:consistent_stiffness}, at equilibrium, the full differentiation of yield function is
\begin{gather}
\pdfrac{f}{\varepsilon_{n+1}}+\pdfrac{f}{\gamma}\pdfrac{\gamma}{\varepsilon_{n+1}}=0.
\end{gather}
Thus,
\begin{gather}
\pdfrac{\gamma}{\varepsilon_{n+1}}=-\left(\pdfrac{f}{\gamma}\right)^{-1}\pdfrac{f}{\varepsilon_{n+1}}.
\end{gather}
In which,
\begin{gather}
\pdfrac{f}{\varepsilon_{n+1}}=nE,
\end{gather}
and $\pdfrac{f}{\gamma}$ should have been computed when local iteration converges.

Finally, the tangent stiffness is
\begin{gather}
\pdfrac{\sigma_{n+1}}{\varepsilon_{n+1}}=E+En\pdfrac{\gamma}{\varepsilon_{n+1}}=E+\left(\pdfrac{f}{\gamma}\right)^{-1}E^2.
\end{gather}
\subsection{Implementation}
The state determination algorithm of this AF model is given in \algoref{algo:af_steel}.
\begin{breakablealgorithm}
\caption{state determination of uniaxial AF steel model}\label{algo:af_steel}
\begin{algorithmic}[1]
\State Parameter: $E$, $K$, $\sigma^s$, $m^s$, $a_i$, $b_i$
\State Input: $\varepsilon_{n+1}$, $\varepsilon_n$, $\varepsilon^p_n$, $\sigma_n$, $\alpha_{i,n}$, $q_n$
\State Output: $E_{n+1}$, $\varepsilon^p_{n+1}$, $\sigma_{n+1}$, $\alpha_{i,n+1}$, $q_{n+1}$
\State compute $\sigma^\text{trial}$, $\eta^\text{trial}$, $f^\text{trial}$ and $n=\sign{\sigma^\text{trial}-\sum\dfrac{\alpha_{i,n}}{1+b_i\gamma}}$
\If {$f^\text{trial}\geqslant0$}
\State $\gamma=0$
\While{true}
\State compute $f$ and $\pdfrac{f}{\gamma}$\Comment{\eqsref{eq:af_f} and \eqsref{eq:af_pfpg}}
\State $\Delta\gamma=\left(\pdfrac{f}{\gamma}\right)^{-1}f$
\If{$\abs{\Delta\gamma}<\text{tolerance}$}
\State break
\EndIf
\State $\gamma\leftarrow\gamma-\Delta\gamma$
\EndWhile
\State $\varepsilon^p_{n+1}=\varepsilon^p_n+\gamma{}n$
\State $q_{n+1}=q_n+\gamma$
\State $\alpha_{i,n+1}=\dfrac{\alpha_{i,n}+a_i\gamma{}n}{1+b_i\gamma}$
\State $\sigma_{n+1}=\sigma_{n+1}^\text{trial}-E\gamma{}n$
\State $E_{n+1}=E+\left(\pdfrac{f}{\gamma}\right)^{-1}E^2$
\Else
\State $\varepsilon^p_{n+1}=\varepsilon^p_n$
\State $q_{n+1}=q_n$
\State $\alpha_{i,n+1}=\alpha_{i,n}$
\State $\sigma_{n+1}=\sigma_{n+1}^\text{trial}$
\State $E_{n+1}=E$
\EndIf
\end{algorithmic}
\end{breakablealgorithm}

So far it is clear that for both simple and complex models, the structure of state determination algorithm remains more or less the same. The core formulation only differs due to different yield function, flow rule and hardening law. Nevertheless, some simplifications are often possible.

We will present the C++ implementation after the enhancement introduced in the coming section.
\section{Armstrong--Fredrick with Strain Memory}
The discussed Armstrong--Fredrick model allows a versatile modelling of kinematic hardening.
However, it is noticed in many experiments that the magnitude of isotropic hardening gradually decreases with increasing number of cycles --- even if cycles with fixed strain range.
To better model such a behaviour, \cite{Ohno1982} introduced the concept of non-hardening strain region.
It is a region that characterises a neighbourhood of a finite size around the current plastic strain. Just like the isotropic hardening of stress, the size of this region grows with the development of plastic strain.
\subsection{Theory}
Instead of solely using $q$ to drive the evolution of isotropic hardening, a new internal variable $r$ is introduced to characterise the size of non-hardening strain region.
\begin{gather}\label{eq:afco}
	\sigma^y=\sigma^i+Kq+\sigma^s\left(1-e^{-m^sq}\right)-\sigma^r\left(1-e^{-m^rr}\right),
\end{gather}
where $m^r$ is a model constant that controls the rate of reduction.
In absence of the linear part, viz., $K=0$, it can be seen that
\begin{gather}
	\lim\limits_{q\rightarrow\infty,r\rightarrow0}\sigma^y=\sigma^i+\sigma^s,\\
	\lim\limits_{q\rightarrow\infty,r\rightarrow\infty}\sigma^y=\sigma^i+\sigma^s-\sigma^r.
\end{gather}
In other words, the reduction of isotropic hardening is controlled by the internal variable $r$, the maximum possible reduction is $\sigma^r$.

Other literature may adopt different forms of reduction of isotropic hardening.
Most of them appear to be too complex for the task.
If one rewrites \eqsref{eq:afco} in the rate form,
\begin{gather}
\dot{\sigma^y}=m^s\sigma^se^{-m^sq}=m^s\left(\sigma^i+\sigma^s-\sigma^r\left(1-e^{-m^rr}\right)-\sigma^y\right),
\end{gather}
it is then clear that the bounding value $\sigma^i+\sigma^s-\sigma^r\left(1-e^{-m^rr}\right)$ is now affected by $r$.
The above expression is in fact identical to the ones proposed originally \cite{Ohno1982}.

A further surface shall be adopted to define the non-hardening strain region.
\begin{gather}
	h=\abs{\varepsilon^p-\theta}-r,
\end{gather}
where $\theta$ is similar to $\alpha$, which, in this case, points to the centre of the non-hardening strain region. It may be called the back plastic strain.

If one chooses a linear evolution law, denoting $n_h=\pdfrac{h}{\varepsilon^p}=\sign{\varepsilon^p-\theta}$, then
\begin{gather}
\dot{r}=c\nu,\qquad
\dot{\theta}=\left(1-c\right)\nu\pdfrac{h}{\varepsilon^p}=n_h\left(1-c\right)\nu,
\end{gather}
in which $c$ ranges from zero to unity, $\nu$ is the plastic multiplier, similar to $\gamma$.

Apart from the existing yield surface defined in stress space, here we have introduced another surface defined in plastic strain space.
With the assist of this surface, it is able to separate plastic strain into `isotropic' and `kinematic` parts.
The isotropic part $r$ is used to define strain memory.
\subsection{Formulation}
The following incremental form
\begin{gather}
\varepsilon^p_{n+1}-\theta_{n+1}=\varepsilon^p_{n+1}-\theta_n-\nu{}n_h\left(1-c\right)
\end{gather}
can be used to derive the following,
\begin{gather}
n_h\left(\abs{\varepsilon^p_{n+1}-\theta_{n+1}}+\nu{}\left(1-c\right)\right)=\varepsilon^p_{n+1}-\theta_n.
\end{gather}
Noting that $\nu\geqslant0$ and $1-c\geqslant0$, one can conclude that
\begin{gather}
\abs{\varepsilon^p_{n+1}-\theta_{n+1}}+\nu{}\left(1-c\right)=\abs{\varepsilon^p_{n+1}-\theta_n},
\end{gather}
and
\begin{gather}
	n_h=\sign{\varepsilon^p_{n+1}-\theta_n}.
\end{gather}

When the non-hardening strain region is developing, that is, $h=0$, one may derive
\begin{gather}
\abs{\varepsilon^p_{n+1}-\theta_n}-\nu{}\left(1-c\right)-r_n-\nu{}c=0,
\end{gather}
so that
\begin{gather}
\nu=\abs{\varepsilon^p_{n+1}-\theta_n}-r_n.
\end{gather}

It is easy to see that, for the excessive plastic strain that exceeds the current non-hardening strain region, this amount of plastic strain will be distributed to $\theta$ and $r$ according to the ratio defined by $c$.
Accounting for the Kuhn--Tucker condition, one may further unify the cases to obtain
\begin{gather}
\dot{r}=c\left\langle\abs{\varepsilon^p_{n+1}-\theta_n}-r_n\right\rangle=c\left\langle\nu\right\rangle,\\
\dot{\theta}=n_h\left(1-c\right)\left\langle\abs{\varepsilon^p_{n+1}-\theta_n}-r_n\right\rangle=n_h\left(1-c\right)\left\langle\nu\right\rangle.
\end{gather}

The original literature \cite{Ohno1982,Nouailhas1985} develops similar concepts.
However, the formulation presented in papers appears to be confusing.
\cite{Nouailhas1985} further modifies the evolution law such that `inside the memory surface, the centre does not move and the radius slowly decreases'.
It is unclear how such an affect could be achieved by the modification introduced.
\subsection{Implementation}
\begin{cppcode}
int ArmstrongFrederick1D::update_trial_status(const vec& t_strain) {
    incre_strain = (trial_strain = t_strain) - current_strain;

    if(norm(incre_strain) <= datum::eps) return SUANPAN_SUCCESS;

    trial_stress = current_stress + (trial_stiffness = initial_stiffness) * incre_strain;

    trial_history = current_history;
    auto& q = trial_history(size);
    const auto& current_r = current_history(size + 2);
    const auto& current_theta = current_history(size + 3);
    auto& ep = trial_history(size + 1);
    auto& r = trial_history(size + 2);
    auto& theta = trial_history(size + 3);

    auto yield_func = fabs(trial_stress(0) - accu(trial_history.head(size))) - std::max(0., yield + hardening * q + saturation * (1. - exp(-ms * q)) - reduction * (1. - exp(-mr * r)));

    if(yield_func < 0.) return SUANPAN_SUCCESS;

    auto gamma = 0.;
    double xi, jacobian, dr = 0.;

    auto counter = 0u;
    auto ref_error = 1.;
    while(true) {
        if(max_iteration == ++counter) {
            suanpan_error("Cannot converge within {} iterations.\n", max_iteration);
            return SUANPAN_FAIL;
        }

        const auto s_term = saturation * exp(-ms * q);
        const auto r_term = reduction * exp(-mr * r);

        auto k = yield + saturation - reduction + hardening * q - s_term + r_term;
        auto dk = hardening + ms * s_term - mr * r_term * dr;
        if(k < 0.) k = dk = 0.;

        auto sum_a = 0., sum_b = 0.;
        for(auto I = 0u; I < size; ++I) {
            const auto denom = 1. + b(I) * gamma;
            sum_a += trial_history(I) / denom;
            sum_b += a(I) / denom;
        }

        yield_func = fabs(xi = trial_stress(0) - sum_a) - (elastic_modulus + sum_b) * gamma - k;

        jacobian = -elastic_modulus - dk;

        if(xi > 0.)
            for(auto I = 0u; I < size; ++I) jacobian += (b(I) * trial_history(I) - a(I)) * pow(1. + b(I) * gamma, -2.);
        else
            for(auto I = 0u; I < size; ++I) jacobian -= (b(I) * trial_history(I) + a(I)) * pow(1. + b(I) * gamma, -2.);

        const auto incre = yield_func / jacobian;
        const auto error = fabs(incre);
        if(1u == counter) ref_error = error;
        suanpan_debug("Local iteration error: {:.5E}.\n", error);
        if(error < tolerance * ref_error || ((fabs(yield_func) < tolerance || error < datum::eps) && counter > 5u)) break;

        gamma -= incre;
        q -= incre;
        ep -= xi > 0. ? incre : -incre;

        r = current_r;
        theta = current_theta;

        if(const auto h = fabs(ep - current_theta) - current_r; h > 0.) {
            const auto nh = ep > current_theta ? 1. : -1.;
            r += memory * h;
            theta += nh * (1. - memory) * h;
            dr = xi > 0. ? memory * nh : -memory * nh;
        }
        else dr = 0.;
    }

    if(xi > 0.) {
        for(auto I = 0u; I < size; ++I) trial_history(I) = (trial_history(I) + a(I) * gamma) / (1. + b(I) * gamma);

        trial_stress -= elastic_modulus * gamma;
    }
    else {
        for(auto I = 0u; I < size; ++I) trial_history(I) = (trial_history(I) - a(I) * gamma) / (1. + b(I) * gamma);

        trial_stress += elastic_modulus * gamma;
    }

    trial_stiffness += elastic_modulus / jacobian * elastic_modulus;

    return SUANPAN_SUCCESS;
}
\end{cppcode}
\subsection{Further Modifications}
Based on the presented framework, it is possible to further modify the specific evolution rules to refine the behaviour.
For example, \cite{Xu2016} proposed a transition mechanism that basically states that, with the development of plasticity characterised by either $q$ or $r$, a portion of isotropic hardening will transit into kinematic hardening.
Such a mechanism can be equivalently expressed by modifying the kinematic hardening rule.

Instead of \eqsref{eq:af_kinematic}, the evolution of back stress can be modified as
\begin{gather}
\dot{\alpha_i}=\left(a_i+\hat{a_i}\right)\dot{\varepsilon^p}-b_i\alpha_i\dot{q}
\end{gather}
This expression includes an additional term $\hat{a_i}=\hat{a_i}\left(r\right)$ that shall evolve with the development of plasticity.

Adopting the same procedure, one can derive
\begin{gather}
\alpha_{n+1}=\sum\dfrac{\alpha_{i,n}+\left(a_i+\hat{a_i}\right)\gamma{}n}{1+b_i\gamma}.
\end{gather}
However, it is not guaranteed that $\alpha_i$ would never exceed the bound $\left(a_i+\hat{a_i}\right)/b_i$.
\section{Subloading Surface Model}\label{sec:uniaxial_subloading}
The subloading surface framework is extensively introduced in monographs \cite{Hashiguchi2023}.
Here we do not repeatedly cite the same references for brevity.
Readers can refer to the book and the references therein.
\subsection{Subloading Surface}
The main modification made by the subloading surface model to the conventional plasticity model is the parameter $z$ such that on the basis of the existing yield surface
\begin{gather}
f=\abs{\eta}-\sigma^y,
\end{gather}
a so-called subloading surface $f_s$ is defined as
\begin{gather}
f_s=\abs{\eta}-z\sigma^y.
\end{gather}
This subloading surface, according to definition, always passes through the current stress point, meaning $f_s\equiv0$ under all loading cases.
In the above, $z\sigma^y$ is expressed as $RF$ in the referenced literature.
Here, we do not adopt the same set of notations as it may cause conflicts.
We keep using what we have been using for better consistency.

The history variable $0\leqslant{}z\leqslant{}1$ is a ratio that adjusts the size of this subloading surface.
It provides a smooth transition from interior to the yield surface $f$.
For loading only, this effectively defines surface that grows with strain increments, and there is no initial elastic region, because when $z=0$, any non-zero $\sigma$ would result in $f_s>0$ thus triggers the evolution of plasticity.
For unloading, if one enforces $f_s=0$, one would eventually end up with $\gamma<0$. In this case, $z$ shall be updated by
\begin{gather}
z=\dfrac{\abs{\eta}}{\sigma^y}.
\end{gather}
Knowing that $f_s<0$ for unloading, $z$ updated according to the above expression is guaranteed to be smaller than unity.
As a result, $f_s$ shrinks such that $\sigma$ always falls on $f_s$.

The evolution of $z$ obeys
\begin{gather}
\dot{z}=U\left(z\right)\dot{q}=U\left(z\right)\gamma,
\end{gather}
where $U\left(z\right)$ is chosen as follows.
\begin{gather}
U\left(z\right)=u\cot\left(\dfrac{\pi}{2}z\right).
\end{gather}
The original model also sets a minimum value for $z$, which is not incorporated here.
The introduction of $z$ is the sole modification by the original subloading surface model.

Later, an extended subloading surface model is proposed.
The extended version further introduces the concept of elastic core, which is characterised by an additional variable $c$ such that the shifted stress $\eta$ shall be computed using the following expression instead.
\begin{gather}
\eta=\sigma-\bar{\alpha},
\end{gather}
where $\bar{\alpha}$ is the new centre of the subloading surface, which shall be defined as
\begin{gather}
\bar{\alpha}-c=z\left(\alpha-c\right).
\end{gather}
The new variable $c$ defines an elastic core.
The shifted stress $\eta$ can equivalently be written as
\begin{gather}
\eta=\sigma-z\alpha+\left(z-1\right)c.
\end{gather}
The above modification essentially pulls the centre of the subloading surface, from the original origin defined by $\alpha$, towards elastic core $c$ by a certain amount governed by $z$.

According to the original formulation, the evolution of $c$ involves not only $\gamma$ but also $\alpha$, which leads to a deeply coupled system that is not easy to solve numerically.
Here, instead of $c$, one can define an equivalent independent variable, say $d$, such that $\sigma^yd=c-\alpha$, then
\begin{gather}
\eta=\sigma-z\alpha+\left(z-1\right)\left(\sigma^yd+\alpha\right)=\sigma-\alpha+\left(z-1\right)\sigma^yd.
\end{gather}
The magnitude of $d$ represents the normalised size of the elastic core.
According to definition, it shall not exceed some limit $0\leqslant{}z_e\leqslant1$.
This leads to the following evolution law for $d$.
\begin{gather}
\dot{d}=c_e\gamma\left(z_en-d\right).
\end{gather}
In the above, we have introduced two constants $c_e$ and $z_e$ to control the behaviour.

Furthermore, we redefine the back stress such that
\begin{gather}
\eta=\sigma-a^y\alpha+\left(z-1\right)\sigma^yd.
\end{gather}
We effectively scale $\alpha$ down by $a^y$ and now $\alpha$ represents some normalised measure of the back stress.
\subsection{Theory}
\subsubsection{Isotropic Hardening}
For isotropic hardening, the same rule combining linear hardening and saturation is adopted.
\begin{gather}
    \sigma^y=\sigma^i+k_\text{iso}q+\sigma^s_\text{iso}\left(1-e^{-m^s_\text{iso}q}\right).
\end{gather}
The model constants are: $\sigma^i$ is the initial yield stress, $k_\text{iso}$ is the linear hardening modulus, $\sigma^s_\text{iso}$ is the saturation stress, and $m^s_\text{iso}$ is the saturation rate.
The subscript $\left(\cdot\right)_\text{iso}$ indicates they are for isotropic hardening, a similar set of parameters will be introduced for kinematic hardening.

Thus,
\begin{gather}
\ddfrac{\sigma^y}{q}=k_\text{iso}+\sigma^s_\text{iso}m^s_\text{iso}e^{-m^s_\text{iso}q}.
\end{gather}
\subsubsection{Kinematic Hardening}
Foe kinematic hardening, we use a modified Armstrong--Frederick rule.
\begin{gather}
    \dot{\alpha}=b\gamma\left(n-\alpha\right),
\end{gather}
with
\begin{gather}
    a^y=a^i+k_\text{kin}q+a^s_\text{kin}\left(1-e^{-m^s_\text{kin}q}\right).
\end{gather}
The parameters $a^i$, $k_\text{kin}$, $a^s_\text{kin}$, and $m^s_\text{kin}$ resemble the ones used in the isotropic hardening rule.
It is possible to associate $a^y$ with $\sigma^y$, say, one sets $a^i=\sigma^i/2$, $k_\text{kin}=k_\text{iso}/2$, $a^s_\text{kin}=\sigma^s_\text{iso}/2$ and $m^s_\text{kin}=m^s_\text{iso}$ to ensure $a^y=\sigma^y/2$.

The normalised back stress can be explicitly expressed as
\begin{gather}
    \alpha=\dfrac{b\gamma{}n+\alpha_n}{1+b\gamma}.
\end{gather}
\subsubsection{Evolution of Elastic Core}
Similar to kinematic hardening, one can obtain the explicit expression for $d$ as
\begin{gather}
d=\dfrac{c_ez_e\gamma{}n+d_n}{1+c_e\gamma}.
\end{gather}
\subsection{Formulation}
It is now possible to summarise all expressions used in this extended subloading surface model.
\begin{table}[ht]
\centering
\begin{tabular}{rl}
\toprule
Constitutive Law&$\sigma=E\left(\varepsilon-\varepsilon^p\right)$\\
Subloading Surface&$f_s=\abs{\eta}-z\sigma^y=\abs{\sigma-a^y\alpha+\left(z-1\right)\sigma^yd}-z\sigma^y$\\
Flow Rule&$\dot{\varepsilon^p}=\gamma{}n=\gamma~\sign{\eta}$\\
Hardening Law&$\dot{q}=\gamma$\\
&$\sigma^y=\sigma^i+k_\text{iso}q+\sigma^s_\text{iso}\left(1-e^{-m^s_\text{iso}q}\right)$\\
&$a^y=a^i+k_\text{kin}q+a^s_\text{kin}\left(1-e^{-m^s_\text{kin}q}\right)$\\
&$\dot{\alpha}=b\gamma\left(n-\alpha\right)$\\
&$\dot{d}=c_e\gamma\left(z_en-d\right)$\\
&$\dot{z}=U\gamma$\\\bottomrule
\end{tabular}
\end{table}

The model constants involve the following.
\begin{itemize}
\item $E$ --- Young's modulus
\item $\sigma^i$, $k_\text{iso}$, $\sigma^s_\text{iso}$, $m^s_\text{iso}$ --- isotropic hardening backbone
\item $a^i$, $k_\text{kin}$, $a^s_\text{kin}$, $m^s_\text{kin}$ --- kinematic hardening backbone
\item $b$ --- kinematic hardening rate
\item $u$ --- evolution rate of $z$
\item $c_e$, $z_e$ --- evolution rate of $d$
\end{itemize}
\subsection{Implementation}
Along with the subloading surface $f_s$ and $z$, one could formulate the local system $\mb{R}$ with two residuals.
\begin{gather}
\mb{R}=\left\{
\begin{array}{l}
\abs{\sigma-a^y\alpha+\left(z-1\right)\sigma^yd}-z\sigma^y,\\[2mm]
z-z_n-U\gamma.
\end{array}
\right.
\end{gather}

Taking local variables $\mb{x}=\begin{bmatrix}
\gamma&z
\end{bmatrix}^\mT$, denoting $n=\sign{\eta}$, the Jacobian $\mb{J}$ shall be written as
\begin{gather}
\mb{J}=\begin{bmatrix}
n\left(-En-\ddfrac{}{\gamma}\left(a^y\alpha\right)+\left(z-1\right)\ddfrac{}{\gamma}\left(\sigma^yd\right)\right)-z\ddfrac{\sigma^y}{\gamma}&n\sigma^yd-\sigma^y\\[2mm]
-U&1-\gamma{}\ddfrac{U}{z}
\end{bmatrix}.
\end{gather}
We also use general forms of evolutions of both kinematic and isotropic hardening, such that $\sigma^y=\sigma^y\left(\gamma\right)$ and $\alpha=\alpha\left(\gamma\right)$.
\subsubsection{Implicit Euler Method}
One shall notice that
\begin{gather}
\eta=\sigma^\text{trial}-E\gamma{}n-a^y\dfrac{b\gamma{}n+\alpha_n}{1+b\gamma}+\left(z-1\right)\sigma^y\dfrac{c_ez_e\gamma{}n+d_n}{1+c_e\gamma},
\end{gather}
by rearranging all $\eta$ related terms (including $n$) to the left, one can obtain
\begin{gather}
\eta+E\gamma{}n+\dfrac{b\gamma{}a^yn}{1+b\gamma}+\left(1-z\right)\dfrac{c_ez_e\gamma{}\sigma^yn}{1+c_e\gamma}=\sigma^\text{trial}-\dfrac{a^y\alpha_n}{1+b\gamma}+\left(z-1\right)\dfrac{\sigma^yd_n}{1+c_e\gamma},
\end{gather}
this is effectively
\begin{gather}
\left(\abs{\eta}+E\gamma{}+\dfrac{b\gamma{}a^y}{1+b\gamma}+\left(1-z\right)\dfrac{c_ez_e\gamma{}\sigma^y}{1+c_e\gamma}\right)n=\sigma^\text{trial}-\dfrac{a^y\alpha_n}{1+b\gamma}+\left(z-1\right)\dfrac{\sigma^yd_n}{1+c_e\gamma}.
\end{gather}
Noting that $\gamma>0$, $a^y>0$, $\sigma^y>0$ and by definition, $1-z>0$, the big bracket is in fact non-negative. This means,
\begin{gather}
n=\sign{\sigma^\text{trial}-\dfrac{a^y\alpha_n}{1+b\gamma}+\left(z-1\right)\dfrac{\sigma^yd_n}{1+c_e\gamma}},
\end{gather}
which is not a function of $\eta$.
\subsubsection{Trapezoidal Method}
One can also use the trapezoidal rule to improve numerical stability.
\begin{gather}
\alpha=\alpha_n+\dfrac{b\gamma}{2}\left(n_n-\alpha_n+n-\alpha\right),
\end{gather}
it can be further rearranged to obtain
\begin{gather}
\alpha=\dfrac{\left(2-b\gamma\right)\alpha_n+b\gamma{}\left(n_n+n\right)}{2+b\gamma}.
\end{gather}
Similarly,
\begin{gather}
d=\dfrac{\left(2-c_e\gamma\right)d_n+c_ez_e\gamma{}\left(n_n+n\right)}{2+c_e\gamma}.
\end{gather}

Repeating the same procedure, one could derive
\begin{multline}
\eta+E\gamma{}n+a^y\dfrac{b\gamma{}n}{2+b\gamma}+\left(1-z\right)\sigma^y\dfrac{c_ez_e\gamma{}n}{2+c_e\gamma}=\\
\sigma^\text{trial}-a^y\dfrac{\left(2-b\gamma\right)\alpha_n+b\gamma{}n_n}{2+b\gamma}+\left(z-1\right)\sigma^y\dfrac{\left(2-c_e\gamma\right)d_n+c_ez_e\gamma{}n_n}{2+c_e\gamma}.
\end{multline}
Then
\begin{gather}
n=\sign{\sigma^\text{trial}-a^y\dfrac{\left(2-b\gamma\right)\alpha_n+b\gamma{}n_n}{2+b\gamma}+\left(z-1\right)\sigma^y\dfrac{\left(2-c_e\gamma\right)d_n+c_ez_e\gamma{}n_n}{2+c_e\gamma}}.
\end{gather}
\subsubsection{Tangent Stiffness}
The tangent stiffness can be computed via the chain rule.
\begin{gather}
\ddfrac{\sigma}{\varepsilon}=E+\dfrac{E^2}{\det\left(\mb{J}\right)}\left(1-\gamma{}\ddfrac{U}{z}\right).
\end{gather}

We summarise the state determination as follows.
\begin{breakablealgorithm}
\caption{state determination of uniaxial subloading surface model}\label{algo:subloading_steel}
\begin{algorithmic}[1]
\State Parameter: $E$
\State Parameter: $\sigma^i$, $k_\text{iso}$, $\sigma^s_\text{iso}$, $m^s_\text{iso}$, $a^i$, $k_\text{kin}$, $a^s_\text{kin}$, $m^s_\text{kin}$
\State Parameter: $b$, $u$, $c_e$, $z_e$
\State Input: $\varepsilon_{n+1}$, $\varepsilon_n$, $\sigma_n$, $\alpha_{n}$, $q_n$, $d_n$, $z_n$
\State Output: $E_{n+1}$, $\sigma_{n+1}$, $\alpha_{n+1}$, $q_{n+1}$, $d_{n+1}$, $z_{n+1}$
\State $\sigma^\text{trial}=\sigma_n+E\left(\varepsilon_{n+1}-\varepsilon_n\right)$
\State set history variables $\left(\cdot\right)_{n+1}=\left(\cdot\right)_{n}$
\State $\gamma=0$
\While{true}
\State update $q$, $\alpha$, $d$ according to $\gamma$
\State formulate $\mb{R}$ and $\mb{J}$
\State $\Delta=\mb{J}^{-1}\mb{R}$
\If{$\norm{\Delta}<\text{tolerance}$}
\State break
\EndIf
\State $\gamma\leftarrow\gamma-\Delta\left(0\right)$
\State $z\leftarrow{}z-\Delta\left(1\right)$
\EndWhile
\If{$\gamma>0$}\Comment{plastic}
\State $\sigma_{n+1}=\sigma^\text{trial}-nE\gamma$
\State $E_{n+1}=E+\dfrac{E^2}{\det\left(\mb{J}\right)}\left(1-\gamma{}\ddfrac{U}{z}\right)$
\Else\Comment{elastic unloading}
\State reset history variables $\left(\cdot\right)_{n+1}=\left(\cdot\right)_{n}$
\State compute $z$ using $f_s=0$
\EndIf
\end{algorithmic}
\end{breakablealgorithm}

A reference implementation can be shown as follows.
\begin{cppcode}
int Subloading1D::update_trial_status(const vec& t_strain) {
    incre_strain = (trial_strain = t_strain) - current_strain;

    if(fabs(incre_strain(0)) <= datum::eps) return SUANPAN_SUCCESS;

    trial_stress = current_stress + (trial_stiffness = initial_stiffness) * incre_strain;

    trial_history = current_history;
    const auto& current_alpha = current_history(0);
    const auto& current_d = current_history(1);
    const auto& current_q = current_history(2);
    const auto& current_z = current_history(3);
    auto& alpha = trial_history(0);
    auto& d = trial_history(1);
    auto& q = trial_history(2);
    auto& z = trial_history(3);

    auto gamma = 0., ref_error = 0.;

    vec2 residual, incre;
    mat22 jacobian;

    const auto current_ratio = yield_ratio(current_z);

    auto counter = 0u;
    while(true) {
        if(max_iteration == ++counter) {
            suanpan_error("Cannot converge within {} iterations.\n", max_iteration);
            return SUANPAN_FAIL;
        }

        q = current_q + gamma;

        const auto exp_iso = saturation_iso * exp(-m_iso * q);
        auto y = initial_iso + saturation_iso + k_iso * q - exp_iso;
        auto dy = k_iso + m_iso * exp_iso;
        if(y < 0.) y = dy = 0.;

        const auto exp_kin = saturation_kin * exp(-m_kin * q);
        auto a = initial_kin + saturation_kin + k_kin * q - exp_kin;
        auto da = k_kin + m_kin * exp_kin;
        if(a < 0.) a = da = 0.;

        const auto n = trial_stress(0) - a * current_alpha / (1. + be * gamma) + (z - 1.) * y * current_d / (1. + ce * gamma) > 0. ? 1. : -1.;

        auto top = be * gamma * n + current_alpha;
        auto bottom = 1. + be * gamma;

        alpha = top / bottom;
        const auto dalpha = (be * n - top * be / bottom) / bottom;

        top = ce * ze * gamma * n + current_d;
        bottom = 1. + ce * gamma;

        d = top / bottom;
        const auto dd = (ce * ze * n - top * ce / bottom) / bottom;

        const auto trial_ratio = yield_ratio(z);
        const auto avg_rate = u * .5 * (current_ratio(0) + trial_ratio(0));

        residual(0) = fabs(trial_stress(0) - elastic * gamma * n - a * alpha + (z - 1.) * y * d) - z * y;

        if(1u == counter && residual(0) < 0.) {
            z = ((trial_stress(0) - a * alpha) / y - d) / (n - d);
            return SUANPAN_SUCCESS;
        }

        residual(1) = z - current_z - gamma * avg_rate;

        jacobian(0, 0) = n * ((z - 1.) * (dd * y + dy * d) - (a * dalpha + alpha * da)) - elastic - z * dy;
        jacobian(0, 1) = n * y * d - y;

        jacobian(1, 0) = -avg_rate;
        jacobian(1, 1) = 1. - u * gamma * .5 * trial_ratio(1);

        if(!solve(incre, jacobian, residual)) return SUANPAN_FAIL;

        const auto error = inf_norm(incre);
        if(1u == counter) ref_error = error;
        suanpan_debug("Local iteration error: {:.5E}.\n", error);
        if(error < tolerance * ref_error || ((error < tolerance || inf_norm(residual) < tolerance) && counter > 5u)) {
            trial_stress -= elastic * gamma * n;
            trial_stiffness += elastic / det(jacobian) * elastic * jacobian(1, 1);

            return SUANPAN_SUCCESS;
        }

        gamma -= incre(0);
        z -= incre(1);
        if(z > 1.) z = 1. - datum::eps;
    }
}
\end{cppcode}
\subsection{Some Further Remarks}
Assuming no hardening and no elastic core, and $U\left(z\right)$ takes a general form, then the Jacobian could be simplified to
\begin{gather}
\mb{J}=\begin{bmatrix}
-E&-\sigma^y\\[2mm]
-U&1-\gamma{}U'
\end{bmatrix}.
\end{gather}
The tangent stiffness shall be then expressed as
\begin{gather}
\ddfrac{\sigma}{\varepsilon}=E-\dfrac{E^2}{E\left(1-\gamma{}U'\right)+U\sigma^y}\left(1-\gamma{}U'\right)=E-\dfrac{E}{1+\dfrac{U\sigma^y}{E\left(1-\gamma{}U'\right)}}.
\end{gather}
We are interested in the continuity condition, that is, when transiting from $z_n=0$ to some non-zero value, would the tangent stiffness be continuous?
The answer is positive according to the definition, and this is exactly one of the important contributions by the subloading surface model.
Here we demonstrate this by exploring the explicit expression for tangent stiffness.
Given that we are interested in the very first step from $z_n=0$ to some non-zero value, then
\begin{gather}
\gamma=\dfrac{z-z_n}{U}=\dfrac{z}{U},
\end{gather}
also since $\sigma^y/E$ is a constant, we shall focus on the following limit,
\begin{gather}
\lim\limits_{z\rightarrow0^+}\dfrac{U}{1-z\dfrac{U'}{U}}=\lim\limits_{z\rightarrow0^+}\dfrac{U^2}{U-zU'}.
\end{gather}
Now if one chooses $U=-u\ln\left(z\right)$ so that $U'=-u/z$, then
\begin{gather}
\lim\limits_{z\rightarrow0^+}\dfrac{u^2\ln^2\left(z\right)}{-u\ln\left(z\right)+u}=\lim\limits_{z\rightarrow0^+}\dfrac{2u\ln\left(z\right)/z}{-1/z}=\lim\limits_{z\rightarrow0^+}-u\ln\left(z\right)=\infty.
\end{gather}
Thus,
\begin{gather}
\lim\limits_{z\rightarrow0^+}\ddfrac{\sigma}{\varepsilon}=E.
\end{gather}

However, numerically, since infinity is not achievable, and $\dfrac{U^2}{U-zU'}$ quickly departs away from infinity, this continuity is not ideal.
In the above example, for $z=\num{e-16}$, $\dfrac{U^2}{U-zU'}\approx35.87$.

One can also choose $U=u\cot\left(\dfrac{\pi}{2}z\right)$ so that $U'=-\dfrac{u\pi}{2}\csc^2\left(\dfrac{\pi}{2}z\right)$, then
\begin{gather}
\lim\limits_{z\rightarrow0^+}\dfrac{U^2}{U-zU'}=\lim\limits_{z\rightarrow0^+}u\dfrac{\cos\left(\pi{}z\right)+1}{\sin\left(\pi{}z\right)+\pi{}z}=\infty.
\end{gather}
For $z=\num{e-16}$, it becomes $\num{e16}$.
This expression presents much better numerical performance.

Now, we shall impose an additional requirement to ensure such a continuity can be achieved numerically.
\begin{gather}
\lim\limits_{z\rightarrow0^+}\dfrac{zU^2}{U-zU'}=\text{constant}.
\end{gather}
We could see that the cotangent function satisfies the above condition.
\section{Uniaxial Model for BRB Steel}\label{sec:uniaxial_brb}
The previous models show symmetric behaviour regardless under tension or compression. For some application such as buckling restrained braces, buckling under compression would result in lower strength. In this section, a model \cite{Zona2012} suitable for BRB steel is presented with a neat implementation.
\subsection{Theory}
To distinguish different responses in tension and compression, one set of flow rule and hardening law is not sufficient. Instead, tension and compression shall be treated separately. Here superscripts $\left(\cdot\right)^+$ and $\left(\cdot\right)^-$ are used to denote tension and compression governing equations.
\subsubsection{Plasticity Activation}
The model adopts a different approach. The activation of plasticity is decided based on the product $\sigma\dot{\varepsilon}$. When loading towards tension (compression), as long as $\sigma$ is in tension (compression), plasticity evolves. In mathematical language,
\begin{gather}
\begin{array}{ll}
\dot{\varepsilon}\sigma>0,&\text{plasticity develops}\\
\dot{\varepsilon}\sigma\leqslant0,&\text{elastic unloading}
\end{array}
\end{gather}
In other words, loading is always plastic while only unloading can be elastic.
\subsubsection{Flow Rule}
The increment of plastic strain is defined to be a portion of the increment of total strain,
\begin{gather}
\dot{\varepsilon^p}=\left\{\begin{array}{ll}
\abs{\dfrac{\sigma-K\varepsilon^p}{\sigma_y^+}}^{\alpha^+}\dot{\varepsilon},&\text{tension evolution}\\
\abs{\dfrac{\sigma-K\varepsilon^p}{\sigma_y^-}}^{\alpha^-}\dot{\varepsilon}.&\text{compression evolution}
\end{array}\right.
\end{gather}
The term $K\varepsilon^p$ defines the linear hardening, which serves as a baseline, given than yield stress $\sigma_y^+$ ($\sigma_y^-)$ defines the boundary, the fraction is always smaller than unity.
\subsubsection{Hardening Law}
The yield stress follows a Voce type rule.
\begin{gather}
\begin{split}
\sigma_y^+&=\sigma_0+\left(\sigma_s^+-\sigma_0\right)\left(1-\exp\left(-\dfrac{q}{n^+}\right)\right)\\
&=\sigma_s^+-\left(\sigma_s^+-\sigma_0\right)\exp\left(-\dfrac{q}{n^+}\right),
\end{split}\\
\begin{split}
\sigma_y^-&=\sigma_0+\left(\sigma_s^--\sigma_0\right)\left(1-\exp\left(-\dfrac{q}{n^-}\right)\right)\\
&=\sigma_s^--\left(\sigma_s^--\sigma_0\right)\exp\left(-\dfrac{q}{n^-}\right).
\end{split}
\end{gather}
In which $n^+$ and $n^-$ are two factors controlling the speed of evolution of yield stress while $\sigma_s^+$ and $\sigma_s^-$ are two saturated stresses. By setting different values, tension response can differ from compression response.
\subsection{Formulation}
The governing equations are summarised as follows.
\begin{table}[ht]
\centering
\begin{tabular}{rl}
\toprule
Constitutive Law&$\sigma=E\left(\varepsilon-\varepsilon^p\right)$\\
Flow Rule&$\dot{\varepsilon^p}=\left\{\begin{array}{ll}
\abs{\dfrac{\sigma-K\varepsilon^p}{\sigma_y^+}}^{\alpha^+}\dot{\varepsilon},&\text{tension evolution,}\\
\abs{\dfrac{\sigma-K\varepsilon^p}{\sigma_y^-}}^{\alpha^-}\dot{\varepsilon},&\text{compression evolution.}
\end{array}\right.$\\
Hardening Law&$\dot{q}=\abs{\dot{\varepsilon^p}}$\\
&$\sigma_y^+=\sigma_s^+-\left(\sigma_s^+-\sigma_0\right)\exp\left(-\dfrac{q}{n^+}\right)$\\
&$\sigma_y^-=\sigma_s^--\left(\sigma_s^--\sigma_0\right)\exp\left(-\dfrac{q}{n^-}\right)$\\\bottomrule
\end{tabular}
\end{table}

The state determination is based on the flow rule. Rearranging it gives
\begin{gather}
R=\dot{\varepsilon^p}-\abs{\dfrac{\sigma-K\varepsilon^p}{\sigma_y^\pm}}^{\alpha^\pm}\dot{\varepsilon}
\end{gather}
with superscript $\left(\cdot\right)^\pm$ covering both tension and compression, whichever suits.

It can be further expanded as
\begin{gather}
R=\dot{\varepsilon^p}-\abs{\dfrac{E\left(\varepsilon_{n+1}-\varepsilon^p_n-\dot{\varepsilon^p}\right)-K\varepsilon^p_n-K\dot{\varepsilon^p}}{\sigma_y^\pm}}^{\alpha^\pm}\dot{\varepsilon}.
\end{gather}
Denoting $\sigma^\text{trial}=E\left(\varepsilon_{n+1}-\varepsilon^p_n\right)-k\varepsilon^p_n$, it is
\begin{gather}\label{eq:brb_r}
R=\dot{\varepsilon^p}-\abs{\dfrac{\sigma^\text{trial}-E\dot{\varepsilon^p}-K\dot{\varepsilon^p}}{\sigma_y^\pm}}^{\alpha^\pm}\dot{\varepsilon}.
\end{gather}

The corresponding derivative is
\begin{gather}\label{eq:brb_dr}
\pdfrac{R}{\dot{\varepsilon^p}}=1+\alpha^\pm\dot\varepsilon\abs{\dfrac{\sigma^\text{trial}-E\dot{\varepsilon^p}-K\dot{\varepsilon^p}}{\sigma_y^\pm}}^{\alpha^\pm}\dfrac{\left(E+K\right)\sigma_y^\pm+\left(\sigma^\text{trial}-E\dot{\varepsilon^p}-K\dot{\varepsilon^p}\right)\ddfrac{\sigma_y^\pm}{\dot{\varepsilon^p}}}{\sigma_y^{\pm}\left(\sigma^\text{trial}-E\dot{\varepsilon^p}-K\dot{\varepsilon^p}\right)}.
\end{gather}
It can be validated the above derivative holds for both positive and negative fraction.

The derivative of yield function can be computed as
\begin{gather}
\ddfrac{\sigma_y^\pm}{\dot{\varepsilon^p}}=\dfrac{\sigma_s^\pm-\sigma_0}{n^\pm}\exp\left(-\dfrac{q}{n^\pm}\right)\sign{\dot{\varepsilon^p}}.
\end{gather}

For tangent stiffness, the following expression will be used.
\begin{gather}
\pdfrac{R}{\varepsilon_{n+1}}=-\abs{\dfrac{\sigma^\text{trial}-E\dot{\varepsilon^p}-K\dot{\varepsilon^p}}{\sigma_y^\pm}}^{\alpha^\pm}\left(1+\dfrac{\alpha^\pm\dot\varepsilon{}E}{\sigma^\text{trial}-E\dot{\varepsilon^p}-K\dot{\varepsilon^p}}\right).
\end{gather}
\subsection{Implementation}
A clear and concise implementation is presented as follows. Compared with other existing bloated implementations, the following one is significantly simpler.
\begin{breakablealgorithm}
\caption{state determination of uniaxial BRB steel model}\label{algo:brb}
\begin{algorithmic}[1]
\State \textbf{Parameter}: $E$, $K$, $\sigma_0$, $\sigma_s^+$, $\sigma_s^-$, $n^+$, $n^-$, $\alpha^+$, $\alpha^-$
\State \textbf{Input}: $\varepsilon_{n+1}$, $\varepsilon_n$, $\varepsilon^p_n$, $\sigma_n$, $q_n$
\State \textbf{Output}: $E_{n+1}$, $\varepsilon^p_{n+1}$, $\sigma_{n+1}$, $q_{n+1}$
\State $\dot{\varepsilon}=\varepsilon_{n+1}-\varepsilon_n$
\State assuming elastic response $\sigma_{n+1}=\sigma_n+E\left(\varepsilon_{n+1}-\varepsilon_n\right)$
\If {$\sigma_{n+1}\dot{\varepsilon}\leqslant0.$}\Comment{elastic}
\State $\varepsilon^p_{n+1}=\varepsilon^p_n$
\State $q_{n+1}=q_n$
\State $E_{n+1}=E$
\Else\Comment{plastic}
\State determine tension/compression plastic loading
\State $\dot{\varepsilon^p}=\dfrac{1}{2}\dot{\varepsilon}$
\While{true}
\State $\varepsilon_{n+1}^p=\varepsilon_n^p+\dot{\varepsilon^p}$
\State $q_{n+1}=q_n+\abs{\dot{\varepsilon^p}}$
\State $\sigma_{n+1}=E\left(\varepsilon_{n+1}-\varepsilon_{n+1}^p\right)$
\State compute $\sigma_y^\pm$ and $\ddfrac{\sigma_y^\pm}{\dot{\varepsilon^p}}$ using the proper function
\State compute $R$ and $\pdfrac{R}{\dot{\varepsilon^p}}$\Comment{\eqsref{eq:brb_r} and \eqsref{eq:brb_dr}}
\State $\Delta=-\left(\pdfrac{R}{\dot{\varepsilon^p}}\right)^{-1}R$
\If{$\abs{\Delta}<\text{tolerance}$}
\State break
\EndIf
\State $\dot{\varepsilon^p}\leftarrow\dot{\varepsilon^p}+\Delta$
\EndWhile
\State $E_{n+1}=E+E\left(\pdfrac{R}{\dot{\varepsilon^p}}\right)^{-1}\pdfrac{R}{\varepsilon_{n+1}}$
\EndIf
\end{algorithmic}
\end{breakablealgorithm}

The idea of the above steel BRB model is fairly straightforward. Given that the target is to generate different response under tension/compression, one can simply adopt two sets of flow rule and hardening law to fulfil this task. For each case, the overall structure of algorithm remains the same, one can switch between two sets of model parameters.

The CPP implementation of the state determination algorithm is shown as follows.
\begin{cppcode}
int SteelBRB::update_trial_status(const vec& t_strain) {
	incre_strain = (trial_strain = t_strain) - current_strain;

	if(fabs(incre_strain(0)) <= datum::eps) return SUANPAN_SUCCESS;

	trial_stress = current_stress + (trial_stiffness = elastic_modulus) * incre_strain;

	trial_history = current_history;
	const auto& current_accumulated_strain = current_history(0); // u
	const auto& current_plastic_strain = current_history(1);     // \delta_1
	auto& accumulated_strain = trial_history(0);                 // u
	auto& plastic_strain = trial_history(1);                     // \delta_1

	if(trial_stress(0) / incre_strain(0) < 0.) return SUANPAN_SUCCESS;

	const auto tension_flag = incre_strain(0) >= 0.;
	const auto& exponent = tension_flag ? t_exponent : c_exponent;
	const auto compute_stress = tension_flag ? std::mem_fn(&SteelBRB::compute_t_yield_stress) : std::mem_fn(&SteelBRB::compute_c_yield_stress);

	auto incre = .5 * incre_strain(0), incre_plastic_strain = 0.;
	auto counter = 0;
	while(true) {
		if(max_iteration == ++counter) {
			suanpan_error("SteelBRB cannot converge within %u iterations.\n", max_iteration);
			return SUANPAN_FAIL;
		}

		incre_plastic_strain += incre;

		plastic_strain = current_plastic_strain + incre_plastic_strain;
		trial_stress = elastic_modulus * (trial_strain - plastic_strain);

		const auto sigma_y = compute_stress(this, accumulated_strain = current_accumulated_strain + fabs(incre_plastic_strain));
		const auto numerator = trial_stress(0) - plastic_modulus * plastic_strain;
		const auto fraction = numerator / sigma_y(0);
		const auto pow_term = pow(fabs(fraction), exponent);
		auto residual = -incre_strain(0) * pow_term;
		const auto jacobian = 1. + exponent / numerator * residual * (s_modulus - fraction * (incre_plastic_strain >= 0. ? sigma_y(1) : -sigma_y(1)));
		residual += incre_plastic_strain;

		const auto error = fabs(incre = -residual / jacobian);

		suanpan_debug("SteelBRB local iteration error: %.5E.\n", error);

		if(error <= tolerance) {
			trial_stiffness *= 1. - (pow_term + incre_strain(0) * elastic_modulus * exponent * pow_term / numerator) / jacobian;

			return SUANPAN_SUCCESS;
		}
	}
}
\end{cppcode}
\section{VAFCRP1D}
Finally, an extension of Armstrong--Fredrick model is presented to close this chapter. A 3D version will be introduced later.
\subsection{Theory}
\subsubsection{Yield Function}
A von Mises yielding function is used.
\begin{gather}
f=\norm{\eta}-k,
\end{gather}
in which $\eta=\sigma-\beta$ is the shifted stress, $\beta$ is the back stress and $k=k\left(q\right)$ is the isotropic hardening stress.
\subsubsection{Flow Rule}
The associated plasticity flow is adopted. The plastic strain rate is then
\begin{gather}
\dot{\varepsilon^p}=\gamma\pdfrac{f}{\sigma}=\gamma{}n=\gamma~\sign{\eta},
\end{gather}
where $n=\dfrac{\eta}{\norm{\eta}}=\sign{\eta}$. The corresponding accumulated plastic strain rate is
\begin{gather}
\dot{p}=\norm{\dot{\varepsilon^p}}=\gamma.
\end{gather}
\subsubsection{Hardening Law}
An exponential function with a linear component is used for isotropic hardening stress.
\begin{gather}
k=\sigma_y+k_lp+k_s-k_se^{-mp}.
\end{gather}
The corresponding derivative is
\begin{gather}
\ddfrac{k}{\gamma}=k_l+k_sme^{-mp}.
\end{gather}
The rate form of back stress $\displaystyle\beta=\sum\beta^i$ is defined as
\begin{gather*}
\dot{\beta^i}=a^i\dot{\varepsilon^p}-b^i\beta^i\dot{p}.
\end{gather*}
In terms of $\gamma$, it is $\dot{\beta^i}=a^i\gamma{}n-b^i\beta^i\gamma$. The incremental form is thus
\begin{gather}
\beta^i=\beta_n^i+a^i\gamma{}n-b^i\beta^i\gamma,\qquad
\beta^i=\dfrac{\beta_n^i+a^i\gamma{}n}{1+b^i\gamma}.
\end{gather}
\subsubsection{Plastic Multiplier}
The rate of plastic multiplier is defined as
\begin{gather}
\dfrac{\gamma}{\Delta{}t}=\dot{\gamma}=\dfrac{1}{\mu}\left(\left(\dfrac{\norm{\eta}}{k}\right)^{\dfrac{1}{\epsilon}}-1\right),
\end{gather}
in which $\mu$ and $\epsilon$ are two material constants. Equivalently, it is
\begin{gather}
\norm{\eta}\left(\dfrac{\Delta{}t}{\Delta{}t+\mu\gamma}\right)^\epsilon-k=0.
\end{gather}
\subsection{Formulation}
\subsubsection{Incremental Form}
The shifted stress can be computed as
\begin{gather}
\eta=\sigma-\beta=E\left(\varepsilon^\text{trial}-\varepsilon^p_n-\gamma{}n\right)-\beta=\sigma^\text{trial}-nE\gamma-\sum\dfrac{\beta_n^i+na^i\gamma}{1+b^i\gamma}
\end{gather}
with $\sigma^\text{trial}=E\left(\varepsilon^\text{trial}-\varepsilon^p_n\right)$. Knowing that $\gamma$ is positive, the following can be obtained by splitting the summation into two parts,
\begin{gather*}
\left(\norm{\eta}+E\gamma+\sum\dfrac{a^i\gamma}{1+b^i\gamma}\right)n=\norm{\sigma^\text{trial}-\sum\dfrac{\beta_n^i}{1+b^i\gamma}}u,
\end{gather*}
where $\displaystyle{}u=\dfrac{\sigma^\text{trial}-\sum\dfrac{\beta_n^i}{1+b^i\gamma}}{\norm{\sigma^\text{trial}-\sum\dfrac{\beta_n^i}{1+b^i\gamma}}}$. This expression is equivalent to $n=u$ and
\begin{gather}
\eta=\left(\norm{\sigma^\text{trial}-\sum\dfrac{\beta_n^i}{1+b^i\gamma}}-E\gamma-\sum\dfrac{a^i\gamma}{1+b^i\gamma}\right)u.
\end{gather}

The reason to find such an identity is that $u$ is only a function of $\gamma$, the derivative of which can be easily computed. Similar derivations can also be seen in the 3D version, leading to the conclusion that the direction of back stress is aligned with some direction.

The corresponding derivatives are
\begin{gather*}
\pdfrac{\norm{\eta}}{\varepsilon^\text{trial}}=uE,\qquad\pdfrac{\norm{\eta}}{\gamma}=\sum\dfrac{b^i\beta_n^iu-a^i}{(1+b^i\gamma)^2}-E.
\end{gather*}
\subsubsection{Scalar Equation Iteration}
With the above expression, it is possible to establish the local residual based on the creep rule, which is
\begin{multline}
R=\left(\norm{\sigma^\text{trial}-\sum\dfrac{\beta_n^i}{1+b^i\gamma}}-E\gamma-\sum\dfrac{a^i\gamma}{1+b^i\gamma}\right)\left(\dfrac{\Delta{}t}{\Delta{}t+\mu\gamma}\right)^\epsilon\\-\left(\sigma_y+k_l\left(p_n+\gamma\right)+k_s\left(1-e^{-m\left(p_n+\gamma\right)}\right)\right).
\end{multline}
The corresponding derivatives are then
\begin{gather}
\pdfrac{R}{\varepsilon^\text{trial}}=uE\left(\dfrac{\Delta{}t}{\Delta{}t+\mu\gamma}\right)^\epsilon,\\
\pdfrac{R}{\gamma}=\left(\sum\dfrac{b^i\beta_n^iu-a^i}{(1+b^i\gamma)^2}-E-\dfrac{\epsilon\mu\norm{\eta}}{\Delta{}t+\mu\gamma}\right)\left(\dfrac{\Delta{}t}{\Delta{}t+\mu\gamma}\right)^\epsilon-\ddfrac{k}{\gamma}.
\end{gather}
\subsubsection{Consistent Tangent Stiffness}
For stiffness, $\varepsilon^\text{trial}$ is now varying, then
\begin{gather}
\pdfrac{R}{\varepsilon^\text{trial}}+\pdfrac{R}{\gamma}\ddfrac{\gamma}{\varepsilon^\text{trial}}=0,\qquad\ddfrac{\gamma}{\varepsilon^\text{trial}}=-\left(\pdfrac{R}{\gamma}\right)^{-1}\pdfrac{R}{\varepsilon^\text{trial}}.
\end{gather}
Since the stress can be written as
\begin{gather}
\sigma=E(\varepsilon^\text{trial}-\varepsilon^p)=E(\varepsilon^\text{trial}-\varepsilon^p_n-\Delta\varepsilon^p)=E(\varepsilon^\text{trial}-\varepsilon^p_n)-E\gamma{}u.
\end{gather}
The derivative is
\begin{gather}
\ddfrac{\sigma}{\varepsilon^\text{trial}}=E-Eu\ddfrac{\gamma}{\varepsilon^\text{trial}}=E+E^2\left(\pdfrac{R}{\gamma}\right)^{-1}\left(\dfrac{\Delta{}t}{\Delta{}t+\mu\gamma}\right)^\epsilon.
\end{gather}
\subsection{Implementation}
\begin{cppcode}
int VAFCRP1D::update_trial_status(const vec& t_strain) {
    trial_stress = current_stress + (trial_stiffness = initial_stiffness) * (incre_strain = (trial_strain = t_strain) - current_strain);

    trial_history = current_history;
    auto& p = trial_history(size);

    if(fabs(trial_stress(0) - accu(trial_history.head(size))) < std::max(0., yield + hardening * p + saturated * (1. - exp(-m * p)))) return SUANPAN_SUCCESS;

    auto gamma = 0.;
    double xi, jacobian, exp_gamma;

    unsigned counter = 0;
    while(true) {
        if(max_iteration == ++counter) {
            suanpan_error("VAFCRP1D cannot converge in %u iterations.\n", max_iteration);
            return SUANPAN_FAIL;
        }

        const auto exp_term = saturated * exp(-m * p);

        auto k = yield + saturated + hardening * p - exp_term;
        auto dk = hardening + m * exp_term;
        if(k < 0.) k = dk = 0.;

        auto sum_a = 0., sum_b = 0.;
        for(unsigned I = 0; I < size; ++I) {
            const auto denom = 1. + b(I) * gamma;
            sum_a += trial_history(I) / denom;
            sum_b += a(I) / denom;
        }

        const auto q = fabs(xi = trial_stress(0) - sum_a) - (elastic_modulus + sum_b) * gamma;

        exp_gamma = pow(*incre_time / (*incre_time + mu * gamma), epsilon);

        jacobian = -elastic_modulus - epsilon * mu * q / (*incre_time + mu * gamma);

        if(xi > 0.) for(unsigned I = 0; I < size; ++I) jacobian += (b(I) * trial_history(I) - a(I)) * pow(1. + b(I) * gamma, -2.);
        else for(unsigned I = 0; I < size; ++I) jacobian -= (b(I) * trial_history(I) + a(I)) * pow(1. + b(I) * gamma, -2.);

        const auto incre = (q * exp_gamma - k) / ((jacobian *= exp_gamma) -= dk);
        suanpan_extra_debug("VAFCRP1D local iterative loop error: %.5E.\n", fabs(incre));
        if(fabs(incre) <= tolerance) break;

        gamma -= incre;
        p -= incre;
    }

    if(xi > 0.) {
        for(unsigned I = 0; I < size; ++I) trial_history(I) = (trial_history(I) + a(I) * gamma) / (1. + b(I) * gamma);

        trial_stress -= elastic_modulus * gamma;
    }
    else {
        for(unsigned I = 0; I < size; ++I) trial_history(I) = (trial_history(I) - a(I) * gamma) / (1. + b(I) * gamma);

        trial_stress += elastic_modulus * gamma;
    }

    trial_stiffness += elastic_modulus / jacobian * elastic_modulus * exp_gamma;

    return SUANPAN_SUCCESS;
}
\end{cppcode}