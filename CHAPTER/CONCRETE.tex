\chapter{Concrete}
\section{Concrete Damage Plasticity (CDP) Model}
In this section, the concrete damage plasticity model proposed by \cite{Lee1998} is presented.
It can be seen as a continuation and further extension of the Barcelona model \cite{Lubliner1989}, in which a unified uniaxial backbone curve that possess some interesting features is proposed.
A slightly different version (with Lode angle dependency and other minor changes) is implemented in ABAQUS.

The CDP model follows Lemaitre's damage theory \cite{Lemaitre1985} and is developed under the assumption of isotropic damage.
Accordingly, the final stress $\bsigma$ can be expressed as the product of the effective stress $\bbsigma$ and some function of damage measure.
\begin{gather}
\bsigma=h\cdot\bbsigma,
\end{gather}
where $h=h\left(d_t,d_c\right)$ is a function of two damage variables $d_t$ and $d_c$, which depend on some internal history variables.
It can be understood by using the two--surface concept, that is, the final stress $\bsigma$ is the projection of $\bbsigma$ onto the final yield surface.

The effective stress $\bbsigma$ itself is bounded by the effective yield surface that fully resembles the conventional plasticity.
The damage measure $h\left(d_t,d_c\right)$ can be independently computed once all internal historic variables are determined.
Thus $h\left(d_t,d_c\right)$ and $\bbsigma$ can be handled in a relatively independent manner.
\subsection{Plasticity Theory}
\subsubsection{Yield Function}
The yield function is defined as
\begin{gather}
f=\alpha\bar{I}_1+\sqrt{\dfrac{3}{2}}\norm{\bar{\bs}}+\beta\left\langle\hat{\sigma}_1\right\rangle-\left(1-\alpha\right)c_c,
\end{gather}
with $\bar{I}_1=\tr{\bbsigma}$ is the first invariant of effective stress tensor $\bbsigma$, $\hat{\sigma}_1$ is the major effective principal stress, $c_c=-\bar{f}_c$ denotes cohesion and $\beta=\dfrac{\bar{f}_c}{\bar{f}_t}(\alpha-1)-(\alpha+1)$ where $\alpha$ is a material constant.
The effective backbone stresses (both positive) $\bar{f}_c$ and $\bar{f}_t$ will be defined later.
It is assumed that $\hat{\sigma}_1\geqslant\hat{\sigma}_2\geqslant\hat{\sigma}_3$.

It shall be noted that, for the given $\bbsigma$, one can perform the eigen decomposition to obtain the principal stress $\hat{\bsigma}$, with which, the following holds
\begin{gather}
\hat{I}_1=\tr{\hat{\bsigma}}=\bar{I}_1,\qquad
\norm{\hat{\bs}}=\norm{\bar{\bs}},
\end{gather}
because they are effectively invariants of the given tensor.
In the following, we do not distinguish between those two forms of invariants.
One shall realise that the same quantity can be computed using different tensors and depending on the context, one form may be more convenient than the other.
\subsubsection{Flow Rule}
The flow potential $g$ is chosen to be
\begin{gather}
g=\sqrt{2\bar{J}_2}+\alpha_p\bar{I}_1=\norm{\bar{\bs}}+\tr{\alpha_p\bbsigma},
\end{gather}
in which $\alpha_p$ is a material constant that controls dilatation.
The flow rule is accordingly defined as
\begin{gather}
\dot{\bvarepsilon^p}=\gamma\pdfrac{g}{\bbsigma}=\gamma\left(\dfrac{\bar{\bs}}{\norm{\bar{\bs}}}+\alpha_p\mb{1}\right)=\gamma\left(\bn+\alpha_p\mb{1}\right).
\end{gather}
It is clearly \textbf{not} associative.
In deviatoric and spherical components,
\begin{gather}
\dot{\bvarepsilon^{d,p}}=\dev{\dot{\bvarepsilon^p}}=\gamma\bn,\quad\dot{\varepsilon^{v,p}}=\tr{\dot{\bvarepsilon^p}}=\gamma3\alpha_p.
\end{gather}

Noting that
\begin{gather}
\bar{\bs}=\bar{\bs}^\text{trial}-2G\dot{\bvarepsilon^{d,p}}=\bar{\bs}^\text{trial}-\gamma2G\bn,\\
\bar{p}=\bar{p}^\text{trial}-K\dot{\varepsilon^{v,p}}=\bar{p}^\text{trial}-\gamma3K\alpha_p,
\end{gather}
equivalently,
\begin{gather}
\norm{\bar{\bs}}+\gamma2G=\norm{\bar{\bs}^\text{trial}},\\
\bar{I}_1+\gamma9K\alpha_p=\bar{I}_1^\text{trial}.
\end{gather}
Furthermore, $\bs$ and $\bs^\text{trial}$ are coaxial (why?), thus,
\begin{gather}
\dfrac{\bar{\bs}}{\norm{\bar{\bs}}}=\dfrac{\bar{\bs}^\text{trial}}{\norm{\bar{\bs}^\text{trial}}}\equiv{}\bn.
\end{gather}
It simply means the flow direction is fixed for all iterations in each sub-step. And due to the coaxiality, $\bar{\bs}$ and $\bar{\bs}^\text{trial}$ share the same eigen space. More importantly, the eigenvectors remain constant for each iteration. Thus,
\begin{gather}
\bn=\dfrac{\bar{\bs}}{\norm{\bar{\bs}}}=\dfrac{\bar{\bs}^\text{trial}}{\norm{\bar{\bs}^\text{trial}}},\qquad
\hat{\bn}=\dfrac{\hat{\bs}}{\norm{\hat{\bs}}}=\dfrac{\hat{\bs}^\text{trial}}{\norm{\hat{\bs}^\text{trial}}},
\end{gather}
where $\hat{\bs}$ denotes the principal stress tensor of the deviatoric stress tensor $\bar{\bs}$. As the yield function $f$ can be equivalently expressed with the principal stresses, in the following derivation, $\hat{\bn}$ is used to represent the unit principal deviatoric stress which has three components. The transformation matrix $\mb{T}$ is defined as
\begin{gather}
\underbrace{\hat{\bsigma}}_\text{\numproduct{3x1}}=\underbrace{\mb{T}}_\text{\numproduct{3x6}}\underbrace{\bbsigma}_\text{\numproduct{6x1}},\qquad
\underbrace{\hat{\bs}}_\text{\numproduct{3x1}}=\underbrace{\mb{T}}_\text{\numproduct{3x6}}\underbrace{\bar{\bs}}_\text{\numproduct{6x1}},
\end{gather}
and can be formulated from eigenanalysis.
\subsubsection{Hardening Law}
Both compression and tension backbones are described by functions of the same form.
Since concrete shows different behaviour under compression and tension, different material parameters are used accordingly.
The subscript $\left(\cdot\right)_\aleph$ may denote either tension $\left(\cdot\right)_t$ or compression $\left(\cdot\right)_c$ in the following expressions.

Internal hardening parameters $\kappa_\aleph$ is governed by the following evolution rule,
\begin{gather}
\dot{\kappa_\aleph}=\gamma{}H_\aleph,
\end{gather}
where $H_\aleph$ defines the hardening law. Different $H_\aleph$ shall be used for compression/tension.
\begin{gather}
H_t=r\dfrac{f_t}{g_t}\left(\hat{n}_1+\alpha_p\right),\\
H_c=\left(1-r\right)\dfrac{f_c}{g_c}\left(\hat{n}_3+\alpha_p\right).
\end{gather}
In which $\hat{n}_1\geqslant\hat{n}_3$ denote the maximum and minimum in $\hat{\bn}$ and $r$ is a scalar--valued function of the effective principal stress. In the original model, it is defined as
\begin{gather}
r\left(\hat{\bsigma}\right)=\dfrac{\left\langle\hat{\sigma}_1\right\rangle+\left\langle\hat{\sigma}_2\right\rangle+\left\langle\hat{\sigma}_3\right\rangle}{\abs{\hat{\sigma}_1}+\abs{\hat{\sigma}_2}+\abs{\hat{\sigma}_3}}.
\end{gather}
The purpose of $r$ is to characterise the proportion of tension in a multiaxial loading case.
Other forms can also be used, for example,
\begin{gather}
r\left(\hat{\bsigma}\right)=\dfrac{\norm{\left\langle\hat{\bsigma}\right\rangle}}{\norm{\hat{\bsigma}}}.
\end{gather}
\subsubsection{Backbone Curve}
The backbone curve $f_\aleph$ is related to the internal parameter $\kappa_\aleph$.
\begin{gather*}
f_\aleph=f_{\aleph,0}\sqrt{\phi_\aleph}\Phi_\aleph,
\end{gather*}
with
\begin{gather*}
\phi_\aleph=1+a_\aleph\left(2+a_\aleph\right)\kappa_\aleph,\qquad
\Phi_\aleph=\dfrac{1+a_\aleph-\sqrt{\phi_\aleph}}{a_\aleph}.
\end{gather*}

The effective counterpart $\bar{f}_\aleph$ is defined as
\begin{gather*}
\bar{f}_\aleph=\dfrac{f_\aleph}{1-d_\aleph}=f_{\aleph,0}\sqrt{\phi_\aleph}\Phi_\aleph^{1-c_\aleph/b_\aleph},
\end{gather*}
with the damage variable $d_\aleph$,
\begin{gather*}
d_\aleph=1-\Phi_\aleph^{c_\aleph/b_\aleph}.
\end{gather*}
One can observe that $f_\aleph=\left(1-d_\aleph\right)\bar{f}_\aleph$ also follows the damage theory.
When $d_\aleph=0$, there is no damage, then $f_\aleph=\bar{f}_\aleph$.
When $d_\aleph=1$, the material has completely failed, then $f_\aleph=0$.
We do not discuss those functions in detail here.
One can refer to the Barcelona model \cite{Lubliner1989} for an extensive discussion.

In general, the backbone curve can be customised. The main algorithm has no interested in how the backbone curve is computed, only $f_\aleph$, $\bar{f}_\aleph$ and $d_\aleph$ and their derivatives with regard to $\kappa_\aleph$ need to be provided. The exponential form adopted in the original model may encounter some numerical difficulties, which in the author's opinion is not ideal. One can always choose other forms, such as providing those quantities in a tabulated fashion. This is also what ABAQUS offers.
\subsection{Damage Theory}
The damage measure takes the form
\begin{gather}
h\left(d_t,d_c\right)=\left(1-d_c\right)\left(1-sd_t\right)
\end{gather}
where $s=s_0+\left(1-s_0\right)r$ is the recovery factor.
When $r=1$, indicating pure tension loading, $s=1$, the tension damage $d_t$ is fully accounted.
When $r=0$, indicating pure compression loading, $s=s_0$, the effective tension damage is $s_0d_t<d_t$.
This is a convenient approach to simulate the strength recovery when the material is initially pulled and later compressed.
No similar counterpart is considered for compression damage as typically once the material is crushed, further tension is not possible.
The alternative to achieve a similar effect is to use anisotropic damage measure.
\subsection{Plasticity Formulation}
The CDP model is driven by three quantities $\kappa_t$, $\kappa_c$ and $\bvarepsilon^p$. The governing equations are listed as follows.
\begin{table}[ht]
\centering
\begin{tabular}{rl}
\toprule
Yield Function&$f=\alpha\bar{I}_1+\sqrt{\dfrac{3}{2}}\norm{\bar{\bs}}+\beta\left\langle\hat{\sigma}_1\right\rangle-\left(1-\alpha\right)c_c$\\
Flow Rule&$\dot{\bvarepsilon^p}=\gamma\left(\dfrac{\bar{\bs}}{\norm{\bar{\bs}}}+\alpha_p\mb{1}\right)$\\
Hardening Law&$\dot{\kappa_\aleph}=\gamma{}H_\aleph$\\
\bottomrule
\end{tabular}
\end{table}
\subsubsection{Elastic Loading/Unloading}
Assuming elastic loading/unloading, the trial state of the effective part can be computed as done in other plasticity models.
\begin{gather}
\bbsigma^\text{trial}=\mb{D}:\left(\bvarepsilon_{n+1}-\bvarepsilon^p_n\right).
\end{gather}
Then by performing eigenanalysis, $\hat{\bsigma}^\text{trial}$ can be computed. The trial yield function is
\begin{gather}\label{eq:cdp_tf}
f^\text{trial}=\alpha\bar{I}_1^\text{trial}+\sqrt{\dfrac{3}{2}}\norm{\bar{\bs}^\text{trial}}+\beta\left\langle\hat{\sigma}_1^\text{trial}\right\rangle-\left(1-\alpha\right)c_c,
\end{gather}
with $\beta$ and $c_c$ computed by using $\kappa_{t,n}$ and $\kappa_{c,n}$. If $f^\text{trial}<0$, indicating elastic loading/unloading, then $\bbsigma_{n+1}=\bbsigma^\text{trial}$, the final stress is simply
\begin{gather}
\bsigma_{n+1}=\left(1-d_{c,n}\right)\left(1-sd_{t,n}\right)\bbsigma^\text{trial}.
\end{gather}

Since $r$ is a function of $\bbsigma_{n+1}$, the corresponding tangent stiffness is \textbf{not} a scaled version of the elastic stiffness.
It can be derived as follows using the chain rule.
\begin{gather}\label{eq:cdp_ed}
\begin{split}
\pdfrac{\bsigma_{n+1}}{\bvarepsilon_{n+1}}&=\pdfrac{\bsigma_{n+1}}{\bbsigma_{n+1}}:\pdfrac{\bbsigma_{n+1}}{\bvarepsilon_{n+1}}\\
&=\left(1-d_{c,n}\right)d_{t,n}\left(s_0-1\right)\bbsigma^\text{trial}\otimes\ddfrac{r}{\bbsigma_{n+1}}:\mb{D}+\left(1-d_{c,n}\right)\left(1-sd_{t,n}\right)\mb{D}\\
&=\left(1-d_{c,n}\right)\left(d_{t,n}\left(s_0-1\right)\bbsigma^\text{trial}\otimes\ddfrac{r}{\bbsigma_{n+1}}+\left(1-sd_{t,n}\right)\mathbb{I}\right):\mb{D}.
\end{split}
\end{gather}
\subsubsection{Plasticity Evolution}
The local system consists of three residuals.
\begin{gather}\label{eq:cdp_r}
\mb{R}=\left\{
\begin{array}{l}
\alpha\bar{I}_1+\sqrt{\dfrac{3}{2}}\norm{\bar{\bs}^\text{trial}}-\gamma\sqrt{6}G+\beta\left\langle\hat{\sigma}_1^\text{trial}-\gamma\left(2G\hat{n}_1+3K\alpha_p\right)\right\rangle+\left(1-\alpha\right)\bar{f}_c,\\[4mm]
\kappa_{t,n}+\gamma{}r\dfrac{f_t}{g_t}\left(\hat{n}_1+\alpha_p\right)-\kappa_t,\\[4mm]
\kappa_{c,n}+\gamma\left(1-r\right)\dfrac{f_c}{g_c}\left(\hat{n}_3+\alpha_p\right)-\kappa_c.
\end{array}
\right.
\end{gather}

By choosing $\mb{x}=\begin{bmatrix}
\gamma&\kappa_t&\kappa_c
\end{bmatrix}^\mT$ as the independent variables and assuming $\hat{\bn}=\dfrac{\hat{\bs}^\text{trial}}{\norm{\hat{\bs}^\text{trial}}}$ that is a function of $\bvarepsilon_{n+1}$ only thus does not contain $\gamma$, the Jacobian can be computed as
\begin{footnotesize}
\begin{gather}\label{eq:cdp_dr}
\mb{J}=\begin{bmatrix}
-9K\alpha\alpha_p-\sqrt{6}G-\beta\left(2G\hat{n}_1+3K\alpha_p\right)H\left(\hat{\sigma}_1\right)&\left\langle\hat{\sigma}_1\right\rangle\pdfrac{\beta}{\kappa_t}&\left(1-\alpha\right)\bar{f}_c'+\left\langle\hat{\sigma}_1\right\rangle\pdfrac{\beta}{\kappa_c}\\[4mm]
f_t\dfrac{\hat{n}_1+\alpha_p}{g_t}\left(r+\gamma\pdfrac{r}{\gamma}\right)&\gamma{}r\dfrac{\hat{n}_1+\alpha_p}{g_t}f_t'-1&\cdot\\[4mm]
f_c\dfrac{\hat{n}_3+\alpha_p}{g_c}\left(1-r-\gamma\pdfrac{r}{\gamma}\right)&\cdot&\gamma\left(1-r\right)\dfrac{\hat{n}_3+\alpha_p}{g_c}f_c'-1
\end{bmatrix},
\end{gather}
\end{footnotesize}
where $H\left(\cdot\right)$ is the heaviside function and
\begin{gather*}
\pdfrac{\beta}{\kappa_t}=\left(1-\alpha\right)\dfrac{\bar{f}_c}{\bar{f}_t^2}\bar{f}_t',\\
\pdfrac{\beta}{\kappa_c}=\left(\alpha-1\right)\dfrac{1}{\bar{f}_t}\bar{f}_c'.
\end{gather*}
In the explicit form, if $\hat{\sigma}_1>0$,
\begin{footnotesize}
\begin{gather}
\mb{J}=\begin{bmatrix}
-9K\alpha\alpha_p-\sqrt{6}G-\beta\left(2G\hat{n}_1+3K\alpha_p\right)&(1-\alpha)\dfrac{\bar{f}_c\hat{\sigma}_1}{\bar{f}_t^2}\bar{f}_t'&\left(1-\alpha\right)\left(1-\dfrac{\hat{\sigma}_1}{\bar{f}_t}\right)\bar{f}_c'\\[4mm]
f_t\dfrac{\hat{n}_1+\alpha_p}{g_t}\left(r+\gamma\pdfrac{r}{\gamma}\right)&r\gamma\dfrac{\hat{n}_1+\alpha_p}{g_t}f_t'-1&\cdot\\[4mm]
f_c\dfrac{\hat{n}_3+\alpha_p}{g_c}\left(1-r-\gamma\pdfrac{r}{\gamma}\right)&\cdot&\left(1-r\right)\gamma\dfrac{\hat{n}_3+\alpha_p}{g_c}f_c'-1
\end{bmatrix},
\end{gather}
\end{footnotesize}
otherwise,
\begin{footnotesize}
\begin{gather}
\mb{J}=\begin{bmatrix}
-9K\alpha\alpha_p-\sqrt{6}G&\cdot&\left(1-\alpha\right)\bar{f}_c'\\[4mm]
f_t\dfrac{\hat{n}_1+\alpha_p}{g_t}\left(r+\gamma\pdfrac{r}{\gamma}\right)&r\gamma\dfrac{\hat{n}_1+\alpha_p}{g_t}f_t'-1&\cdot\\[4mm]
f_c\dfrac{\hat{n}_3+\alpha_p}{g_c}\left(1-r-\gamma\pdfrac{r}{\gamma}\right)&\cdot&\left(1-r\right)\gamma\dfrac{\hat{n}_3+\alpha_p}{g_c}f_c'-1
\end{bmatrix}.
\end{gather}
\end{footnotesize}
\subsection{Damage Formulation}
There is no local iteration required in the damage part. Once $\kappa_t$ and $\kappa_c$ are determined, the effective part $\bar\bsigma$ can be determined. Damage measures $d_t$ and $d_c$ can be computed accordingly.
\subsection{Consistent Tangent Stiffness}
In order to take derivatives with regard to the trial strain $\bvarepsilon_{n+1}$, one can replace $\norm{\hat{\bs}^\text{trial}}$ and $\norm{\bar{\bs}^\text{trial}}$, which yields
\begin{gather}
\pdfrac{\mb{R}}{\bvarepsilon_{n+1}}=\left\{
\begin{array}{l}
3K\alpha{}\mb{1}+\sqrt{6}G\bn+H\left(\hat{\sigma}_1\right)\beta\ddfrac{\hat\sigma_1}{\bbsigma}:\pdfrac{\bbsigma}{\bvarepsilon_{n+1}},\\[4mm]
\gamma\dfrac{f_t}{g_t}\left(r\pdfrac{\hat{n}_1}{\bvarepsilon_{n+1}}+\left(\hat{n}_1+\alpha_p\right)\ddfrac{r}{\bbsigma}:\pdfrac{\bbsigma}{\bvarepsilon_{n+1}}\right),\\[4mm]
\gamma\dfrac{f_c}{g_c}\left(\left(1-r\right)\pdfrac{\hat{n}_3}{\bvarepsilon_{n+1}}-\left(\hat{n}_3+\alpha_p\right)\ddfrac{r}{\bbsigma}:\pdfrac{\bbsigma}{\bvarepsilon_{n+1}}\right).
\end{array}
\right.
\end{gather}
so that
\begin{gather}
\ddfrac{\mb{x}}{\bvarepsilon_{n+1}}=\begin{bmatrix}
\ddfrac{\gamma}{\bvarepsilon_{n+1}}\\[3mm]
\ddfrac{\kappa_t}{\bvarepsilon_{n+1}}\\[3mm]
\ddfrac{\kappa_c}{\bvarepsilon_{n+1}}
\end{bmatrix}=-\left(\pdfrac{\mb{R}}{\mb{x}}\right)^{-1}\pdfrac{\mb{R}}{\bvarepsilon_{n+1}}.
\end{gather}

The effective stress $\bbsigma_{n+1}$ only depends on $\bvarepsilon_{n+1}$ and $\gamma$. The partial derivative is
\begin{gather}
\begin{split}
\pdfrac{\bbsigma_{n+1}}{\bvarepsilon_{n+1}}&=\pdfrac{}{\bvarepsilon_{n+1}}\left(\bs^\text{trial}-2G\gamma\dfrac{\bs^\text{trial}}{\norm{\bs^\text{trial}}}+\left(p^\text{trial}-3K\alpha_p\gamma\right)\mb{1}\right)\\
&=\mb{D}-\dfrac{4G^2\gamma}{\norm{\bs^\text{trial}}}\left(\mathbb{I}^\text{dev}-\bn\otimes{}\bn\right).
\end{split}
\end{gather}
The full derivative is
\begin{gather}
\begin{split}
\ddfrac{\bbsigma_{n+1}}{\bvarepsilon_{n+1}}&=\ddfrac{}{\bvarepsilon_{n+1}}\left(\bs^\text{trial}-2G\gamma\dfrac{\bs^\text{trial}}{\norm{\bs^\text{trial}}}+\left(p^\text{trial}-3K\alpha_p\gamma\right)\mb{1}\right)\\
&=\mb{D}-\dfrac{4G^2\gamma}{\norm{\bs^\text{trial}}}\left(\mathbb{I}^\text{dev}-\bn\otimes{}\bn\right)-\left(2G\bn+3K\alpha_p\mb{1}\right)\otimes\pdfrac{\gamma}{\bvarepsilon_{n+1}}\\
&=\pdfrac{\bbsigma_{n+1}}{\bvarepsilon_{n+1}}-\left(2G\bn+3K\alpha_p\mb{1}\right)\otimes\pdfrac{\gamma}{\bvarepsilon_{n+1}}.
\end{split}
\end{gather}

The derivative of the damage factor can be expressed as
\begin{gather}
\begin{split}
\ddfrac{h}{\bvarepsilon_{n+1}}&=\left(1-d_c\right)\ddfrac{\left(1-sd_t\right)}{\bvarepsilon_{n+1}}+\left(1-sd_t\right)\ddfrac{\left(1-d_c\right)}{\bvarepsilon_{n+1}}\\
&=\left(d_c-1\right)\left(s\ddfrac{d_t}{\bvarepsilon_{n+1}}+d_t\ddfrac{s}{\bvarepsilon_{n+1}}\right)+\left(sd_t-1\right)\ddfrac{d_c}{\bvarepsilon_{n+1}},
\end{split}
\end{gather}
with
\begin{gather}
\ddfrac{d_t}{\bvarepsilon_{n+1}}=\ddfrac{d_t}{\kappa_t}\ddfrac{\kappa_t}{\bvarepsilon_{n+1}},\\
\ddfrac{d_c}{\bvarepsilon_{n+1}}=\ddfrac{d_c}{\kappa_c}\ddfrac{\kappa_c}{\bvarepsilon_{n+1}},\\
\ddfrac{s}{\bvarepsilon_{n+1}}=\left(1-s_0\right)\ddfrac{r}{\bbsigma_{n+1}}:\ddfrac{\bbsigma_{n+1}}{\bvarepsilon_{n+1}}.
\end{gather}

The following derivatives would be useful.
\begin{gather}
\ddfrac{d_\aleph}{\kappa_\aleph}=\dfrac{c_\aleph}{b_\aleph}\dfrac{a_\aleph+2}{2\sqrt{\phi_\aleph}}\Phi_\aleph^{c_\aleph/b_\aleph-1},\\
\ddfrac{f_\aleph}{\kappa_\aleph}=f_{\aleph,0}\dfrac{a_\aleph+2}{2\sqrt{\phi_\aleph}}\left(a_\aleph-2\sqrt{\phi_\aleph}+1\right),\\
\ddfrac{\bar{f}_\aleph}{\kappa_\aleph}=f_{\aleph,0}\dfrac{a_\aleph+2}{2\sqrt{\phi_\aleph}}\dfrac{\left(a_\aleph+1+\left(\dfrac{c_\aleph}{b_\aleph}-2\right)\sqrt{\phi_\aleph}\right)}{\Phi_\aleph^{c_\aleph/b_\aleph}}.
\end{gather}

Given that the stress update is computed as follows,
\begin{gather}
\bsigma_{n+1}=h\bbsigma_{n+1},
\end{gather}
the consistent tangent stiffness is then
\begin{gather}
\begin{split}
\ddfrac{\bsigma_{n+1}}{\bvarepsilon_{n+1}}&=\bbsigma_{n+1}\otimes\ddfrac{h}{\bvarepsilon_{n+1}}+h\ddfrac{\bbsigma_{n+1}}{\bvarepsilon_{n+1}}\\
&=\bbsigma_{n+1}\otimes\left(\left(d_c-1\right)\left(s\ddfrac{d_t}{\bvarepsilon_{n+1}}+d_t\ddfrac{s}{\bvarepsilon_{n+1}}\right)+\left(sd_t-1\right)\ddfrac{d_c}{\bvarepsilon_{n+1}}\right)+h\ddfrac{\bbsigma_{n+1}}{\bvarepsilon_{n+1}},
\end{split}
\end{gather}
which is equivalently,
\begin{multline}\label{eq:cdp_pd}
\ddfrac{\bsigma_{n+1}}{\bvarepsilon_{n+1}}=\bbsigma_{n+1}\otimes\underbrace{\left(\begin{bmatrix}
s\left(d_c-1\right)\ddfrac{d_t}{\kappa_t}&
\left(sd_t-1\right)\ddfrac{d_c}{\kappa_c}
\end{bmatrix}
\begin{bmatrix}
\ddfrac{\kappa_t}{\bvarepsilon_{n+1}}\\[4mm]
\ddfrac{\kappa_c}{\bvarepsilon_{n+1}}
\end{bmatrix}\right)}_\text{dot product in vector representation}\\+\left(d_t\left(d_c-1\right)\left(1-s_0\right)\bbsigma_{n+1}\otimes\ddfrac{r}{\bbsigma_{n+1}}+h\mathbb{I}\right)\ddfrac{\bbsigma_{n+1}}{\bvarepsilon_{n+1}}.
\end{multline}

The CDP model has no kinematic hardening, the effective part resembles the bounding surface concept. Due to the coaxiality, the model can be constructed in the eigen space, which brings some convenience in terms of implementation.
\subsection{Implementation}
The state determination of the CDP model is shown in \algoref{algo:cdp_model}. Compared with the original implementation, the presented one is much more concise and is able to avoid lengthy computation of consistent tangent stiffness.

It is worth noting some quantities remain constant in the local iteration, for example,
\begin{gather}
2G\hat{n}_1+3K\alpha_p,\qquad
\dfrac{\hat{n}_1+\alpha_p}{g_t},\qquad
\dfrac{\hat{n}_3+\alpha_p}{g_c}.
\end{gather}
They can be computed as stored before entering local iteration.
\begin{breakablealgorithm}
\caption{state determination of the CDP model}\label{algo:cdp_model}
\begin{algorithmic}
\State \textbf{Parameter}: $\lambda$, $G$
\State \textbf{Input}: $\bvarepsilon_{n+1}$, $\bvarepsilon_n$, $\bvarepsilon^p_n$, $\bsigma_n$, $\kappa_{\aleph,n}$
\State \textbf{Output}: $\mb{D}_{n+1}$, $\bvarepsilon^p_{n+1}$, $\bsigma_{n+1}$, $\kappa_{\aleph,n+1}$
\State $\bbsigma^\text{trial}=\mb{D}:\left(\bvarepsilon_{n+1}-\bvarepsilon^p_n\right)$
\State $\bs^\text{trial}=\dev{\bbsigma^\text{trial}}$
\State perform eigenanalysis on $\bbsigma^\text{trial}$ and compute $\mb{T}$, $\bar{\bs}^\text{trial}$
\State $\hat{\bn}=\dfrac{\bar{\bs}^\text{trial}}{\norm{\bar{\bs}^\text{trial}}}$
\State compute $f^\text{trial}$\Comment{\eqsref{eq:cdp_tf}}
\State $\kappa_{\aleph,n+1}=\kappa_{\aleph,n}$
\If {$f^\text{trial}\geqslant0$}\Comment{plasticity evolution}
\State compute $f_\aleph$, $\bar{f}_\aleph$ and $d_\aleph$ and their derivatives\Comment{This can be an independent overridable method.}
\While{true}
\State $\gamma=0$
\State compute $\mb{R}$ and $\pdfrac{\mb{R}}{\mb{x}}$\Comment{\eqsref{eq:cdp_r} and \eqsref{eq:cdp_dr}}
\State $\Delta=\left(\pdfrac{\mb{R}}{\mb{x}}\right)^{-1}\mb{R}$\Comment{$\Delta=\begin{bmatrix}
\delta\gamma&\delta\kappa_t&\delta\kappa_c
\end{bmatrix}$}
\If {$\norm{\Delta}<\text{tolerance}$}
\State break
\EndIf
\State $\gamma\leftarrow\gamma-\delta\gamma$
\State $\kappa_{t,n+1}\leftarrow\kappa_{t,n+1}-\delta\kappa_t$
\State $\kappa_{t,n+1}\leftarrow\kappa_{t,n+1}-\delta\kappa_c$
\EndWhile
\State $\bsigma_{n+1}=\bsigma^\text{trial}-\gamma\left(2G\bn+3K\alpha_p\mb{1}\right)$
\State $\bvarepsilon^p_{n+1}=\bvarepsilon^p_n+\gamma\left(\bn+\alpha_p\mb{1}\right)$
\State compute $\mb{D}_{n+1}$\Comment{\eqsref{eq:cdp_pd}}
\Else\Comment{elastic loading/unloading}
\State $\bsigma_{n+1}=\bsigma^\text{trial}$
\State $\bvarepsilon^p_{n+1}=\bvarepsilon^p_n$
\State compute $\mb{D}_{n+1}$\Comment{\eqsref{eq:cdp_ed}}
\EndIf
\end{algorithmic}
\end{breakablealgorithm}

The CPP implementation of state determination can be found as follows.
\begin{cppcode}
NonlinearCDP::update_trial_status
\end{cppcode}
\section{CDPM2 Model}
The CDP adopts an isotropic damage, which leads to, for example, degradation of compressive/tensile strength due to tensile/compressive damage. In cyclic loading cases, it may not be ideal. The CDPM2 model uses a different approach that applies tensile damage to tensile part of stress and compressive damage to compressive part of stress.

Compared to the original formulation \cite{Grassl2013}, here the dependency of the Lode angle is removed for brevity.
\subsection{Plasticity}
The model is driven by the norm of the deviatoric stress $s=\norm{\mb{s}}$, the hydrostatic stress $p$ and the internal  hardening variable $\kappa_p$.
\subsubsection{Yield Function}
The yield function $F$ is defined as
\begin{gather}
F=g_1^2+m_0q^2_{h1}q_{h2}g_3-q^2_{h1}q^2_{h2},
\end{gather}
where the friction parameter $m_0$
\begin{gather}
m_0=\dfrac{3f_c^2-3f_t^2}{f_cf_t}\dfrac{e}{1+e}
\end{gather}
depends on material strengths $f_c$ and $f_t$, $e$ is the eccentricity constant which depends on the previous two strengths and equibiaxial compression strength $f_{bc}$. The hardening functions $q_{h1}$ and $q_{h2}$ will be introduced later. The helper functions $g_1$ and $g_3$ are also used to define the plastic potential $G$. If the Lode angle shall be considered, $g_3$ needs to be replaced by $g_4$ which is defined as
\begin{gather}
g_4=\dfrac{s}{\sqrt{6}f_c}r+\dfrac{p}{f_c},
\end{gather}
with $r=r\left(\theta\right)$ being a function of the lode angle $\theta$.

The partial derivatives of $F$ are
\begin{gather}
\pdfrac{F}{p}=2g_1\pdfrac{g_1}{p}+m_0q^2_{h1}q_{h2}\pdfrac{g_3}{p},\\
\pdfrac{F}{s}=2g_1\pdfrac{g_1}{s}+m_0q^2_{h1}q_{h2}\pdfrac{g_3}{s},\\
\pdfrac{F}{\kappa_p}=2g_1\pdfrac{g_1}{\kappa_p}+2m_0q_{h1}q_{h2}g_3\ddfrac{q_{h1}}{\kappa_p}+m_0q^2_{h1}g_3\ddfrac{q_{h2}}{\kappa_p}-2q_{h1}q_{h2}\left(q_{h2}\ddfrac{q_{h1}}{\kappa_p}+q_{h1}\ddfrac{q_{h2}}{\kappa_p}\right).
\end{gather}
\subsubsection{Flow Rule}
The plastic potential $G$ is defined as
\begin{gather}
G=g_1^2+q^2_{h1}g_2,
\end{gather}
where
\begin{gather}
g_1=\left(1-q_{h1}\right)g_3^2+\sqrt{\dfrac{3}{2}}\dfrac{s}{f_c},\qquad
g_2=\dfrac{m_0s}{\sqrt{6}f_c}+\dfrac{m_g}{f_c},\qquad
g_3=\dfrac{s}{\sqrt{6}f_c}+\dfrac{p}{f_c},\\
m_g=A_gB_gf_c\exp\left(\dfrac{3p-q_{h2}f_t}{3B_gf_c}\right),\\
A_g=\dfrac{3f_tq_{h2}}{f_c}+\dfrac{m_0}{2},\\
B_g=\dfrac{q_{h2}\left(1+f_t/f_c\right)/3}{\ln{}A_g-\ln\left(2D_f-1\right)-\ln\left(3q_{h2}+m_0/2\right)+\ln\left(D_f+1\right)}.
\end{gather}
With the derivatives of auxiliary functions expressed as
\begin{gather}
\pdfrac{g_3}{p}=\dfrac{1}{f_c},\qquad\pdfrac{g_3}{s}=\dfrac{1}{\sqrt{6}f_c},\qquad
\pdfrac{g_2}{p}=\dfrac{1}{f_c}\pdfrac{m_g}{p},\qquad\pdfrac{g_2}{s}=\dfrac{m_0}{\sqrt{6}f_c},\\
\pdfrac{g_1}{p}=2\left(1-q_{h1}\right)g_3\pdfrac{g_3}{p},\qquad\pdfrac{g_1}{s}=2\left(1-q_{h1}\right)g_3\pdfrac{g_3}{s}+\sqrt{\dfrac{3}{2}}\dfrac{1}{f_c}.
\end{gather}
The flow rule can be derived as
\begin{gather}
G_p=\pdfrac{G}{p}=2g_1\pdfrac{g_1}{p}+q^2_{h1}\pdfrac{g_2}{p}=\dfrac{4\left(1-q_{h1}\right)g_1g_3+q^2_{h1}A_g\exp\left(\dfrac{p-q_{h2}f_t/3}{B_gf_c}\right)}{f_c},\\
G_s=\pdfrac{G}{s}=2g_1\pdfrac{g_1}{s}+q^2_{h1}\pdfrac{g_2}{s}=\dfrac{4\left(1-q_{h1}\right)g_1g_3+6g_1+m_0q^2_{h1}}{\sqrt{6}f_c}.
\end{gather}
\subsubsection{Hardening Law}
The variables $q_{h1}$ and $q_{h2}$ are functions of the hardening variable $\kappa_p$.
\begin{gather}
q_{h1}=\left\{
\begin{array}{ll}
q_{h0}+\left(1-q_{h0}\right)\left(\kappa_p^3-3\kappa_p^2+3\kappa_p\right)-H_p\left(\kappa_p^3-3\kappa_p^2+2\kappa_p\right)&\text{for~}\kappa_p<1,\\
1&\text{for~}\kappa_p\geqslant1.
\end{array}
\right.\\
q_{h2}=\left\{
\begin{array}{ll}
1&\text{for~}\kappa_p<1,\\
1+H_p\left(\kappa_p-1\right)&\text{for~}\kappa_p\geqslant1.
\end{array}
\right.
\end{gather}

The evolution of $\kappa_p$ is defined as
\begin{gather}
\dot{\kappa_p}=\dfrac{\norm{\dot{\varepsilon_{p}}}}{x_h}4\cos^2\left(\theta\right).
\end{gather}
Here the dependency on the Lode angle $\theta$ is removed such that
\begin{gather}
x_h\dot{\kappa_p}=\norm{\dot{\varepsilon_{p}}}=\gamma{}G_\kappa,
\end{gather}
with $G_\kappa=\sqrt{G_s^2+G_p^2/3}$. This effectively means $\theta=2\pi/3$ is a constant.
\subsubsection{Plasticity Residual}
Collecting the yield function, evolutions of $s$, $p$ and $\kappa_p$, the local residual of the plasticity part can be expressed as
\begin{gather}
\mb{R}=\left\{
\begin{array}{l}
g_1^2+m_0q^2_{h1}q_{h2}g_3-q^2_{h1}q^2_{h2},\\[4mm]
s+2G\gamma{}G_s-s^\text{trial},\\[4mm]
p+K\gamma{}G_p-p^\text{trial},\\[4mm]
x_h\kappa_p^n+\gamma{}G_\kappa-x_h\kappa_p.
\end{array}
\right.
\end{gather}
The local system consists of four scalar equations.

The Jacobian thus has a size of \num{4} and reads
\begin{gather}
\mb{J}=\begin{bmatrix}
\cdot&\pdfrac{F}{s}&\pdfrac{F}{p}&\pdfrac{F}{\kappa_p}\\[6mm]
2GG_s&1+2G\gamma\pdfrac{G_s}{s}&2G\gamma\pdfrac{G_s}{p}&2G\gamma\pdfrac{G_s}{\kappa_p}\\[6mm]
KG_p&K\gamma\pdfrac{G_p}{s}&1+K\gamma\pdfrac{G_p}{p}&K\gamma\pdfrac{G_p}{\kappa_p}\\[6mm]
G_\kappa&\gamma\pdfrac{G_\kappa}{s}&\left(\kappa_p^n-\kappa_p\right)\ddfrac{x_h}{p}+\gamma\pdfrac{G_\kappa}{p}&\gamma\pdfrac{G_\kappa}{\kappa_p}-x_h
\end{bmatrix}.
\end{gather}
\subsection{Damage}
\subsubsection{Equivalent Strain}
The equivalent strain is defined as
\begin{gather}
\tilde{\varepsilon}=g_5+\sqrt{g_5^2+g_6^2},
\end{gather}
with
\begin{gather}
g_5=\dfrac{\varepsilon_0m_0}{2}g_3,\qquad
g_6=\sqrt{\dfrac{3}{2}}\dfrac{s}{f_c}.
\end{gather}
Again, if the Lode angle needs to be considered, replace $g_3$ with $g_4$. The equivalent strain is further split into tensile and compressive parts.
\begin{gather}
\tilde{\varepsilon}_t=\alpha_t\tilde{\varepsilon},\qquad
\tilde{\varepsilon}_c=\alpha_c\tilde{\varepsilon},
\end{gather}
where $\alpha_t+\alpha_c=1$ are two parameters that characterise the tensile/compressive portion of the current loading step.

The damage history variables $\kappa_{dt}$ and $\kappa_{dc}$ track the maximum values of $\tilde{\varepsilon}_t$ and $\tilde{\varepsilon}_c$, respectively.
\subsubsection{Tension}
The inelastic strain is expressed as a function of damage factors.
\begin{gather}
\varepsilon_i=\kappa_{dt1}+\omega_t\kappa_{dt2}.
\end{gather}
While the uniaxial stress response can be expressed as
\begin{gather}
\sigma=\left(1-\omega_t\right)E\kappa_{dt}.
\end{gather}
Assume an exponential degradation curve,
\begin{gather}
\sigma=f_t\exp\left(-\dfrac{\varepsilon_i}{\varepsilon_{ft}}\right)
\end{gather}

The local residual is
\begin{gather}
R=f_t\exp\left(-\dfrac{\varepsilon_i}{\varepsilon_{ft}}\right)-\left(1-\omega_t\right)E\kappa_{dt}.
\end{gather}
For a given set of $\kappa_{dt}$, $\kappa_{dt1}$ and $\kappa_{dt2}$, the above residual is a function of unknown $\omega_t$. It can be solved by, for example, Newton's method.
The corresponding derivatives are
\begin{gather}
\pdfrac{R}{\omega_t}=E\kappa_{dt}-\dfrac{\kappa_{dt2}}{\varepsilon_{fc}}f_t\exp\left(-\dfrac{\varepsilon_i}{\varepsilon_{ft}}\right),\\
\pdfrac{R}{\kappa_{dt}}=-\left(1-\omega_t\right)E,\\
\pdfrac{R}{\kappa_{dt1}}=-\dfrac{1}{\varepsilon_{ft}}f_t\exp\left(-\dfrac{\varepsilon_i}{\varepsilon_{ft}}\right),\\
\pdfrac{R}{\kappa_{dt2}}=-\dfrac{\omega_t}{\varepsilon_{ft}}f_t\exp\left(-\dfrac{\varepsilon_i}{\varepsilon_{ft}}\right).
\end{gather}
\subsubsection{Compression}
For compression, the definition fully resembles its tensile counterpart. The inelastic strain is driven by two internal parameters $\kappa_{dc1}$ and $\kappa_{dc2}$,
\begin{gather}
\varepsilon_i=\kappa_{dc1}+\omega_c\kappa_{dc2}.
\end{gather}
The uniaxial response is
\begin{gather}
\sigma=\left(1-\omega_c\right)E\kappa_{dc}.
\end{gather}
With the exponential backbone, the local residual is
\begin{gather}
R=f_t\exp\left(-\dfrac{\varepsilon_i}{\varepsilon_{fc}}\right)-\left(1-\omega_c\right)E\kappa_{dc}.
\end{gather}
\subsubsection{Stress}
The final stress can be expressed as
\begin{gather}
\bsigma=\left(1-\omega_t\right)\left(1-\omega_c\right)\bbsigma.
\end{gather}
This form is identical to the one used in the CDP model. It represents the isotropic damage.
Alternatively, it can also be expressed as
\begin{gather}\label{eq:aniso_damage}
\bsigma=\left(1-\omega_t\right)\bbsigma_t+\left(1-\omega_c\right)\bbsigma_c,
\end{gather}
which stands for the anisotropic damage. The tensile and compressive part of the stress tensor $\bbsigma_t$ and $\bbsigma_c$ are obtained by performing an eigenanalysis such that
\begin{gather}
\bbsigma_t=\sum_{i=1}^3\left\langle\hat{\sigma}_i\right\rangle\mb{p}_i\otimes\mb{p}_i,\qquad\bbsigma_c=\bbsigma-\bbsigma_t,
\end{gather}
where $\hat{\sigma}_i$ is the eigenvalue of $\bbsigma$ and $\mb{p}_i$ is the associated eigenvector.

\eqsref{eq:aniso_damage} can be equivalently expressed as
\begin{gather}
\begin{split}
\bsigma&=\left(1-\omega_t\right)\bbsigma_t+\left(1-\omega_c\right)\left(\bbsigma-\bbsigma_t\right)\\
&=\left(1-\omega_c\right)\bbsigma+\left(\omega_c-\omega_t\right)\bbsigma_t.
\end{split}
\end{gather}
With the above expression, the tangent stiffness can be expressed as
\begin{gather}
\ddfrac{\bsigma}{\bvarepsilon}=\left(1-\omega_c\right)\ddfrac{\bbsigma}{\bvarepsilon}+\left(\omega_c-\omega_t\right)\ddfrac{\bbsigma_t}{\bbsigma}\ddfrac{\bbsigma}{\bvarepsilon}-\bbsigma\otimes\ddfrac{\omega_c}{\bvarepsilon}+\bbsigma_t\otimes\left(\ddfrac{\omega_c}{\bvarepsilon}-\ddfrac{\omega_t}{\bvarepsilon}\right).
\end{gather}
\subsection{Formulation}
\subsection{Implementation}
The main body of state determination consists of two main tasks: 1) compute the plasticity part and 2) compute the damage part. The implementation can be found as follows.
\begin{cppcode}
CDPM2::update_trial_status
\end{cppcode}

Here, all relevant quantities (mainly derivatives) are stored in array \texttt{data} that is passed to the corresponding functions. For example, in order to compute plasticity, the following method is used. It is lengthy and tedious, mainly computing derivatives via the chain rule.
\begin{cppcode}
CDPM2::compute_plasticity
\end{cppcode}

The converged \texttt{data} is then passed to the method to compute damage factors. It is similar to the plasticity part.
\begin{cppcode}
CDPM2::compute_damage
\end{cppcode}

Finally, the damage part involves a local iteration procedure.
\begin{cppcode}
CDPM2::compute_damage_factor
\end{cppcode}
