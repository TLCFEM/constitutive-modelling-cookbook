\chapter{Miscellaneous Topics}
In this chapter, we discuss some miscellaneous topics that are relevant to numerical implemenntations of consitutive models.
\section{Equivalent But Preferred?}
Some (yield) functions are mathematically equivalent, is there a preferred one?
Or any form can be used?
Consider for example, we want to define a yield function that enforces the magnitude of stress $\sigma$ to be less than a yield stress $\sigma_y\geqslant0$, one can write
\begin{gather}
    f=\abs{\sigma}-\sigma_y,
\end{gather}
or alternatively, one can write
\begin{gather}
    f=\sigma^2-\sigma_y^2.
\end{gather}
They are indeed mathematically equivalent as we have defined the yield stress to be non-negative.
However, they may have different implications in terms of numerical implementation.
First, the involvement of squares is discouraged.
It easily leads to precision issues, especially when the local nonlinear system also contains some nondimentionsal history variables.
Second, if associative flow is used, the above two yield functions lead to
\begin{gather}
    \dot{\varepsilon^p}=\gamma\pdfrac{f}{\sigma}=\left\{\begin{array}{ll}
        \gamma\sign{\sigma}, & \text{for~}f=\abs{\sigma}-\sigma_y, \\[4mm]
        \gamma2\sigma,       & \text{for~}f=\sigma^2-\sigma_y^2.
    \end{array}\right.
\end{gather}
It is easy to see that in the first case, $\abs{\dot{\varepsilon^p}}=\gamma$, meaning that $\gamma<\abs{\dot{\varepsilon}}$ is bounded as the increment of plastic strain can never exceed the increment of total strain.
However, in the second case, it is not easy to determine the upper bound of $\gamma$.
Since the whole plasticity development is driven by the plastic multiplier $\gamma$, it is possible to use root finding algorithms to solve for $\gamma$ when an explicit bound can be found.
This is particularly useful when the local Newton--Raphson iteration fails to converge.

For numerical robustness, it is preferred to use the first form of yield function in this case.
Similar arguments can be made for other more complex formulations.

Can you analyze the pros and cons of the third option shown below?
\begin{gather}
    f=\dfrac{\abs{\sigma}}{\sigma_y}-1.
\end{gather}