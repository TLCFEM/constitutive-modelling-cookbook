\section{Orthotropic Hoffman Model}\label{sec:hoffman}
Here we introduce an anisotropic model that adopts the orthotropic Hoffman yielding criterion.
This framework resembles the isotropic von Mises model.
As a matter of fact, the von Mises model can be seen as a degenerated version of the Hoffman model.
It can be used to model orthtropic materials such sheet steel and timber.
One can further refer to the corresponding section (\S~10.3) in \cite{SouzaNeto2008}.
\subsection{Theory}
\subsubsection{Yield Function}
The yield function adopts the Hoffman criterion, which is expressed in compressed matrix form,
\begin{gather}\label{eq:orthotropic_yield_function_proto}
    f=\dfrac{1}{2}\bsigma^\mT\mb{P}\bsigma+\mb{q}^\mT\bsigma-\sigma_y^2,
\end{gather}
where $\mb{P}=\mb{P}^\mT$ and $\bq$ are constant scaling matrix/vector.
Various other orthotropic yielding criteria can also fit in this form, see \cite{Oller2003} for details.
Apparently, $\mb{P}$ and $\bq$ represent two tensors of order four and two, respectively.
But since \cite{Oller2003} uses matrix notation, we shall follow this convention.
For example, the Hoffman criterion can be expressed as
\begin{tiny}
    \begin{gather}\label{eq:hoffman_criterion}
        \dfrac{1}{2}\mb{P}=\begin{bmatrix}
            T_1                     & -\dfrac{T_1+T_2-T_3}{2} & -\dfrac{T_3+T_1-T_2}{2} &                     &                     &                     \\[4mm]
            -\dfrac{T_1+T_2-T_3}{2} & T_2                     & -\dfrac{T_2+T_3-T_1}{2} &                     &                     &                     \\[4mm]
            -\dfrac{T_3+T_1-T_2}{2} & -\dfrac{T_2+T_3-T_1}{2} & T_3                     &                     &                     &                     \\[4mm]
                                    &                         &                         & \dfrac{1}{f_{12}^2} &                     &                     \\[4mm]
                                    &                         &                         &                     & \dfrac{1}{f_{23}^2} &                     \\[4mm]
                                    &                         &                         &                     &                     & \dfrac{1}{f_{13}^2}
        \end{bmatrix},\quad
        \mb{q}=\begin{bmatrix}
            \left(f_{11}^c-f_{11}^t\right)T_1 \\[4mm]
            \left(f_{22}^c-f_{22}^t\right)T_2 \\[4mm]
            \left(f_{33}^c-f_{33}^t\right)T_3 \\[4mm]
            0                                 \\[4mm]
            0                                 \\[4mm]
            0
        \end{bmatrix},
    \end{gather}
\end{tiny}
in which,
\begin{gather}
    T_1=\dfrac{1}{f_{11}^tf_{11}^c},\qquad
    T_2=\dfrac{1}{f_{22}^tf_{22}^c},\qquad
    T_3=\dfrac{1}{f_{33}^tf_{33}^c},
\end{gather}
with $f_{ij}^\aleph$ representing the yielding stress along different directions.

Unlike other models introduced in this book, because the yield function is expressed in matrix form, all the following are thus mainly based in matrix notation.
An equivalent tensor notation may also exist, but as it is not used in literature, we shall not introduce it here.
\subsubsection{Flow Rule}
The associated plasticity is assumed such that the plastic potential $g$ is identical to $f$. The plastic flow direction is then
\begin{gather}
    \bn=\pdfrac{g}{\bsigma}=\pdfrac{f}{\bsigma}=\mb{P}\bsigma+\mb{q}.
\end{gather}
The flow rule can be defined as
\begin{gather}
    \dot{\bvarepsilon^p}=\gamma\bn.
\end{gather}
It is worth noting that since the yield function is expressed as a function of the stress (Voigt) vector rather than the stress tensor, the partial derivative $\mb{P}\bsigma+\mb{q}$ is the derivative with regard to the vector $\bsigma$, which already contains the scaling factor for shear terms.
This is different from other models that use the tensor notation.
\subsubsection{Hardening Law}
The reference yield stress $\sigma_y$ is defined as a function of the accumulated equivalent plastic strain $\varepsilon_{p}$.
\begin{gather}
    \sigma_y=H\left(\varepsilon^{p}\right).
\end{gather}
It must be noted that here it is called the `reference' yield stress --- there is no unique uniaxial equivalence as along different directions the yield stress may vary.

The evolution of $\varepsilon^{p}$ is defined by
\begin{gather}
    \dot{\varepsilon^p}=\sqrt{\dfrac{2}{3}\dot{\bvarepsilon^p}:\dot{\bvarepsilon^p}}=\gamma\sqrt{\dfrac{2}{3}\bn:\bn},
\end{gather}
where $\bn:\bn$ shall be treated as double contraction of strain tensors, that is,
\begin{gather}
    \underbrace{\bn:\bn}_\text{tensor}=\underbrace{\bn^\mT\mb{T}\bn}_\text{vector},
\end{gather}
with $\mb{T}=\diag{\begin{matrix}
            1 & 1 & 1 & \dfrac{1}{2} & \dfrac{1}{2} & \dfrac{1}{2}
        \end{matrix}}$.
This is due to the fact that \eqsref{eq:orthotropic_yield_function_proto} is expressed in vectors/matrices rather than tensors!

Although \textbf{not} correct, we'll use $\norm{\bn}_e$ to represent $\sqrt{\dfrac{2}{3}\bn:\bn}$ for brevity in the following.
\subsection{Formulation}
To some extend, the model is even simpler than the von Mises model as there is no back stress to support kinematic hardening.
This is reasonable as there is no experimental evidence that timber exhibits the Bauschinger effect.
For other materials, specific models can be developed, in which kinematic hardening can be introduced.
Furthermore, the yield function involves only matrix--vector operations, the corresponding derivatives are relatively easy to compute.
\begin{table}[ht]
    \centering
    \begin{tabular}{rl}
        \toprule
        Constitutive Law & $\bsigma=\mb{E}\left(\bvarepsilon-\bvarepsilon^p\right)$              \\
        Yield Function   & $f=\dfrac{1}{2}\bsigma^\mT\mb{P}\bsigma+\mb{q}^\mT\bsigma-\sigma_y^2$ \\
        Flow Rule        & $\dot{\bvarepsilon^p}=\gamma\bn$                                      \\
        Hardening Law    & $\dot{\varepsilon^p}=\gamma\norm{\bn}_e$                              \\\bottomrule
    \end{tabular}
\end{table}
\subsubsection{Elastic Loading/Unloading}
Compared with the von Mises framework, there is no essential difference in terms of elastic loading/unloading, the plasticity is frozen at the beginning of each substep, allowing one to compute the trial stress such that
\begin{gather}
    \bsigma^\text{trial}=\mb{E}\left(\bvarepsilon_{n+1}-\bvarepsilon^p_n\right)\xcancel{=\bsigma_n+\mb{E}\left(\bvarepsilon_{n+1}-\bvarepsilon_n\right)}.
\end{gather}
One shall note that the second form is not used here.
In most cases, both forms are equivalent.
However, some models may further apply damage mechanics to the result of plasticity, making the second form incorrect (as $\bsigma_n$ may contain damage reductions).
Here we demonstrate the usage of the first form.

The trial yield function can then be computed using the unchanged plastic strain.
\begin{gather}
    f^\text{trial}=\dfrac{1}{2}\bsigma^\text{trial,T}\mb{P}\bsigma^\text{trial}+\mb{q}^\mT\bsigma^\text{trial}-\sigma_{y,n}^2.
\end{gather}
\subsubsection{Plastic Evolution}
The discretised evolution of plastic strain is written as
\begin{gather}
    \bvarepsilon^p_{n+1}=\bvarepsilon^p_{n}+\Delta\bvarepsilon^p=\bvarepsilon^p_{n}+\gamma\bn_{m}.
\end{gather}
Note that we do not directly use $\bn_{n+1}$ here, instead, we use $\bn_{m}$ to allow multiple different implementations.
In specific, if $\bn_{m}=\bn_{n+1}$, it is effectively the implicit Euler method.
\subsubsection{Local Residual}
The residual is chosen as follows. For brevity, all subscripts $\left(\cdot\right)_{n+1}$ are dropped.
\begin{gather}
    \mb{R}=\left\{
    \begin{array}{l}
        \dfrac{1}{2}\bsigma^\mT\mb{P}\bsigma+\mb{q}^\mT\bsigma-\sigma_y^2, \\[4mm]
        \bsigma+\gamma\mb{E}\bn_m-\bsigma^\text{trial}.
    \end{array}
    \right.
\end{gather}
We use $\mb{E}$ to denote the elasticity tensor $\mb{D}$ expressed in matrix form.
They represent the same quantity.

By choosing $\mb{x}=\begin{bmatrix}
        \gamma & \bsigma
    \end{bmatrix}^\mT$ as the independent variables, the Jacobian can be then computed as
\begin{gather}
    \mb{J}=\pdfrac{\mb{R}}{\mb{x}}=\begin{bmatrix}
        -2\sigma_y\ddfrac{\sigma_y}{\varepsilon_{p}}\norm{\bn_m}_e & \bn^\mT-2\sigma_y\ddfrac{\sigma_y}{\varepsilon_{p}}\gamma\ddfrac{\norm{\bn_m}_e}{\bsigma} \\[4mm]
        \mb{E}\bn_m                                                & \mb{I}+\gamma\mb{E}\ddfrac{\bn_m}{\bsigma}
    \end{bmatrix}.
\end{gather}

Some references would further derive a scalar local residual at the cost of complicating gradient. Here we choose to increase the size of local system in order to express the Jacobian in a simpler form. Performance wise, a scalar local residual does not necessarily leads to faster state determination.
\subsubsection{Consistent Tangent Stiffness}
The consistent tangent stiffness can be directly computed from the local residual, given that $\bsigma_{n+1}$ is chosen as the variable. Differentiating $\mb{R}$ at equilibrium/convergence $\mb{R}=\mb{0}$ gives
\begin{gather}\label{eq:universal_consistent_tangent}
    \pdfrac{\mb{R}}{\mb{x}}\pdfrac{\mb{x}}{\bvarepsilon_{n+1}}+\pdfrac{\mb{R}}{\bvarepsilon_{n+1}}=\mb{0},
\end{gather}
rearranging which gives
\begin{gather}
    \pdfrac{\mb{x}}{\bvarepsilon_{n+1}}=-\left(\pdfrac{\mb{R}}{\mb{x}}\right)^{-1}\pdfrac{\mb{R}}{\bvarepsilon_{n+1}}=-\mb{J}^{-1}\pdfrac{\mb{R}}{\bvarepsilon_{n+1}}.
\end{gather}

Since the left hand side $\pdfrac{\mb{x}}{\bvarepsilon_{n+1}}$ contains
\begin{gather}
    \pdfrac{\mb{x}}{\bvarepsilon_{n+1}}=\begin{bmatrix}
        \pdfrac{\gamma}{\bvarepsilon_{n+1}} \\[4mm]
        \pdfrac{\bsigma_{n+1}}{\bvarepsilon_{n+1}}
    \end{bmatrix},
\end{gather}
then,
\begin{gather}\label{eq:ortho_consistent_tangent}
    \pdfrac{\bsigma_{n+1}}{\bvarepsilon_{n+1}}=-\left(\mb{J}^{-1}\pdfrac{\mb{R}}{\bvarepsilon_{n+1}}\right)^{\langle2-7\rangle},
\end{gather}
where $\left(\cdot\right)^{\langle2-7\rangle}$ denotes the second to the seventh row of target quantity $\left(\cdot\right)$.

Unlike the von Mises framework, in which the analytical expression for the consistent tangent stiffness matrix is derived. Here we take advantage of the fact that when the local equilibrium is achieved, $\mb{R}=\mb{0}$ or at least $\mb{R}\approx\mb{0}$, allowing one to take full differentiation to obtain some useful quantities that otherwise may be difficult to compute. If $\bsigma_{n+1}$ is directly involved as one of the independent variables in local iteration, the consistent stiffness can be directly obtained. Otherwise, often additional simple steps (chain rule) shall be applied to the stress update formula to compute $\pdfrac{\bsigma_{n+1}}{\bvarepsilon_{n+1}}$.

This method avoid the computation of lengthy, cumbersome analytical expressions of consistent tangent. In most cases, it is also very simple to implement as $\mb{J}$ is already available when the local iteration converges, and $\pdfrac{\mb{R}}{\bvarepsilon_{n+1}}$ often is very easy to compute. Readers shall try to grasp the beauty of \eqsref{eq:universal_consistent_tangent}, as this method will be frequently used in the models introduced later in this book.
\subsection{Implementation}
\subsubsection{Trapezoidal Integration}
In this particular model, we demonstrate the less common trapezoidal rule.
By the trapezoidal rule, we mean that the update of the plastic strain shall be expressed as
\begin{gather}
    \bvarepsilon^p_{n+1}=\bvarepsilon^p_{n}+\gamma\bn_{m},
\end{gather}
as presented earlier.
In a conventional context, it can be further expressed as
\begin{gather}
    \bvarepsilon^p_{n+1}=\bvarepsilon^p_{n}+\gamma\dfrac{\bn_{n}+\bn_{n+1}}{2},\qquad\text{(WRONG!)}
\end{gather}
such that $\bn_{n}$ and $\bn_{n+1}$ are the flow directions at the beginning and the end of the step, respectively.
However, one must notice that the current state $\left(\cdot\right)_n$ may be elastic!
The above expression is correct only when the current state is on the yield surface.
To be precise, as we know that the new state $\left(\cdot\right)_{n+1}$ must be plastic, the above expression has to be modified as
\begin{gather}
    \bvarepsilon^p_{n+1}=\bvarepsilon^p_{n}+\gamma\dfrac{\bn_\text{onset}+\bn_{n+1}}{2},
\end{gather}
where $\bn_\text{onset}$ is the flow direction at the onset of plasticity in the current step.
It thus implies that
\begin{gather}
    \bn_{m}=\dfrac{\bn_\text{onset}+\bn_{n+1}}{2}.
\end{gather}

Assuming proportional loading, finding $\bn_\text{onset}$ is equivalent to finding a ratio $r$ such that the onset stress $\bsigma_\text{onset}$, which is defined as
\begin{gather}
    \bsigma_\text{onset}=\mb{E}\left(\bvarepsilon_n+r\Delta\bvarepsilon-\bvarepsilon_n^p\right),
\end{gather}
is on the yield surface, that is,
\begin{gather}
    f_\text{onset}=f\left(\bsigma_\text{onset}\right)=\dfrac{1}{2}\bsigma_\text{onset}^\mT\mb{P}\bsigma_\text{onset}+\mb{q}^\mT\bsigma_\text{onset}-\sigma_{y,n}^2=0.
\end{gather}
Note in this procedure, plasticity is frozen, the only variable is the ratio $r$.
This is a well bounded problem --- $r$ is guaranteed to be greater than zero and less than one.
The solution $r$ can be computed using a root finding method.
Alternatively, noting that $f_\text{onset}$ is a quadratic function of $r$, one can compute the roots of the equation directly.
Once $r$ is known, $\bsigma_\text{onset}$ can then be computed, as well as $\bn_\text{onset}$.

It is necessary to compute the derivative as well.
From the equilibrium of $f_\text{onset}$, one has
\begin{gather}
    \pdfrac{f_\text{onset}}{r}\md{r}+\pdfrac{f_\text{onset}}{\bvarepsilon_{n+1}}\md{\bvarepsilon_{n+1}}=0,
\end{gather}
thus,
\begin{gather}
    \ddfrac{r}{\bvarepsilon_{n+1}}=\dfrac{-1}{\pdfrac{f_\text{onset}}{r}}\pdfrac{f_\text{onset}}{\bvarepsilon_{n+1}},
\end{gather}
with
\begin{gather}
    \pdfrac{f_\text{onset}}{r}=\bn_\text{onset}^\mT\mb{E}\Delta\bvarepsilon,\qquad
    \pdfrac{f_\text{onset}}{\bvarepsilon_{n+1}}=\bn_\text{onset}^\mT\mb{E}r.
\end{gather}
With the above, one can further derive
\begin{gather}
    \ddfrac{\bn_\text{onset}}{\bvarepsilon_{n+1}}=\mb{PE}\left(\Delta\bvarepsilon\ddfrac{r}{\bvarepsilon_{n+1}}+r\mb{I}\right)=r\left(\mb{PE}-\mb{PE}\dfrac{\Delta\bvarepsilon\bn_\text{onset}^\mT\mb{E}}{\bn_\text{onset}^\mT\mb{E}\Delta\bvarepsilon}\right).
\end{gather}
A slight rearrangement gives
\begin{gather}
    \ddfrac{\bn_\text{onset}}{\bvarepsilon_{n+1}}=r\mb{P}\left(\mb{I}-\dfrac{\mb{E}\Delta\bvarepsilon\bn_\text{onset}^\mT}{\bn_\text{onset}^\mT\mb{E}\Delta\bvarepsilon}\right)\mb{E}=r\mb{P}\left(\mb{I}-\dfrac{\Delta\bsigma\bn_\text{onset}^\mT}{\bn_\text{onset}^\mT\Delta\bsigma}\right)\mb{E}.
\end{gather}
It is clear that when $r\rightarrow0$, the derivative approaches constant as $\ddfrac{\bn_\text{onset}}{\bvarepsilon_{n+1}}\rightarrow\mb{0}$.

With $\bn_\text{onset}$, it is possible to compute the explicit expression of $\pdfrac{\mb{R}}{\bvarepsilon_{n+1}}$ as
\begin{gather}
    \pdfrac{\mb{R}}{\bvarepsilon_{n+1}}=\begin{bmatrix}
        -2\sigma_y\ddfrac{\sigma_y}{\varepsilon_{p}}\gamma\ddfrac{\norm{\bn_m}_e}{\bvarepsilon_{n+1}} \\[4mm]
        \gamma\mb{E}\ddfrac{\bn_m}{\bvarepsilon_{n+1}}-\mb{E}
    \end{bmatrix}=\begin{bmatrix}
        -\sigma_y\ddfrac{\sigma_y}{\varepsilon_{p}}\gamma\ddfrac{\norm{\bn_m}_e}{\bn_m}r\mb{P}\left(\mb{I}-\dfrac{\Delta\bsigma\bn_\text{onset}^\mT}{\bn_\text{onset}^\mT\Delta\bsigma}\right)\mb{E} \\[4mm]
        \dfrac{\gamma}{2}\mb{E}r\mb{P}\left(\mb{I}-\dfrac{\Delta\bsigma\bn_\text{onset}^\mT}{\bn_\text{onset}^\mT\Delta\bsigma}\right)\mb{E}-\mb{E}
    \end{bmatrix}.
\end{gather}
Rearranging the above gives
\begin{gather}
    \pdfrac{\mb{R}}{\bvarepsilon_{n+1}}=\begin{bmatrix}
        -\sigma_yr\gamma\ddfrac{\sigma_y}{\varepsilon_{p}}\ddfrac{\norm{\bn_m}_e}{\bn_m}\mb{P}\left(\mb{I}-\dfrac{\Delta\bsigma\bn_\text{onset}^\mT}{\bn_\text{onset}^\mT\Delta\bsigma}\right) \\[4mm]
        \dfrac{r\gamma}{2}\mb{EP}\left(\mb{I}-\dfrac{\Delta\bsigma\bn_\text{onset}^\mT}{\bn_\text{onset}^\mT\Delta\bsigma}\right)-\mb{I}
    \end{bmatrix}\mb{E}.
\end{gather}
The above can be inserted into \eqsref{eq:ortho_consistent_tangent} to obtain the consistent tangent stiffness.
There is no tensor algebra involved, all expressions are in matrix form.
\subsubsection{Line Search}
Since it is an anisotropic model, the local iteration may have difficulties in convergence, especially when a high anisotropy is defined.
Some implementations \cite{Krasnovskiy2004,SouzaNeto2008,Scherzinger2017} adopt line search, which effectively becomes a damped Newton--Raphson method.
It indeed improves numerical stability in certain cases, but does not guarantee global convergence.
The additional cost of searching is not necessarily negligible, as a desired step size is not guaranteed to be found in a fixed number of searches.
What we end up with is a more costly algorithm that may not converge.

Instead of line search, noting that there must exist a solution for some $\gamma>0$, it is possible to use a bracketing method that guarantees a solution can be found for arbitrary specified accuracy.
The idea can be explained as follows.
For a given $\gamma$, it is possible to compute the stress $\bsigma$ via
\begin{gather}
    \bsigma=\bsigma^\text{trial}-\gamma\mb{E}\bn_m,
\end{gather}
thus
\begin{multline}
    \left(\mb{I}+\dfrac{\gamma}{2}\mb{E}\mb{P}\right)\bsigma=\bsigma^\text{trial}-\dfrac{\gamma}{2}\mb{E}\left(\bn_\text{onset}+\mb{q}\right) \\\longrightarrow\qquad
    \bsigma=\left(\mb{I}+\dfrac{\gamma}{2}\mb{E}\mb{P}\right)^{-1}\left(\bsigma^\text{trial}-\dfrac{\gamma}{2}\mb{E}\left(\bn_\text{onset}+\mb{q}\right)\right).
\end{multline}
Once $\bsigma$ is known, the yield function can be evaluated.
This effectively turns $f=f\left(\gamma\right)$ into a univariate function of $\gamma$, and noting that utilizing the above procedure, no iteration is required to evaluate $f$ for a given $\gamma$, and such an evaluation is precise.
Knowing that $f\left(0\right)>0$, to use a root finding method, one may start with a small $\gamma_b>0$ and compute $f\left(\gamma_b\right)$.
If $f\left(\gamma_b\right)>0$, double $\gamma_b$ by, for example, $\gamma_b\leftarrow2\gamma_b$ and continue evaluating $f\left(\gamma_b\right)$ until $f\left(\gamma_b\right)\leqslant0$.
This ensures that the solution shall be bracketed by $0$ and $\gamma_b$.
One can then use root finding methods to find the solution $\gamma$.
\subsubsection{Return Mapping Implementation}
The following is a reference implementation that switches between the implicit Euler method and the trapezoidal method based on the size of strain increment.
The criterion used to determine which method to use is purely heuristic, and it is not guaranteed to be optimal.
The general idea is to use the second order method when the strain increment is small so that the asymptotic convergence can be retained.
\begin{cppcode}
    NonlinearOrthotropic::update_trial_status
\end{cppcode}

The two integration methods follow a similar structure.
Note the computation of the onset state.
\begin{cppcode}
    NonlinearOrthotropic::trapezoidal_return
\end{cppcode}

The Euler method is simpler.
\begin{cppcode}
    NonlinearOrthotropic::euler_return
\end{cppcode}
