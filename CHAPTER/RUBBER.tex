\chapter{Rubber}
\section{Basic Quantities}
The Green--Lagrangian strain tensor $\mb{E}$ can be defined using the deformation gradient tensor $\mb{F}$ as
\begin{gather}
    \mb{E}=\dfrac{1}{2}\left(\mb{F}^\mT\mb{F}-\mb{1}\right),
\end{gather}
Noting that the right Cauchy--Green deformation tensor $\mb{C}$ is given by
\begin{gather}
    \mb{C}=\mb{F}^\mT\mb{F},
\end{gather}
one can obtain
\begin{gather}
    \mb{E}=\dfrac{1}{2}\left(\mb{C}-\mb{1}\right).
\end{gather}
If $\mb{E}$ is given, $\mb{C}$ can be obtained as
\begin{gather}
    \mb{C}=2\mb{E}+\mb{1}.
\end{gather}
If $\mb{E}$ is expressed in the Voigt notation as
\begin{gather}
    \mb{E}=\begin{bmatrix}
        E_{11} & E_{22} & E_{33} & 2E_{12} & 2E_{23} & 2E_{31}
    \end{bmatrix}^\mT,
\end{gather}
then $\mb{C}$ can be expressed as
\begin{gather}
    \mb{C}=\begin{bmatrix}
        C_{11} \\
        C_{22} \\
        C_{33} \\
        C_{12} \\
        C_{23} \\
        C_{31}
    \end{bmatrix}=\begin{bmatrix}
        2 &   &   &   &   &   \\
          & 2 &   &   &   &   \\
          &   & 2 &   &   &   \\
          &   &   & 1 &   &   \\
          &   &   &   & 1 &   \\
          &   &   &   &   & 1 \\
    \end{bmatrix}\begin{bmatrix}
        E_{11}  \\
        E_{22}  \\
        E_{33}  \\
        2E_{12} \\
        2E_{23} \\
        2E_{31}
    \end{bmatrix}+\begin{bmatrix}
        1 \\1\\1\\0\\0\\0
    \end{bmatrix}.
\end{gather}

Hyperelastic material models are often defined in terms of the reduced invariants of the right Cauchy--Green deformation tensor $\mb{C}$.
\begin{gather}
\bar{J_1}=I_1I_3^{-1/3},\quad
\bar{J_2}=I_2I_3^{-2/3},\quad
\bar{J_3}=I_3^{1/2}.
\end{gather}
In which, $I_1$, $I_2$, and $I_3$ are the invariants of $\mb{C}$.
\section{Mooney--Rivlin Model}
\subsection{Theory}
The Mooney--Rinvlin model for compressible material uses the following strain energy density function
\begin{gather}
W=A_{10}\left(\bar{J_1}-3\right)+A_{01}\left(\bar{J_2}-3\right)+\dfrac{K}{2}\left(\bar{J_3}-1\right)^2,
\end{gather}
where $A_{10}$, $A_{01}$ and $K$ are material parameters.
As one may observe, $\bar{J_3}$ is involved, the bulk modulus $K$ can be set to a large value such that it approaches so called near incompressible behaviour.

To compute the second Piola--Kirchhoff stress tensor $\mb{S}$, the derivatives are required.
From \eqsref{eq:tensor_invariant_derivative}, using the Voigt notation, one can derive, for example,
\begin{gather}
\begin{split}
    \ddfrac{I_2}{\mb{E}}=\ddfrac{I_2}{\mb{C}}\ddfrac{\mb{C}}{\mb{E}}&=\begin{bmatrix}
        C_{22}+C_{33} \\
        C_{11}+C_{33} \\
        C_{11}+C_{22} \\
        -2C_{12}      \\
        -2C_{23}      \\
        -2C_{31}
    \end{bmatrix}^\mT\begin{bmatrix}
        2 &   &   &   &   &   \\
          & 2 &   &   &   &   \\
          &   & 2 &   &   &   \\
          &   &   & 1 &   &   \\
          &   &   &   & 1 &   \\
          &   &   &   &   & 1
    \end{bmatrix}=2\begin{bmatrix}
        C_{22}+C_{33} \\
        C_{11}+C_{33} \\
        C_{11}+C_{22} \\
        -C_{12}       \\
        -C_{23}       \\
        -C_{31}
    \end{bmatrix}^\mT\\&=2\left(I_1\mb{1}-\mb{C}\right).
\end{split}
\end{gather}
One shall realize the above result agrees with the expression derived using tensor algebra.
Here is one more example.
\begin{gather}
\begin{split}
\ddfrac{I_3}{\mb{E}}=\ddfrac{I_3}{\mb{C}}\ddfrac{\mb{C}}{\mb{E}}&=\begin{bmatrix}
    C_{22}C_{33}-C_{23}^2       \\
    C_{11}C_{33}-C_{31}^2       \\
    C_{11}C_{22}-C_{12}^2       \\
    2C_{23}C_{31}-2C_{12}C_{33} \\
    2C_{12}C_{31}-2C_{23}C_{11} \\
    2C_{12}C_{23}-2C_{31}C_{22}
\end{bmatrix}^\mT
\begin{bmatrix}
    2 &   &   &   &   &   \\
      & 2 &   &   &   &   \\
      &   & 2 &   &   &   \\
      &   &   & 1 &   &   \\
      &   &   &   & 1 &   \\
      &   &   &   &   & 1
\end{bmatrix}\\
&=2\begin{bmatrix}
    C_{22}C_{33}-C_{23}^2     \\
    C_{11}C_{33}-C_{31}^2     \\
    C_{11}C_{22}-C_{12}^2     \\
    C_{23}C_{31}-C_{12}C_{33} \\
    C_{12}C_{31}-C_{23}C_{11} \\
    C_{12}C_{23}-C_{31}C_{22}
\end{bmatrix}^\mT=2I_3\mb{C}^{-1}.
\end{split}
\end{gather}

The second Piola--Kirchhoff stress tensor $\mb{S}$ can be computed via the chain rule,
\begin{gather}
    \mb{S}=A_{10}\ddfrac{\bar{J_1}}{\mb{E}}+A_{01}\ddfrac{\bar{J_2}}{\mb{E}}+K\left(\bar{J_3}-1\right)\ddfrac{\bar{J_3}}{\mb{E}}.
\end{gather}
Please derive the explicit expression for $\mb{S}$.

The tangent stiffness tensor $\mb{D}$ can be computed by further differentiating $\mb{S}$ with respect to $\mb{E}$.
\begin{gather}
    \mb{D}=A_{10}\ddfrac{^2\bar{J_1}}{\mb{E}^2}+A_{01}\ddfrac{^2\bar{J_2}}{\mb{E}^2}+K\left(\bar{J_3}-1\right)\ddfrac{^2\bar{J_3}}{\mb{E}^2}+K\ddfrac{\bar{J_3}}{\mb{E}}\otimes\ddfrac{\bar{J_3}}{\mb{E}}.
\end{gather}

This involves the second order derivatives of invariants $I_1$, $I_2$ and $I_3$.
\begin{gather}
\ddfrac{^2I_1}{\mb{E}^2}=\mb{0},\qquad
\ddfrac{^2I_2}{\mb{E}^2}=4\left(\mb{1}\otimes\mb{1}-\mathbb{I}\right),\\
\begin{split}
\ddfrac{^2I_3}{\mb{E}^2}=2\begin{bmatrix}
             & 2C_{33}  & 2C_{22}  &          & -2C_{23} &          \\
    2C_{33}  &          & 2C_{11}  &          &          & -2C_{31} \\
    2C_{22}  & 2C_{11}  &          & -2C_{12} &          &          \\
             &          & -2C_{12} & -C_{33}  & C_{31}   & C_{23}   \\
    -2C_{23} &          &          & C_{31}   & -C_{11}  & C_{12}   \\
             & -2C_{31} &          & C_{23}   & C_{12}   & -C_{22}
\end{bmatrix}.
\end{split}
\end{gather}
\subsection{Implementation}
The implementation is straightforward.
One shall faithfully calculate the derivatives and then assemble the stress and stiffness tensors.
\begin{cppcode}
MooneyRivlin::update_trial_status
\end{cppcode}
\section{Blatz--Ko Model}
\subsection{Theory}
\subsection{Formulation}
\subsection{Implementation}
\begin{cppcode}
BlatzKo::update_trial_status
\end{cppcode}
\section{Yeoh Model}
\subsection{Theory}
The Yeoh's model for compressible materials is given by
\begin{gather}
    W=\sum_{i=1}^nA_{0,i}\left(J_1-3\right)^i+\sum_{i=1}^nA_{1,i}\left(J_3-1\right)^{2i},
\end{gather}

The second Piola--Kirchhoff stress tensor $\mb{S}$ can be computed via the chain rule,
\begin{gather}
    \mb{S}=\ddfrac{W}{\mb{E}}=\pdfrac{W}{J_1}\ddfrac{J_1}{\mb{E}}+\pdfrac{W}{J_3}\ddfrac{J_3}{\mb{E}}.
\end{gather}
In which,
\begin{gather}
    \pdfrac{W}{J_1}=\sum_{i=1}^nA_{0,i}i\left(J_1-3\right)^{i-1},\qquad
    \pdfrac{W}{J_3}=\sum_{i=1}^nA_{1,i}2i\left(J_3-1\right)^{2i-1}.
\end{gather}

The tangent stiffness can be obtained by further differentiating $\mb{S}$ with respect to $\mb{E}$.
\begin{gather}
    \mb{D}=\ddfrac{\mb{S}}{\mb{E}}=\pdfrac{^2W}{J_1^2}\ddfrac{J_1}{\mb{E}}\otimes\ddfrac{J_1}{\mb{E}}+\pdfrac{^2W}{J_3^2}\ddfrac{J_3}{\mb{E}}\otimes\ddfrac{J_3}{\mb{E}}+\pdfrac{W}{J_1}\ddfrac{^2J_1}{\mb{E}^2}+\pdfrac{W}{J_3}\ddfrac{^2J_3}{\mb{E}^2},
\end{gather}
where
\begin{gather}
    \pdfrac{^2W}{J_1^2}=\sum_{i=1}^nA_{0,i}i\left(i-1\right)\left(J_1-3\right)^{i-2},\qquad
    \pdfrac{^2W}{J_3^2}=\sum_{i=1}^nA_{1,i}2i\left(2i-1\right)\left(J_3-1\right)^{2i-2}.
\end{gather}
\subsection{Implementation}
Noting that the scalar derivatives can be computed independently, the following method can be used to simplify the main body of state determination.
\begin{cppcode}
Yeoh::compute_derivative
\end{cppcode}

The state determination is straightforward.
\begin{cppcode}
Yeoh::update_trial_status
\end{cppcode}
