\chapter{Preface}
Most books/papers present constitutive models analytically, leaving the corresponding implementation to readers to figure out.
It, however, could be a challenge for readers to implement the models correctly since it requires not only a good understanding of the model itself but also a good grasp of the numerical methods and programming.

This book aims to provide a practical guidance to numerical implementation of plasticity models.

Readers are expected to have undergraduate level background of linear algebra, numerical analysis, elasticity and some programming knowledge.
\section{Overview}
The book presents a collection of constitutive models covering both uniaxial and triaxial and a wide range of common materials used in (civil) engineering.
The implementation details are derived for each model.

The ultimate target is to provide an easy--to--follow, error and confusion free reference for readers who are interested in implementing constitutive models under the modern plasticity framework.

A wide range of different constitutive models (both formulations and implementations) are covered in this book, some models are very lengthy.
There are a huge amount of symbols and formulae.

As the title indicates, this book is drafted as a reference book on constitutive modelling, it shall be self--complete in terms of the theoretical part.
For practical implementation, algorithms in pseudo code are given for most models.
Core CPP implementations are also provided as references/comparisons for readers who are familiar with CPP and scientific computation with relevant tools.
The code snippets are taken from the implementations from \texttt{suanPan} \cite{Chang2022}.
In this regard, this book can also be used as the program manual.
\section{Implementation}
\subsection{Which Programming Language?}
In early days, \texttt{C} and \texttt{Fortran} were the most popular languages for scientific computation, mainly due to their performance.

For \texttt{C}, vectors/matrices/tensors are represented by plain arrays.
Some metadata, such as dimensions, can, at best, be stored in structs.
Operating on arrays is cumbersome and error--prone.
As there is no operator overloading, all linear algebra operations have to be implemented as functions.
The implementation quickly becomes lengthy and hard to read, especially for 2D/3D elements and material models.

For \texttt{Fortran}, since matrix is natively supported, the implementation would be more concise.
However, the language itself does not have a rich feature set, writing \texttt{Fortran} is boring (my very subjective opinion) and often ends up with writing one dull loop after another.
Besides, it has a smaller ecosystem.

It appears that, as of writing, \texttt{C++} is still a better choice for scientific computation.
Despite its monstrous complexity, \texttt{C++} has 1) operator overloading (which makes scientific code much more concise), 2) meta-programming (templating that saves boilerplate code), 3) object--oriented programming (which offers a decent way to organise and structure data) and 4) a rich ecosystem.
It also has a good performance, comparable to \texttt{C} and \texttt{Fortran}.

Thus, in this book, \texttt{C++} is chosen as the primary language for implementation.
However, which language is actually used is not very important in this context.
At the end of the day, one would realise that the implementation highly resembles its mathematical counterpart.
Once the mathematical model is understood, the implementation in any language should not be a great concern.
\subsection{State Determination}
In all the state determination algorithms presented in this book, by default it is assumed all local iterations eventually converge.
Thus the handling of failure of convergence is not presented for brevity.
One shall always bear in mind that not all algorithms converge, and for robust, practical algorithms, more numerical processes are required.
\subsection{Internal Storage}
Accounting for the Cauchy stress theory, under which strain and stress tensors are symmetric, there seems to be no need to actually store second order tensors in matrices and fourth order tensors in 4D cubes.
Tensor operations on those high dimensional arrays are cumbersome and may have a performance penalty.
While the Voigt notation is used, row/column vectors and 2D matrices are sufficient to implement most models.

In this book, all models are implemented in this way.
However, some other applications choose to use the full tensor representation.