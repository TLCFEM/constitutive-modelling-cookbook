\chapter{Preface}
Most monographs and academic papers present constitutive models analytically with little to no emphasis on implementation, leaving it to readers to figure out on their own.
It, however, could be a challenge to implement various models, especially those complex ones, correctly since it requires not only a good understanding of the theoretical formulation of models but also a good grasp of the relevant numerical methods and programming details.

This book aims to provide a practical guidance to numerical implementations of plasticity models.

Readers are expected to have undergraduate level background of linear algebra, numerical analysis, elasticity and some programming knowledge.
\section{Overview}
The book presents a collection of constitutive models, covering both uniaxial and triaxial versions, of a wide range of common materials used in (civil/mechanical) engineering.
The implementation details are derived for each model.

The ultimate target is to provide an easy to follow, error and confusion free reference for readers who are interested in implementing constitutive models under the modern conventional and novel plasticity framework.

A wide range of different constitutive models (both formulations and implementations) are covered in this book, some models are very lengthy.
There are a huge amount of symbols and formulae.

As the title indicates, this book is drafted as a reference book on constitutive modelling, it shall be self--complete in terms of the theoretical part.
For practical implementation, algorithms in pseudo code are given for most models.
Core \texttt{C++} implementations are also provided as references/comparisons for readers who are familiar with \texttt{C++} and scientific computation with relevant tools.
The code snippets are taken from the implementations in \texttt{suanPan} \cite{Chang2022}.
This book can also be used as the program manual.
\section{What Knowledge Is Needed?}
Constitutive modelling effectively concerns numerical implementations of various plasticity theories.
Readers are expected to have a good understanding of undergraduate level of linear algebra and shall feel comfortable with matrix/vector operations and analysis tools such as eigenanalysis.
Readers shall also be fairly familiar with elasticity theories, especially the small strain theory.
Furthermore, as can be seen later, plasticity theories can essentially be interpreted as differential equations, thus a good grasp of numerical methods for differential equations (ODEs and PDEs) is also required.
It, however, shall be pointed out that, in most cases, only low order (very basic) methods are needed.

If 2D and 3D models are of interest, it is likely that basic understanding of tensor algebra (literally I mean the first one or two chapters of any monographs on tensor algebra and nothing more) is also required.
Chapter \ref{chap:tensor} provides a brief summary of all necessary tensor algebra knowledge needed in this book.
If one is able to understand the contents of Chapter \ref{chap:tensor}, then it is sufficient.
\section{Implementation}
To get hands on the actual implementation, one shall then consider practical programming issues such as which programming language to use, which paradigm to follow, etc.
\subsection{Which Programming Language?}
In early days, \texttt{C} and \texttt{Fortran} were the most popular languages for scientific computation.
One can easily find a large amount of libraries written in either \texttt{C} or \texttt{Fortran} that are quietly driving the whole scientific computation community.

However, the procedural programming may not necessarily be the ideal paradigm to be exercised in scientific computation.
Consider, for example, in those languages, vectors/matrices/tensors are represented by plain arrays.
The corresponding metadata, such as dimensions, can, at best, be stored in \texttt{struct}s.
Operating on arrays is cumbersome and error--prone.
As there is no operator overloading, all linear algebra operations have to be implemented as functions.
The implementation quickly becomes lengthy and hard to read, especially for 2D/3D elements and material models.

For \texttt{Fortran}, since matrix operations are natively supported, the corresponding implementation would be more concise.
However, when interfacing with external libraries (which is often necessary because they typically offer optimised performance), following the signature is often lengthy and tedious.
Furthermore, the language itself does not have a rich feature set, writing \texttt{Fortran} is boring (my very subjective opinion) and often ends up with writing one dull loop after another.

When choosing a proper language for scientific computation, the following three aspects are important.
\begin{itemize}
\item Operator Overloading\\
Operator overloading is a mechanism that allows one to define custom behaviour for standard operators (e.g., $+$, $-$, $*$, $/$.) when applied to custom types.
This is particularly useful in scientific computing, where mathematical operations on vectors, matrices, and tensors are common.
With operator overloading, one can write code that closely resembles mathematical notation, making it more readable and easier to understand.
It also offers possible performance optimisation.
Consider the matrix multiplication of three matrices, with operator overloading it can be concisely expressed as
\begin{cppcode}
    mat A, B, C;
    auto D = A * B * C;
\end{cppcode}
However, if the chosen language does not support it, it can only be done via conventional function calls.
\begin{cppcode}
    mat matmul(const mat&, const mat&);
    mat A, B, C;
    auto D = matmul(matmul(A, B), C);
\end{cppcode}
\item Ecosystem\\
For implementing standalone material models, whether there is a rich ecosystem may not be a big concern.
However, at higher levels, material models need to be integrated into larger analysis frameworks, a rich ecosystem can significantly avoid reinventing the wheel and reduce the overall workload.
\item Multiparadigm Support\\
Although not strictly necessary, multiparadigm support can be very useful in scientific computation.
\end{itemize}
\subsection{Notes on Miscellaneous Things}
\subsubsection{State Determination}
In all the state determination algorithms presented in this book, by default it is assumed all local iterations eventually converge.
However, this is often not what would happen in practice.
For example, most algorithms solve a local nonlinear system with the Newton--Raphson method, which is known to only have local convergence guarantee.
This implies that the algorithm would happily fail if the initial starting point is far enough from the desired solution.
To improve robustness, it is always possible to add more numerical processes to safeguard the algorithm.
Thus, the presented implementations are the `reference' ones, not the `golden' ones.
\subsubsection{Internal Storage}
The mathematical objects, such as tensors, can be loyally represented in programs just like what they imply in mathematics.
This means, for example, second order tensors can be represented as 2D matrices, fourth order tensors can be represented as 4D cubes.
A complete set of rules can be defined in order to do tensor algebra.

This can be seen in many other applications.
However, accounting for the Cauchy stress theory, with the Voigt notation, row/column vectors and 2D matrices are sufficient to implement most models.

In this book, all models are implemented in this way.