\chapter{Geomaterial}
In geotechnical engineering, compression positive convention is used in existing literatures. This brings some convenience to geotechnical engineering specific applications but conflicts against the common convention adopted in continuum mechanics (tension positive).

To be consistent with other models, in this chapter the continuum mechanics convention is adopted. Thus the formulations presented here may differ from what readers can found in other literatures.
\section{Drucker--Prager Model}
The Drucker--Prager model is a simple model that resembles the J2 model and can be used in simple preliminary stability/capacity analysis.
\subsection{Theory}
\subsubsection{Yield Function}
\begin{gather}
f=\sqrt{J_2}+\eta{}p-\xi{}c,
\end{gather}
where $J_2=\dfrac{1}{2}\bs:\bs$ is the J2 invariant of the deviatoric stress tensor $\bs$, $p=\dfrac{1}{3}\tr{\bsigma}$ is the hydrostatic stress, $\eta$ and $\xi$ are two model parameters which can be adjusted to match different shapes, $c$ is the cohesion.
\subsubsection{Flow Rule}
The plastic potential possesses a form that resembles the yield function.
\begin{gather}
g=\sqrt{J_2}+\bar{\eta}p.
\end{gather}
When $\bar{\eta}=\eta$, an associative flow rule is implied. The corresponding rate form of plastic strain is then
\begin{gather}
\dot{\bvarepsilon^p}=\gamma\pdfrac{g}{\bsigma}=\gamma\left(\dfrac{\sqrt{2}}{2}\bn+\dfrac{\bar{\eta}}{3}\mb{1}\right).
\end{gather}
\subsubsection{Hardening Law}
The accumulated equivalent plastic strain $\alpha$ can be defined as
\begin{gather}
\dot{\alpha}=\gamma.
\end{gather}
Then the cohesion can be defined as a function of $\alpha$.
\begin{gather}
c=c\left(\alpha\right).
\end{gather}

More complex laws to allow $\eta$ and $\xi$ to evolve with the development of plasticity are also possible. Here for simplicity, it is assumed that $\eta$ and $\xi$ are two constants.
\subsection{Formulation}
As the cohesion $c$ is not explicitly defined in this model, it can be customised to produce various hardening responses.

The only local variable is the plasticity consistency parameter $\gamma$.
\subsubsection{Elastic Loading/Unloading}
The trial state can be computed as
\begin{gather}
\bsigma^\text{trial}=\mb{D}:\left(\bvarepsilon_{n+1}-\bvarepsilon^p_n\right).
\end{gather}
The deviatoric stress and hydrostatic stress can be computed accordingly,
\begin{gather}
\bs^\text{trial}=\dev{\bsigma^\text{trial}},\\
p^\text{trial}=\dfrac{1}{3}\tr{\bsigma^\text{trial}}.
\end{gather}
\subsection{Implementation}
\begin{cppcode}
int NonlinearDruckerPrager::update_trial_status(const vec& t_strain) {
    incre_strain = (trial_strain = t_strain) - current_strain;

    if(norm(incre_strain) <= datum::eps) return SUANPAN_SUCCESS;

    trial_stress = current_stress + (trial_stiffness = initial_stiffness) * incre_strain;

    trial_history = current_history;
    auto& plastic_strain = trial_history(0);

    const auto dev_stress = tensor::dev(trial_stress);
    const auto hydro_stress = tensor::mean3(trial_stress);
    const auto sqrt_j2 = sqrt(std::max(datum::eps, tensor::stress::invariant2(dev_stress)));

    const auto yield_const = sqrt_j2 + eta_yield * hydro_stress;

    if(yield_const <= xi * compute_c(plastic_strain)) return SUANPAN_SUCCESS;

    auto gamma = 0.;
    double denominator;

    auto counter = 0u;
    auto ref_error = 1.;
    while(true) {
        if(max_iteration == ++counter) {
            suanpan_error("Cannot converge within {} iterations.\n", max_iteration);
            return SUANPAN_FAIL;
        }

        const auto residual = yield_const - factor_a * gamma - xi * compute_c(plastic_strain);
        const auto incre_gamma = residual / (denominator = factor_a + xi * xi * compute_dc(plastic_strain));
        const auto error = fabs(incre_gamma);
        if(1u == counter) ref_error = error;
        suanpan_debug("Local iteration error: {:.5E}.\n", error);
        if(error < tolerance * ref_error || (fabs(residual) < tolerance && counter > 5u)) break;
        plastic_strain = current_history(0) + xi * (gamma += incre_gamma);
    }

    if(sqrt_j2 >= shear * gamma) {
        const auto norm_s = tensor::stress::norm(dev_stress);

        const auto t_factor = shear / sqrt_j2 * gamma;

        trial_stress -= t_factor * dev_stress + bulk * eta_flow * gamma * tensor::unit_tensor2;

        trial_stiffness += double_shear * (t_factor - shear / denominator) / norm_s / norm_s * dev_stress * dev_stress.t() - double_shear * t_factor * unit_dev_tensor - factor_d / denominator * unit_x_unit;

        const mat t_mat = eta_yield * factor_c / denominator / norm_s * dev_stress * tensor::unit_tensor2.t();

        associated ? trial_stiffness -= t_mat + t_mat.t() : trial_stiffness -= t_mat + eta_flow / eta_yield * t_mat.t();
    }
    else {
        // apex return
        gamma = 0.; // volumetric strain reuse variable
        plastic_strain = current_history(0);

        counter = 0u;
        while(true) {
            if(max_iteration == ++counter) {
                suanpan_error("Cannot converge within {} iterations.\n", max_iteration);
                return SUANPAN_FAIL;
            }

            const auto residual = compute_c(plastic_strain) * xi / eta_flow - hydro_stress + bulk * gamma;
            const auto incre_gamma = residual / (denominator = factor_b * compute_dc(plastic_strain) + bulk);
            const auto error = fabs(incre_gamma);
            if(1u == counter) ref_error = error;
            suanpan_debug("Local iteration error: {:.5E}.\n", error);
            if(error < tolerance * ref_error || (fabs(residual) < tolerance && counter > 5u)) break;
            plastic_strain = current_history(0) + xi / eta_yield * (gamma -= incre_gamma);
        }

        trial_stress = (hydro_stress - bulk * gamma) * tensor::unit_tensor2;

        trial_stiffness = (bulk - bulk * bulk / denominator) * unit_x_unit;
    }

    return SUANPAN_SUCCESS;
}
\end{cppcode}
\section{Modified Cam Clay Model}
This model is identical to the one documented in \cite[see \S~10.1]{SouzaNeto2008} apart from the definition of hardening variable. In this section, we present an equivalent formulation.
\subsection{Theory}
\subsubsection{Basics}
The model adopts a two-scalar formulation that depends on hydrostatic stress $p$ and von Mises equivalent stress $q$, which are defined as
\begin{gather}
p=\dfrac{\tr{\bsigma}}{3}=\dfrac{I_1}{3},\qquad{}q=\sqrt{3J_2}=\sqrt{\dfrac{3}{2}\bs:\bs}=\sqrt{\dfrac{3}{2}}\norm{\bs}.
\end{gather}
They are effectively two invariants $I_1$ and $J_2$.
\subsubsection{Yield Function}
The yield function is defined as
\begin{gather}
f=\dfrac{p_e^2}{b^2}+\dfrac{q^2}{M^2}-a^2,
\end{gather}
where $p_t\geqslant0$ is the tensile yield strength, $b=1$ if $p_e\geqslant0$ and $b=\beta$ if $p_e<0$ with $p_e=p-p_t+a$ denotes relative stress to the centre of ellipse, $\beta$ is a constant that controls the shape of negative-wards half of yielding ellipse, $M$ is the ratio between two radii of yielding ellipse.

The above yield function essentially possesses the form of an ellipse, for example,
\begin{gather*}
\dfrac{x^2}{r_x^2}+\dfrac{y^2}{r_y^2}=1.
\end{gather*}
The centre of the ellipse is shifted to $p_t-a$ so that the majority of the ellipse is in the compression region.
The radius $b$ alters its value depending on the sign of $p_e$.
The yield surface is effectively two half ellipses glued together.
\subsubsection{Flow Rule}
The associative plasticity is assumed so that
\begin{gather}
\dot{\bvarepsilon^p}=\gamma\pdfrac{f}{\bsigma}=\gamma\left(\dfrac{2p_e}{3b^2}\mb{1}+\dfrac{3}{M^2}\bs\right).
\end{gather}
The following relationship is used,
\begin{gather*}
\ddfrac{q^2}{\bsigma}=\dfrac{3}{2}\ddfrac{\left(\bs:\bs\right)}{\bsigma}=\dfrac{3}{2}\left(\ddfrac{\bs}{\bsigma}:\bs+\bs:\ddfrac{\bs}{\bsigma}\right)=3\bs,
\end{gather*}
since $\ddfrac{\bs}{\bsigma}:\bs=\mathbb{I}^\text{dev}:\bs=\bs$.
Accordingly,
\begin{gather}
\dot{\varepsilon^p_v}=\gamma\dfrac{2p_e}{b^2},\qquad\dot{\bvarepsilon^p_d}=\gamma\dfrac{3}{M^2}\bs,
\end{gather}
where $\dot{\varepsilon^p_v}=\tr{\dot{\bvarepsilon^p}}$ is the increment of the volumetric strain scalar and $\dot{\bvarepsilon^p_d}$ is the increment of the deviatoric strain tensor.
\subsubsection{Hardening Rule}
The hardening variable $\alpha$ is defined as the volumetric strain $\varepsilon^p_v$ so that
\begin{gather}
\alpha=\varepsilon^p_v.
\end{gather}
The corresponding incremental form is then
\begin{gather}
\alpha=\alpha_n+\gamma\dfrac{2p_e}{b^2}.
\end{gather}
The hardening rule is then defined as a function of $\alpha$,
\begin{gather}
a=a\left(\alpha\right)\geqslant0.
\end{gather}

Note here we simple take $\alpha=\varepsilon^p_v$ instead of $\alpha=-\varepsilon^p_v$ in the original literature \cite{SouzaNeto2008}.
\subsection{Formulation}
\subsubsection{Local Residual}
By using the elastic relationship,
\begin{gather}
\bs=2G\bvarepsilon^e_d=2G\left(\bvarepsilon^\text{trial}_d-\Delta\bvarepsilon^p_d\right)=\bs^\text{trial}-\gamma\dfrac{6G}{M^2}\bs,
\end{gather}
one can rearrange and obtain the following,
\begin{gather}
\bs=\dfrac{M^2}{M^2+6G\Delta\gamma}\bs^\text{trial},\qquad
q=\dfrac{M^2}{M^2+6G\Delta\gamma}q^\text{trial}.
\end{gather}

The governing residual equations for independent variables $\mb{x}=\begin{bmatrix}\gamma&\alpha\end{bmatrix}^\mathrm{T}$ can be expressed as
\begin{gather}
\mb{R}=\left\{\begin{array}{l}
\dfrac{p_e^2}{b^2}+\dfrac{q^2}{M^2}-a^2=0,\\[4mm]
\alpha-\alpha_n-\gamma\dfrac{2}{b^2}p_e=0.
\end{array}\right.
\end{gather}
where $p_e=p^\text{trial}-K\alpha+K\alpha_n-p_t+a$ since
\begin{gather}
p=K\varepsilon^e_v=K\left(\varepsilon^\text{trial}_v-\Delta\varepsilon^p_v\right)=p^\text{trial}-K\left(\alpha-\alpha_n\right),
\end{gather}
and $q=\dfrac{M^2}{M^2+6G\Delta\gamma}q^\text{trial}$.
\subsubsection{Local Iteration}
The Jacobian can be formed accordingly.
\begin{gather}
\mb{J}=\pdfrac{\mb{R}}{\mb{x}}=\begin{bmatrix}
\dfrac{-12GM^2q^{\text{trial},2}}{(M^2+6G\gamma)^3}&\dfrac{2}{b^2}p_e(a'-K)-2aa'\\[4mm]
-\dfrac{2}{b^2}p_e&1-\gamma\dfrac{2}{b^2}(a'-K)
\end{bmatrix}.
\end{gather}
\subsubsection{Tangent Stiffness}
At local iteration, $\bvarepsilon^\text{trial}$ (or $\bvarepsilon^{n+1}$, or just $\bvarepsilon$ for brevity) is fixed and $\mb{R}$ is iterated out. Noting that in the global iteration, $\bvarepsilon^\text{trial}$ is also a variable that changes. If local iteration is converged, then $\mb{R}=\mb{0}$, so
\begin{gather}
\ddfrac{\mb{R}}{\bvarepsilon}=\pdfrac{\mb{R}}{\bvarepsilon}+\pdfrac{\mb{R}}{\mb{x}}\ddfrac{\mb{x}}{\bvarepsilon}=\mb{0},
\end{gather}
consequently,
\begin{gather}\label{eq:mcc_tangent}
\ddfrac{\mb{x}}{\bvarepsilon}=
\begin{bmatrix}
\ddfrac{\gamma}{\bvarepsilon}\\[4mm]\ddfrac{\alpha}{\bvarepsilon}
\end{bmatrix}
=-\left(\pdfrac{\mb{R}}{\mb{x}}\right)^{-1}\pdfrac{\mb{R}}{\bvarepsilon}=-\mb{J}^{-1}\pdfrac{\mb{R}}{\bvarepsilon}.
\end{gather}

Taking derivatives about $\bvarepsilon$ gives
\begin{gather}
\pdfrac{\mb{R}}{\bvarepsilon}=\begin{bmatrix}
\dfrac{2p_eK}{b^2}\mb{1}+\dfrac{6G}{M^2+6G\gamma}\bs\\[4mm]
\dfrac{-2\gamma{}K}{b^2}\mb{1}
\end{bmatrix}.
\end{gather}

The stress can be expressed as
\begin{gather}
\bsigma=\bs+p\mb{1}=\dfrac{M^2}{M^2+6G\Delta\gamma}\bs^\text{trial}+\left(p^\text{trial}-K\left(\alpha-\alpha_n\right)\right)\mb{1}.
\end{gather}
Direct differentiation gives
\begin{gather}
\begin{split}
\ddfrac{\bsigma}{\bvarepsilon}&=\dfrac{M^2}{M^2+6G\gamma}\ddfrac{\bs^\text{trial}}{\bvarepsilon}+\bs^\text{trial}\otimes\ddfrac{\dfrac{M^2}{M^2+6G\gamma}}{\bvarepsilon}+\mb{1}\otimes\ddfrac{\left(p^\text{trial}-K\left(\alpha-\alpha_n\right)\right)}{\bvarepsilon}\\
&=\dfrac{2GM^2}{M^2+6G\gamma}\mathbb{I}^\text{dev}+K\mb{1}\otimes{}\mb{1}-K\mb{1}\otimes\ddfrac{\alpha}{\bvarepsilon}-\dfrac{6GM^2}{\left(M^2+6G\gamma\right)^2}\bs^\text{trial}\otimes\ddfrac{\gamma}{\bvarepsilon},\\
&=\mb{D}-\dfrac{12G^2\gamma}{M^2+6G\gamma}\mathbb{I}^\text{dev}-K\mb{1}\otimes\ddfrac{\alpha}{\bvarepsilon}-\dfrac{6G}{M^2+6G\gamma}s\otimes\ddfrac{\gamma}{\bvarepsilon}.
\end{split}
\end{gather}
In which $\ddfrac{\alpha}{\bvarepsilon}$ and $\ddfrac{\gamma}{\bvarepsilon}$ are obtained from \eqsref{eq:mcc_tangent}.
The above derivation also takes advantage of the expression
\begin{gather}
\mb{D}=2G\mathbb{I}^\text{dev}+K\mb{1}\otimes{}\mb{1}.
\end{gather}
\subsection{Implementation}
\begin{cppcode}
int NonlinearCamClay::update_trial_status(const vec& t_strain) {
    incre_strain = (trial_strain = t_strain) - current_strain;

    if(norm(incre_strain) <= datum::eps) return SUANPAN_SUCCESS;

    trial_stress = current_stress + (trial_stiffness = initial_stiffness) * incre_strain;

    trial_history = current_history;
    auto& alpha = trial_history(0);
    const auto& current_alpha = current_history(0);

    auto trial_s = tensor::dev(trial_stress);
    const auto trial_q = sqrt_three_two * tensor::stress::norm(trial_s);
    const auto p = tensor::mean3(trial_stress);

    auto gamma = 0.;

    vec residual(2), incre;
    mat jacobian(2, 2);

    auto counter = 0u;
    auto rel_error = 1.;
    while(true) {
        if(max_iteration == ++counter) {
            suanpan_error("Cannot converge within {} iterations.\n", max_iteration);
            return SUANPAN_FAIL;
        }

        const auto a = compute_a(alpha);
        const auto da = compute_da(alpha);
        const auto incre_alpha = alpha - current_alpha;
        const auto trial_p = p - bulk * incre_alpha;
        const auto rel_p = trial_p - pt + a;
        const auto square_b = rel_p >= 0. ? 1. : square_beta;
        const auto denom = square_m + six_shear * gamma;
        const auto square_qm = pow(m * trial_q / denom, 2.);

        residual(0) = rel_p * rel_p / square_b + square_qm - a * a;

        if(1u == counter && residual(0) < 0.) return SUANPAN_SUCCESS;

        residual(1) = incre_alpha - 2. * gamma / square_b * rel_p;

        jacobian(0, 0) = -2. * six_shear / denom * square_qm;
        jacobian(1, 0) = -2. * rel_p / square_b;
        jacobian(0, 1) = jacobian(1, 0) * (bulk - da) - 2. * a * da;
        jacobian(1, 1) = 1. - 2. * gamma / square_b * (da - bulk);

        if(!solve(incre, jacobian, residual, solve_opts::equilibrate)) return SUANPAN_FAIL;

        const auto error = inf_norm(incre);
        if(1u == counter) rel_error = error;
        suanpan_debug("Local iteration error: {:.5E}.\n", error);
        if(error < tolerance * rel_error || (inf_norm(residual) < tolerance && counter > 5u)) {
            mat left(6, 2);

            rel_error = 2. * bulk / square_b; // reuse variable
            left.col(0) = rel_error * rel_p * tensor::unit_tensor2 + six_shear / denom * (trial_s *= square_m / denom);
            left.col(1) = -rel_error * gamma * tensor::unit_tensor2;

            trial_stress = trial_s + trial_p * tensor::unit_tensor2;

            rel_error = six_shear / denom; // reuse variable
            trial_stiffness -= 2. * shear * gamma * rel_error * unit_dev_tensor - join_rows(rel_error * trial_s, bulk * tensor::unit_tensor2) * solve(jacobian, left.t());

            return SUANPAN_SUCCESS;
        }

        gamma -= incre(0);
        alpha -= incre(1);
    }
}
\end{cppcode}
\section{Simple Sand Model}
This sand model is a hybrid of several existing models with significant simplifications. Readers should not expect this model to give accurate results. Instead, this model only captures the characteristics of sand material. It is presented to craft a basic framework from which more complex sand models can be developed.

Compared to other models for sand, this model is more like a preliminary model just like the bilinear hardening von Mises model compared to other models for metals.

A distinctive feature of sand models is that, often a non-associative plastic flow is used and the hardening law is significantly more complex, involving heavy tensor operations, than aforementioned models. Often the hardening laws are tensor--valued functions, and cannot be simplified to scalar--valued functions. The numerical complexity is higher.
\subsection{Theory}
\subsubsection{Elastic Constitutive Relation}
Often for sand, the elastic response is nonlinear (to be specific, hyper-elastic). The shear modulus and bulk modulus often are functions of hydrostatic pressure. In this model, for simplicity, an linear elastic relation is assumed. Readers shall be warned this does no reflect the real behaviour of sand.

The decomposition of total strain into elastic/plastic strain components is still valid. Thus the trial elastic state can be computed via the same expression.
\begin{gather}\label{eq:ss_tsigma}
\bsigma^\text{trial}=\mb{D}:\left(\bvarepsilon_{n+1}-\bvarepsilon^p_n\right)=\bsigma_n+\mb{D}:\left(\bvarepsilon_{n+1}-\bvarepsilon_n\right).
\end{gather}
Furthermore,
\begin{gather}
\bsigma^\text{trial}=\bs^\text{trial}+p^\text{trial}\mb{1}.
\end{gather}
\subsubsection{Yield Function}
A wedge--like function is chosen to be the yield surface.
\begin{gather*}
f=\norm{\bs+p\balpha}+mp=\norm{\mb{\eta}}+mp,
\end{gather*}
where $\bs=\dev{\bsigma}$ is the deviatoric stress, $p=\dfrac{1}{3}\tr{\bsigma}$ is the hydrostatic stress, $\balpha$ is the so called back stress and $m$ characterises the size of the wedge. For simplicity, $m$ is assumed to be a constant in this model.
\subsubsection{Flow Rule}
A non-associated plastic flow is used, the corresponding flow rule is defined as follows.
\begin{gather*}
\dot{\mb{\varepsilon}^p}=\gamma\left(\bn+\dfrac{1}{3}A\left(\alpha^d-m-\balpha:\bn\right)\mb{1}\right),
\end{gather*}
where $D=A\left(\alpha^d-m-\balpha:\bn\right)$ is the dilatancy parameter, $\bn=\dfrac{\beeta}{\norm{\beeta}}$ is the unit directional tensor, $A$ is a model constant, it can also be defined as a function of current state. Note since the sign convention is changed, a positive $A$ leads to dilatant behaviour.

Since the linear elasticity is applied, the following expressions hold.
\begin{gather*}
\bs=\bs^\text{trial}-2G\gamma\bn,\qquad
p=p^\text{trial}-KA\gamma\left(\alpha^d-m-\balpha:\bn\right).
\end{gather*}
\subsubsection{Hardening Law}
The evolution rate of the back stress $\balpha$ is defined in terms of a proper distance measure from the bounding surface, Here, such a distance measure is chosen to be $h\left(\left(\alpha^b-m\right)\bn-\balpha\right)$, where $h$ is a model constant. Thus,
\begin{gather*}
\dot{\balpha}=\gamma{}h\left(\left(\alpha^b-m\right)\bn-\balpha\right).
\end{gather*}
Essentially, this is similar to the Armstrong--Frederick type kinematic hardening rule. It shall be noted that $\bn$ is a deviatoric stress, and $\balpha$ stays deviatoric but may not be coaxial with $\bn$. Thus in tensor notion,
\begin{gather}
\bn=\mathbb{I}:\bn=\mathbb{I}^\text{dev}:\bn,\qquad
\balpha=\mathbb{I}:\balpha=\mathbb{I}^\text{dev}:\balpha.
\end{gather}
\subsubsection{Critical State}
The state parameter is defined to be
\begin{gather*}
\psi=v_0\left(1+\varepsilon^{v,\text{trial}}\right)-v_c+\lambda_c\ln\left(\dfrac{p}{p_c}\right),
\end{gather*}
where $v_0$ is the initial specific volume, $v_c$ is the corresponding specific volume on the critical line at $p_c$ and $\lambda_c$ is absolute value of the slope of the critical line in $v$--$\ln(-p)$ space.

The Lode angle dependence is not considered for simple derivations of the corresponding terms. Hence both the dilatancy surface and bounding surface will be circular cones in the principal stress space.

The dilatancy surface is defined as
\begin{gather*}
\alpha^d=\alpha^c\exp\left(n^d\psi\right).
\end{gather*}
The corresponding derivatives are
\begin{gather*}
\pdfrac{\alpha^d}{\bvarepsilon^\text{trial}}=\alpha^dn^dv_0\mb{1}^\mT,\qquad
\pdfrac{\alpha^d}{p}=\alpha^dn^d\dfrac{\lambda_c}{p}.
\end{gather*}

The bounding surface is defined as
\begin{gather*}
\alpha^b=\alpha^c\exp\left(-n^b\psi\right).
\end{gather*}
The corresponding derivatives are
\begin{gather*}
\pdfrac{\alpha^b}{\bvarepsilon^\text{trial}}=-\alpha^bn^bv_0\mb{1}^\mT,\qquad
\pdfrac{\alpha^b}{p}=-\alpha^bn^b\dfrac{\lambda_c}{p}.
\end{gather*}

The symbol $\mb{1}$ denotes the unit second order tensor. In explicit compressed matrix representation, it is $\mb{1}=\begin{bmatrix}
1&1&1&0&0&0
\end{bmatrix}^\mT$.
\subsection{Formulation}
Some tensorial derivatives are presented first. They will be used in both formulation and implementation. According to \eqsref{eq:unit_derivative}, the following expressions can be derived.
\begin{gather*}
\pdfrac{\norm{\beeta}}{}=\bn:\pdfrac{\mb{\eta}}{},\qquad
\pdfrac{\bn}{}=\dfrac{1}{\norm{\beeta}}\left(\mathbb{I}-\bn\otimes\bn\right):\pdfrac{\mb{\eta}}{}.
\end{gather*}
Hence, in compressed matrix representation with the scaling vector $\bc=\begin{bmatrix}
1&1&1&2&2&2
\end{bmatrix}^\mT$,
\begin{gather*}
\begin{array}{ll}
\pdfrac{\norm{\beeta}}{p}=\bn^\mT\diag{\bc}\balpha,
&
\pdfrac{\bn}{p}=\dfrac{1}{\norm{\beeta}}\left(\balpha-\bn\bn^\mT\diag{\bc}\balpha\right),
\\[4mm]
\pdfrac{\norm{\beeta}}{\bs}=\bn\circ\bc,
&
\pdfrac{\bn}{\bs}=\dfrac{1}{\norm{\beeta}}\left(\mb{I}-\bn\bn^\mT\diag{\bc}\right),
\\[4mm]
\pdfrac{\norm{\beeta}}{\balpha}=p~\bn\circ\bc,
&
\pdfrac{\bn}{\balpha}=\dfrac{p}{\norm{\beeta}}\left(\mb{I}-\bn\bn^\mT\diag{\bc}\right).
\end{array}
\end{gather*}

The governing equations of this model are
\begin{table}[ht]
\centering
\begin{tabular}{rl}
\toprule
Constitutive Law&$\bsigma=\mb{D}:\left(\bvarepsilon-\bvarepsilon^p\right)$\\
Yield Function&$f=\norm{\bs+p\balpha}+mp=\norm{\mb{\eta}}+mp$\\
Flow Rule&$\dot{\bvarepsilon^p}=\gamma\left(\bn+\dfrac{1}{3}A\left(\alpha^d-m-\balpha:\bn\right)\mb{1}\right)$\\
Hardening Law&$\dot{\balpha}=\gamma{}h\left(\left(\alpha^b-m\right)\bn-\balpha\right)$\\
Critical State&$\psi=v_0\left(1+\varepsilon^{v,\text{trial}}\right)-v_c+\lambda_c\ln\left(\dfrac{p}{p_c}\right)$\\
&$\alpha^d=\alpha^c\exp\left(n^d\psi\right)$\\
&$\alpha^b=\alpha^c\exp\left(-n^b\psi\right)$\\\bottomrule
\end{tabular}
\end{table}
\subsubsection{Elastic Loading/Unloading}
The trial yield function can be computed from trial stress as
\begin{gather}\label{eq:ss_yield}
f^\text{trial}=\norm{\bs^\text{trial}+p^\text{trial}\balpha}+mp^\text{trial}=\norm{\mb{\eta}^\text{trial}}+mp^\text{trial}.
\end{gather}
\subsubsection{Plastic Evolution}
In summary, there are four residual equations.
\begin{gather}\label{eq:ss_f}
\mb{R}=\left\{\begin{array}{l}
\norm{\beeta_{n+1}}+mp_{n+1},\\[4mm]
p_{n+1}-p^\text{trial}+KA\gamma\left(\alpha^d_{n+1}-m-\balpha_{n+1}:\bn_{n+1}\right),\\[4mm]
\bs_{n+1}-\bs^\text{trial}+2G\gamma\bn_{n+1},\\[4mm]
\balpha_n+\gamma{}h\left(\alpha^b_{n+1}-m\right)\bn_{n+1}-\left(\gamma{}h+1\right)\balpha_{n+1}.
\end{array}\right.
\end{gather}

To avoid complex computation of derivatives, the above four result equations are not further combined. By defined $\mb{x}=\begin{bmatrix}\gamma&p_{n+1}&\bs_{n+1}&\balpha_{n+1}\end{bmatrix}^\mT$, the local Jacobian can be derived and expressed in the compressed matrix form. It shall be noted for simplicity the subscript $\left(\cdot\right)_{n+1}$ is dropped.
\begin{gather}\label{eq:ss_df}
\pdfrac{\mb{R}}{\mb{x}}=\begin{bmatrix}
\pdfrac{\mb{R}}{\gamma}&\pdfrac{\mb{R}}{p}&\pdfrac{\mb{R}}{\bs}&\pdfrac{\mb{R}}{\balpha}
\end{bmatrix},
\end{gather}
with
\begin{gather}
\pdfrac{\mb{R}}{\gamma}=\begin{bmatrix}
0\\[5mm]
KA\left(\alpha^d-m-\balpha:\bn\right)\\[5mm]
2G\bn\\[5mm]
h\left(\alpha^b-m\right)\bn-h\balpha
\end{bmatrix},
\end{gather}
\begin{gather}
\pdfrac{\mb{R}}{p}=\begin{bmatrix}
\balpha:\bn+m\\[5mm]
1+KA\gamma\left(\alpha^dn^d\dfrac{\lambda_c}{p}-\dfrac{\balpha:\balpha-\left(\balpha:\bn\right)^2}{\norm{\beeta}}\right)\\[5mm]
\dfrac{2G\gamma}{\norm{\beeta}}\left(\balpha-\left(\balpha:\bn\right)\bn\right)\\[5mm]
\dfrac{\gamma{}h\left(\alpha^b_{n+1}-m\right)}{\norm{\beeta}}\balpha-\left(\gamma{}h\alpha^bn^b\dfrac{\lambda_c}{p}+\dfrac{\gamma{}h\left(\alpha^b_{n+1}-m\right)}{\norm{\beeta}}\left(\balpha:\bn\right)\right)\bn
\end{bmatrix},
\end{gather}
\begin{gather}
\pdfrac{\mb{R}}{\bs}=\begin{bmatrix}
\bn^\mT\circ\bc^\mT\\[5mm]
\dfrac{KA\gamma}{\norm{\beeta}}\left(\left(\balpha:\bn\right)\bn-\balpha\right)^\mT\circ\bc^\mT\\[5mm]
\mb{I}+\dfrac{2G\gamma}{\norm{\beeta}}\left(\mb{I}-\bn\bn^\mT\diag{\bc}\right)\\[5mm]
\dfrac{\gamma{}h\left(\alpha^b_{n+1}-m\right)}{\norm{\beeta}}\left(\mb{I}-\bn\bn^\mT\diag{\bc}\right)
\end{bmatrix},
\end{gather}
\begin{gather}
\pdfrac{\mb{R}}{\balpha}=\begin{bmatrix}
p~\bn^\mT\circ\bc^\mT\\[5mm]
KA\gamma\left(\left(1+\dfrac{p}{\norm{\beeta}}\left(\balpha:\bn\right)\right)\bn-\dfrac{p}{\norm{\beeta}}\balpha\right)^\mT\circ\bc^\mT\\[5mm]
2G\gamma\dfrac{p}{\norm{\beeta}}\left(\mb{I}-\bn\bn^\mT\diag{\bc}\right)\\[5mm]
\dfrac{\gamma{}h\left(\alpha^b_{n+1}-m\right)p}{\norm{\beeta}}\left(\mb{I}-\bn\bn^\mT\diag{\bc}\right)-\left(\gamma{}h+1\right)\mb{I}
\end{bmatrix}.
\end{gather}

For simplicity, the double contraction is not explicitly expressed, one shall always use $\mb{A}^\mT\diag{\mb{c}}\mb{B}$ (in compressed matrix representation) for $\mb{A}:\mb{B}$. The Jacobian has a size of $1+1+6+6=14$.
\subsubsection{Consistent Tangent Stiffness}
The consistent tangent stiffness can be derived from the decomposition
\begin{gather}
\bsigma_{n+1}=\bs_{n+1}+p_{n+1}\mb{1}.
\end{gather}
From which,
\begin{gather}\label{eq:ss_stiffness}
\pdfrac{\bsigma_{n+1}}{\bvarepsilon_{n+1}}=\pdfrac{\bs_{n+1}}{\bvarepsilon_{n+1}}+\mb{1}\otimes\pdfrac{p_{n+1}}{\bvarepsilon_{n+1}}.
\end{gather}
Since $\bs$ and $p$ are local independent variables, from local residual,
\begin{gather}
\pdfrac{\mb{R}}{\bvarepsilon_{n+1}}+\pdfrac{\mb{R}}{\mb{x}}\pdfrac{\mb{x}}{\bvarepsilon_{n+1}}=\mb{0},
\end{gather}
then
\begin{gather}
\pdfrac{\mb{x}}{\bvarepsilon_{n+1}}=\begin{bmatrix}
\pdfrac{\gamma}{\bvarepsilon_{n+1}}\\[4mm]
\pdfrac{p_{n+1}}{\bvarepsilon_{n+1}}\\[4mm]
\pdfrac{\bs_{n+1}}{\bvarepsilon_{n+1}}\\[4mm]
\pdfrac{\balpha_{n+1}}{\bvarepsilon_{n+1}}
\end{bmatrix}=-\left(\pdfrac{\mb{R}}{\mb{x}}\right)^{-1}\pdfrac{\mb{R}}{\bvarepsilon_{n+1}}.
\end{gather}

From the above equation, $\pdfrac{\bs_{n+1}}{\bvarepsilon_{n+1}}$ and $\pdfrac{p_{n+1}}{\bvarepsilon_{n+1}}$ can be extracted. The term $\pdfrac{\mb{R}}{\bvarepsilon_{n+1}}$ can be computed as
\begin{gather}
\pdfrac{\mb{R}}{\bvarepsilon_{n+1}}=\begin{bmatrix}
0\\[4mm]
\left(KA\gamma\alpha^dn^dv_0-K\right)\mb{1}^\mT\\[4mm]
-2G\mathbb{I}^\text{dev}\\[4mm]
-\gamma{}h\alpha^bn^bv_0\mb{1}^\mT
\end{bmatrix}
\end{gather}
\subsection{Implementation}
The implementation of this simple sand model is presented in \algoref{algo:ss_model}. Please note since the local iteration uses $\bs$ and $p$ as independent variables, the plastic strain history does not need to be explicitly stored. It can always be recovered via $\bvarepsilon^p=\bvarepsilon-\mb{D}^{-1}:\sigma$.
\begin{breakablealgorithm}
\caption{state determination of simple sand model}\label{algo:ss_model}
\begin{algorithmic}
\State \textbf{Parameter}: $K$, $G$, $A$, $m$, $h$, $n^d$, $n^b$, $\alpha_c$, $v_0$, $v_c$, $\lambda_c$, $p_c$
\State \textbf{Input}: $\bvarepsilon_{n+1}$, $\bvarepsilon_n$, $\bsigma_n$, $\balpha_n$
\State \textbf{Output}: $\mb{D}_{n+1}$, $\bsigma_{n+1}$, $\balpha_{n+1}$
\State $\bsigma^\text{trial}=\bsigma_n+\mb{D}:\left(\bvarepsilon_{n+1}-\bvarepsilon_n\right)$
\State $\bs^\text{trial}=\dev{\bsigma^\text{trial}}$
\State $p^\text{trial}=\dfrac{1}{3}\tr{\bsigma^\text{trial}}$
\State $\beeta^\text{trial}=\bs^\text{trial}+p^\text{trial}\balpha_n$
\State compute $f^\text{trial}$\Comment{\eqsref{eq:ss_yield}}
\If {$f^\text{trial}\geqslant0$}
\State $\gamma=0$
\State $p_{n+1}=p^\text{trial}$
\State $\bs_{n+1}=\bs^\text{trial}$
\State $\balpha_{n+1}=\balpha^\text{trial}$
\While{true}
\State  $\bn=\dfrac{\beeta_{n+1}}{\norm{\beeta_{n+1}}}$
\State compute $\mb{R}$ and $\pdfrac{\mb{R}}{\mb{x}}$\Comment{\eqsref{eq:ss_f} and \eqsref{eq:ss_df}}
\State $\Delta\mb{x}=\left(\pdfrac{\mb{R}}{\mb{x}}\right)^{-1}\mb{R}$
\If {$\norm{\Delta\mb{x}}<\text{tolerance}$}
\State break
\EndIf
\State update $\gamma$, $p_{n+1}$, $\bs_{n+1}$ and $\balpha_{n+1}$ using the increment $\Delta\mb{x}$
\EndWhile\Comment{Once this while exists, $p_{n+1}$, $\bs_{n+1}$ and $\balpha_{n+1}$ are all new states.}
\State $\bsigma_{n+1}=\bs_{n+1}+p_{n+1}\mb{1}$
\State compute $\mb{D}_{n+1}$\Comment{\eqsref{eq:ss_stiffness}}
\Else
\State $\bsigma_{n+1}=\bsigma^\text{trial}$
\State $\balpha^p_{n+1}=\balpha^p_n$
\State $\mb{D}_{n+1}=\mb{D}$
\EndIf
\end{algorithmic}
\end{breakablealgorithm}

It can be seen that with such a framework, the state determination can be kept quite concise.
\section{Dafalias--Manzari Sand Model}
\subsection{Theory}
\subsubsection{Hyperelasticity}
The hyperelastic response is defined as
\begin{gather}
G=G_0\dfrac{\left(2.97-e\right)^2}{1+e}\sqrt{pp_{at}},\qquad
K=\dfrac{2}{3}\dfrac{1+\nu}{1-2\nu}G.
\end{gather}
The corresponding derivatives are
\begin{gather}
\pdfrac{G}{e}=G_0\sqrt{pp_{at}}\dfrac{e^2+2e-14.7609}{(1+e)^2},\qquad
\pdfrac{G}{p}=G_0\dfrac{\left(2.97-e\right)^2}{1+e}\dfrac{1}{2}\sqrt{\dfrac{p_{at}}{p}}.
\end{gather}
The void ratio can be associated to strain so that
\begin{gather}
e=e_0+\left(1+e_0\right)\tr{\bvarepsilon_{n+1}}.
\end{gather}

The strain increment can be decomposed into elastic and plastic parts.
\begin{gather}
\bvarepsilon_{n+1}=\bvarepsilon_n+\dot\bvarepsilon=\bvarepsilon_n+\dot{\bvarepsilon^{e}}+\dot{\bvarepsilon^{p}}.
\end{gather}
As such, the stress increment can be expressed accordingly,
\begin{gather}
\bsigma_{n+1}=\bsigma_n+\dot{\bsigma}=\bsigma_n+2G\left(\dot{\bvarepsilon^d}-\dot{\bvarepsilon^{d,p}}\right)+K\left(\dot{\varepsilon^v}-\dot{\varepsilon^{v,p}}\right)\mb{1}.
\end{gather}
We do not use trial state any more since in this model the elastic part is not linear.

In derivatoric and spherical components,
\begin{gather}
\bsigma_{n+1}=\bs_{n+1}+p_{n+1}\mb{1},\\[2mm]
p_{n+1}=p_n+K\left(\dot{\varepsilon^v}-\dot{\varepsilon^{v,p}}\right),\\[2mm]
\bs_{n+1}=\bs_n+2G\left(\dot{\bvarepsilon^d}-\dot{\bvarepsilon^{d,p}}\right),
\end{gather}
with
\begin{gather}
\dot{\bvarepsilon}=\dot{\bvarepsilon^d}+\dfrac{1}{3}\dot{\varepsilon^v}\mb{1},
\end{gather}
where $\bs=\dev{\bsigma}$ is the deviatoric stress, $p=\dfrac{1}{3}\tr{\bsigma}$ is the hydrostatic stress.
\subsubsection{Critical State}
The critical state parameter is chosen as
\begin{gather}
\psi=e-e_0+\lambda_c\left(\dfrac{p}{p_{at}}\right)^\xi.
\end{gather}
The derivatives are
\begin{gather}
\pdfrac{\psi}{e}=1,\qquad
\pdfrac{\psi}{p}=\lambda_c\xi\left(\dfrac{p}{p_{at}}\right)^{\xi-1}\dfrac{1}{p_{at}}.
\end{gather}

The dilatancy surface is defined as
\begin{gather}
\alpha^d=\alpha^c\exp\left(n^d\psi\right).
\end{gather}

The bounding surface is defined as
\begin{gather}
\alpha^b=\alpha^c\exp\left(-n^b\psi\right).
\end{gather}

The corresponding derivatives are
\begin{gather}
\pdfrac{\alpha^d}{\psi}=n^d\alpha^d,\qquad\pdfrac{\alpha^b}{\psi}=-n^b\alpha^b.
\end{gather}
\subsubsection{Yield Function}
A wedge-like function is chosen to be the yield surface.
\begin{gather}
F=\norm{\bs+p\balpha}+mp=\norm{\beeta}+mp,
\end{gather}
where $\balpha$ is the so called back stress and $m$ characterises the size of the wedge. For simplicity, $m$ is assumed to be a constant in this model.

By denoting $\beeta=\bs+p\balpha$, the directional unit tensor is defined as
\begin{gather}
\bn=\dfrac{\beeta}{\norm{\beeta}}.
\end{gather}
\subsubsection{Flow Rule}
A non-associated plastic flow is used, the corresponding flow rule is defined as follows.
\begin{gather}
\dot{\bvarepsilon^p}=\gamma\left(\bn+\dfrac{1}{3}D\mb{1}\right),
\end{gather}
where $D$ is the dilatancy parameter.
\begin{gather}
D=A_d\left(\alpha^d-m-\balpha:\bn\right)=A_0\left(1+\left\langle\mb{z}:\bn\right\rangle\right)\left(\alpha^d-m-\balpha:\bn\right).
\end{gather}

For $\mb{z}:\bn\geqslant0$,
\begin{align}
\pdfrac{D}{p}&=A_0\left(\alpha^d-m-\balpha:\bn\right)\left(\mb{z}:n_p\right)+A_0\left(1+\mb{z}:\bn\right)\left(\pdfrac{\alpha^d}{p}-\balpha:n_p\right),\\
\pdfrac{D}{\bs}&=A_0\left(\alpha^d-m-\balpha:\bn\right)\left(\mb{z}:n_s\right)-A_0\left(1+\mb{z}:\bn\right)\left(\balpha:n_s\right),\\
\pdfrac{D}{\balpha}&=A_0\left(\alpha^d-m-\balpha:\bn\right)\left(\mb{z}:n_\alpha\right)-A_0\left(1+\mb{z}:\bn\right)\left(\balpha:n_\alpha+\bn:\mathbb{I}\right),\\
\pdfrac{D}{\mb{z}}&=A_0\left(\alpha^d-m-\balpha:\bn\right)\left(\bn:\mathbb{I}\right).
\end{align}

For $\mb{z}:\bn<0$,
\begin{align}
\pdfrac{D}{p}&=A_0\left(\pdfrac{\alpha^d}{p}-\balpha:n_p\right),\\
\pdfrac{D}{\bs}&=-A_0\left(\balpha:n_s\right),\\
\pdfrac{D}{\balpha}&=-A_0\left(\balpha:n_\alpha+\bn:\mathbb{I}\right),\\
\pdfrac{D}{\mb{z}}&=\mb{0}.
\end{align}

In the above expressions, $n_p$, $n_s$ and $n_\alpha$ are partial derivatives of $\bn$ against $p$, $\bs$ and $\balpha$. Note due to the change of sign convention, a negative $D$ leads to contractive response. Thus, $A_0$ often needs to be negative.
\subsubsection{Hardening Law}
The evolution rate of the back stress $\balpha$ is defined in terms of a proper distance measure from the bounding surface,
\begin{gather}
\dot{\balpha}=\gamma{}h\left(\left(\alpha^b-m\right)\bn-\balpha\right),
\end{gather}
where $h$ controls the hardening rate,
\begin{gather}
h=b_0\exp\left(h_1\left(\balpha_{in}:\bn-\balpha:\bn\right)\right).
\end{gather}
The constant $\balpha_{in}$ is updated whenever load reversal occurs.

The parameter $b_0$ is defined as a function of current state,
\begin{gather}
b_0=G_0h_0\left(1-c_he\right)\sqrt{\dfrac{p_{at}}{p}}.
\end{gather}
The derivatives are
\begin{gather}
\pdfrac{b_0}{e}=-G_0h_0c_h\sqrt{\dfrac{p_{at}}{p}},\qquad
\pdfrac{b_0}{p}=-G_0h_0\left(1-c_he\right)\dfrac{\sqrt{pp_{at}}}{2p^2}=-\dfrac{b_0}{2p}.
\end{gather}
Hence,
\begin{align}
\pdfrac{h}{p}&=\pdfrac{b_0}{p}\exp\left(h_1\left(\balpha_{in}:\bn-\balpha:\bn\right)\right),\\
\pdfrac{h}{\bs}&=hh_1\left(\balpha_{in}-\balpha\right):n_s,\\
\pdfrac{h}{\balpha}&=hh_1\left(\left(\balpha_{in}-\balpha\right):n_a-\bn:\mathbb{I}\right).
\end{align}
\subsubsection{Fabric Effect}
The fabric tensor changes when $\Delta\varepsilon^p_v$ is positive,
\begin{gather}
\dot{\mb{z}}=c_z\left\langle\dot{\varepsilon^{v,p}}\right\rangle\left(z_m\bn-\mb{z}\right)=c_z\gamma\left\langle{}D\right\rangle\left(z_m\bn-\mb{z}\right).
\end{gather}
The sign is flipped compared to the original definition for consistency with other parts.
\subsection{Formulation}
The governing equations of this model are
\begin{table}[ht]
\centering
\renewcommand{\arraystretch}{1.5}
\begin{tabular}{rl}
\toprule
Constitutive Law&$\bsigma=\bsigma_n+2G\left(\dot{\bvarepsilon^d}-\dot{\bvarepsilon^{d,p}}\right)+K\left(\dot{\varepsilon^v}-\dot{\varepsilon^{v,p}}\right)\mb{1}$\\
Yield Function&$f=\norm{\bs+p\balpha}+mp=\norm{\mb{\eta}}+mp$\\
Flow Rule&$\dot{\bvarepsilon^p}=\gamma\left(\bn+\dfrac{1}{3}\left(A_0\left(1+\left\langle\mb{z}:\bn\right\rangle\right)\left(\alpha^d-m-\balpha:\bn\right)\right)\mb{1}\right)$\\
Hardening Law&$\dot{\balpha}=\gamma{}b_0\exp\left(h_1\left(\balpha_{in}:\bn-\balpha:\bn\right)\right)\left(\left(\alpha^b-m\right)\bn-\balpha\right)$\\
Fabric Effect&$\dot{\mb{z}}=c_z\left\langle\dot{\varepsilon^{v,p}}\right\rangle\left(z_m\bn-\mb{z}\right)$\\
Critical State&$\psi=e-e_0+\lambda_c\left(\dfrac{p}{p_{at}}\right)^\xi$\\
&$\alpha^d=\alpha^c\exp\left(n^d\psi\right)$\\
&$\alpha^b=\alpha^c\exp\left(-n^b\psi\right)$\\\bottomrule
\end{tabular}
\end{table}
\subsubsection{Elastic Residual}
The new state shall be computed assuming there is no plasticity. In which case,
\begin{gather}
p_{n+1}=p_n+K\dot{\varepsilon^v},\\
\bs_{n+1}=\bs_n+2G\dot{\bvarepsilon^d}.
\end{gather}
\subsubsection{Elastic Local Iteration}
The independent variables are chosen to be $\mb{x}=\begin{bmatrix}
p&\bs
\end{bmatrix}^\mT$, then the local residual is
\begin{gather}\label{eq:dm_er}
\mb{R}=\left\{
\begin{array}{l}
p_{n+1}-p_n-K\dot{\varepsilon^v},\\
\bs_{n+1}-\bs_n-2G\dot{\bvarepsilon^d}.
\end{array}
\right.
\end{gather}
The Jacobian can be expressed as
\begin{gather}\label{eq:dm_edr}
\mb{J}=\begin{bmatrix}
1-\dot{\varepsilon^v}\dfrac{2+2\nu}{3-6\nu}\pdfrac{G}{p}&\mb{0}\\[4mm]
-2\dot{\bvarepsilon^d}\pdfrac{G}{p}&\mb{I}
\end{bmatrix}.
\end{gather}
and
\begin{gather}\label{eq:dm_estiffness}
\pdfrac{\mb{R}}{\bvarepsilon_{n+1}}=
\begin{bmatrix}
-K\mb{1}^\mT-\dot{\varepsilon^v}\pdfrac{K}{\bvarepsilon_{n+1}}\\[4mm]
-2G\mathbb{I}^\text{dev}-2\dot{\bvarepsilon^d}\pdfrac{G}{\bvarepsilon_{n+1}}
\end{bmatrix},
\end{gather}
which will be used in computation of consistent tangent stiffness.

When elastic state is computed, the trial yield function can be calculated.
\begin{gather}\label{eq:dm_trial}
f^\text{trial}=\norm{\bs_{n+1}+p_{n+1}\balpha_n}+mp_{n+1}.
\end{gather}
\subsubsection{Plastic Residual}
The are five local residual equations.
\begin{gather}\label{eq:dm_r}
\mb{R}=\left\{\begin{array}{ll}
\norm{\beeta}+mp,\\[2mm]
p-p_n+K\left(\gamma{}D-\dot{\bvarepsilon^v}\right),\\[2mm]
\bs-\bs_n+2G\left(\gamma\bn-\dot{\bvarepsilon^d}\right),\\[2mm]
\balpha-\balpha_n+\gamma{}h\left(\balpha-\left(\alpha^b-m\right)\bn\right),\\[2mm]
\mb{z}-\mb{z}_n+c_z\gamma{}\left\langle{}D\right\rangle\left(\mb{z}-z_m\bn\right).
\end{array}
\right.
\end{gather}
\subsubsection{Plastic Local Iteration}
By choosing $\mb{x}=\begin{bmatrix}\gamma&p&\bs&\balpha&\mb{z}\end{bmatrix}^\mT$, the Jacobian consists of the entries that can be listed as follows.
\begin{gather}\label{eq:dm_dr}
\pdfrac{\mb{R}}{\mb{x}}=\begin{bmatrix}
\pdfrac{\mb{R}}{\gamma}&\pdfrac{\mb{R}}{p}&\pdfrac{\mb{R}}{\bs}&\pdfrac{\mb{R}}{\balpha}&\pdfrac{\mb{R}}{\mb{z}}
\end{bmatrix}
\end{gather}
with
\begin{gather}
\pdfrac{\mb{R}}{\gamma}=
\begin{bmatrix}
	\cdot                                           \\[3mm]
	DK                                              \\[3mm]
	2G\bn                                  \\[3mm]
	h\left(\balpha-\alpha^{bm}\bn\right)   \\[3mm]
	c_z\left\langle{}D\right\rangle{}\mb{z}_z
\end{bmatrix},
\end{gather}
\begin{gather}
\pdfrac{\mb{R}}{p}=
\begin{bmatrix}
	\eta_p+m                                                                                                                                \\[3mm]
	1+\pdfrac{K}{p}\left(\gamma{}D-\dot{\varepsilon^v}\right)+K\gamma\pdfrac{D}{p}                                                          \\[3mm]
	2\pdfrac{G}{p}\left(\gamma\bn-\dot{\bvarepsilon^d}\right)+2G\gamma{}n_p                                                          \\[3mm]
	\gamma{}\pdfrac{h}{p}\left(\balpha-\alpha^{bm}\bn\right)-\gamma{}h\left(\pdfrac{\alpha^b}{p}\bn+\alpha^{bm}n_p\right) \\[3mm]
	c_z\gamma{}\pdfrac{\left\langle{}D\right\rangle}{p}\mb{z}_z-c_z\gamma{}\left\langle{}D\right\rangle{}z_m{}n_p
\end{bmatrix},
\end{gather}
\begin{gather}
\pdfrac{\mb{R}}{\bs}=
\begin{bmatrix}
	\eta_s                                                                                                              \\[3mm]
	\gamma{}K\left(\pdfrac{D}{\bs}\right)^\mT                                                                                            \\[3mm]
	\mb{I}+2G\gamma{}n_s                                                                                            \\[3mm]
	\gamma\left(\balpha-\alpha^{bm}\bn\right)\otimes\pdfrac{h}{\bs}-\gamma{}h\alpha^{bm}n_s                           \\[3mm]
	c_z\gamma{}\mb{z}_z\otimes\pdfrac{\left\langle{}D\right\rangle}{\bs}-c_z\gamma{}\left\langle{}D\right\rangle{}z_mn_s
\end{bmatrix},
\end{gather}
\begin{gather}
\pdfrac{\mb{R}}{\balpha}=
\begin{bmatrix}
	\eta_\alpha                                                                                                                      \\[3mm]
	\gamma{}K\left(\pdfrac{D}{\balpha}\right)^\mT                                                                                                     \\[3mm]
	2G\gamma{}n_\alpha                                                                                                               \\[3mm]
	\left(1+\gamma{}h\right)\mb{I}+\gamma\left(\balpha-\alpha^{bm}\bn\right)\otimes\pdfrac{h}{\balpha}-\gamma{}h\alpha^{bm}n_a \\[3mm]
	c_z\gamma{}\mb{z}_z\otimes\pdfrac{\left\langle{}D\right\rangle}{\balpha}-c_z\gamma{}\left\langle{}D\right\rangle{}z_mn_\alpha
\end{bmatrix},
\end{gather}
\begin{gather}
\pdfrac{\mb{R}}{\mb{z}}=
\begin{bmatrix}
	\cdot                                                                                                                             \\[3mm]
	\gamma{}K\left(\pdfrac{D}{\mb{z}}\right)^\mT                                                                                                       \\[3mm]
	\cdot                                                                                                                             \\[3mm]
	\cdot                                                                                                                             \\[3mm]
	\left(1+c_z\gamma{}\left\langle{}D\right\rangle\right)\mb{I}+c_z\gamma{}\mb{z}_z\otimes\pdfrac{\left\langle{}D\right\rangle}{\mb{z}}
\end{bmatrix}.
\end{gather}
In which,
\begin{gather}
\alpha^{bm}=\alpha^b-m,\\
\mb{z}_z=\mb{z}-z_m\bn.
\end{gather}
And $\eta_p$, $\eta_s$ and $\eta_\alpha$ are partial derivatives of $\norm{\beeta}$ against $p$, $\bs$ and $\balpha$.
\subsubsection{Consistent Tangent Operator}
\begin{gather}
\pdfrac{\mb{R}}{\bvarepsilon_{n+1}}=
\begin{bmatrix}
\cdot\\[3mm]
\left(\left(\gamma{}D-\dot{\varepsilon^v}\right)\pdfrac{K}{\bvarepsilon_{n+1}}+K\left(\gamma{}\pdfrac{D}{\bvarepsilon_{n+1}}-\mb{1}\right)\right)^\mT,\\[3mm]
2\left(\gamma\bn-\dot{\bvarepsilon^d}\right)\otimes\pdfrac{G}{\bvarepsilon_{n+1}}-2G\mathbb{I}^\text{dev},\\[3mm]
\gamma{}\left(\balpha-\alpha^{bm}\bn\right)\otimes\pdfrac{h}{\bvarepsilon_{n+1}}-\gamma{}h\bn\otimes\pdfrac{\alpha^b}{\bvarepsilon_{n+1}},\\[3mm]
c_z\gamma{}\mb{z}_z\otimes\pdfrac{\left\langle{}D\right\rangle}{\bvarepsilon_{n+1}}
\end{bmatrix}.
\end{gather}

Then,
\begin{gather}
\pdfrac{\mb{x}}{\bvarepsilon_{n+1}}=-\mb{J}^{-1}\pdfrac{\mb{R}}{\bvarepsilon_{n+1}}.
\end{gather}

Similar to the previous procedure,
\begin{gather}\label{eq:dm_stiffness}
\pdfrac{\bsigma_{n+1}}{\bvarepsilon_{n+1}}=\pdfrac{\bs_{n+1}}{\bvarepsilon_{n+1}}+\mb{1}\otimes\pdfrac{p_{n+1}}{\bvarepsilon_{n+1}}.
\end{gather}
The corresponding quantities can be extracted and be used to formulate consistent tangent stiffness.
\subsection{Implementation}
\begin{breakablealgorithm}
\caption{state determination of Dafalias--Manzari sand model}\label{algo:dm_model}
\begin{algorithmic}
\State \textbf{Parameter}: $G_0$, $v$, $p_{at}$, $e_0$, $\lambda_c$, $\xi$, $n^d$, $n^b$, $\alpha_c$, $m$, $A_0$, $h_0$, $h_1$, $c_z$, $z_m$
\State \textbf{Input}: $\bvarepsilon_{n+1}$, $\bvarepsilon_n$, $\bsigma_n$, $\balpha_n$, $\mb{z}_n$
\State \textbf{Output}: $\mb{D}_{n+1}$, $\bsigma_{n+1}$, $\balpha_{n+1}$, $\mb{z}_{n+1}$
\State $\bs_{n+1}=\bs_n=\dev{\bsigma_n}$
\State $p_{n+1}=p_n=\dfrac{1}{3}\tr{\bsigma_n}$
\While{true}
\State compute $\mb{R}$ and $\pdfrac{\mb{R}}{\mb{x}}$\Comment{\eqsref{eq:dm_er} and \eqsref{eq:dm_edr}}
\State $\Delta\mb{x}=\left(\pdfrac{\mb{R}}{\mb{x}}\right)^{-1}\mb{R}$
\If {$\norm{\Delta{}\mb{x}}<\text{tolerance}$}
\State break
\EndIf
\State update $p_{n+1}$ and $\bs_{n+1}$ using the increment $\Delta\mb{x}$
\EndWhile\Comment{Once this while exists, $p_{n+1}$ and $\bs_{n+1}$ are new elastic states.}
\State compute $f^\text{trial}$\Comment{\eqsref{eq:dm_trial}}
\State $\balpha_{n+1}=\balpha_n$
\State $\mb{z}_{n+1}=\mb{z}_n$
\If {$f^\text{trial}\geqslant0$}\Comment{plastic}
\State $\gamma=0$
\While{true}
\State  $\bn=\dfrac{\beeta_{n+1}}{\norm{\beeta_{n+1}}}$
\State compute $\mb{R}$ and $\pdfrac{\mb{R}}{\mb{x}}$\Comment{\eqsref{eq:dm_r} and \eqsref{eq:dm_dr}}
\State $\Delta\mb{x}=\left(\pdfrac{\mb{R}}{\mb{x}}\right)^{-1}\mb{R}$
\If {$\norm{\Delta{}x}<\text{tolerance}$}
\State break
\EndIf
\State update $\gamma$, $p_{n+1}$, $\bs_{n+1}$, $\balpha_{n+1}$ and $\mb{z}_{n+1}$ using the increment $\Delta\mb{x}$
\EndWhile\Comment{Once this while exists, $p_{n+1}$, $\bs_{n+1}$, $\balpha_{n+1}$ and $\mb{z}_{n+1}$ are all new states.}
\State $\bsigma_{n+1}=\bs_{n+1}+p_{n+1}\mb{1}$
\State compute plastic $\mb{D}_{n+1}$\Comment{\eqsref{eq:dm_stiffness}}
\Else\Comment{elastic}
\State $\bsigma_{n+1}=\bs_{n+1}+p_{n+1}\mb{1}$
\State compute elastic $\mb{D}_{n+1}$\Comment{\eqsref{eq:dm_estiffness}}
\EndIf
\end{algorithmic}
\end{breakablealgorithm}

The Dafalias--Manzari sand model is probably the most complex model so far as it involves heavy computation of tensor quantities. Here is a example implementation of state determination.
\begin{cppcode}
int DafaliasManzari::update_trial_status(const vec& t_strain) {
    incre_strain = (trial_strain = t_strain) - current_strain;

    if(norm(incre_strain) <= tolerance) return SUANPAN_SUCCESS;

    const auto current_p = tensor::mean3(current_stress);
    const auto current_s = tensor::dev(current_stress);
    const auto incre_ev = tensor::trace3(incre_strain);
    const vec incre_ed = unit_dev_tensor * incre_strain;

    // assume no plasticity
    // compute trial stress

    auto p = current_p + pr * gi * incre_ev;
    vec s = current_s + 2. * gi * incre_ed;

    const auto void_ratio = e0 + (1. + e0) * tensor::trace3(trial_strain);
    const auto v_term_a = pow(2.97 - void_ratio, 2.) / (1. + void_ratio);
    const auto v_term_b = (void_ratio * (void_ratio + 2.) - 14.7609) * pow(1. + void_ratio, -2.) * (1. + e0);

    double g, pgpe, pgpp;

    vec residual(7, fill::none), incre;
    mat jacobian(7, 7, fill::eye);

    auto counter = 0u;
    auto ref_error = 1.;

    while(true) {
        if(max_iteration == ++counter) return SUANPAN_FAIL;

        const auto sqrt_term = shear_modulus * sqrt(std::max(datum::eps, pc * p));

        g = sqrt_term * v_term_a;

        if(g > gi) {
            pgpe = sqrt_term * v_term_b;
            pgpp = .5 * g / p;
        }
        else {
            g = gi;
            pgpe = pgpp = 0.;
        }

        residual(sa) = p - current_p - pr * g * incre_ev;
        residual(sb) = s - current_s - 2. * g * incre_ed;

        jacobian(sa, sa) = 1. - pr * incre_ev * pgpp;
        jacobian(sb, sa) = -2. * pgpp * incre_ed;

        if(!solve(incre, jacobian, residual)) return SUANPAN_FAIL;

        auto error = norm(residual);
        if(1 == counter) ref_error = std::max(1., error);
        suanpan_debug("DafaliasManzari local elastic iteration error: %.5E.\n", error /= ref_error);
        if(error <= tolerance) break;

        p -= incre(sa);
        s -= incre(sb);
    }

    // check if yield

    const vec current_alpha(&current_history(0), 6, false, true);

    vec eta = s + p * current_alpha;
    auto norm_eta = tensor::stress::norm(eta);

    if(norm_eta + m * p < 0.) {
        trial_stress = s + p * tensor::unit_tensor2;

        mat left(7, 6, fill::none), right;

        left.row(sa) = pr * (incre_ev * pgpe + g) * tensor::unit_tensor2.t();
        left.rows(sb) = 2. * pgpe * incre_ed * tensor::unit_tensor2.t() + 2. * g * unit_dev_tensor;

        if(!solve(right, jacobian, left)) return SUANPAN_FAIL;

        trial_stiffness = right.rows(sb);
        trial_stiffness.row(0) += right.row(sa);
        trial_stiffness.row(1) += right.row(sa);
        trial_stiffness.row(2) += right.row(sa);

        return SUANPAN_SUCCESS;
    }

    // yield function violated

    const vec current_z(&current_history(6), 6, false, true);

    trial_history = current_history;
    vec alpha(&trial_history(0), 6, false, true);
    vec z(&trial_history(6), 6, false, true);
    vec ini_alpha(&trial_history(12), 6, false, true);

    residual.set_size(20);
    jacobian.set_size(20, 20);
    jacobian(si, si) = 0.;
    jacobian(si, sm).zeros();
    jacobian(sk, sm).zeros();
    jacobian(sl, sm).zeros();

    counter = 0u;

    vec n, zz, aabmn;
    auto gamma = 0.;
    double pabpe, d, pdpe, h, phpe;
    auto update_ini_alpha = false;

    while(true) {
        if(max_iteration == ++counter) return SUANPAN_FAIL;

        // shear modulus

        auto tmp_term = shear_modulus * sqrt(std::max(datum::eps, pc * p));
        g = tmp_term * v_term_a;

        if(g > gi) {
            pgpe = tmp_term * v_term_b;
            pgpp = .5 * g / p;
        }
        else {
            g = gi;
            pgpe = pgpp = 0.;
        }

        // state parameter

        tmp_term = lc * pow(std::max(datum::eps, p / pc), xi);
        const auto psi = void_ratio - e0 + tmp_term;
        const auto ppsipp = xi * tmp_term / p;

        // surface

        const auto ad = ac * exp(nd * psi);
        const auto ab = ac * exp(-nb * psi);
        const auto adm = ad - m;
        const auto abm = ab - m;

        auto padpe = nd * ad;
        pabpe = -nb * ab;
        const auto padpp = padpe * ppsipp;
        const auto pabpp = pabpe * ppsipp;
        padpe *= 1. + e0;
        pabpe *= 1. + e0;

        // yield function

        eta = s + p * alpha;
        norm_eta = tensor::stress::norm(eta);

        n = eta / norm_eta;
        const vec unit_n = n % tensor::stress::norm_weight;
        const vec unit_alpha = alpha % tensor::stress::norm_weight;
        const auto alpha_n = dot(n, unit_alpha);
        aabmn = alpha - abm * n;

        const vec np = (alpha - alpha_n * n) / norm_eta;
        const mat ns = (eye(6, 6) - n * unit_n.t()) / norm_eta;

        // dilatancy

        const vec unit_z = z % tensor::stress::norm_weight;
        const auto zn = dot(n, unit_z);

        d = a * (adm - alpha_n);

        double pdpp;
        rowvec pdps, pdpa, pdpz;
        if(zn > 0.) {
            const auto term_a = a * (1. + zn);

            pdpe = term_a * padpe;
            pdpp = term_a * padpp + dot(d * unit_z - term_a * unit_alpha, np);
            pdps = (d * unit_z - term_a * unit_alpha).t() * ns;
            pdpa = p * pdps - term_a * unit_n.t();
            pdpz = d * unit_n.t();

            d *= 1. + zn;
        }
        else {
            pdpe = a * padpe;
            pdpp = a * (padpp - dot(unit_alpha, np));
            pdps = -a * unit_alpha.t() * ns;
            pdpa = p * pdps - a * unit_n.t();
            pdpz.zeros(6);
        }

        // hardening

        tmp_term = shear_modulus * h0 * sqrt(std::max(datum::eps, pc / p));
        const auto b0 = tmp_term * (1. - ch * void_ratio);
        const auto pb0pe = -ch * tmp_term * (1. + e0);
        const auto pb0pp = -.5 * b0 / p;

        update_ini_alpha = false;
        vec diff_alpha = (ini_alpha - alpha) % tensor::stress::norm_weight;
        tmp_term = exp(h1 * dot(diff_alpha, n));

        if(tmp_term > 1.) {
            update_ini_alpha = true;
            diff_alpha = (current_alpha - alpha) % tensor::stress::norm_weight;
            tmp_term = exp(h1 * dot(diff_alpha, n));
        }

        h = tmp_term * b0;

        phpe = tmp_term * pb0pe;
        const auto phpp = tmp_term * pb0pp;
        const rowvec phps = h * h1 * diff_alpha.t() * ns;
        const rowvec phpa = p * phps - h * h1 * unit_n.t();

        // local iteration

        residual(si) = norm_eta + m * p;
        residual(sj) = p - current_p + pr * g * (gamma * d - incre_ev);
        residual(sk) = s - current_s + 2. * g * (gamma * n - incre_ed);
        residual(sl) = alpha - current_alpha + gamma * h * aabmn;
        residual(sm) = z - current_z;

        jacobian(si, sj) = alpha_n + m;
        jacobian(si, sk) = unit_n.t();
        jacobian(si, sl) = p * jacobian(si, sk);

        const auto gk = gamma * pr * g;
        jacobian(sj, si) = d * pr * g;
        jacobian(sj, sj) = 1. + pr * pgpp * (gamma * d - incre_ev) + gk * pdpp;
        jacobian(sj, sk) = gk * pdps;
        jacobian(sj, sl) = gk * pdpa;
        jacobian(sj, sm) = gk * pdpz;

        jacobian(sk, si) = 2. * g * n;
        jacobian(sk, sj) = 2. * pgpp * (gamma * n - incre_ed) + 2. * g * gamma * np;
        jacobian(sk, sk) = 2. * g * gamma * ns;
        jacobian(sk, sl) = p * jacobian(sk, sk);
        jacobian(sk, sk) += eye(6, 6);

        jacobian(sl, si) = h * aabmn;
        jacobian(sl, sj) = gamma * phpp * aabmn - gamma * h * (pabpp * n + abm * np);
        jacobian(sl, sk) = gamma * aabmn * phps - gamma * h * abm * ns;
        jacobian(sl, sl) = (1. + gamma * h) * eye(6, 6) + gamma * aabmn * phpa - gamma * h * abm * p * ns;

        jacobian(sm, sm) = eye(6, 6);

        if(d > 0.) {
            const auto factor_a = cz * gamma;
            const auto factor_b = factor_a * d;
            const auto factor_c = factor_b * zm;

            zz = z - zm * n;

            residual(sm) += factor_b * zz;

            jacobian(sm, si) = cz * d * zz;
            jacobian(sm, sj) = factor_a * pdpp * zz - factor_c * np;
            jacobian(sm, sk) = factor_a * zz * pdps - factor_c * ns;
            jacobian(sm, sl) = factor_a * zz * pdpa - factor_c * p * ns;
            jacobian(sm, sm) += factor_b * eye(6, 6) + factor_a * zz * pdpz;
        }
        else {
            jacobian(sm, si).zeros();
            jacobian(sm, sj).zeros();
            jacobian(sm, sk).zeros();
            jacobian(sm, sl).zeros();
        }

        if(!solve(incre, jacobian, residual)) return SUANPAN_FAIL;

        auto error = norm(residual);
        if(1 == counter) ref_error = std::max(1., error);
        suanpan_debug("DafaliasManzari local plastic iteration error: %.5E.\n", error /= ref_error);
        if(error <= tolerance) break;

        gamma -= incre(si);
        p -= incre(sj);
        s -= incre(sk);
        alpha -= incre(sl);
        z -= incre(sm);
    }

    trial_stress = s + p * tensor::unit_tensor2;

    mat left(20, 6, fill::none), right;

    left.row(si).zeros();
    left.row(sj) = pr * (pgpe * (incre_ev - gamma * d) + g - g * gamma * pdpe) * tensor::unit_tensor2.t();
    left.rows(sk) = 2. * g * unit_dev_tensor + 2. * pgpe * (incre_ed - gamma * n) * tensor::unit_tensor2.t();
    left.rows(sl) = (gamma * h * n * pabpe - gamma * aabmn * phpe) * tensor::unit_tensor2.t();

    if(d > 0.) left.rows(sm) = -cz * gamma * pdpe * zz * tensor::unit_tensor2.t();
    else left.rows(sm).zeros();

    if(!solve(right, jacobian, left)) return SUANPAN_FAIL;

    trial_stiffness = right.rows(sk);
    trial_stiffness.row(0) += right.row(sj);
    trial_stiffness.row(1) += right.row(sj);
    trial_stiffness.row(2) += right.row(sj);

    if(update_ini_alpha) ini_alpha = current_alpha;

    return SUANPAN_SUCCESS;
}
\end{cppcode}
\section{Duncan Soil Model}
In this section, we present a plane strain soil model.
Unlike other models that are typically strain--driven, it is a non-linear elastic phenomenological stress--driven model.
\subsection{Theory}
The incremental form of the constitutive relationship is defined as
\begin{gather}\label{eq:duncan_incremental}
\dot{\bsigma}=\mb{D}\dot{\bvarepsilon}=\dfrac{3B}{9B-E}\begin{bmatrix}
3B+E&3B-E&0\\
3B-E&3B+E&0\\
0&0&E\\
\end{bmatrix}\dot{\bvarepsilon}
\end{gather}
where $\bvarepsilon=\begin{bmatrix}\varepsilon_{x}&\varepsilon_{y}&\gamma_{xy}\end{bmatrix}^\mT$ and $\bsigma=\begin{bmatrix}\sigma_{x}&\sigma_{y}&\tau_{xy}\end{bmatrix}^\mT$ are the strain and stress vector for conditions of plane strain.
As the original model is non-linear elastic, it can be fully defined by the stress state.
Thus, the elastic modulus $E$ and the bulk modulus $K$ can be determined solely by the stress vector $\bsigma$.

Before proceeding, one shall note that the stress vector $\bsigma$ can be converted into principal stresses via either Mohr's circle or eigenanalysis.
\begin{gather}
\begin{array}{rr}
\sigma_1\\
\sigma_3
\end{array}=\dfrac{\sigma_x+\sigma_y}{2}\pm\sqrt{\left(\dfrac{\sigma_x-\sigma_y}{2}\right)^2+\tau_{xy}^2}.
\end{gather}
In the context of geotechnical engineering, typically $\abs{\sigma_1}>\abs{\sigma_3}$ and both take compression as positive.
Depending on how $\bsigma$ is defined, the signs presented in the above expression may be flipped.

The corresponding derivatives $\ddfrac{\sigma_1}{\bsigma}$ and $\ddfrac{\sigma_3}{\bsigma}$ can be determined accordingly.

The difference between $\sigma_1$ and $\sigma_3$ is taken as the deviatoric stress.
\begin{gather}
\sigma_d=\abs{\sigma_1-\sigma_3}.
\end{gather}
\subsubsection{Elastic Modulus}
The elastic modulus $E$ can be expressed as the product of two components, namely, the initial part and the reduction part.
\begin{gather}\label{eq:duncan_reduction}
E=E_i\left(1-\dfrac{\sigma_d}{\sigma_d^{\text{max}}}\right)^2.
\end{gather}

The initial part $E_i$ increases when $\sigma_3$ grows larger.
\begin{gather}
E_i=E_r\abs{\dfrac{\sigma_3}{p_a}}^n,
\end{gather}
where $E_r$ is the reference elastic modulus, $p_a$ is the atmospheric pressure, and $n$ is a model parameter that is typically between zero and unity.

The reduction part physically implies that the elastic modulus decreases when the current deviatoric stress approaches the maximum deviatoric stress.
This is effectively the Mohr--Coulomb failure criterion.
As a result, the deviatoric stress would asymptotically converge to the maximum deviatoric stress.
\begin{gather}
\sigma_d^\text{max}=\dfrac{\sigma_d^\text{ult}}{r_f},
\end{gather}
with the ultimate deviatoric stress $\sigma_d^\text{ult}$ being defined as
\begin{gather}
\sigma_d^\text{ult}=2\dfrac{c\cos\phi+\sigma_3\sin\phi}{1-\sin\phi}.
\end{gather}
It can be solved graphically using the following trigonometry.
\begin{gather}
\left(\sigma_d^\text{ult}/2+\sigma_3-c\tan\phi\right)\sin\phi=\sigma_d^\text{ult}/2-\dfrac{c}{\cos\phi}.
\end{gather}
\begin{figure}[ht]
\centering
\begin{tikzpicture}
\draw[->](-1,0)--++(10,0)node[below]{$\sigma$};
\draw[->](0,-.5)--++(0,4)node[right]{$\tau$};
\draw(0,0.86602540378)--++(6,3.46410161514);
\draw[dashed](0,0.86602540378)--++(4,0)--++(0,-0.86602540378);
\draw(7.5,0)arc(0:180:3);
\draw(4.5,0)--(3,2.59807621135);
\draw[|<->|](0,-.4)--++(1.5,0)node[midway,fill=white]{$\sigma_3$};
\draw[|<->|](1.5,-.4)--++(3,0)node[midway,fill=white]{$\sigma_d^\text{ult}/2$};
\draw[|<->|](-.3,0)--++(0,0.86602540378)node[midway,fill=white]{$c$};
\node at($(0,0.86602540378)+(15:.8)$){\scriptsize$\phi$};
\node at($(4,0.86602540378)+(-75:.5)$){\scriptsize$\phi$};
\end{tikzpicture}
\caption{$\sigma_d^\text{ult}$ determined by Mohr's circle}
\end{figure}
The ratio $r_f<1$ is introduced to fine--tune the behaviour.
\subsubsection{Bulk Modulus}
The bulk modulus $B$ takes a similar approach. However, it only monotonically increases with an increasing $\sigma_3$.
\begin{gather}
B=B_i=B_r\abs{\dfrac{\sigma_3}{p_a}}^m,
\end{gather}
where $B_r$ is the reference bulk modulus and $m$ is another modal parameter similar to $n$.
\subsection{Formulation}
\subsubsection{Local Iteration}
The incremental form \eqsref{eq:duncan_incremental} resembles the elastic Hooke's law.
However, since $E$ and $B$ rely on $\bsigma$, the consistent tangent stiffness requires the computation of the full derivative.
\begin{gather}
\ddfrac{E}{\bsigma}=\pdfrac{E}{\sigma_1}\ddfrac{\sigma_1}{\bsigma}+\pdfrac{E}{\sigma_3}\ddfrac{\sigma_3}{\bsigma},\qquad
\ddfrac{B}{\bsigma}=\ddfrac{E}{\sigma_3}\ddfrac{\sigma_3}{\bsigma}.
\end{gather}

Adopting the implicit Euler method, the local residual can be expressed as
\begin{gather}
\mb{R}=\bsigma-\bsigma_n-\mb{D}\Delta\bvarepsilon.
\end{gather}
The Jacobian can be derived as
\begin{gather}
\mb{J}=\pdfrac{\mb{R}}{\bsigma}=\mb{I}-\ddfrac{\mb{D}}{\bsigma}\Delta\bvarepsilon.
\end{gather}

The consistent tangent stiffness is thus
\begin{gather}
\ddfrac{\bsigma}{\bvarepsilon}=\mb{C}=\mb{J}^{-1}\mb{D}.
\end{gather}
One shall note the following two points.
\begin{enumerate}
\item The derivative $\ddfrac{\mb{D}}{\bsigma}$ is a cube, since $\mb{D}$ is a matrix.
\item Although \eqsref{eq:duncan_incremental} is symmetric and resembles the elastic Hooke's law, the consistent tangent is not symmetric.
\end{enumerate}
\subsubsection{Plasticity}
The original model is a non-linear elastic model, implying no plasticity whatsoever.
Similar to the concept discussed in \cite{Katona2015}, to enable plasticity--like behaviour, one could introduce the concept of yield surface. Since the Mohr--Coulomb failure criterion is used in \eqsref{eq:duncan_reduction}, the yield function $f\left(\bsigma\right)$ can be simply chosen as
\begin{gather}
f\left(\bsigma\right)=\sigma_d-\sigma_d^\text{max}.
\end{gather}
Similar to conventional plasticity theory based models, a return mapping style algorithm can be applied.

The maximum $\sigma_d^\text{max}$ shall be recorded, and it is only updated when local iteration converges and the converged stress yields a larger $\sigma_d$.
Then, whenever $\sigma_d>\sigma_d^\text{max}$, $f>0$ indicating the development of plasticity, the elastic modulus shall be computed according to \eqsref{eq:duncan_reduction}.
For $\sigma_d\leqslant\sigma_d^\text{max}$, $f<0$ indicating elastic loading/unloading, then the reduction in $E$ is not considered, viz., $E=E_i$.
\subsection{Implementation}
The state determination algorithm is shown in \algoref{algo:duncan_model}.
One must note the fact that, since the model is stress driven, it is not as straightforward to obtain plastic strain alike quantities as in conventional strain driven models.
If one wants, the plastic strain can be defined as the residual strain when fully unloaded (strain is zero).
However, due to its non-linearity nature, it is not guaranteed that the development of plasticity is irreversible, which mainly depends on model parameters.
\begin{breakablealgorithm}
\caption{state determination of Duncan soil model}\label{algo:duncan_model}
\begin{algorithmic}
\State \textbf{Parameter}: $p_a$, $E_r$, $n$, $B_r$, $m$, $\phi_i$, $\Delta\phi$, $r_f$, $c$
\State \textbf{Input}: $\bvarepsilon_{n+1}$, $\bvarepsilon_n$, $\bsigma_n$, $\sigma_d^{\text{max}}$
\State \textbf{Output}: $\mb{D}_{n+1}$, $\bsigma_{n+1}$
\State $\bsigma_{n+1}=\bsigma_n$
\While{true}
\State compute $\sigma^d$ based on $\bsigma_{n+1}$
\If {$\sigma^d\geqslant\sigma_d^{\text{max}}$}
\State compute $E$ and $B$\Comment{with reduction}
\Else
\State compute $E$ and $B$\Comment{without reduction}
\EndIf
\State compute $\mb{D}_{n+1}$ based on $E$ and $B$
\State compute $\mb{R}$ and $\mb{J}$
\State $\Delta\bsigma=\mb{J}^{-1}\mb{R}$
\If {convergence}
\State update $\sigma_d^{\text{max}}$ if $\sigma_d$ exceeds $\sigma_d^{\text{max}}$
\State $\mb{C}=\mb{J}^{-1}\mb{D}_{n+1}$
\State \Return
\EndIf
\State $\bsigma_{n+1}\leftarrow\bsigma_{n+1}-\Delta\bsigma$
\EndWhile
\end{algorithmic}
\end{breakablealgorithm}

The main body of the implementation can be shown as follows.
\begin{cppcode}
int DuncanSelig::update_trial_status(const vec& t_strain) {
    incre_strain = (trial_strain = t_strain) - current_strain;

    if(norm(incre_strain) <= datum::eps) return SUANPAN_SUCCESS;

    trial_history = current_history;
    auto& max_dev_stress = trial_history(0);

    trial_stress = current_stress + (trial_stiffness = current_stiffness) * incre_strain;

    auto ref_error = 0.;
    vec3 incre;

    auto counter = 0u;
    while(true) {
        if(max_iteration == ++counter) {
            suanpan_error("Cannot converge within {} iterations.\n", max_iteration);
            return SUANPAN_FAIL;
        }

        const auto trial_dev_stress = dev(trial_stress);

        const auto branch = trial_dev_stress > max_dev_stress ? &DuncanSelig::compute_plastic_moduli : &DuncanSelig::compute_elastic_moduli;

        const auto [elastic, bulk, deds, dkds] = (this->*branch)();

        const auto factor_a = 3. * bulk * (3. * bulk + elastic);
        const auto factor_b = 3. * bulk * (3. * bulk - elastic);

        mat33 right(fill::zeros);
        right(0, 0) = right(1, 1) = factor_a;
        right(0, 1) = right(1, 0) = factor_b;
        right(2, 2) = 3. * bulk * elastic;

        const vec3 residual = (9. * bulk - elastic) * (trial_stress - current_stress) - right * incre_strain;

        const rowvec3 dfads = (18. * bulk + 3. * elastic) * dkds + 3. * bulk * deds;
        const rowvec3 dfbds = (18. * bulk - 3. * elastic) * dkds - 3. * bulk * deds;

        mat33 jacobian = (9. * bulk - elastic) * eye(3, 3) + (trial_stress - current_stress) * (9. * dkds - deds);
        jacobian.row(0) -= incre_strain(0) * dfads + incre_strain(1) * dfbds;
        jacobian.row(1) -= incre_strain(0) * dfbds + incre_strain(1) * dfads;
        jacobian.row(2) -= incre_strain(2) * 3. * (bulk * deds + elastic * dkds);

        if(!solve(incre, jacobian, residual)) return SUANPAN_FAIL;

        const auto error = inf_norm(incre);
        if(1u == counter) ref_error = error;
        suanpan_debug("Local iteration error: {:.5E}.\n", error / ref_error);
        if(error < tolerance * ref_error || ((error < tolerance || inf_norm(residual) < tolerance) && counter > 5u)) {
            if(!solve(trial_stiffness, jacobian, right)) return SUANPAN_FAIL;

            if(trial_dev_stress > max_dev_stress) max_dev_stress = trial_dev_stress;

            return SUANPAN_SUCCESS;
        }

        trial_stress -= incre;
    }
}
\end{cppcode}

The computation of two moduli follows the formulation with extra cumbersome computation of derivatives.
To improve numerical robustness, additional limits are applied. Stress components are bounded to the atmospheric pressure.
\begin{cppcode}
std::tuple<double, double> DuncanSelig::compute_elastic(const double s3) const {
    double elastic, deds3;
    if(s3 < -min_ratio * p_atm) {
        elastic = ref_elastic * std::pow(.01, n);
        deds3 = 0.;
    }
    else if(s3 < min_ratio * p_atm) {
        elastic = ref_elastic * std::pow(min_ratio, n);
        deds3 = 0.;
    }
    else {
        elastic = ref_elastic * std::pow(s3 / p_atm, n);
        deds3 = n * elastic / s3;
    }

    return {elastic, deds3};
}

std::tuple<double, double> DuncanSelig::compute_bulk(const double s3) const {
    double bulk, dkds3;
    if(s3 < -min_ratio * p_atm) {
        bulk = ref_bulk * std::pow(.01, m);
        dkds3 = 0.;
    }
    else if(s3 < min_ratio * p_atm) {
        bulk = ref_bulk * std::pow(min_ratio, m);
        dkds3 = 0.;
    }
    else {
        bulk = ref_bulk * std::pow(s3 / p_atm, m);
        dkds3 = m * bulk / s3;
    }

    return {bulk, dkds3};
}

std::tuple<double, double, rowvec3, rowvec3> DuncanSelig::compute_elastic_moduli() {
    // principal stresses

    const auto radius = .5 * dev(trial_stress);
    rowvec3 drds(fill::zeros);
    if(radius > datum::eps) drds = der_dev(trial_stress) / radius * .25;

    const auto center = -.5 * (trial_stress(0) + trial_stress(1));
    const rowvec3 dcds{-.5, -.5, 0.};

    const auto s3 = center - radius;

    const rowvec3 ds3ds = dcds - drds;

    // for elastic modulus

    const auto [elastic, deds3] = compute_elastic(s3);

    const rowvec3 deds = deds3 * ds3ds;

    // for bulk modulus

    auto [bulk, dkds3] = compute_bulk(s3);

    rowvec3 dkds = dkds3 * ds3ds;

    if(3. * bulk < elastic) {
        bulk = elastic / 3.;
        dkds = deds / 3.;
    }

    return {elastic, bulk, deds, dkds};
}

std::tuple<double, double, rowvec3, rowvec3> DuncanSelig::compute_plastic_moduli() {
    // principal stresses

    const auto center = -.5 * (trial_stress(0) + trial_stress(1));
    const rowvec3 dcds{-.5, -.5, 0.};

    auto radius = .5 * dev(trial_stress);
    rowvec3 drds(fill::zeros);
    if(radius > datum::eps) drds = der_dev(trial_stress) / radius * .25;

    const auto s1 = center + radius, s3 = center - radius;

    const rowvec3 ds1ds = dcds + drds, ds3ds = dcds - drds;

    // for elastic modulus

    double phi, dphids3;
    if(s3 < p_atm) {
        phi = ini_phi;
        dphids3 = 0.;
    }
    else {
        phi = ini_phi - ten_fold_phi_diff * log10(s3 / p_atm);
        dphids3 = -ten_fold_phi_diff / (s3 * log(10));
        if(phi < 0.) phi = dphids3 = 0.;
    }

    const auto denom = 1. - std::sin(phi);
    auto max_dev_stress = 2. / r_f * (cohesion * std::cos(phi) + s3 * std::sin(phi)) / denom;
    auto dmdsds3 = 0.;
    if(max_dev_stress > min_ratio * p_atm) {
        const auto pmdspphi = 2. / r_f * (s3 * std::cos(phi) / denom / denom + cohesion / denom);
        const auto pmdsps3 = 2. / r_f * std::sin(phi) / denom;
        dmdsds3 = pmdspphi * dphids3 + pmdsps3;
    }
    else max_dev_stress = min_ratio * p_atm;

    const auto dev_stress = s1 - s3;
    const auto pdsps1 = 1.;
    const auto pdsps3 = -1.;

    const auto [ini_elastic, deids3] = compute_elastic(s3);

    const auto pepei = std::pow(1. - dev_stress / max_dev_stress, 2.);
    const auto elastic = ini_elastic * pepei;
    const auto pepds = -2. * ini_elastic * (1. - dev_stress / max_dev_stress) / max_dev_stress;
    const auto pepmds = 2. * ini_elastic * (1. - dev_stress / max_dev_stress) * dev_stress / max_dev_stress / max_dev_stress;

    const auto peps1 = pepds * pdsps1;
    const auto peps3 = pepei * deids3 + pepds * pdsps3 + pepmds * dmdsds3;

    const rowvec3 deds = peps1 * ds1ds + peps3 * ds3ds;

    // for bulk modulus

    auto [bulk, dkds3] = compute_bulk(s3);

    rowvec3 dkds = dkds3 * ds3ds;

    if(3. * bulk < elastic) {
        bulk = elastic / 3.;
        dkds = deds / 3.;
    }

    return {elastic, bulk, deds, dkds};
}
\end{cppcode}