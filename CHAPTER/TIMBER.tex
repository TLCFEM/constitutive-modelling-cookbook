\chapter{Timber}
\section{TimberPD}
Here we present a 3D model based on the Hoffman yielding criterion and damage mechanics. It is suitable for modelling timber. The model is based on the split of the effective stress $\bbsigma$ such that
\begin{gather}
\bbsigma=\bbsigma_t+\bbsigma_c,
\end{gather}
with
\begin{gather}
\bbsigma_t=\sum_{i=1}^{3}\left\langle\hat{\sigma}_i\right\rangle\bp_i\otimes\bp_i,\qquad
\bbsigma_c=\sum_{i=1}^{3}\left(\hat{\sigma}_i-\left\langle\hat{\sigma}_i\right\rangle\right)\bp_i\otimes\bp_i,
\end{gather}
where $\bp_i$ and $\hat{\sigma}_i$ are eigenvectors and eigenvalues of the second order tensor $\bbsigma$. The above expression shall be interpreted with the tensor notation.

The effective stress $\bbsigma$ obeys the conventional hardening model using the Hoffman yielding criterion, see \secref{sec:hoffman} for details. The final stress can be obtained after applying damage factors on tensile and compressive part of $\bbsigma$.
\begin{gather}
\bsigma=\left(1-\omega_t\right)\bbsigma_t+\left(1-\omega_c\right)\bbsigma_c.
\end{gather}
\subsection{Damage}
The damage part follows the one proposed in \cite{SirumbalZapata2018}. The damage evolution is governed by the equivalent stress $\bar{\tau}_{\aleph}$.
\begin{gather}\label{eq:timber_damage}
\bar{\tau}_{\aleph}=\sqrt{\dfrac{1}{2}\bbsigma_\aleph^\mT\mb{H}_\aleph\bbsigma_\aleph},
\end{gather}
that covers both tensile and compressive cases. The matrix $\mb{H}_\aleph$ is the projection matrix of the Hill criterion, which is a special case of \eqsref{eq:hoffman_criterion}. For $\aleph=t$, set $f_{ii}^c=f_{ii}^t$. For $\aleph=c$, set $f_{ii}^t=f_{ii}^c$. It could be noted that in either case, $\mb{q}=\mb{0}$, thus, \eqsref{eq:timber_damage} does not contain a second term.

The damage variables are updated based on the maximum history of $\bar{\tau}_{\aleph}$, that is
\begin{gather}
r_\aleph=\max_t\bar{\tau}_{\aleph},
\end{gather}
and
\begin{gather}
\omega_t=1-\dfrac{r_{t,0}}{r_t}\left(1-n+n\exp\left(b\left(r_{t,0}-r_t\right)\right)\right),\\
\omega_c=\beta\left(1-\dfrac{r_{c,0}}{r_c}\right)^m.
\end{gather}

The final stress is
\begin{gather}
\bsigma=\left(1-\omega_t\right)\bbsigma_t+\left(1-\omega_c\right)\bbsigma_c.
\end{gather}
\subsection{Consistent Tangent Stiffness}
In the case of activation of damage evolution,
\begin{gather}
\pdfrac{\bsigma}{\bvarepsilon}=\left(\left(\left(1-\omega_t\right)\mb{I}-\bbsigma_t\ddfrac{\omega_t}{r_t}\ddfrac{r_t}{\bbsigma_t}\right)\ddfrac{\bbsigma_t}{\bbsigma}
+\left(\left(1-\omega_c\right)\mb{I}-\bbsigma_c\ddfrac{\omega_c}{r_c}\ddfrac{r_c}{\bbsigma_c}\right)\ddfrac{\bbsigma_c}{\bbsigma}\right)\pdfrac{\bbsigma}{\bvarepsilon}.
\end{gather}
\subsection{Implementation}
\begin{cppcode}
TimberPD::update_trial_status
\end{cppcode}

\begin{cppcode}
    TimberPD::update_damage_t
\end{cppcode}

\begin{cppcode}
    TimberPD::update_damage_c
\end{cppcode}

\begin{cppcode}
    TimberPD::compute_damage_t
\end{cppcode}

\begin{cppcode}
TimberPD::compute_damage_c
\end{cppcode}